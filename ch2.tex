%****************************************************
%	CHAPTER 2 - Prototype Design
%****************************************************
\chapter{Prototype Design}
\label{ch:proto}
%====================================================
\section{Conventions Used}
\label{sec:proto.conventions}
Attitude conventions used for the plants' dynamic derivations in the following Chapter:\ref{ch:dynamics} are briefly discussed here. Often these aspects are omitted or assumed to be known already, it's important to clearly and unambiguously define a standard set frames to avoid uncertainty in the kinematics. Rotation matrices and angles are included briefly for the sake of completeness but the focus is on \emph{contrast} between rotation and transformation operations. Both \cite{spacecraftattitutdequaternions} and \cite{rigidbodylecture} provide an in depth and thorough explanation of rotational matrices and DCM attitude representation if such concepts are unfamiliar to the reader.
%====================================================
\subsection{Reference Frames Convention}
\label{subsec:proto.conventions.frames}
%====================================================
\begin{figure}[htbp]
\centering
\includegraphics[width=0.6\textwidth]{figs/reference_frame}
\caption{Inertial and Body Reference Frames}
\label{fig:ref_frame}
\end{figure}
Regular aerospace (Euler) frames are used for principle inertial and body directions. Shown in Fig: \ref{fig:ref_frame}, the inertial frame,~$\mathcal{F}^i$, is aligned such that the $\vec{X}_i$ axis is in the $\hat{N}$orth direction, $\vec{Y}_i$ is in the $\hat{E}$ast direction and $\vec{Z}_i$ is  in the $\hat{D}$ownward direction\footnote{In orbital sequences this would be toward the Earths' center. Sometimes referred to as the NED convention}. The body frame, $\mathcal{F}^b$, then has both $\vec{X}_b$ and $\vec{Y}_b$ aligned with two perpendicular arms of the quadrotors' body and finally the $\vec{Z}_b$ axis pointing in the body's normal direction. The body frames' axes and their relation to the prototype design are highlighted next in Section:\ref{subsec:proto.conventions.motoraxis}. Frame superscripts $i$ and $b$ represent inertial and body frames respectively. Vector subscripts imply the reference frame in which the vectors' coordinates exists in. 
\par
The relative angular displacement between the two frames is commonly measured by the three angle Euler set. The Euler set $[\psi ~\theta ~\psi]^T$ represents rotations about the $\vec{X}$,$\vec{Y}$ and $\vec{Z}$ axes respectively. Depending on how the rotation sequence is formulated, those angles can be used to construct rotation matrices which give relation to vectors or can transform coordinates. The generic equation for a rotation of a vector $\vec{v}$ about a (normalized) axis $\hat{n}$ by some angle $\mu$ is given by\footnote{Derived in \cite{quaddynamics}}:
\begin{equation}\label{eq:genrotationmatrix}
\vec{v}~'=\big(1-cos(\mu)\big)\big(\vec{v}\cdot \hat{n}\big)\hat{n}+cos(\mu)\vec{v}+sin(\mu)\big(\hat{n}\times\vec{v}\big)
\end{equation}
Which, when $\hat{n}$ is either $\vec{X}$,$\vec{Y}$ or $\vec{Z}$ axes, can be simplified to produce the common rotation matrices $\mathbb{R}_x(\psi)$,$\mathbb{R}_y(\theta)$ and $\mathbb{R}_z(\phi)$. Multiplication by a rotation matrix $\mathbb{R}(\cdot)$ applies a \emph{rotation} operator, the resultant vector still exists in the same reference frame, for a vector $\vec{v}\in\mathcal{F}^i$;
\begin{subequations} \label{eq:rotationoperator}
\begin{equation}\label{eq:rotationoperator.a}
\vec{v}~'=\mathbb{R}_{x}(\psi)\vec{v}
\end{equation}
\vspace{-20pt}
\begin{equation}\label{eq:rotationoperator.b}
\vec{v}~',\vec{v}\in\mathcal{F}^i
\end{equation}
\end{subequations}
\emph{\color{Gray} No subscripts are used in Eq: \ref{eq:rotationoperator} to indicate reference frame ownership because all vectors are in the same frame}
\par
A \emph{transformation} changes the vectors reference frame. The transformation is a rotation by a transformation angle which is difference between the resulting reference frame and the principle reference frame. A transformation from frame $\mathcal{F}^i$ to $\mathcal{F}^b$ by an angle of $\psi$ about the $\vec{X}$ axis is then:
\begin{subequations}\label{eq:transformationoperator}
\begin{equation}\label{eq:transformationoperator.a}
\vec{v}_b=\mathbb{R}_x(-\psi)\vec{v}_i
\end{equation}
\vspace{-20pt}
\begin{equation}\label{eq:transformationoperator.b}
\vec{v}_b\in\mathcal{F}^b~\text{and}~\vec{v}_i\in\mathcal{F}^i
\end{equation}
\end{subequations}
The distinction between Eq:\ref{eq:rotationoperator} and Eq:\ref{eq:transformationoperator} is the sense of the angular operand, and hence the effect it has on the argument vector. The transformation of a vector from $\mathcal{F}^i$ to $\mathcal{F}^b$ is the product of three sequential operations about each of the axes. The sequence of those operations will effect the Euler set and the consequences of chosen order is well documented in \emph{Quaternions and Rotation Sequence}, \cite{rotationsequences}.
 In this dissertation the ZYX sequence is used and so a transformation of a vector $\vec{v}$ from the inertial to the body frame is applied by:
\begin{subequations}
\begin{equation}\label{eq:inertialbodytransformation}
\vec{v}_b=\mathbb{R}_z(-\phi)\mathbb{R}_y(-\theta)\mathbb{R}_x(-\psi)\vec{v}_i
\end{equation}
\end{subequations}
\par
An inherent singularity does exists with such attitude representations. Indeed Quaternions are used later in Sec: \ref{subsec:dynamics.rigidbody.quaternion} in lieu of Euler angles, but they are easily understood and well suited to conventional definitions made here.
\par

\subsection{Motor Axis Layout}
\label{subsec:proto.conventions.motoraxis}
%****************************************************

%****************************************************
\section{Design}
\label{sec:proto.design}
%****************************************************
\subsection{Gimbal Articulation}
\label{subsec:proto.design.actuation}
%****************************************************
\subsection{Inertial Matrix Function}
\label{subsec:proto.design.inertia}
%****************************************************
\subsection{Overall Aspects}
\label{subsec:proto.design.aspects}
%****************************************************
\subsubsection{Vibration Damping}
%****************************************************
\subsubsection{Duct}
%****************************************************
\subsubsection{Landing Skids}
%****************************************************
\subsubsection{Motors \& ESCs}
%****************************************************

%****************************************************
\section{System Layout}
\label{sec:proto.layout}
%****************************************************
