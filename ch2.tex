%****************************************************
%	CHAPTER 2 - Prototype Design
%****************************************************
\chapter{Prototype Design}
\label{ch:proto}
%====================================================
\section{Design}
\label{sec:proto.design}
%====================================================
\begin{figure}[htbp]
\centering
\includegraphics[width=0.93\textwidth]{figs/iso-design}
\caption{Isometric view of the prototype design}
\label{fig:iso-design}
\end{figure}
The final prototype (Fig:\ref{fig:iso-design}) went through a series of different design iterations, all aimed at optimizing engineering time spent on construction and reducing the associated component costs thereof. A significant aspect of consideration for the design process was the net weight whose upper limit, as mentioned before, is inherently limited by the thrust produced from lift motors. Some consequences of the design decisions taken, like inertias \& mass centers are discussed later, where pertinent in the context of multibody dynamics (Section:\ref{subsec:proto.design.inertia}). The reference frame orientations which the rigid body dynamics are developed with respect to are detailed here. A brief overview of the electrical systems layout is given with the components associated and their electrical characteristics included. Finally the actuator suite's functionality and transfer characteristics are also quantified. A review of the physical prototype realized and control loop(s) implemented is detailed in Chapter:\ref{ch:flight}~along with actual flight test results.
\newpage
%====================================================
\subsection{Actuation}
\label{subsec:proto.design.actuation}
%====================================================
The novel component of the design is the manner of articulation for each concentric gimbal ring which forms the motor module structures. The control objective is to produce a thrust vectoring actuation set for a quadrotor's control plant. The outcome was a module which independently redirects the thrust generated by the lift propellers (Fig:\ref{fig:motor-assembly}). Within each module are servos affixed onto sequential support rings to pitch and roll the substructure's axes. The gyroscope-like frame that surrounds each motor/propeller pair accommodates that relative movement. Aligned with each servo is a coaxial support bearing. The bearing and actuator servos have a mass disparity which results in an eccentric center of mass, producing a gravitational torque arm. Unfortunately, due to weight constraints, counter balance measures cannot be introduced. Consequences from the center of mass variations must be either compensated for (\emph{plant dependent solution}) or exploited in the dynamics (\emph{additional non-linear actuator plants}). The precise effects are quantified numerically later in Section:\ref{subsec:proto.design.inertia}.
\par
\begin{figure}[hbtp]
\begin{subfigure}{.5\textwidth}
\centering
\includegraphics[width=\textwidth]{figs/motor-assembly}
\caption{Motor module assembly}
\label{fig:motor-assembly}
\end{subfigure}
\begin{subfigure}{.5\textwidth}
\centering
\includegraphics[width=\textwidth]{figs/motor-support}
\caption{Motor frame damping support assemblies}
\label{fig:motor_support}
\end{subfigure}
\caption{}
\end{figure}
Each motor module is positioned such that its produced thrust vector coincides with the intersection of its two rotational axes. As a result, there's only a perpendicular displacement $\vec{L}_{arm}$ co-planar to the body frames X-Y-Z origin $\vec{\mathbf{O}}_b$ (See subsequent Fig:\ref{fig:motor-frame}). That length directly affects the differential torque plant $\tau_{diff}=\vec{L}_{arm}\times\vec{T}$. An off-center thrust vector line would make that arm displacement a non-orthogonal vector. The center of gravity for each module is time varying and depends on its two servo rotational positions. It's more prudential to ensure intersection of the thrust vector with the rotational center than to balance the masses undergoing rotation. A thrust varying torque is harder to approximate and hence compensate for than a gravitational torque, given the complexity with modeling a propeller's aerodynamic thrust (Section:\ref{subsec:dynamics.aero.bem}).
\par
The primary body structure, similar to a traditional quadcopter '+' configuration, suspends each motor module's rotational assembly with silicon damping balls (Fig:\ref{fig:motor_support}). For damping to be effective there has to be roughly equatable relative masses between the two damped bodies. A smaller damping assembly in the center of the frame houses all the electronics and power distribution circuitry. All the mounting brackets which affix the motor module rings are 3D printed from CAD models using an Ultimaker V2+\cite{ultimaker}. There is a complete bill of materials for all parts used, including working drawings for each 3D printed bracket and the laser cut frame(s), in Appendix:\ref{app:bom}.
\par
\begin{figure}[hbtp]
\centering
\includegraphics[width=0.9\textwidth]{figs/motor-prop}
\caption{Difference between propeller and motor planes}
\label{fig:motor_prop}
\end{figure}
The propellers rotational plane is not aligned exactly with the plane made by the $\hat{X}_{M_i}$ and $\hat{Y}_{M_i}$ rotational servo axes (Fig:\ref{fig:motor_prop}). The offset is approximately 23.39 mm and must be considered when evaluating pitch/roll gyroscopic torque responses later in Section:\ref{subsec:dynamics.nonlinearities.gyrotorques}. The propellers are 6 inch ($6 \times 4.5$) 3-Blade plastic Gemfam propellers, powered by Cobra CM2208-2000KV Brushless DC motors. The thrust produced as a function of angular velocity (in RPS) for the propellers is derived in Section:\ref{subsec:dynamics.aero.bem}. 
\newpage
The BLDC motors are controlled with LDPower 20A ESC\footnote{Flashed with BLHeli\cite{BLHeli} firmware. The default firmware on the speed controllers has an unsatisfactory input response.} modules with an inline OrangeRx RPM Sensor. The transfer function for the combined unit is presented next in Section:\ref{subsec:proto.design.transfer}. Power for the quadrotor is supplied not from a battery bank but from a power tether. Tethered power will ensure consistent flight time and reduce the concern of payload restriction on the available lift actuation. Power lines to both the BLDC motors and servos are supplied through conventional wiring, however an ideal construction would see slip-rings for each module's power supply. 
\begin{figure}[htbp]
\centering
\begin{subfigure}{0.49\textwidth}
\centering
\includegraphics[width=0.9\textwidth]{figs/motor-bldc}
\caption{Cobra CM2208-2000KV BLDC Motor Module}
\label{fig:bldc-motor}
\end{subfigure}
\begin{subfigure}{0.49\textwidth}
\centering
\includegraphics[width=0.9\textwidth]{figs/motor-servo}
\caption{Corona DS-339MG Servo Bracket}
\label{fig:motor-servo}
\end{subfigure}
\end{figure}
\par
Metal gear Corona DS-339MG digital servos are used for the two axes of rotation (Fig:\ref{fig:motor-servo}). Each servo has a rotational range of $\approx\pi$, positioned such that a $\text{zero}^{\text{th}}$ offset aligns the motor modules, adjacent to the body frame, and has a $\pm\pi/2$ range. A digital servo updates at 330 Hz, faster than a 50 Hz analogue servo equivalent (Table:\ref{tab:servo}). This means the otherwise $20$ms zero-order "analogue" sampling becomes a less significant $3.30$ms zero-order holding time. Both the $\hat{X}_{M_i}$ and $\hat{Y}_{M_i}$ axis servos will be rotating a large loading mass and so their \emph{open loop} plant dynamics are determined empirically in Section:\ref{subsec:proto.design.transfer}.
\begin{table}[h]
\centering
\fbox{
\begin{minipage}{0.7\textwidth}
\begin{tikztimingtable}
50 Hz: &[C] 31{C} G\\
Analogue Servo &1{L} 1.5{H} 18.5{L} 1.5{H} 10{L}\\
Digital Servo&1{L} 15{1.5{H} 1.53{L}}\\
\end{tikztimingtable}
\end{minipage}
}
\caption{Analogue \& Digital Timing Signals}
\label{tab:servo}
\end{table}
%====================================================
\section{Conventions Used}
\label{sec:proto.conventions}
The attitude conventions used for the system's dynamic derivations, in the next Chapter:\ref{ch:dynamics},~are first briefly discussed here. Often these aspects are assumed to be obvious enough that they're omitted. It's important to clearly and unambiguously define a standard set of framing conventions to avoid uncertainty later. Rotation matrices are included but the focus is on the \emph{contrast} between a rotation and transformation operation. Both \cite{spacecraftattitutdequaternions} and \cite{rigidbodylecture} provide an in depth and thorough explanation of rotation matrices and DCM attitude representation if such concepts are unfamiliar to the reader. Quaternions are introduced to replace rotation matrices in Section:\ref{subsec:dynamics.rigidbody.quaternion}.
%====================================================
\subsection{Reference Frames Convention}
\label{subsec:proto.conventions.frames}
%====================================================
\begin{figure}[htbp]
\centering
\includegraphics[width=0.65\textwidth]{figs/reference-frame}
\caption{Inertial and Body Reference Frames}
\label{fig:ref_frame}
\end{figure}
Euler (aerospace) frames are used for principle cartesian inertial and body coordinate representation (Fig:\ref{fig:ref_frame}). The inertial frame,~$\mathcal{F}^I$, is aligned such that the $\hat{X}_I$ axis is in the $\hat{N}$orth direction, $\hat{Y}_I$ is in the $\hat{E}$ast direction and $\hat{Z}_I$ is  in the $\hat{D}$ownward direction\footnote{In orbital sequences this would be toward the Earth's center. Sometimes referred to as the N\^{E}D convention}. The body frame, $\mathcal{F}^b$, then has both $\hat{X}_b$ and $\hat{Y}_b$ aligned obliquely between two perpendicular arms of the quadrotor's body and the $\hat{Z}_b$ axis in the body's normal direction (Fig:\ref{fig:body-frame}). The body frame's axes and their relation to the prototype design are highlighted next in Section:\ref{subsec:proto.conventions.motoraxis}. Frame superscripts $I$ and $b$ represent inertial and body frames respectively whilst vector subscripts imply the reference frame in which the vector's coordinates exists or taken relative to.
\par
Relative angular displacement between two frames is commonly measured by the three angle Euler set. The Euler angles $\vec{\eta}=[\phi ~\theta ~\psi]^T$ represent rotations about the $\hat{X}$,$\hat{Y}$ and $\hat{Z}$ axes respectively. Depending on how the rotation sequence is formulated, those angles can be used to construct rotation matrices which give relation to vectors or can transform coordinates. The generic equation to \emph{rotate} a vector $\vec{v}$ about some (normalized) axis $\hat{n}$ by some angle $\mu$ is given by\footnote{Derived and proven in \emph{Quadrotor Dynamics and Control}\cite{quaddynamics}}:
\begin{equation}\label{eq:genrotationmatrix}
\vec{v}~'=\big(1-cos(\mu)\big)\big(\vec{v}\cdot \hat{n}\big)\hat{n}+cos(\mu)\vec{v}+sin(\mu)\big(\hat{n}\times\vec{v}\big)
\end{equation}
Which, when $\hat{n}$ is either $\hat{X}$,$\hat{Y}$ or $\hat{Z}$ axes, can be simplified to produce the fundamental rotation matrices $\mathbb{R}_x(\phi)$,$\mathbb{R}_y(\theta)$ and $\mathbb{R}_z(\psi)$.
\newpage
Multiplication by a rotation matrix $\mathbb{R}(\cdot)$ applies a left-handed \emph{rotation} operator, the resultant vector still exists in the same reference frame. An $\hat{X}$ axis rotation by $\phi$ is;
\begin{subequations} \label{eq:rotationoperator}
\begin{equation}\label{eq:rotationoperator.a}
\vec{v}~'=\mathbb{R}_{x}(\phi)\vec{v}
\end{equation}
\vspace{-15pt}
\begin{equation}\label{eq:rotationoperator.b}
\vec{v}~',\vec{v}\in\mathcal{F}^1
\end{equation}
\end{subequations}
\emph{\color{Gray} No subscripts are used in Eq: \ref{eq:rotationoperator} to indicate reference frame ownership because all vectors are in the same frame}
\par
A vector \emph{transformation} changes the resultant vector's reference frame. The transformation is then a rotation by an angle of the \emph{difference} between the resulting and principle reference frames. A transformation from frame $\mathcal{F}^1$ to $\mathcal{F}^2$, differing by an angle of $\phi$ about the $\hat{X}$ axis is then:
\begin{subequations}\label{eq:transformationoperator}
\begin{equation}\label{eq:transformationoperator.a}
\vec{v}_2=\mathbb{R}_x(-\phi)\vec{v}_1
\end{equation}
\vspace{-15pt}
\begin{equation}\label{eq:transformationoperator.b}
\vec{v}_2\in\mathcal{F}^2~\text{and}~\vec{v}_1\in\mathcal{F}^1
\end{equation}
\end{subequations}
The distinction between Eq:\ref{eq:rotationoperator} and Eq:\ref{eq:transformationoperator} is the directional sense of the angular operand $\phi$, and hence the effect it has on the argument vector. The transformation or rotation of a vector from $\mathcal{F}^I$ to $\mathcal{F}^b$ is the product of three sequential operations about each axis. Each subsequent rotation is applied relative to a new intermediate frame; hence each Euler angle is taken relative to a specific intermediate frame. The sequence of axial rotation operations does indeed effect the Euler set. Any consequences of that chosen order is something discussed indepth in \emph{Quaternions and Rotation Sequence}, \cite{rotationsequences}. In this dissertation the Z-Y-X sequence is used. A transformation of the vector $\vec{v}$ from the inertial to the body frame, $\mathcal{F}^I\rightarrow\mathcal{F}^b$, is then applied by:
\\
\begin{subequations}
\begin{equation}\label{eq:inertialbodytransformation.a}
\mathbb{R}_{I}^{b}\triangleq\mathbb{R}_z(\psi)\mathbb{R}_y(\theta)\mathbb{R}_x(\phi)
\end{equation}
\vspace{-10pt}
\begin{equation}\label{eq:inertialbodytransformation.b}
\vec{v}_b=\mathbb{R}_I^b(-\psi,-\theta,-\phi)\vec{v}_I
\end{equation}
\vspace{-10pt}
\begin{equation}\label{eq:inertialbodytransformation.c}
\Rightarrow\vec{v}_b=\mathbb{R}_z(-\psi)\mathbb{R}_y(-\theta)\mathbb{R}_x(-\phi)\vec{v}_I
\end{equation}
\vspace{-10pt}
\begin{equation} \label{eq:inertialbodytransformation.d}
\mathbb{R}_z(-\psi)\mathbb{R}_y(-\theta)\mathbb{R}_x(-\phi) \iff \mathbb{R}_x(\phi)\mathbb{R}_y(\theta)\mathbb{R}_z(\psi)=\mathbb{R}_{b}^{I}
\end{equation}
\vspace{-10pt}
\begin{equation}\label{eq:inertialbodytransformation.e}
\mathbb{R}_I^b=\big(\mathbb{R}_b^I\big)^{-1}=\big(\mathbb{R}_b^I\big)^T
\end{equation}
\end{subequations}
The relationship in Eq:\ref{eq:inertialbodytransformation.e} is an inversion property (\emph{transpose}) of the rotation matrix. A rotation matrix's inverse can be used interchangeably with its negative counterpart to maintain a positive sense of the argument angle. To ensure clarity throughout this dissertation's mathematics, a negative angular sense implies a \emph{transformation} to a different reference frame. Where applicable, the order of rotation will indicate the sequence direction and an angular sign differentiates the rotation and transformation operators.
\par
The body frame's angular velocity is taken relative to the inertial frame, represented by $\vec{\omega}_{b/I}\Rightarrow \vec{\omega}_b$. Seeing that each Euler angle is measured with respect to an intermediary frame, a distinction must then be made between $d\vec{\eta}/dt$ and $\vec{\omega}_b$. All three Euler angles need to be transformed to one common frame. Exploiting vehicle frames 1 \& 2, or rather $\mathcal{F}^{v1}$ \& $\mathcal{F}^{v2}$, as intermediate frames to respectively describe post $\mathbb{R}_x(\phi)$ and $\mathbb{R}_y(\theta)$ operations.
\begin{subequations}
\begin{equation}\label{eq:angular-rates.a}
\vec{\omega}_b=\frac{d}{dt_b}\vec{\eta}=\frac{d\phi}{dt}\mathbb{R}_{v2}^b(\phi)\begin{bmatrix}
\phi\\
0\\
0\\
\end{bmatrix}
+
\frac{d\theta}{dt}\mathbb{R}_{v2}^{b}(\phi)\mathbb{R}_{v1}^{v2}(\theta)\begin{bmatrix}
0\\
\theta\\
0
\end{bmatrix}
+
\frac{d\psi}{dt}\mathbb{R}_{v2}^{b}(\phi)\mathbb{R}_{v1}^{v2}(\theta)\mathbb{R}_{I}^{v1}(\psi)\begin{bmatrix}
0\\
0\\
\psi
\end{bmatrix}
\end{equation}
\emph{\color{Gray}The vehicle frames in Eq:\ref{eq:angular-rates.a} and the subsequent rotations between each frame don't necessarily have to be in that order. The equation could change depending on what rotation sequence was used.}
\newpage
Which then simplifies to the formal relationship between two rotating frames, with $\vec{\omega}_b=[p~q~r]^T$ in $rad.s^{-1}$:
\begin{equation}\label{eq:angular-rates.b}
\begin{bmatrix}
p\\
q\\
r\\
\end{bmatrix}
=
\begin{bmatrix}
1 & 0 & -sin(\theta)\\
0 & cos(\phi) & sin(\phi)cos(\theta)\\
0 & -sin(\theta) & cos(\phi)sin(\theta)\\
\end{bmatrix}
\begin{bmatrix}
\dot{\phi}\\
\dot{\theta}\\
\dot{\psi}\\
\end{bmatrix}
\end{equation}
\vspace{-10pt}
\begin{equation}\label{eq:angular-rates.c}
\Rightarrow\vec{\omega}_b=\Psi(\eta)\dot{\eta}
\end{equation}
\vspace{-10pt}
\begin{equation}\label{eq:angular-rates.d}
\Psi(\eta)=
\begin{bmatrix}
1 & 0 & -sin(\theta)\\
0 & cos(\phi) & sin(\phi)cos(\theta)\\
0 & -sin(\theta) & cos(\phi)sin(\theta)\\
\end{bmatrix}
\end{equation}
\vspace{-5pt}
\begin{equation}\label{eq:angular-rates.e}
\Rightarrow\dot{\eta}=\Psi^{-1}(\eta)\vec{\omega}_b=\Phi(\eta)\vec{\omega}_b
\end{equation}
\vspace{-10pt}
\begin{equation}\label{eq:angular-rates.f}
\Phi(\mathcal{E})=\begin{bmatrix}
1 & sin(\phi)tan(\theta) & cos(\phi)tan(\theta)\\
0 & cos(\phi) & -sin(\phi)\\
0 & sin(\phi)sec(\theta) & cos(\phi)sec(\theta)\\
\end{bmatrix}
\end{equation}
\end{subequations}
\par
The termed \emph{Euler} matrix, $\Phi(\eta)$, contains a well known and problematic singularity at $\theta=\pm\pi/2$; because $tan(\theta),sec(\theta)\rightarrow\infty$ as $\theta\rightarrow\pi/2$. The effect of the rotation matrix singularity is further explored later in Section:\ref{subsec:dynamics.rigidbody.singularity}. Its manifestation in the $\theta$ angle here is a direct consequence of the Z-Y-X sequence used. Each Euler angle can potentially suffer a singularity depending on how the rotations are sequenced. Indeed quaternions are used for kinematics later in lieu of Euler angles. Euler angular attitude representation is, however, easily understood and well suited to the conventional distinctions made in this Chapter.
\par
Quaternion operations are similarly sequenced in the Z-Y-X order:
\begin{subequations}
\begin{equation}\label{eq:quaternion-rotation-equivalence}
\mathbb{R}_I^b\iff Q_b \otimes (.) \otimes Q_b^*
\end{equation}
\vspace{-15pt}
\begin{equation}
Q_b \triangleq Q_z Q_y Q_x~\text{and}~Q_b \triangleq Q_x^* Q_y^* Q_z^*
\end{equation}
\end{subequations}
With $\otimes$ being the Hamilton product (or quaternion multiplication). Each quaternion, $Q_i$, is a unit quaternion about that $\hat{i}^{th}$ axis. It is important to note that a quaternion rotation operates on an argument vector with a zero quaternion scalar component. So then for some vector $\vec{v}$, the quaternion rotation operation in Eq:\ref{eq:quaternion-rotation-equivalence} is equivalent to;
\begin{subequations}
\begin{equation}\label{eq:quaternion-operator.a}
Q_{\vec{v}}~'=Q \otimes (Q_{\vec{v}}) \otimes Q^*
\end{equation}
\vspace{-10pt}
\begin{equation}\label{eq:quaternion-operator.b}
\text{Where}~Q_{\vec{v}}=\begin{bmatrix}
0\\
\vec{v}\\
\end{bmatrix},~Q_{\vec{v}~'}=\begin{bmatrix}
0\\
\vec{v}~'\\
\end{bmatrix}
\end{equation}
\end{subequations}
The quaternion representation in Eq:\ref{eq:quaternion-operator.b} ensures that the operation is entirely in $\mathbb{R}^4$ space. However it is usually omitted, despite $\mathbb{R}^4$ being implied and as such, Eq:\ref{eq:quaternion-operator.a} is then simply:
\begin{equation}
\vec{v}~'=Q \otimes (\vec{v}) \otimes Q^*
\end{equation}
Quaternion dynamics, and the quaternion operator, are later expanded upon to replace the use of Euler angles and Rotation matrices as a convention for attitude representation later in Chapter:\ref{ch:dynamics}
\newpage
%====================================================
\subsection{Motor Axis Layout}
\label{subsec:proto.conventions.motoraxis}
%====================================================
Fundamentally the whole structure, although treated as fixed and rigid in the kinematics, consists of multiple rigid bodies with relative rotations to one another, illustrated previously in the design description in Section:\ref{sec:proto.design}. Those rigid bodies are divided into four inter-connected motor modules and a single body (\emph{frame}) structure. Each module consists of two sequential gimbal rings, each with one degree of relative rotation between itself and the next subsequent ring. There needs to be distinct nomenclature used for describing these motor modules. 
\begin{figure}[htbp]
\centering
\includegraphics[width=0.85\textwidth]{figs/motor-axes}
\caption{Aligned Motor Frame Axes}
\label{fig:motor-axes}
\end{figure}
\par
Every propeller/motor pair is actuated by two servos. The $i^{th}$ propeller, directly driven by the motor's rotor, has a rotational speed $\Omega_i~[RPS]$ about the $\hat{Z}$ stator axis. Two servos are aligned \emph{at rest} with $\hat{Y}$ and $\hat{X}$ axes to pitch and roll the propeller away from its principle rotational axis. Each motor has its own reference frame, $\mathcal{F}^{M_i}$, aligned in Fig:\ref{fig:motor-axes} and highlighted with the rotational rings in Fig:\ref{fig:motor-frame}.
\par
\begin{minipage}{\textwidth}
\begin{wrapfigure}{r}{0.55\textwidth}
\centering
\includegraphics[width=0.55\textwidth]{figs/motor-frame}
\caption{Intermediate Motor Frames}
\label{fig:motor-frame}
\end{wrapfigure}
Motor frames, numbered $1-4$, transform to the body frame first by an angle of $\lambda_i$ about the $\hat{X}_{M_i}$ axis. Then by $\alpha_i$ about the $\hat{Y}_{M_i'}$ axis in an intermediate $M_i'$ frame. The first servo actuates $\lambda_i$, rotating $\mathcal{F}^{M_i}$ to an intermediate $\mathcal{F}^{M_i'}$ frame. Secondly, the next servo actuates $\alpha_i$ to produce a second intermediate frame $M_i''$. That second servo is affixed in the $M_i''$ frame. Lastly there's a relative orthogonal rotation about $\hat{Z}_{M_i''}$ between $\mathcal{F}^b$ and $\mathcal{F}^{M_i''}$. Each module's actuation state is fully described by $[\Omega_{i},~\lambda_{i},~\alpha_{i}]^{T}$ for $i\in [1:4]$. The four motor modules are aligned relative to the body's XYZ axes as shown in Fig:\ref{fig:body-frame}. Modules 1 and 3 have their X-axes in the positive and negative $\hat{X}$ directions of the body frame respectively. Similarly Modules 2 and 4 have their X-axes in the positive and negative $\hat{Y}$ directions of the body frame.
\end{minipage}
\newpage
\begin{figure}[htbp]
\centering
\includegraphics[width=0.9\textwidth]{figs/body-frame}
\caption{Body Frame Axes Layout}
\label{fig:body-frame}
\end{figure}
\par
\emph{\color{Gray}Not shown in Fig:\ref{fig:body-frame} is the relative $\hat{Z}$ axis position with respect to the structure. The $\hat{Z}$ height of the body's motion centroid is such that its origin is co-planar with the four motor modules rotational centers. The center of motion is \underline{not} the center of mass, an aspect which is investigated next in Section:\ref{subsec:proto.design.inertia}.}
\\
Vector transformations from each of the four motor frames to the body frame are characterized as:
\begin{subequations}
\begin{equation}\label{eq:motor-module-rotation.a}
\vec{v}_b=\mathbb{R}_z(-\sigma_i)\mathbb{R}_y(-\alpha_i)\mathbb{R}_x(-\lambda_i)\vec{v}_{M_i},~~\sigma_i\in\frac{1}{2}[0,~\pi,~2\pi,~3\pi]
\end{equation}
With orthogonal rotation matrices $\mathbb{R}_z$:
\begin{equation}\label{eq:motor-module-rotation.b}
\mathbb{R}_z=\begin{bmatrix}
1 & 0 & 0\\
0 & 1 & 0\\
0 & 0 & 1
\end{bmatrix}, \begin{bmatrix}
0 & -1 & 0\\
1 & 0 & 0\\
0 & 0 & 1
\end{bmatrix}, \begin{bmatrix}
-1 & 0 & 0\\
0 & -1 & 0\\
0 & 0 & 1
\end{bmatrix}, \begin{bmatrix}
0 & 1 & 0\\
-1 & 0 & 0\\
0 & 0 & 1
\end{bmatrix}~\text{for}~i\in[1,2,3,4]~\text{respectively}
\end{equation}
\label{eq:motor-module-rotation}
\end{subequations}
\\
The entire actuator space, including propeller speed $\Omega_i~[RPS]$, is then $\in\mathbb{R}^{12}$, or rather $\mathbb{U}\in\mathbb{R}^{12}$, in contrast with $\mathbb{U}\in\mathbb{R}^4$ for a normal quadrotor. The actuator input set $u \in \mathbb{U}$ is then structured as:
\begin{equation}
u_{\in\mathbb{U}}=\big[ \Omega_{1} ~ \lambda_{1} ~ \alpha_{1} ~ \ldots ~ \Omega_{4} ~ \lambda_{4} ~ \alpha_{4}  \big]^T
\end{equation}
\newpage
\section{Electronics}
\label{sec:proto.layout}
%====================================================
{\centering
\fbox{
\begin{minipage}{\textwidth}
\centering
\includegraphics[width=0.95\textwidth]{pdfpages/electrical-schematic.pdf}
\label{fig:electrical-schematic}
\end{minipage}
}
\captionof{figure}{Hardware Schematic Diagram}
}
%-----------------------------------------------------
\newpage
%-----------------------------------------------------
An abstracted hardware diagram for the (electronic) system layout is shown in Fig:\ref{fig:electrical-schematic}. It's an illustration for the connection of different electronic peripherals used to aid the on-board control system. The structure of the autopilot system and control loops are addressed later in Chapter:\ref{ch:flight}. This section aims to provide a brief overview of the specific modules used, their purpose and a description of how they're interfaced. No code structure or control loops are considered yet\ldots
\par
\begin{figure}[htbp]
\begin{subfigure}{0.5\textwidth}
\centering
\includegraphics[width=0.9\textwidth]{figs/f3-deluxe}
\caption{SPRacing F3 Deluxe Flight Controller}
\end{subfigure}
\begin{subfigure}{0.5\textwidth}
\centering
\includegraphics[width=0.9\textwidth]{figs/ppm-sbus}
\caption{SBUS Converter \& 6CH Receiver Modules}
\end{subfigure}
\caption{}
\end{figure}
The entire system is centered around an ARM STM32F303\cite{stm32f303} based $\mu$C. The micro-processor board is a commercial flight control board, specifically an SPRacing F3 Deluxe\cite{spracing}\footnote{CleanFlight opensource software is regularly used for the F3 but its hardware specifications are not openly avaiable.\\The reverse engineered electrical schematic for the board is included in Appendix:\ref{app:deluxe-diagram}}, which has had its bootloader removed and custom firmware, unique to this project, developed for it. That software is later described in Chapter:\ref{ch:flight}; the I/O for all the peripherals are detailed here however. The flight-controller has the following onboard peripherls: an I2C MPU-6050\cite{mpu6050} 6-axis gyroscope \& accelerometer with a connected HMC5883\cite{hmc5883} magnetometer compass, an SPI MS5611\cite{ms5611} barometer and similarly 64 Mb of SPI flash memory. The electrical schematic of those peripherals and the core STM32F303 is detailed in Appendix:\ref{app:deluxe-diagram} but their connection(s) are shown in Fig:\ref{fig:electrical-schematic}. 
\\
\emph{\color{Gray} The combination of above sensors fused for state estimation and their associated algorithms are dealt with in Section:\ref{sec:simulation.state} in Chapter:\ref{ch:simulation}.}
\begin{figure}[hbtp]
\centering
\includegraphics[width=\textwidth]{figs/sbus}
\caption{S.BUS Data Stream}
\label{fig:sbus}
\end{figure}
\\
Two wireless communication peripherals are used. First the system relays full state information, for a complete 6-DOF autopilot system, from a ground control station using 2.4 GHz XBEE S1 module(s)\cite{xbees1}, USART connected. Secondly, an augmented pilot control input system, fail safe and secondary to the autopilot loop, is transmitted through 6 Channel 2.4 GHz R/F comms. The 6 CH received signals, otherwise permeated as six individual 20 KHz PWM signals via an OrangeRx R615x\cite{r615x} receiver, are encoded into a single proprietary S.BUS data stream. 
\newpage
The need for an S.BUS encoder \cite{sbusencoder} comes about as a consequence of the introduction of the 8 additional servos. As a result, there are no longer 6 free additional timer I/O channels which can be allocated for input capture. Encoding the received data to a serial communication protocol means the 6CH data can be processed on a single serial RX line. The S.Bus encoder implements a USART derivative communications standard, Fig:\ref{fig:sbus} shows the sampled data stream used to ascertain the standard's following parameters:
\par
\begin{tabularx}{\textwidth}{X X}
\begin{minipage}{\textwidth}
\begin{itemize}[itemsep=0em]
\item 25 Bytes per packet
\item 8-Bit byte length
\item 1 Start byte 0x240
\item 1 Byte of flags
\item 1 Stop byte 0x0
\item Bytes are:
\vspace{-5pt}
\begin{itemize}[itemsep=0em]
\item MSB First
\item 1 start \& 2 stop bits
\item Even parity bit
\item Inverted
\item 100000 baud (bps)
\end{itemize}
\vspace{-5pt}
\end{itemize}
\end{minipage}
&
\begin{minipage}{\textwidth}
\begin{itemize}[itemsep=0em]
\item 22 bytes of CH data 
\item Each channel's data is 11 bits long
\item 16CH encoded
\item Channel data is little endian prioritized
\item 14 ms idle time between packets
\item Packets are arranged:
\end{itemize}
{
$\overbrace{[0x240]}^{Start~byte}\overbrace{[8B_1][3B_2}^{CH1}|\overbrace{5B_2][6B_3}^{CH2}|\overbrace{2B_3][8B_4][1B_5}^{CH3}|\ldots$
\\
$\overbrace{7B_5][4B_6|}^{CH4}~\longrightarrow~\overbrace{3B_22][8B_23]}^{CH16}\overbrace{[8B_24]}^{Flags}\overbrace{[0x00]}^{Stop~byte}$
}
\end{minipage}
\\
\end{tabularx}
\par
{\color{red}
The received information from the transmitted 6 channels is filtered through a moving average IIR$^\dagger$ filter. The filters difference equation is as follows: 
\begin{equation}
y_n = \big(1-\frac{1}{N}\big)y_{n-1}+\frac{1}{N}x_n
\end{equation}
Moving over an average of $N=5$ samples. The signal's sample rate is sufficiently fast enough such that the digital filter's frequency response isn't of consequence. Similarly all the measured RPM signals are filtered as well. Any received signals referred to are all post filtration. Filtering for state estimates is separately performed on the Ground Control Station computer.}
\par
Each of the eight digital servo actuators are driven individually from 330 Hz PWM timer output compare channels (TIM2:CH1$\rightarrow$CH4 \& TIM3:CH1$\rightarrow$CH4). Output pulses typically range from 1ms - 2ms to linearly control the rotary position. The exact range and transfer function is empirically determined next in Subsection:\ref{subsec:proto.design.transfer}. The four 20A brushless DC speed controllers (ESCs) are each driven from a 20 Hz PWM output (TIM4:CH1$\rightarrow$CH4), similarly with 1ms - 2ms pulse widths. There are a total of 12 PWM output compare signals drawn from the $\mu$C. Servos are powered by a regulated 6V DC 10A power supply \cite{rotorstar} whilst the ESCs switch unregulated 15.1 V DC from an externally tethered power supply. The DC supply could potentially be drawn from an on-board battery bank but that would add significant weight to an already heavy platform.
\par
There's no integrated feedback for instantaneous RPM values from the ESCs. Using discrete OrangeRX BLDC RPM sensors \cite{orangerpm}, that measure switching phases across two of the three motor phases, the exact RPM can be ascertained. The switching signal of a 3-Phase \footnote{Although called BLDC motors, the motors are actually 3-Phase IM motors which, when combined with an ESC, behave in closed loop like BLDC motors.} is\cite{.}:
\begin{equation}
F_{rps}=\frac{2\times F_{poles}}{\text{No. of rotor poles}}~~[Hz]
\end{equation}
The signal produced by the RPM sensors varies a 50\% duty cycle square wave's period, directly proportional to that pole switching frequency. The RPM sensor's output signal is then calibrated to a gain of 7 for the 14 pole BLDC Cobra motors used. That gain is verified with the linear relationship(s) is shown in Figs:\ref{fig:rpm-sensor}. Knowing exact RPM rates means the subsequent thrust and aerodynamic torques for the control plant inputs can be calculated.
\newpage
\begin{figure}[htbp]
\begin{subfigure}{0.5\textwidth}
\centering
\includegraphics[width=\textwidth]{graphs/rpm-sensor-noload}
\caption{RPM sensor plot - no load}
\label{fig:rpm-sensor-noload}
\end{subfigure}
\begin{subfigure}{0.5\textwidth}
\centering
\includegraphics[width=\textwidth]{graphs/rpm-sensor-prop}
\caption{RPM sensor plot - 6" prop}
\label{fig:rpm-sensor-prop}
\end{subfigure}
\caption{}
\label{fig:rpm-sensor}
\end{figure}
\par
The speed controllers, although LDPower 20A devices, are all re-flashed with BLHeli\footnote{LDPower 20A ESCs(Fig:\ref{fig:ldpower-20A}) match Hobbywing Xrotor 20A speed controllers (Fig:\ref{fig:xrotor-20A}), they both use SiLabs F396 MCUs. Physical rotational values in the plots Fig:\ref{fig:rpm-sensor} were measured with optical encoders.}\cite{BLHeli} firmware. The custom software on the ESC's $\mu$controller can provide greater refinement over parameter configuration like the deflection range of inputs, however, default values were used. The plot in Fig:\ref{fig:rpm-sensor-noload} shows the linear RPS (in Hz) speed line for an unloaded motor, similarly in Fig:\ref{fig:rpm-sensor-prop} shows the speed curve when loaded for a 6-inch prop. It's interesting to note that the loaded speed curve is slightly parabolic, this is from the aerodynamic drag term which is quadratic with respect to rotational velocity, Section:\ref{subsec:dynamics.aero.bem}. Moreover, when the motor is torque loaded by the propeller, the ESC current limits rotational speeds at just over 13 000 RPM. The sensor feedback is used for minor loop RPM control.
\begin{figure}[hbtp]
\begin{subfigure}{0.5\textwidth}
\centering
\includegraphics[width=0.9\textwidth]{figs/xrotor-20A}
\caption{XRotor 20A ESC connection guide\cite{xrotor}}
\label{fig:xrotor-20A}
\end{subfigure}
\begin{subfigure}{0.5\textwidth}
\centering
\includegraphics[width=0.9\textwidth]{figs/ldpower-20A}
\caption{LDPower 20A ESC}
\label{fig:ldpower-20A}
\end{subfigure}
\caption{}
\end{figure}
\\
Timers channels are used to measure the varying frequency of the RPM sensors. General purpose Timers 15 (TIM15:CH1$\rightarrow$CH2), 16 (TIM16:CH1) and 17 (TIM17:CH1) are configured to capture the input PWM signal generated by the speed sensors. Included on the I2C communciation line is an I2C O-LED display for debugging and status update purposes.
\par
Any STM32 $\mu$controller is programmed through a dedicated debugging device. The ST-Link V2\cite{st-link} is the current proprietary device which, itself, is a specially programmed STM32F10 chip. The chip connects to the dedicated \textbf{S}erial \textbf{W}ire \textbf{D}ebugging ports of the target STM (\emph{SWD-CLK, SWD-IO} \& \emph{SWD-NRST}) and is interfaced via regular USBD+ and USBD- data lines. 
%====================================================
\subsection{Actuator Transfer Functions}
\label{subsec:proto.design.transfer}
%====================================================
\subsubsection*{Servo Transfer Functions}
%====================================================
The full scale deflection of each digital servo is in fact greater than its quoted 180\textdegree ~range. Each servo has a rotational range of around 230\textdegree ~(Fig:\ref{fig:servo-range}). The exact characteristics for every servo differ slightly and thus individual transfer functions for each of the 8 servos are used in simulation. In the prototype control loop the servos are left in open loop; the major loop controller coefficients are expected to account for minor loop actuator dynamics. With that being said, for such an expectation the simulation would need to accurately represent the servo's response. Seeing that the 180\textdegree ~limitation was imposed as a design decision; one of the first points of contention is the effect such a constraint would have on the feasible operating trajectories. The control algorithms developed in Chapter:\ref{ch:control} are first tested with an ideal, continuous rotation servo actuator with similar rate limits and transfer characteristics, later the servo limitations are introduced.
\begin{figure}[htbp]
\centering
\begin{subfigure}{0.49\textwidth}
\centering
\includegraphics[width=\textwidth]{graphs/servo-range}
\caption{DS339-MG Full Range}
\label{fig:servo-range}
\end{subfigure}
\begin{subfigure}{0.49\textwidth}
\centering
\includegraphics[width=\textwidth]{graphs/servo-step}
\caption{DS339-MG Step Response}
\label{fig:servo-step}
\end{subfigure}
\caption{}
\label{fig:servo-no-load}
\end{figure}
\par
For the servo \footnote{Servo $\lambda_1$} whose data range and response are shown in Fig:\ref{fig:servo-no-load}, the relationship the input pulse-width $x~[m.s]$ and the rotational output position $y~degrees$ is given by:
\begin{equation}\label{eq:servo-range}
y(x)=
\begin{cases}\begin{array}{ll}
0\text{\textdegree} & ~~x<0.65~m.s\\
129.12x-82.64 & ~~0.64~m.s \leq x \leq 2.46~m.s\\
230\text{\textdegree} & ~~x>2.46~m.s\\
\end{array}
\end{cases}
\end{equation}\par
\begin{figure}[hbtp]
\centering
\includegraphics[width=0.8\textwidth]{figs/servo-block}
\caption{Servo block diagram}
\end{figure}
Although, in practice, the equation Eq:\ref{eq:servo-range} is changed such that 0\textdegree ~is taken at around a 50\% input and the operational range is $\pm 90$\textdegree . Each servo is mechanically rate limited to $60\text{\textdegree}/0.15s$ or $400 RPS$ with a dead time of $1.2~m.s$ and a mechanical deadband of $\leq4\mu s$. The net transfer block for the servo then has the following structure, including non-linearities but neglecting the deadband;
\par
Each servo has an approximate (\emph{critically damped}) second order transfer function$^{\dagger}$:
\begin{subequations}\label{eq:servo-transfer}
\begin{equation}
G_{servo}(s)=e^{-t_d s}\frac{w_n^2}{s^2+2\zeta w_n s + w_n^2}
=e^{-0.012s}\frac{(15.717)^2}{s^2+2(1)(15.717)+(15.717)^2}
\end{equation}
\begin{equation}
Y_{servo}(s)=
\begin{cases}\begin{array}{ll}
0\text{\textdegree} & ~~|U(s)|<0.65\\
G(s) & 0.65 \leq |U(s)| \leq 2.46\\
230\text{\textdegree} & ~~|U(s)|>2.46\\
\end{array}
\end{cases}
\end{equation}
\end{subequations}
\par
The plot in Fig:\ref{fig:servo-step} is that of an unloaded servo's response. When loaded by an inner ring assembly (Fig:\ref{fig:servo-step-inner}) the plant response {\color{Blue}$\mathbf{y(t)}$} is consistent with Eq:\ref{eq:servo-transfer}. Despite rotating a load mass and hence requiring a greater torque, the servo remains unchanged, even when the BLDC motor (with a 6$\times$4.5" prop) is spun an average rate of 6500 RPM, {\color{Red}$\mathbf{y'(t)}$}, further loading the assembly.
\begin{figure}[hbtp]
\begin{subfigure}{0.5\textwidth}
\centering
\includegraphics[width=0.98\textwidth]{graphs/servo-step-inner}
\caption{Inner Ring Servo Response}
\label{fig:servo-step-inner}
\end{subfigure}
\begin{subfigure}{0.5\textwidth}
\centering
\includegraphics[width=0.98\textwidth]{graphs/servo-step-middle}
\caption{Middle Ring Servo Response}
\label{fig:servo-step-middle}
\end{subfigure}
\caption{}
\end{figure}
\par
However, in Fig:\ref{fig:servo-step-middle} the mechanical response remains the same but oscillations are introduced by the larger mass being driven. These are product of the structure's flex within the middle ring assembly\footnote{The rotational position was measured with respect to the bearing supported output shaft, coaxial to the servos, and \emph{not} the servo's output shaft}, \emph{not from the servo plant}. The oscillations are still present under load, {\color{Red}$\mathbf{y'(t)}$}, despite the frame being tensioned by a thrust vector. The harmonics can be reduced by either introducing a more rigid sub-frame, limiting the maximum angular rate or applying a damping minor loop controller. The latter would be a \emph{virtual} closed loop with an approximated error rate as the prototype doesn't support position feedback for each motor module.
\newpage
\subsubsection*{BLDC Transfer Functions}
Each Cobra 2208 BLDC motor, when loaded with a 6$\times$4.5 propeller has a quadratic speed curve, Fig:\ref{fig:bldc-range}. This is as a result of the propeller's opposing aerodynamic drag, \emph{appromixately} proportional to the square of the propellers angular velocity (Section:\ref{subsec:dynamics.aero.bem}). The relationship\footnote{The input range can be adjusted in BLHeli ESC software to improve input resolution.} between input pulse-width to the ESCs and output RPM sensor signal (Fig:\ref{fig:bldc-range}) is:
\begin{equation}
y(x)=
\begin{cases}\begin{array}{ll}
0 & ~~x<1.065~m.s\\
-20593x^2 + 80187x - 60004 & ~~1.065~m.s \leq x \leq 1.655~m.s\\
16300\text{\textdegree} & ~~x>1.655~m.s\\
\end{array}
\end{cases}
\label{eq:bldc-range}
\end{equation}
\par
\begin{figure}[hbtp]
\begin{subfigure}{0.5\textwidth}
\centering
\includegraphics[width=0.9\textwidth]{graphs/bldc-range}
\caption{BLDC RPM range}
\label{fig:bldc-range}
\end{subfigure}
\begin{subfigure}{0.5\textwidth}
\centering
\includegraphics[width=0.9\textwidth]{graphs/BLDC-step}
\caption{Cobra BLDC step response}
\label{fig:bldc-step}
\end{subfigure}
\caption{}
\end{figure}
The upper limit in Eq:\ref{eq:bldc-range} and the motor's step response are both governed by the ESC's maximum current limit; in this case 20A. Imposing 10A current limiting (a consequence of using lower power ESCs), the plot for {\color{YellowGreen}$\mathbf{c(t)}$} in Fig:\ref{fig:bldc-step}, significantly restricts the motor's transient and steady-state performance. The motor's step response, {\color{Purple}$\mathbf{y(t)}$} has a negligible dead time and 2$^{nd}$ order dynamics\footnote{It can't be stressed enough how much the BLHeli ESC firmware improved dynamic response of the motors}, far faster than the servo's plant. The motors transfer function for speed in RPM is:
\begin{subequations}\label{eq:bldc-transfer}
\begin{equation}
G_{BLDC}(s)=\frac{1}{\big(1+1.7583s\times 10^{-3}\big)\big(1+1.7494s\times 10^{-3}\big)}
\end{equation}
\begin{equation}
Y_{BLDC}(s)=
\begin{cases}\begin{array}{ll}
0\text{\textdegree} & ~~|U(s)|<1.065\\
G(s) & 1.065 \leq |U(s)| \leq 1.655\\
16300\text{\textdegree} & ~~|U(s)|>1.655\\
\end{array}
\end{cases}
\end{equation}
\end{subequations}
%====================================================
