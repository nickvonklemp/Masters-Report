%====================================================
%	CHAPTER 4 - Control
%====================================================
\chapter{Controller Development}
\label{ch:control}
%====================================================
\section{Control Loop}
\label{sec:control.loop}
%====================================================
The control problem is, as outlined in Ch:\ref{ch:intro}, to achieve non-zero setpoint tracking (for both \emph{attitude} and \emph{position} states) on a quadrotor by solving the problem of its inherent underactuation. For the intents and purposes of the subsequent controller development, the plant for some state $\vec{\mathbf{x}}$ is described in the following typical nonlinear state-space form in the time domain:
\begin{subequations}\label{eq:control-loop-states}
\begin{equation}
\frac{d}{dt}{\vec{\mathbf{x}}}=f(\vec{\mathbf{x}},t)+g(\vec{\mathbf{x}},\vec{\nu},t)
\end{equation}
\vspace{-12pt}
\begin{equation}
\vec{y} = c(\vec{\mathbf{x}},t)+d(\vec{\mathbf{x}},\vec{\nu},t)
\end{equation}
\end{subequations}
Where the plant's dynamics are governed by state progression $f(\vec{\mathbf{x}},t)$ and the plant's input response $g(\vec{\mathbf{x}},\vec{\nu},t)$ for a given control input $\vec{\nu}$. The latter could take the affine form; $g(\vec{\mathbf{x}},t)\vec{\nu}$. Setpoint tracking aims for the output to track the plant's state; namely $\vec{y} = c(\vec{\mathbf{x}},t)\equiv\vec{\mathbf{x}}$. The control problem is then to design a stabilizing control law $\mathcal{H}$ for some error state $\vec{\mathbf{x}}_e$:
\begin{equation}
\vec{\nu}_d=\mathcal{H}(\vec{\mathbf{x}}_e,\dot{\vec{\mathbf{x}}}_e,t)=\mathcal{H}(\vec{\mathbf{x}}_b,\dot{\vec{\mathbf{x}}}_b,\vec{\mathbf{x}}_d,\dot{\vec{\mathbf{x}}}_d,t)=\begin{bmatrix}
\vec{F}_d\\
\vec{\tau}_d
\end{bmatrix}
\end{equation}
Such that the controlled plant is asymptotically stabilizing or that $\lim_{t\rightarrow\infty}\vec{\mathbf{x}}_e=0$. Trajectory stability conditions are defined next in Sec:\ref{sec:control.stability}. Note that it is possible to combine attitude and position states into a single common trajectory state such that:
\\
\vspace{-5pt}
\begin{equation}
\vec{\mathbf{x}}_b=\begin{bmatrix}\vec{\mathcal{E}}_I&Q_b\end{bmatrix}^T
\end{equation}
The body's trajectory is then fully described by $\vec{\mathbf{x}}_b(t)$. Separate control laws are developed for attitude and position tracking and hence those states are not combined in the context of this control project. However for the purposes of describing the control plant, a single major loop control is structure considered.
\par
Because of the plant's overactuatedness the control loop is split into two blocks; first a higher level \emph{setpoint tracking} controller designs a virtual control input $\vec{\nu}_d$. That being net forces $\vec{F}_d$ and torques $\vec{\tau}_d$ to act on the body. A lower level \emph{allocator} then solves for explicit actuator positions from $\vec{\nu}_d$ to physically actuate that \emph{virtual} control input. The actuator set then implements a commanded control input $\vec{\nu}_c$ through its effectiveness function (Eq:\ref{eq:dynamic-plant-inputs}) with an actuator estimate $\hat{u}$ subject to transfer functions $C(s)$:
\begin{equation}\label{eq:control-effectiveness}
\vec{\nu}_c=B(\vec{\mathbf{x}},\hat{u},t)=\begin{bmatrix}
\vec{F}_c(u) & \vec{\tau}_c(u)
\end{bmatrix}^T
\end{equation}
The allocator solves for commanded actuator values $u_c$ such that $\vec{\nu}_c\rightarrow\vec{\nu}_d$. That allocation function, $B^\dagger$, can be \emph{roughly} referred to as the effectiveness inverse:
\begin{equation}
u_c=B^{\dagger}(\vec{\mathbf{x}},\vec{\nu}_d,t)~~~~\in\mathbb{U}
\end{equation}
This chapter derives higher level controllers for $\vec{\nu}_d=\mathcal{H}(\vec{\mathbf{x}}_e,\dot{\vec{\mathbf{x}}}_e,t)$; allocation rules are discussed next in Ch:\ref{ch:allocation}. A collection of attitude and position controllers are presented here whose stability is proven with Lyapunov theorem \cite{bojelayupanov,lyapunovstabilitytheorem,noteonlyapunov}. Each controller is compared in the context of an overactuated quadrotor plant, similarly a series of allocation schemes are presented. For the most part, it is assumed errors between commanded actuator setpoints $u$ and their state estimates $\hat{u}$ affected by transfer functions $C(s)$ are negligible. That plant dependent error is small and diminishing over time but is later incorporated into adaptive control in Sec:\ref{subsubsec:control.attitude.nonlinear.adaptivebackstep}. Resultant controller comparisons, their details and efficacy are evaluated subsequently in Ch:\ref{ch:simulation}. 
\par
A generalized overactuated control loop consists of a series of cascaded control blocks (Fig:\ref{fig:control-loop}). From the trajectory's error state $\vec{\mathbf{x}}_e$, a control law designs a virtual control input $\vec{\nu}_d$ which is applied to the allocation block. The allocation law $B^{\dagger}(\vec{\mathbf{x}},\vec{\nu}_d,t)$ solves for physical actuator positions $u_c\in\mathbb{U}$. Commanded actuator (\emph{estimate}) positions affect a physical input $\vec{\nu}_c=B(\vec{\mathbf{x}},\hat{u},t)$ which is an input applied to the state's dynamics, Eq:\ref{eq:control-loop-states}. Finally the output tracking state is estimated with some filter paradigm $\hat{\mathbf{x}}=A(\vec{\mathbf{x}},t)$ which is fed back for error state calculation (Sec:\ref{sec:simulation.state}).
\begin{figure}[htbp]
\centering
\includegraphics[width=0.97\textwidth]{figs/control-loop}
\vspace{-28pt}
\caption{Generalized control loop with allocation}
\vspace{-16pt}
\label{fig:control-loop}
\end{figure}
\par
Fig:\ref{fig:control-loop} shows a generalized overactuated control loop's structure; the plant's dynamics from Eq:\ref{eq:quaternion-states} include state derivative feedback. Moreover aspects of the state transfer function includes multibody nonlinearities dependent on actuator positions and rates as detailed in Sec:\ref{sec:dynamics.nonlinearities}. That generalized case is now refined in the context of an overactuated quadcopter.
%====================================================
\section{Control Plant Inputs}
\label{sec:control.inputs}
%====================================================
Control inputs for the differential state equations, from Eq:\ref{eq:quaternion-states}, have mostly been described with net forces and torques; $\vec{F}_\mu(\hat{u})$ and $\vec{\tau}_\mu(\hat{u})$. The relationship of the \emph{effectiveness function} between each propeller's rotational speed and servo positions with the produced thrust vector is calculated from Eq:\ref{eq:quaternion-inputs}.
\begin{subequations}\label{eq:control-input}
\begin{equation}
\vec{\nu}_c\triangleq\begin{bmatrix}
\vec{F}_\mu(\hat{u}) & \vec{\tau}_\mu(\hat{u})
\end{bmatrix}^T=B(\vec{\mathbf{x}},\hat{u},t)~~~~\in\mathbb{R}^6,~u\in\mathbb{U}
\end{equation}
\vspace{-10pt}
\begin{equation}
\vec{F}_\mu(\hat{u})=\sum_{i=1}^4 Q_{M_i}^*(\lambda_i,\alpha_i)\otimes \vec{T}(\Omega_i)\otimes Q_{M_i}(\lambda_i,\alpha_i)~~~~\in\mathcal{F}^b
\end{equation}
\vspace{-4pt}
\begin{equation}
\vec{\tau}_\mu(\hat{u})=\sum_{i=1}^4 \vec{L}_i\times\big(Q_{M_i}^*(\lambda_i,\alpha_i)\otimes \vec{T}(\Omega_i)\otimes Q_{M_i}(\lambda_i,\alpha_i)\big)~~~~\in\mathcal{F}^b
\end{equation}
\end{subequations}
As mentioned previously, a higher level controller $\mathcal{H}(\vec{\mathbf{x}}_e,\dot{\vec{\mathbf{x}}}_e,t)$ designs desired net plant inputs $\vec{\nu}_d=\begin{bmatrix}\vec{F}_d&\vec{\tau}_d\end{bmatrix}^T$ whilst a lower level allocator commands actuator positions $u_c=B^{\dagger}(\vec{\mathbf{x}},\vec{\nu}_d,t)$. This separation allows for independent comparison of proposed control and allocation laws. However, typical allocation rules like pseudo-inversion require an affine relationship between plant and control inputs, detailed in Sec:\ref{subsubsec:intro.lit.control.allocation}).
\par
The relationship in Eq:\ref{eq:control-input} is not reducible to a single multiplicative relationship with the actuator matrix $u\in\mathbb{U}$. So the effectiveness function needs an extra layer of abstraction to incorporate a multiplicative relationship. Rather than calculating explicit actuator positions directly from $\vec{\nu}_d$; a set of four 3-D thrust vectors $\vec{T}_{1\rightarrow 4}$ for each motor module are first calculated.
\begin{subequations}\label{eq:4.7}
\begin{equation}
\vec{\nu}_c=\begin{bmatrix}
\vec{F}_c(u)\\
\vec{\tau}_c(u)
\end{bmatrix}
= 
\begin{bmatrix}
\mathbb{I}_{3\times 3} & \mathbb{I}_{3\times 3} & \mathbb{I}_{3\times 3} & \mathbb{I}_{3\times 3}\\
[\vec{L}_1]_\times & [\vec{L}_2]_\times & [\vec{L}_3]_\times & [\vec{L}_4]_\times
\end{bmatrix}
\begin{bmatrix}
\vec{T}_1&
\vec{T}_2&
\vec{T}_3&
\vec{T}_4
\end{bmatrix}^T
\end{equation}
\begin{equation}
\rightarrow\vec{\nu}_c=B'(\vec{\mathbf{x}},t)\begin{bmatrix}
\vec{T}_1&
\vec{T}_2&
\vec{T}_3&
\vec{T}_4
\end{bmatrix}^T
\end{equation}
\end{subequations}
Where $[\vec{L}_i]_\times$ is the cross product vector of the $i^{th}$ torque arm from Eq:\ref{eq:cross-product-matrix}. Explicit actuator positions for each module, $[\Omega_i,\lambda_i,\alpha_i]^T$, can then be solved from those thrust vectors $\vec{T}_i$ for $i\in[1:4]$ with some trigonometry, ``undoing" the transformation applied in Eq:\ref{eq:control-input}. That trigonometric inversion is detailed later in Sec:\ref{sec:allocation.inversion} but is described as the function $R^\dagger$:
\begin{equation}\label{eq:4.8}
[\Omega_i,~\lambda_i,~\alpha_i]^T=R^\dagger(\vec{\mathbf{x}},\vec{T}_i,t)~~~~\text{for}~i\in[1:4]
\end{equation}
The generalized control loop illustrated in Fig:\ref{fig:control-loop} is extended to include the abstracted allocation blocks of Eq:\ref{eq:4.7} and Eq:\ref{eq:4.8}, shown in Fig:\ref{fig:control-block}. The net control block still solves for the same actuator matrix $u\in\mathbb{U}$. The entire loop accommodates for comparison of various $B^\dagger(\vec{\mathbf{x}},\vec{\nu}_d,t)$ allocation rules without having to redesign the remainder of the loop's structure.
\begin{figure}[htbp]
\vspace{-8pt}
\centering
\includegraphics[width=\textwidth]{figs/control-block}
\caption{Extended control loop with overactuation}
\label{fig:control-block}
\vspace{-16pt}
\end{figure}
\par
Certain blocks in Fig:\ref{fig:control-block} use actuator estimates $\hat{u}$ to calculate responses for feedback compensation, other blocks are affected by instantaneous actuator positions $u$. The two are, for the most part, interchangeable as minor loop transfer dynamics of $C(s)$ in Sec:\ref{subsec:proto.design.transfer}. In summary; each controller designs a net force and torque to act on the body. Allocation rules decompose that virtual input into four separate 3-D thrust vectors, or twelve directional components. The force components are an abstracted allocation layer in place of explicit actuator positions, which are subsequently solved for\ldots
\begin{equation}
B^{\dagger}(\mathbf{x},\vec{\nu}_d,t)=\big[ T_{1x},~T_{1y},~T_{1z},~\ldots~T_{4x},~T_{4y},~T_{4z}\big]^T
\end{equation}
Each control law is co-dependent on an accompanying allocation algorithm. Traditional control loops (underactuated or well matched) typically have a unity allocation rule and as such require no consideration so they're mostly disregarded. Separate control laws for attitude ad position control are presented in Section:\ref{sec:control.attitude} and \ref{sec:control.position} respectively. Thereafter a series of allocation rules are proposed in Ch:\ref{ch:allocation}. Although presented independently, the controller and allocation laws are mutually inclusive. The stability of each controller is proven objectively but explicit controller coefficients are optimized in the subsequent Ch:\ref{ch:simulation}, in Sec:\ref{sec:simulation.tuning}.
\newpage
%====================================================
\section{Stability}
\label{sec:control.stability}
%====================================================
Before undertaking the control plant derivations, it is worth outlining definitions of the control objective's stability. The research question aims to achieve non-zero setpoint tracking of state's trajectory. A control loop then aims to \emph{stabilize} the dynamics described previously in Sec:\ref{sec:dynamics.model} whilst tracking particular trajectories for attitude and position setpoints, $\vec{\mathbf{x}}_d(t)=[\vec{\mathcal{E}}_d(t)~Q_d(t)]^T$. 
\par
The entire system's control loop was previously detailed in Sec:\ref{sec:control.loop}. Stability in the context of trajectory tracking must first be defined. Generalized trajectory stability definitions are not uncommon in the context of energy based control design, or Lyapunov theorem (Sec:\ref{sec:control.lyapunov}). Stability definitions pertinent to Lyapunov's stability theorem are briefly presented here; the following is adapted from \cite{bojelayupanov,lyapunovstabilitytheorem}. In general for some autonomous trajectory $\vec{\mathbf{x}}(t)$, an equilibrium point $\vec{\mathbf{x}}(t_0)$ is said to be stable (\textbf{S}) if and only if (\emph{iff}) the following is true:
\begin{subequations}\label{eq:basic-stability}
\begin{equation}
\forall\varepsilon>0,~\exists~\delta_0(t_0,\varepsilon):~\norm{\vec{\mathbf{x}}(t_0)}<\delta_0(t_0,\varepsilon)
\end{equation}
\vspace{-20pt}
\begin{equation}
\Rightarrow\norm{\vec{\mathbf{x}}(t)}<\varepsilon,~~\forall t\geq t_0
\end{equation}
\end{subequations}
The implication of which is that if, for some initial condition $\vec{\mathbf{x}}(t_0)$ whose magnitude is bound by the manifold $\delta_0(t_0,\varepsilon)$, the entire subsequent trajectory of $\vec{\mathbf{x}}(t)$ is bound from above by some other manifold $\varepsilon$. Basic stability is illustrated in Fig:\ref{fig:basic-stability} for a 2-D trajectory.
\begin{figure}[hbtp]
\vspace{-6pt}
\centering
\begin{subfigure}{0.49\textwidth}
\centering
\includegraphics[width=\textwidth]{figs/basic-stability}
\vspace{-16pt}
\caption{Basic stability}
\label{fig:basic-stability}
\end{subfigure}
\begin{subfigure}{0.49\textwidth}
\centering
\includegraphics[width=\textwidth]{figs/uniform-stability}
\vspace{-16pt}
\caption{Uniform stability}
\label{fig:uniform-stability}
\end{subfigure}
\vspace{-8pt}
\caption{Trajectory illustrations for $\mathbf{S}$ and $\mathbf{US}$}
\vspace{-18pt}
\end{figure}
\par
An equilibrium point is further said to be uniformly stable (\textbf{US}) \emph{iff} for the time $t\in[t_0,\infty)$ the following criteria, being an extension of basic stability, is met:
\begin{subequations}\label{eq:uniform-stability}
\begin{equation}
\forall\varepsilon>0,~\exists~\delta_0(\varepsilon)>0:~\norm{\vec{\mathbf{x}}(t_1)}<\delta_0(\varepsilon),~~t_1>t_0
\end{equation}
\vspace{-18pt}
\begin{equation}
\Rightarrow\norm{\vec{\mathbf{x}}(t)}<\varepsilon,~~\forall
t\geq t_1
\end{equation}
\end{subequations}
\textbf{US} similarly bounds a trajectory from above by $\varepsilon$ if the trajectory originates from within $\delta_0(\varepsilon)$. The difference is that the principle trajectory region $\delta_0(\varepsilon)$ is independent of $t_0$ in the case of \textbf{US}. The two surfaces are non-concentric; a $\mathbf{US}$ trajectory is illustrated in Fig:\ref{fig:uniform-stability}. Uniform stability is a subset of general stability, $\mathbf{US}\subset\mathbf{S}$, however the converse is not true. Furthermore \textbf{US} is a stronger qualification of stability, each subsequent stability presented represents a stronger assertion of stability.
\par
Extending stability definitions to include settling; an equilibrium point is said to be asymptotically stable (\textbf{AS}) \emph{iff} conditions for \textbf{S} are met (Eq:\ref{eq:basic-stability}) and that the following holds true:
\begin{subequations}\label{eq:asymptotic-stability}
\begin{equation}
\exists~\delta_1(t_0,\varepsilon) >0:~\norm{\vec{\mathbf{x}}(t_0)}<\delta_1(t_0,\varepsilon)
\end{equation}
\vspace{-18pt}
\begin{equation}
\Rightarrow \lim_{t\rightarrow\infty}\norm{\vec{\mathbf{x}}(t)}\rightarrow 0
\end{equation}
\end{subequations}
This asserts that trajectories originating within some finer region $\delta_1(t_0,\varepsilon)$, being a subset of $\delta_0(t_0,\varepsilon)$, tends to and \emph{asymptotically} settles at the origin. In the case of \textbf{AS} the origin is both \emph{stable} and \emph{attractive} (shown in Fig:\ref{fig:asymptotic-stability}). Asymptotic stability is typically the first requirement for any control law, being a stronger stability than both \textbf{US} and \textbf{S}, typically stabilizing a control setpoint's error\ldots
\begin{figure}[hbtp]
\centering
\begin{subfigure}{0.49\textwidth}
\centering
\includegraphics[width=\textwidth]{figs/asymptotic-stability}
\vspace{-8pt}
\caption{Asymptotic stability}
\label{fig:asymptotic-stability}
\end{subfigure}
\begin{subfigure}{0.49\textwidth}
\centering
\includegraphics[width=\textwidth]{figs/uniform-asymptotic-stability}
\vspace{-8pt}
\caption{Uniform asymptotic stability}
\label{fig:uniform-asymptotic-stability}
\end{subfigure}
\vspace{-4pt}
\caption{Trajectory illustrations for $\mathbf{AS}$ and $\mathbf{UAS}$}
\vspace{-14pt}
\end{figure}
\par
Uniform asymptotic stability (\textbf{UAS}), an extension of asymptotic stability $\mathbf{UAS}\subset\mathbf{AS}$, occurs when the asymptotically stable bound region $\delta_1(\epsilon)$ is independent of the principle starting $t_0$. An equilibrium point is \textbf{UAS} \emph{iff} conditions for \textbf{S} are met and that:
\begin{subequations}\label{eq:uniform-asymptotic-stability}
\begin{equation}
\exists~\delta_1(\varepsilon)>0:~\norm{\vec{\mathbf{x}}(t_1)}<\delta_1(\varepsilon),~~t_1\geq t_0
\end{equation} 
\vspace{-16pt}
\begin{equation}
\Rightarrow \lim_{t\rightarrow\infty}\norm{\vec{\mathbf{x}}(t)}\rightarrow 0
\end{equation}
\end{subequations}
A uniformly asymptotic equilibrium point implies stability from a non-concentric ball of attraction; settling to the origin (illustrated in Fig:\ref{fig:uniform-asymptotic-stability}). 
\par
An equilibrium point is regarded as exponentially stable (\textbf{UES}) if conditions for \textbf{UAS} are met and that there exist $\exists~a,b,r$ that bound the settling of the trajectory such that:
\begin{equation}\label{eq:exponential-stability}
\norm{\vec{\mathbf{x}}(t,t_0,\vec{\mathbf{x}}_0)}\leq a\norm{\vec{\mathbf{x}}_0}e^{-bt},~~\forall\norm{\vec{\mathbf{x}}_0}\leq r
\end{equation}
The term $a\norm{\vec{\mathbf{x}}_0}e^{-bt}$ bounds the rate at which the trajectory settles to the origin, illustrated in Fig:\ref{fig:exponential-stability}. The initial point of the trajectory, $\vec{\mathbf{x}}_0$ is bound from above by some $r\triangleq \delta_1(\varepsilon)$. Moreover uniform stability is \emph{implied} with exponential stability.
\begin{figure}[hbtp]
\centering
\begin{subfigure}{0.49\textwidth}
\includegraphics[width=\textwidth]{figs/exponential-stability}
\vspace{-14pt}
\caption{Exponential stability}
\label{fig:exponential-stability}
\end{subfigure}
\begin{subfigure}{0.49\textwidth}
\centering
\includegraphics[width=\textwidth]{figs/global-exponential-stability}
\vspace{-14pt}
\caption{Global exponential stability}
\label{fig:global-exponential-stability}
\end{subfigure}
\vspace{-4pt}
\caption{Trajectory illustrations for \textbf{UES} and \textbf{GUES}}
\vspace{-14pt}
\end{figure}
\par
The above definitions of stabilities are only locally defined, and so the stabilities hold true only for local trajectories, only in the case of $\vec{\mathbf{x}}(t_0)\leq\varepsilon$. Extending \textbf{UAS} to global uniform asymptotic stability (\textbf{GUAS}); the origin's equilibrium point is \textbf{GUAS} \emph{iff} conditions for \textbf{UAS} are first met, the origin is only the equilibrium point and the asymptotic approach can be extended such that:
\begin{subequations}
\begin{equation}
\exists~\delta_1(\varepsilon)>0:~\norm{\vec{\mathbf{x}}(t_1)}<\delta_1(\varepsilon),~~t_1\geq t_0
\end{equation}
\vspace{-16pt}
\begin{equation}
\Rightarrow \lim_{t\rightarrow\infty}\norm{\vec{\mathbf{x}}(t)}\rightarrow 0,~~\forall \vec{\mathbf{x}}(t_0)
\end{equation}
\end{subequations}
Similarly exponential stability can extend to the global case, shown in Fig:\ref{fig:global-exponential-stability}, but only \emph{iff} \textbf{UES} conditions are first met. In the global case, the origin can be the \emph{only equilibrium point}. Stability from Eq:\ref{eq:exponential-stability} is then globally:
\begin{equation}\label{eq:global-exponential-stability}
\norm{\vec{\mathbf{x}}(t,t_0)}\leq a\norm{\vec{\mathbf{x}}_0}e^{-bt},~~\forall\norm{\vec{\mathbf{x}}_0}
\end{equation}
Initial trajectory conditions are dropped in Eq:\ref{eq:global-exponential-stability}, for any number of trajectories until $\vec{\mathbf{x}}_n(t)$; each trajectory is bound by an exponential $a_n\norm{\vec{\mathbf{x}}_t(0)}e^{-b_nt}$. It follows that, irrespective of the starting point $\vec{\mathbf{x}}_n(t_0)$ for the trajectory, the system \emph{always} settles to the origin. \textbf{GUES} is the strongest sense of stability and provides insight into the trajectory stabilizing rate. The most desirable control design outcome is a controller which applies globally uniform exponential stability to a plant.
%====================================================
\section{Lyapunov Stability Theorem}
\label{sec:control.lyapunov}
%====================================================
Lyapunov's stability theory is an important aspect of nonlinear controller design. An abundance of literature exists on the subject, included in almost every meritable textbook or control paper. If the reader is unfamiliar with Lyapunov's theorem,  \cite{noteonlyapunov,nonlinearsystems,bojelayupanov} each explains the concept in detail. The following is adapted from \cite{bojelayupanov} and \cite{lyapunovstabilitytheorem} and briefly outlines how Lyapunov's stability theorem is used to prove (\emph{global}) asymptotic stability for continuous time invariant systems, linear or otherwise. The theory analyzes a generalized energy function of a system's autonomous trajectory. If the trajectory has a negative energy derivative that implies the system's energy will always dissipate towards a state of zero energy or stable equilibrium point. Lyapunov analysis is a powerful tool for stability verification because the system's trajectory itself need not be explicitly defined for stability to be determined. Proof of Lyapunov's theorem is done with a contradiction disproof and, as such, the theoretical underpinning is somewhat cumbersome.
\par
It is worth reiterating its fundamentals given that backstepping controllers are proposed later in Sec:\ref{subsec:control.attitude.nonlinear} for attitude control. A backstepping controllers enforce Lyapunov stability criteria onto the system through interative control structure design, \cite{backstepping,adaptivebackstep,intelligentbackstep}. In general, given a nonlinear time invariant system that follows some continually differentiable trajectory $\vec{\mathbf{x}}(t)$, typically the trajectory is going to progress subject to some rule:
\begin{equation}\label{eq:4.17}
\dot{\vec{\mathbf{x}}}(t)=f\big(\vec{\mathbf{x}}(t),u\big)
\end{equation}
Then, constructing a generalized positive-definite function generalized energy or \emph{Lyapunov function candidate} (\emph{LFC}) $V(\vec{\mathbf{x}})$ for a trajectory $\vec{\mathbf{x}}(t)$. A positive definite matrix $M$ is defined such that:
\begin{equation}
\mathbf{z}^TM\mathbf{z}\geq 0~~~\forall \mathbf{z}
\end{equation}
As such an LFC typically, but not exclusively, has the quadratic and positive-definite form with some positive square matrix $P\in\mathbb{R}^{n\times n}>0$:
\begin{equation}
V(\vec{\mathbf{x}})=\vec{\mathbf{x}}\hspace{1pt}^TP\vec{\mathbf{x}},~~\vec{\mathbf{x}}\in\mathbb{R}^{n}
\end{equation}
An LFC could simply be positive semi-definite over the trajectory's path, the quadratic form is just convenient for the use of backstepping. From its definition the trajectory Eq:\ref{eq:4.17} is continually differentiable; there is then a gradient matrix for each element of $V(\vec{\mathbf{x}})$ in the form:
\begin{equation}\label{eq:4.20}
\nabla V(\vec{\mathbf{x}})\triangleq\bigg[\frac{\partial V(\vec{\mathbf{x}})}{\partial x_1}~\frac{\partial V(\vec{\mathbf{x}})}{\partial x_2}~\ldots~\frac{\partial V(\vec{\mathbf{x}})}{\partial x_n}\bigg]~~~~\vec{\mathbf{x}}\in\mathbb{R}^n
\end{equation}
The energy function's derivative, otherwise referred to as the \emph{Lie derivative} in some texts;\cite{noteonlyapunov,nonlinearsystems}, is calculated from partial derivatives in Eq:\ref{eq:4.20} as follows:
\begin{equation}
\dot{V}(\vec{\mathbf{x}})\triangleq\nabla V(\vec{\mathbf{x}})^T\frac{d}{dt}f(\vec{\mathbf{x}})=\frac{\delta V(\vec{\mathbf{x}})}{\delta x_1}\frac{df_1(x_1)}{dt}+\frac{\delta V(\vec{\mathbf{x}})}{\delta x_2}\frac{df_2(x_2)}{dt}+~\ldots~+\frac{\delta V(\vec{\mathbf{x}})}{\delta x_n}\frac{df_n(x_b)}{dt}
\end{equation}
Lyapunov's theorem states that \emph{iff} the candidate function $V(\vec{\mathbf{x}})$ is positive definite with $V(\vec{0})=0$ and its derivative is strictly negative; $\dot{V}(\vec{\mathbf{x}})< 0~~\forall \vec{\mathbf{x}}(t) \not= 0$, the system is then asymptotically stable ($\mathbf{AS}$ from Eq:\ref{eq:asymptotic-stability}). Mathematically that means, for any $\vec{\mathbf{x}}(t)$ with $t\geq t_0$:
\begin{equation}\label{eq:4.22}
V\big(\vec{\mathbf{x}}(t)\big)=V\big(\vec{\mathbf{x}}(t_0)\big)+\int_{t_0}^t \dot{V}\big(\vec{\mathbf{x}}(t)\big).dt \leq V\big(\vec{\mathbf{x}}(t_0)\big)
\end{equation}
Which can be physically interpreted as the system's generalized energy function dissipating, irrespective of the trajectory path taken. With a strictly decreasing energy function, the system will stabilize to a state of zero energy which, naturally, is a stable equilibrium point.
\begin{equation}
\lim_{t\rightarrow\infty}\norm{V\big(\vec{\mathbf{x}}(t)\big)}\rightarrow 0
\end{equation}
The trajectory's asymptotic stability can be extended to exponential stability boundedness, such that \emph{iff} the same conditions are met for asymptotic stability in Eq:\ref{eq:4.22} and there exists some positive coefficient $\alpha>0$ such that $\dot{V}(\vec{\mathbf{x}})\leq-\alpha V(\vec{\mathbf{x}})$. That implies the system is globally exponentially stable as is bound in such a way that:
\begin{equation}
\norm{V\big(\vec{\mathbf{x}}(t)\big)}\leq Me^{-\alpha t/2}\norm{V\big(\vec{\mathbf{x}}(t_0)\big)}
\end{equation}
%====================================================
\section{Model Dependent \& Independent Controllers}
%====================================================
Two classes of controllers are included for a full state trajectory tracking control loop; both attitude and position control laws. Attitude setpoint tracking is the primary focus of this research project (Sec:\ref{subsec:control.attitude.problem}) and incorporates a more detailed schedule of controller design and evaluation. 
\par
The allocation law combines both virtual control inputs from attitude and position controllers, $\vec{\nu}_d=[\vec{F}_d~\vec{\tau}_d]^T$, to solve for explicit actuator positions. Controller dependency on the plant's state is as a consequence of the actuator responses and complex inertial dynamics, as derived previously in Sec:\ref{subsec:dynamics.nonlinearities.gyrotorques}. Whilst not a prerequisite for stability, plant dependent compensation obviously improves controller performances. Independent and dependent cases are only considered for one type of controller; the most basic case proportional-derivative controller in Section:\ref{subsubsec:control.attitude.controllers.pd} and tested in Sec:\ref{subsec:simulation.attitude.pd}. All other control laws compensate for unwanted plant dynamics in a feedback configuration.
\par
The plant dependency makes backstepping controllers an effective controller choice for this dissertation's context. The proposed plant dependent control laws compensate for undesirable dynamics their design, basic PD and PID control structures (\emph{and the like}) will not. The first and most basic control solution, used as a reference case, is a PD controller for attitude and position with direct-inversion (Pseudo or Moore-Penrose inversion) allocation. 
%====================================================
\section{Attitude Control}
\label{sec:control.attitude}
%====================================================
\subsection{The Attitude Control Problem}
\label{subsec:control.attitude.problem}
%====================================================
The setpoint tracking control problem (\cite{attitudecontrolproblem}) for the attitude plant is to design a stabilizing control torque $\vec{\tau}_d=h(\vec{\mathbf{x}}_e,\dot{\vec{\mathbf{x}}}_e,t)$ such that for any desired attitude quaternion $\forall~Q_d\in\mathbb{Q}$ and an instantaneous attitude body quaternion $Q_b\in\mathbb{Q}$ the error state asymptotically stabilizes to the origin; $Q_e\rightarrow[1~\vec{0}~]^T$. Or that:
\begin{equation}
\vec{\tau}_d=h(Q_d,~\dot{Q}_d,~Q_b,~\dot{Q}_b)~~\text{such that}~~\underset{t\rightarrow\infty}{\lim}Q_b\rightarrow Q_d
\end{equation}
Quaternion attitude error states are defined as the Hamilton product or \emph{difference} between the desired and instantaneous quaternion attitude states, previously in Eq:\ref{eq:quaternion-error}. Quaternion error states are multiplicative, in contrast with the subtractive relationship for Euler angle error states. The attitude error state is defined as:
\begin{equation}\label{eq:quaternion-error-control}
Q_e\triangleq Q_b^*\otimes Q_d
\end{equation}
The relative angular velocity error between the body frame $\mathcal{F}^b$ and the trajectory's desired frame $\mathcal{F}^d$ is given as $\vec{\omega}_e$. The body angular velocity $\vec{\omega}_b$ is subject to the differential Eq:\ref{eq:quaternion-states-angular}. As such there is an angular rate error:
\begin{subequations}
\begin{equation}\label{eq:angular-error}
\vec{\omega}_e\triangleq\vec{\omega}_d-\vec{\omega}_b\hspace{-2pt}'~~~~\in\mathcal{F}^d
\end{equation}
The desired angular velocity $\vec{\omega}_d$ is taken with respect to the desired angular attitude frame, and so it must be transformed back onto the existing body frame.
\begin{equation}
\vec{\omega}_e=Q_e^*\otimes\vec{\omega}_d\otimes Q_e-\vec{\omega}_b~~~~\in\mathcal{F}^{b}
\end{equation}
Typically for the trajectories generated here only first order setpoints are commanded, hence the desired angular velocity is zero; $\vec{\omega}_d=\vec{0}$. It follows that the angular rate error is then simply the negative body angular velocity. It would be easy to incorporate a non-zero angular velocity setpoints to accommodate for higher order state derivative tracking trajectories.
\begin{equation}\label{eq:4.27c}
\vec{\omega}_e=-\vec{\omega}_b\Big|_{\vec{\omega}_d=\vec{0}}
\end{equation}
\end{subequations}
The time derivative for the quaternion error state is calculated from the quaternion derivative definition Eq:\ref{eq:quaternion-deriv}. The derivative $\dot{Q}_e$ is then dependent on the angular velocity error and calculated as follows:
\begin{equation}
\dot{Q}_e=\frac{1}{2}Q_e\otimes\vec{\omega}_e=-\frac{1}{2}Q_e\otimes\vec{\omega}_b\Big|_{\vec{\omega}_d=\vec{0}}
\end{equation}
%====================================================
\subsection{Linear Controllers}
\label{subsec:control.attitude.controllers}
%====================================================
\subsubsection{PD Controller}
\label{subsubsec:control.attitude.controllers.pd}
%====================================================
The following control law is used as a reference case for comparing the remaining controllers derived. It is a simple proportional-derivative (\emph{PD}) attitude controller, adapted from \cite{fullquaternion} and following a stability proof similar to the one derived in \cite{attitudecontrolproblem}. An attitude PD controller is proportional only to the \emph{vector quaternion error}. Such that the error is the same dimension as the angular velocity error; $\vec{q}_e\in\mathbb{R}^3$. A PD controller designs the commanded torque input:
\begin{equation}\label{eq:independent-pd}
\vec{\tau}_{_{PD}}=K_d\vec{\omega}_e+K_p\vec{q}_e~~~~\in\mathcal{F}^b
\end{equation}
Where both $K_d$ and $K_p$ are \emph{positive symmetrical} $3\times 3$ gain coefficient matrices to be determined at a later stage. Eq:\ref{eq:independent-pd} neglects the quaternion scalar error and is therefore susceptible to unwinding. Using a positive-definite Lyapunov function candidate $V_{_{PD}}$ for the attitude trajectory:
\begin{equation}\label{eq:lyapunov-pd}
V_{_{PD}}(Q_e,\vec{\omega}_e)=\vec{q}_e\text{}^T\vec{q}_e+(1-q_0)^2+\frac{1}{2}\vec{\omega}_e\text{}^TJ_b(u)K_p^{-1}\vec{\omega}_e~~>0,~\forall(Q_e,\vec{\omega}_e)
\end{equation}
Note also that $V_{_{PD}}([\pm 1~\vec{0}\hspace{1pt}]^T,\vec{0}\hspace{1pt})=0$, making it a suitable LFC. Exploiting the unit quaternion's inherent magnitude property, it follows that:
\begin{equation}\label{eq:4.17}
\norm{Q}=\vec{q}\text{}^{\hspace{3pt}T}\vec{q}+q_0\text{}^2=\vec{q}\text{}^{\hspace{3pt}2}+q_0\text{}^2=1
\end{equation}
Substituting that and the angular velocity's error state, $\vec{\omega}_e=-\vec{\omega}_b$; the proportional derivative LFC in Eq:\ref{eq:lyapunov-pd} reduces to:
\begin{subequations}\label{eq:4.18}
\begin{equation}
V_{_{PD}}=\vec{q}_e\text{}^2+\big(q_0\text{}^2 -2q_0 + 1\big) +\frac{1}{2}\vec{\omega}_e\text{}^TJ_b(u)K_p^{-1}\vec{\omega}_e
\end{equation}
\vspace{-10pt}
\begin{equation}
=2(1-q_0)+\frac{1}{2}\vec{\omega}_b\text{}^TJ_b(u)K_p^{-1}\vec{\omega}_b\Big|_{\vec{\omega}_e=-\vec{\omega}_b}
\end{equation}
\end{subequations}
Taking the derivative of that Lyapunov Function candidate then yields:
\begin{equation}\label{eq:lyapunov-pd-deriv}
\dot{V}_{_{PD}}(\vec{q}_e,\vec{\omega}_e)=-2\dot{q}_0+\vec{\omega}_b\text{}^TJ_b(u)K_p^{-1}\dot{\vec{\omega}}_b
\end{equation}
Recalling the angular velocity differential equation from Eq:\ref{eq:quaternion-states-angular}, $\dot{\vec{\omega}}_b$ with a control torque input $\vec{\tau}_{_{PD}}$ from Eq:\ref{eq:independent-pd}:
\begin{equation}
\dot{\vec{\omega}}_b=J_b\text{}^{-1}(u)\big(-\vec{\omega}_b\times J_b(u)\vec{\omega}_b-\vec{\tau}_b(u)+\vec{\tau}_g+\vec{\tau}_H+\vec{\tau}_{_{PD}}\big)~~~~\in\mathcal{F}^b
\end{equation}
Where $\vec{\tau}_H$ is a simplified representation of the net aerodynamic torque experienced by the body from the rotating propellers, from Eq:\ref{eq:consolidated-h-torque}. Then, using the fact that a quaternion's derivative by definition is:
\begin{equation}\label{eq:quat-derivative}
\dot{Q}=\begin{bmatrix}
-\frac{1}{2}\vec{q}^{\hspace{3pt}T}\vec{\omega}\\
\frac{1}{2}\big([\vec{q}\hspace{3pt}]_\times+q_0\mathbb{I}_{3\times 3}\big)\vec{\omega}
\end{bmatrix}
\end{equation}
Substituting the above into the LFC derivative $\dot{V}_{_{PD}}$ in Eq:\ref{eq:lyapunov-pd-deriv} with an expanded $\vec{\tau}_{_{PD}}$ yields:
\begin{subequations}
\begin{equation}
\rightarrow\dot{V}_{_{PD}}=\vec{q}_e\text{}^T\vec{\omega}_e+\vec{\omega}_b\text{}^TJ_b(u)K_p^{-1}\Big(J_b(u)^{-1}\big(-\vec{\omega}_b\times J_b(u)\vec{\omega}_b-\vec{\tau}_b+\vec{\tau}_g+\vec{\tau}_H+K_d\vec{\omega}_e+K_p\vec{q}_e\big)\Big)
\end{equation}
\vspace{-10pt}
\begin{equation}
=-\vec{q}_e\text{}^T\vec{\omega}_b+\vec{\omega}_b\text{}^T\vec{q}_e-\vec{\omega}_b\text{}^TK_p^{-1}K_d\vec{\omega}_b+\vec{\omega}_b\text{}^TK_p^{-1}\big(-\vec{\omega}_b\times J_b(u)\vec{\omega}_b-\vec{\tau}_b(u)+\vec{\tau}_g+\vec{\tau}_H\big)
\end{equation}
It follows that the transpose term $\vec{q}_e\text{}^T\vec{\omega}_b\iff\vec{\omega}_b\text{}^T\vec{q}_e$ is interchangeable as its resultant product is the same. The LCF derivative then simplifies:
\begin{equation}
\therefore\dot{V}_{_{PD}}=-\vec{\omega}_b\text{}^TK_p^{-1}K_d\vec{\omega}_b+\vec{\omega}_b\text{}^TK_p^{-1}\big(-\vec{\omega}_b\times J_b(u)\vec{\omega}_b-\vec{\tau}_b(u)+\vec{\tau}_g+\vec{\tau}_Q\big)
\end{equation}
\end{subequations}
Then, as long as $\big(-\vec{\omega}_b\times J_b(u)\vec{\omega}_b-\vec{\tau}_b(u)+\vec{\tau}_g+\vec{\tau}_Q\big)\leq \vec{0}$, some basic stability is guaranteed. Under specific circumstances the following assumptions can be made to ensure the asymptotic stability proof can be applied. The stability obviously breaks down if any of the assumptions fail, as such the stability is not global\ldots
\vspace{-10pt}
\begin{enumerate}[itemsep=0em]
\item The inertial matrix, $J_b(u)$, is approximately diagonal which, given the inertia ranges from Eq:\ref{eq:inertia-max} and Eq:\ref{eq:inertia-min}, is reasonable. Similarly that the angular rate can be made small with appropriately slow trajectory updates such that the torque gyroscopic cross-product is negligible:
\begin{center}
\vspace{-10pt}
$\big(\vec{\omega}_b\times J_b(u)\vec{\omega}_b\big)\approx\vec{0}$
\vspace{-8pt}
\end{center}
\item The actuator rate torque response, $\vec{\tau}_b(u)$, is a second order effect dependent on $d/dt(u)$. Typically the actuator rates are going to be kept small, Fig:\ref{fig:tau-body} shows torque magnitudes $|\vec{\tau}_b(u)|$ for a typical trajectory. For slow attitude steps those position changes are small enough to be considered negligible. The approximation is made:
\begin{center}
\vspace{-10pt}
$\vec{\tau}_b(u)\approx\vec{0}$
\vspace{-8pt}
\end{center}
\item Finally, for the sake of the stability proof, the eccentric gravitational torque arm is neglected. Such a situation only holds true if $u\approx\vec{0}$ or that servo actuator positions are close to their zero positions.
\begin{center}
\vspace{-10pt}
$\vec{\tau}_g\approx\vec{0}$
\vspace{-8pt}
\end{center}
\end{enumerate}
{\emph{\color{Gray}All of the above assumptions are made under extraneous circumstances and can not be assumed for almost all of the prototype's flight envelope. The plant independent case is considered and simulated in Sec\ref{subsubsec:simulation.atttiude.pd.independent} purely for contrition; mainly to demonstrate the need for plant dependent compensation.}
\par
If each of the assumptions made hold true, then the Lyapunov function's derivative is approximately negative definite. The stability proof for a very local trajectory is then:
\begin{subequations}
\begin{equation}
\dot{V}_{_{PD}}\approx-\vec{q}_e\text{}^T\vec{\omega}_b+\vec{\omega}_b\text{}^TK_p^{-1}\big(-K_d\vec{\omega}_b+K_p\vec{q}_e\big)
\end{equation}
\vspace{-14pt}
\begin{equation}\label{eq:pd-local-stability}
\rightarrow\dot{V}_{_{PD}}=-\vec{\omega}_b\text{}^TK_p^{-1}K_d\vec{\omega}_b=-K_p^{-1}K_d\norm{\vec{\omega}_b}\text{}^2~~<0,~\exists~(K_p^{-1},K_d)>0
\end{equation}
\end{subequations}
From Lyapunov stability theorem there then exists the limits for \emph{local} asymptotic stability: 
\begin{subequations}
\begin{equation}
\lim_{t\rightarrow\infty}\vec{\omega}_e\rightarrow\vec{0}~~\therefore~~\lim_{t\rightarrow\infty}\vec{\omega}_b\rightarrow\vec{0}^{-}
\end{equation}
\vspace{-14pt}
\begin{equation}
\lim_{t\rightarrow\infty}\vec{q}_e\rightarrow \vec{0}~~\text{and}~~\lim_{t\rightarrow\infty}(1-q_0)\rightarrow 0
\end{equation}
\end{subequations}
Hence the quaternion error stabilizes $Q_e\rightarrow[1~\vec{0}\hspace{3pt}]^{T}$ as $t\rightarrow\infty$. The stability shown in Eq:\ref{eq:pd-local-stability} is only local; introducing plant dependent compensation to the PD control law in Eq:\ref{eq:independent-pd} alleviates the stringent requirements on assumptions 1 through 3. 
\begin{equation}\label{eq:dependent-pd}
\vec{\tau}_{_{PD}}=\underbrace{K_p\vec{q}_e+K_d\vec{\omega}_e}_{Independent}+\underbrace{\hat{\omega}_b\times J_b(\hat{u})\hat{\omega}_b+\vec{\tau}_b(\hat{u})-\vec{\tau}_g-\vec{\tau}_Q}_{Compensation}
\end{equation}
Obviously controller errors and compensation terms are \emph{state estimates}, where inertias and torque responses are calculated using sampled $\hat{u}$ actuator states. Moreover the quaternion attitude and angular velocity states $\hat{Q}_b$ and $\hat{\omega}_b$ are both estimates and so a small degree of uncertainty exists; robust stability in the case of plant dependent uncertainty is investigated in Sec:\ref{sec:simulation.disturbance} but for now the estimates are \emph{assumed to be error free}.
\par
The resultant stability proof for the plant dependent case Eq:\ref{eq:dependent-pd} is much the same as that for the independent controller, Eq:\ref{eq:independent-pd}. The same LFC from Eq:\ref{eq:lyapunov-pd} shows that Eq:\ref{eq:pd-local-stability} holds globally:
\begin{equation}\label{eq:dependent-global-stability}
\rightarrow\dot{V}_{_{PD}}=-\vec{\omega}_b\text{}^TK_p^{-1}K_d\vec{\omega}_b=-K_p^{-1}K_d\norm{\vec{\omega}_b}\text{}^2<\vec{0},~~\forall(Q_e,\vec{\omega}_b),~~\exists~(K_p^{-1},K_d)>0
\end{equation}
Note that the inverse qualifier of $K_p^{-1}$ in the above is redundant given that $K_p$ is a symmetrical coefficient matrix. The plant dependent rule is not reliant on the same limiting assumptions needed for independent asymptotic stability to be achieved. Dynamic compensation in Eq:\ref{eq:dependent-pd} is simple to implement; especially considering the unwanted dynamics have already been quantified and corroborated in Sec:\ref{subsec:dynamics.nonlinearities.torque-tests} together with state estimate terms in Sec:\ref{subsec:proto.design.transfer}.
%====================================================
\subsubsection{Auxiliary Plant Controller}
\label{subsubsec:control.attitude.controllers.auxpd}
%====================================================
Expanding on what has, in practice (Table:\ref{tab:controllers} from Sec:\ref{subsec:intro.lit.related}), proven to be a popular and effective controller for attitude stabilization, \cite{attitudestabilization} proposed an auxiliary plant term to a P-D attitude controller. Most significantly, the altered PD controller adds auxiliary terms proportional to the quaternion rate error (Eq:\ref{eq:quaternion-deriv}). Moreover part of the auxiliary plant is proportional to the \emph{quaternion scalar} $q_0$, a term that is otherwise neglected in the previous PD control law (Sec:\ref{subsubsec:control.attitude.controllers.pd}) and prevents unwinding if incorporated. The \emph{auxilliarly} PD control torque is a function of errors states:
\begin{equation}\label{eq:control-aux-pd}
\vec{\tau}_{_{XPD}}=\underbrace{\Gamma_2{\widetilde{\Omega}}+\Gamma_3\vec{q}_e-J_b(u)\dot{\bar{\Omega}}}_{Independent}+\underbrace{\vec{\omega}_b\times J_b(u)\vec{\omega}_b+\hat{\tau}_b(u)-\vec{\tau}_g-\vec{\tau}_Q}_{Compensation}
\end{equation}
Wherein the coefficients $\Gamma_2$ and $\Gamma_3$ are both diagonal positive coefficient matrices whilst $\Gamma_1$, used next in Eq:\ref{eq:aux-pd-1}, is a symmetrical matrix. Each coefficient matrix is explicitly determined later. Auxiliary plants $\widetilde{\Omega}$ and $\dot{\bar{\Omega}}$ are defined as follows. 
\par
The first auxiliary plant $\bar{\Omega}$ is proportional to the quaternion error and hence its derivative $\dot{\bar{\Omega}}$ is a quaternion rate:
\begin{subequations}\label{eq:aux-pd-1}
\begin{equation}
\bar{\Omega}\triangleq-\Gamma_1\vec{q}_e~~\therefore~~\dot{\bar{\Omega}}=-\Gamma_1\dot{\vec{q}}_e
\end{equation}
\vspace{-15pt}
\begin{equation}
\rightarrow\dot{\bar{\Omega}}=-\frac{1}{2}\Gamma_1\big([\vec{q}_e]_{\times}+q_0\mathbb{I}_{3X3}\big)\vec{\omega}_e
\end{equation}
\vspace{-10pt}
\begin{equation}
=\frac{1}{2}\Gamma_1\big([\vec{q}_e]_{\times}+q_0\mathbb{I}_{3X3}\big)\vec{\omega}_b\Big|_{\vec{\omega}_e=-\vec{\omega}_b}
\end{equation}
\end{subequations}
The second auxiliary plant, $\widetilde{\Omega}$, is proportional to both quaternion vector and angular velocity errors.
\begin{subequations}\label{eq:aux-pd-2}
\begin{equation}
\widetilde{\Omega}\triangleq\vec{\omega}_e-\bar{\Omega}=\vec{\omega}_e+\Gamma_1\vec{q}_e
\end{equation}
\vspace{-15pt}
\begin{equation}
=-\vec{\omega}_b+\Gamma_1\vec{q}_e\Big|_{\vec{\omega_e}=-\vec{\omega}_b}
\end{equation}
\end{subequations}
Using an LFC similar to the basic $V_{_{PD}}$ function candidate from Eq:\ref{eq:lyapunov-pd}, but substituting an auxiliary term $\widetilde{\Omega}$ for the body's angular velocity $\vec{\omega}_b$ into the LFC $V_{_{XPD}}$:
\begin{equation}\label{eq:lyapunov-xpd}
V_{_{XPD}}\big(\vec{q}_e,~\widetilde{\Omega}\big)=\vec{q}_e\text{}^T\vec{q}_e+\big(1-q_0\big)^2+\frac{1}{2}\widetilde{\Omega}\text{}^{\hspace{1pt}T}\Big(\Gamma_3^{-1}J_b(u)\Big)\widetilde{\Omega}~~>0,~\forall(\vec{q}_e,\widetilde{\Omega})
\end{equation}
Again using the simplification from a quaternion's inherent properties in Eq:\ref{eq:4.17}, the LFC from Eq:\ref{eq:lyapunov-xpd} then simplifies with the following derivative:
\begin{subequations}
\vspace{-5pt}
\begin{equation}
V_{_{XPD}}=2(1-q_0)+\frac{1}{2}\widetilde{\Omega}\text{}^{\hspace{1pt}T}\Big(\Gamma_3^{-1}J_b(u)\Big)\widetilde{\Omega}
\end{equation}
\vspace{-11pt}
\begin{equation}
\dot{V}_{_{XPD}}=2\frac{1}{2}\vec{q}_e\text{}^T\vec{\omega}_e+\frac{1}{2}\dot{\widetilde{\Omega}}\text{}^{\hspace{1pt}T}\Big(\Gamma_3^{-1}J_b(u)\Big)\widetilde{\Omega}+\frac{1}{2}\widetilde{\Omega}\text{}^{\hspace{1pt}T}\Big(\Gamma_3^{-1}J_b(u)\Big)\dot{\widetilde{\Omega}}
\end{equation}
\vspace{-8pt}
\begin{equation}\label{eq:4.34c}
\dot{V}_{_{XPD}}=-\vec{q}_e\text{}^T\vec{\omega}_b+\frac{1}{2}\dot{\widetilde{\Omega}}\text{}^{\hspace{1pt}T}\Big(\Gamma_3^{-1}J_b(u)\Big)\widetilde{\Omega}+\frac{1}{2}\widetilde{\Omega}\text{}^{\hspace{1pt}T}\Big(\Gamma_3^{-1}J_b(u)\Big)\dot{\widetilde{\Omega}}\Big|_{\vec{\omega}_e=-\vec{\omega}_b}
\end{equation}
\end{subequations}
It then follows, substituting $\dot{\vec{\omega}}_b$ from Eq:\ref{eq:aux-pd-2}, the auxiliary plant's derivative $\dot{\widetilde{\Omega}}$ is:
\begin{subequations}
\vspace{-6pt}
\begin{equation}
\dot{\widetilde{\Omega}}=-\dot{\vec{\omega}}_b+\Gamma_1\dot{q_e}=-\vec{\omega}_b-\dot{\bar{\Omega}}
\end{equation}
\vspace{-15pt}
\begin{equation}
\rightarrow\dot{\vec{\omega}}_b=J_b^{-1}(u)\big(-\vec{\omega}_b\times J_b(u)\vec{\omega}_b-\hat{\tau}_b+\vec{\tau}_g+\vec{\tau}_Q+\vec{\tau}_{_{XPD}}\big)
\end{equation}
\vspace{-10pt}
\begin{equation}
\therefore\dot{\widetilde{\Omega}}=-J_b^{-1}(u)\big(-\vec{\omega}_b\times J_b(u)\vec{\omega}_b-\hat{\tau}_b+\vec{\tau}_g+\vec{\tau}_Q+\vec{\tau}_{_{XPD}}\big)-\dot{\bar{\Omega}}
\end{equation}
Substituting the auxiliary PD control law, $\vec{\tau}_{_{XPD}}$ from Eq:\ref{eq:control-aux-pd}, into the auxiliary derivative $\dot{\widetilde{\Omega}}$:
\begin{equation}
\rightarrow\dot{\widetilde{\Omega}}=-J_b^{-1}(u)\Big(\Gamma_2\widetilde{\Omega}+\Gamma_3\vec{q}_e-J_b(u)\dot{\bar{\Omega}}\Big)-\dot{\bar{\Omega}}
\end{equation}
\vspace{-15pt}
\begin{equation}
=J_b^{-1}(u)\Big(-\Gamma_2\widetilde{\Omega}-\Gamma_3\vec{q}_e\Big)
\end{equation}
\end{subequations}
From the \emph{approximately} diagonal inertial matrix $J_b(u)$ (Eq:\ref{eq:inertia-max} and Eq:\ref{eq:inertia-min}) and the positive symmetric (or \emph{diagonal}) properties of the coefficient matrices $\Gamma_1$,$\Gamma_2$ and $\Gamma_3$; the auxiliary plant $\dot{\widetilde{\Omega}}$ has a transpose:
\begin{equation}
\dot{\widetilde{\Omega}}\text{}^{\hspace{1pt}T}=J_b^{-1}\Big(-\Gamma_2\widetilde{\Omega}^{\hspace{1pt}T}-\Gamma_3\vec{q}_e\text{}^T\Big)
\end{equation}
The PD auxiliary plant component(s) in the LFC derivative $\dot{V}_{_{XPD}}$ in Eq:\ref{eq:lyapunov-xpd} simplifies:
\begin{subequations}
\begin{equation}
\frac{1}{2}\dot{\widetilde{\Omega}}\text{}^{\hspace{1pt}T}\Big(\Gamma_3^{-1}J_b(u)\Big)\widetilde{\Omega}=\frac{1}{2}\Big(-\Gamma_2\widetilde{\Omega}\text{}^{\hspace{1pt}T}-\Gamma_3\vec{q}_e\text{}^T\Big)\Gamma_3^{-1}\widetilde{\Omega}
\end{equation}
\vspace{-12pt}
\begin{equation}\label{eq:4.48b}
=\frac{1}{2}\Big(-\widetilde{\Omega}\text{}^{\hspace{1pt}T}\Gamma_2\Gamma_3^{-1}\widetilde{\Omega}-\vec{q}_e\text{}^{T}\widetilde{\Omega}\Big)
\end{equation}
Substituting Eq:\ref{eq:aux-pd-2} for $\vec{q}_e\text{}^T\widetilde{\Omega}$ into Eq:\ref{eq:4.48b}:
\begin{equation}\label{eq:4.48c}
\rightarrow\frac{1}{2}\dot{\widetilde{\Omega}}^T\Big(\Gamma_3^{-}J_b(u)\Big)\widetilde{\Omega}=\frac{1}{2}\Big(-\widetilde{\Omega}\text{}^{\hspace{1pt}T}\Gamma_2\Gamma_3^{-1}\widetilde{\Omega}+\vec{q}_e\text{}^T\vec{\omega}_b-\vec{q}_e\text{}^T\Gamma_1\vec{q}_e\Big)\Big|_{\vec{q}_e\text{}^{T}\widetilde{\Omega}=-\vec{q}_e\text{}^{T}\vec{\omega}_b+\Gamma_1\vec{q}_e\text{}^{T}}
\end{equation}
Similarly, for the transposed counterpart of Eq:\ref{eq:4.48c} in Eq:\ref{eq:4.34c}:
\begin{equation}
\frac{1}{2}\widetilde{\Omega}\text{}^{\hspace{1pt}T}\Big(\Gamma_3^{-1}J_b(u)\Big)\dot{\widetilde{\Omega}}=\frac{1}{2}\Big(-\widetilde{\Omega}\Gamma_2\Gamma_3^{-1}\widetilde{\Omega}\text{}^{\hspace{1pt}T}+\vec{q}_e\vec{\omega}_b\text{}^T-\vec{q}_e\Gamma_1\vec{q}_e\text{}^T\Big)
\end{equation}
Which, when substituted back into Eq:\ref{eq:4.34c}, then simplifies the LFC derivative to negative definite:
\end{subequations}
\begin{equation}
\Rightarrow\dot{V}_{_{XPD}}=-\vec{q}_e\text{}^T\Gamma_1\vec{q}_e-\widetilde{\Omega}\Gamma_2\Gamma_3^{-1}\widetilde{\Omega}\text{}^{\hspace{1pt}T}~~<0,~\forall(\vec{q}_e,\widetilde{\Omega}),~\exists(\Gamma_1,\Gamma_2,\Gamma_3)>0
\end{equation}
As such, the control law $\vec{\tau}_{_{XPD}}$ asymptomatically stabilizes the attitude plant globally. Both $\widetilde{\Omega}$ and $\vec{q}_e$ tend to $\vec{0}$, or more specifically the following global stability limits exist:
\begin{subequations}
\begin{equation}
\underset{t\rightarrow\infty}{\lim}\vec{q}_e=0~~\text{and}~~\underset{t\rightarrow\infty}{\lim}\widetilde{\Omega}=0
\end{equation}
Then, from the auxiliary plant definition(s) in Eq:\ref{eq:aux-pd-2}, the extended limits present themselves;
\begin{equation}
\underset{t\rightarrow\infty}{\lim}\vec{\omega}_b=\vec{0}\Big|_{\vec{\omega}_d=\vec{0}}~~~\text{and}~~~\underset{t\rightarrow\infty}{\lim}\bar{\Omega}=\vec{0}
\end{equation}
\end{subequations}
Whilst global asymptotic stability is indeed satisfactory, faster exponential stability is obviously more desirable. The stability proof for $V_{_{XPD}}$ can be extended to a stable exponentially bounded trajectory. From a quaternion's inherent definition it follows that $0\leq |q_0| \leq 1$. It can then be stated that:
\begin{equation}\label{eq:4.34}
1-|q_0|\leq 1-q_0^2=\norm{\vec{q}_e}^2
\end{equation}
Exponential stability is a maximum boundedness proof; the relationship Eq:\ref{eq:4.34} can then replace the quaternion scalar term $2(1-q_0)$ in $V_{_{XPD}}$ as an upper bound. The LFC is then rewritten in terms of its component's norm(s) to produce a bounding inequality:
\begin{subequations}\label{eq:xpd-ibc}
\begin{equation}
V_{_{XPD}}=\vec{q}_e\text{}^T\vec{q}_e+(q_0-1)^2+\frac{1}{2}\widetilde{\Omega}\text{}^{\hspace{1pt}T}\big(\Gamma_3^{-1}J_b(u)\big)\widetilde{\Omega}
\end{equation}
\vspace{-10pt}
\begin{equation}
\rightarrow V_{_{XPD}}\leq 2\norm{\vec{q}_e}\text{}^2+\frac{1}{2}\Gamma_3^{-1}J_b(u)||\widetilde{\Omega}||\text{}^2
\end{equation}
Similarly the LFC's derivative can be written in terms of its norms as:
\begin{equation}
\dot{V}_{_{XPD}}\leq-\Gamma_2\Gamma_3^{-1}||\widetilde{\Omega}||\text{}^2-\Gamma_1\norm{\vec{q}_e}\text{}^2
\end{equation}
\end{subequations}
The LFC, $V_{_{XPD}}$, has a maximum such that:
\begin{equation}
V_{_{XPD}}\leq max \bigg\{ 2,\frac{\lambda_{max}(\Gamma_3^{-1}J_b(u))}{2}\bigg\}\big(\norm{\vec{q}_e}\text{}^2+||\widetilde{\Omega}||\text{}^2\big)
\end{equation}
Where the function $\lambda_{max}$ represents the maximum eigenvalue of its argument; in this case $\Gamma_3^{-1}J_b(u)$. Similarly the \emph{negative definite} LCF derivative is bound by the minimum:
\begin{equation}
\dot{V}_{_{XPD}} \leq -min \big\{ \lambda_{min}(\Gamma_1),\lambda_{min}(\Gamma_2\Gamma_3\text{}^{-1})\big\}\big(\norm{\vec{q}_e}\text{}^2+||\widetilde{\Omega}||^2 \big)
\end{equation}
Therefore there exists some ratio $\alpha>0$ that satisfies the relationship requirement between the LCF and its derivative; $\dot{V}_{_{XPD}}\leq -\alpha V_{_{XPD}}$, where $\alpha$ is defined as the ratio:
\begin{equation}
\alpha=\frac{min\big\{\lambda_{min}(\Gamma_1),\lambda_{min}(\Gamma_2\Gamma_3\text{}^{-1})\big\}}{max\big\{2,\frac{\lambda_{max}(\Gamma_3\text{}^{-1}J_b(u))}{2}\big\}}
\end{equation}
The attitude trajectory $\big(\vec{q}_e(t),\widetilde{\Omega}(t)\big)$ is then exponentially bounded by:
\begin{equation}\label{eq:exponential-pd}
\big(\norm{\vec{q}_e(t)},||\widetilde{\Omega}(t)||\big)\leq Me^{-\alpha t/2}\big(\norm{\vec{q}_e(0)},||\widetilde{\Omega}(0)||\big)
\end{equation}
The bounding exponential coefficient $\alpha$ can be found using maximum Eigen values for the maximum inertia $J_b(u_{\Lambda})$ from Eq:\ref{eq:inertia-max}. Using the relationship in Eq:\ref{eq:exponential-pd} and testing proposed controller coefficients for $\Gamma_1,~\Gamma_2,~\Gamma_3$ the settling rate can be optimized.
\par
\emph{\color{Gray}The above stability proof for the auxiliary attitude controller was expanded upon and derived from \cite{attitudestabilization}, adapted to fit attitude setpoint tracking. Introduction of the quaternion error, which is dependent on the quaternion scalar, dramatically improves controller performance. The exponential stability notably improves settling times and overshoot errors, demonstrated in Sec:\ref{subsec:simulation.attitude.xpd}.}
\newpage
Interestingly a previous \cite{robustattitude} was the precursor for PD based attitude plants with asymptotic exponential stability. That proposed control law first did not make use of any defined \emph{auxiliary plants}, unlike Eq:\ref{eq:control-aux-pd}; however equivalent terms were effectively incorporated. The control law was developed for spacecraft attitude tracking and proposed a very similar exponentially stabilizing control scheme to that of $\vec{\tau}_{_{XPD}}$. That controller, when changed to the notational convention used here, designs body torque as:
\begin{equation}\label{eq:control-auxp-pd}
\vec{\tau}^{\hspace{3pt}'}_{_{XPD}}=-\frac{1}{2}\Big[\big([\vec{q}_e]_\times+q_0\mathbb{I}_{3\times 3}\big)\Gamma_1+\alpha\big(1-q_0\mathbb{I}_{3\times 3}\big)\Big]\vec{q}_e-\Gamma_2\vec{\omega}_b\in\mathcal{F}^b
\end{equation} 
Eq:\ref{eq:control-auxp-pd} could easily incorporate plant dependent compensation to accommodate for unwanted nonlinear dynamics. Both exponentially stabilizing PD controllers, from Eq:\ref{eq:control-aux-pd} and above Eq:\ref{eq:control-auxp-pd}, bear a striking similarity to the ideal backstepping controllers derived in the sequel, Eq:\ref{eq:control-ibc}. 
%====================================================
\subsection{Nonlinear Controllers}
\label{subsec:control.attitude.nonlinear}
%====================================================
Backstepping controllers(\cite{satellitebackstepping,intelligentbackstep,backstepslidingmode},etc\ldots) are a popular choice for nonlinear attitude control plants. The process, through iterative design, enforces Lyapunov stability criteria to ensure asymptotic stability. A report \cite{backstepping} surveys the fundamentals of backstepping procedure. Ideal backstepping control (\emph{IBC}) is a precise control solution which requires exact plant matching, something that is difficult to achieve in practice. Considering that some state estimate, $\hat{\mathbf{x}}(t)$ is used for feedback control. 
\par
The caveat of IBC control is poor robust stability performance; being especially susceptible to state dependent uncertainty. Unmodelled disturbances could easily drive the energy function away from stability conditions. The ideal backstepping algorithm can then be extended to incorporate such uncertainties. Adatively including disturbance and \emph{estimate} uncertainty into the LFC energy function improves the stability's robustness (Adaptive backstepping control, \emph{ABC}). By Lyapunov's theorem the respective estimation error terms are stabilized.
%====================================================
\subsubsection{Ideal Backstepping Controller}
\label{subsubsec:control.attitude.nonlinear.idealbackstep}
%====================================================
Starting with the ideal case for the first proposed backstepping controller, similar to \cite{satellitebackstepping}; it is assumed the attitude plant described in Eq:\ref{eq:quaternion-states-angular} from the consolidated model in Sec:\ref{sec:dynamics.model} exactly matches the dynamics of the physical prototype. The ideal backstepping controller aims to compensate for the plant's dynamic response to trajectory inputs perfectly. Neglecting uncertainties associated with the dynamic model, the aim here is to apply a stabilizing torque design. Recalling the quaternion tracking error from Eq:\ref{eq:quaternion-error}; $Q_e=Q_b^*\otimes Q_e$, consider the first LFC proposal for a quaternion error $Q_e$:
\begin{equation}\label{eq:ibc-lfc-1}
V_1(\vec{q}_e)=\vec{q}_e\text{}^T\vec{q}_e+(1-q_0)^2~~>0,~\forall(\vec{q}_e)
\end{equation}
After substituting in the quaternion rates but \emph{without} using the quaternion reduction proposed in Eq:\ref{eq:4.18}, $V_1(\vec{q}_e)$ has a Lie derivative:
\begin{subequations}
\begin{equation}
\dot{V}_1=2\vec{q}_e\text{}^T\frac{1}{2}\big([\vec{q}_e]_\times+q_0\mathbb{I}_{3X3}\big)\vec{\omega}_e-2\big(1-q_0\big)\dot{q}_0
\end{equation}
\vspace{-5pt}
\begin{equation}
=\vec{q}_e\text{}^T\Big([\vec{q}_e]_\times+q_0\mathbb{I}_{3\times 3}\Big)\vec{\omega}_e+\big(1-q_0\big)\vec{q}_e\text{}^T\vec{\omega}_e
\end{equation}
Simplifying further and then substituting the angular velocity set point; $\vec{\omega}_e=\vec{\omega}_d-\vec{\omega}_b=\vec{0}-\vec{\omega}=-\vec{\omega}_b$:
\begin{equation}
=\vec{q}_e\text{}^T[\vec{q}_e]_\times\vec{\omega}_e+\vec{q}_e\text{}^T\vec{\omega}_e
\end{equation}
\vspace{-5pt}
\begin{equation}
=-\vec{q}_e\text{}^T[\vec{q}_e]_\times\vec{\omega}_b-\vec{q}_e\text{}^T\vec{\omega}_b\Big|_{\vec{\omega}_e=-\vec{\omega}_b}
\end{equation}
\end{subequations}
Then choosing the first virtual backstepping control input $\gamma_d$. Note that $\gamma_d$ is used here to \emph{differentiate the backstepping design} variable from the trajectory commanded $\vec{\omega}_d$, Eq:\ref{eq:angular-error}. Choosing $\gamma_d$ such that the first LFC Eq:\ref{eq:ibc-lfc-1} is negative definite, $\dot{V}_1<0$:
\begin{equation}
\vec{\omega}_b\Rightarrow\gamma_d=\Gamma_1\vec{q}_e
\end{equation}
Where $\Gamma_1$ is the first symmetric positive definite gain matrix, a fact that is important to stress due to positive definite matrix's invertability. That backstepping input simplifies the LFC derivative $\dot{V}_1$ to the negative definite term:
\begin{subequations}
\begin{equation}
\dot{V}_1=-\vec{q}_e\text{}^T[\vec{q}_e]_\times\gamma_d-\vec{q}_e\text{}^T\gamma_d
\end{equation}
\vspace{-10pt}
\begin{equation}
=-\vec{q}_e\text{}^T[\vec{q}_e]_\times\Gamma_1\vec{q}_e-\vec{q}_e\text{}^T\Gamma_1\vec{q}_e
\end{equation}
And considering a vector cross product with itself has a zero resultant, $\vec{q}_e\text{}^T[\vec{q}_e]_\times=\vec{0}$, $\dot{V}_1$ then reduces:
\begin{equation}
=-\vec{q}_e\text{}^T\Gamma_1\vec{q}_e<0
\end{equation}
\end{subequations}
However, that backstepping input $\gamma_d$ has its own associated error. A stabilizing law $z_1$ needs to control that error:
\begin{subequations}
\begin{equation}
z_1\triangleq\gamma_d-\vec{\omega}_b=\Gamma_1\vec{q}_e-\vec{\omega}_b\Big|_{\gamma_d=\Gamma_1\vec{q}_e}
\end{equation}
\vspace{-8pt}
\begin{equation}
\rightarrow\vec{\omega}_b=\Gamma_1\vec{q}_e-z_1
\end{equation}
\vspace{-10pt}
\begin{equation}
\therefore\dot{V}_1=-\vec{q}_e\text{}^{T}\vec{\omega}_b=-\vec{q}_e\text{}^{T}\big(\Gamma_1\vec{q}_e-z_1\big)\Big|_{\vec{\omega}_b\Rightarrow\gamma_d}
\end{equation}
\vspace{-10pt}
\begin{equation}
=-\vec{q}_e\text{}^{T}\Gamma_1\vec{q}_e+\vec{q}_e\text{}^{T}z_1
\end{equation}
\end{subequations}
Introducing that error $z_1$ into a second LCF, which expands the first proposed $V_1$. And exploiting the fact that $\Gamma_1$ is symmetrical:
\begin{subequations}
\begin{equation}\label{eq:ibc-lfc}
V_2(\vec{q}_e,z_1)=V_1(\vec{q}_e)+\frac{1}{2}z_1\text{}^Tz_1
\end{equation}
\vspace{-6pt}
\begin{equation}
=\vec{q}_e\text{}^T\vec{q}_e+(1-q_0)^2+\frac{1}{2}z_1\text{}^Tz_1~~>0,~\forall(\vec{q}_e,z_1)
\end{equation}
\end{subequations}
That first error $z_1$ has its own time derivative, and recalling the body's angular acceleration $\dot{\vec{\omega}}_b$ from earlier with an undefined input $\vec{\tau}_{IBC}$, which still has plant dependency compensation.
\begin{subequations}
\begin{equation}
\dot{z}_1=\Gamma_1\dot{\vec{q}}_e-\dot{\vec{\omega}}_b
\end{equation}
\vspace{-12pt}
\begin{equation}
=\frac{\Gamma_1}{2}\big([\vec{q}_e]_\times+q_0\mathbb{I}_{3X3}\big)\vec{\omega}_e-\dot{\vec{\omega}}_b
\end{equation}
\vspace{-10pt}
\begin{equation}
=-\frac{\Gamma_1}{2}\big([\vec{q}_e]_\times+q_0\mathbb{I}_{3\times 3}\big)\vec{\omega}_b-\dot{\vec{\omega}}_b\Big|_{\vec{\omega}_e=-\vec{\omega}_b}
\end{equation}
\vspace{-4pt}
\begin{equation}\label{eq:z-deriv}
=-\frac{\Gamma_1}{2}\big([\vec{q}_e]_\times+q_0\mathbb{I}_{3\times 3}\big)\vec{\omega}_b-J_b(u)^{-1}\big(-\vec{\omega}_b\times J_b(u)\vec{\omega}_b-\hat{\tau}_b(u)+\vec{\tau}_g+\vec{\tau}_Q+\vec{\tau}_{IBC}\big)
\end{equation}
\end{subequations}
So then, following from Eq:\ref{eq:z-deriv}, finding the derivative of $\dot{V}_2$, with $\dot{V}_1=-\vec{q}_e\text{}^T(\Gamma_1\vec{q}_e-z_1)$:
\begin{subequations}\label{eq:ibc-deriv}
\begin{multline}
\dot{V}_2=-\vec{q}_e\text{}^T\big(\Gamma_1\vec{q}_e-z_1\big)+z_1\text{}^T\bigg(-\frac{\Gamma_1}{2}\big([\vec{q}_e]_\times+q_0\mathbb{I}_{3X3}\big)\vec{\omega}_b
\\
-J_b^{-1}(u)\big(-\vec{\omega}_b\times J_b(u)\vec{\omega}_b-\hat{\tau}_b(u)+\vec{\tau}_g+\vec{\tau}_Q+\vec{\tau}_{IBC}\big)\bigg)
\end{multline}
\vspace{-25pt}
\begin{multline}
=-\vec{q}_e\text{}^T\Gamma_1\vec{q}_e+z_1\text{}^T\bigg(\vec{q}_e-\frac{\Gamma_1}{2}\big([\vec{q}_e]_\times+q_0\mathbb{I}_{3X3}\big)\vec{\omega}_b
\\
-J_b^{-1}(u)\big(-\vec{\omega}_b\times J_b(u)\vec{\omega}_b-\hat{\tau}_b(u)+\vec{\tau}_g+\vec{\tau}_Q+\vec{\tau}_{IBC}\big)
\bigg)
\end{multline}
\end{subequations}
So then proposing the exactly matched stabilizing backstepping control law:
\begin{subequations}\label{eq:control-ibc}
\begin{equation}
\vec{\tau}_{_{IBC}}=J_b(u)\vec{q}_e-\frac{J_b(u)\Gamma_1}{2}\big([\vec{q}_e]_\times+q_0\mathbb{I}_{3X3}\big)\vec{\omega}_b+J_b(u)\Gamma_2z_1+\vec{\omega}_b\times J_b(u)\vec{\omega}_b+\hat{\tau}_b(u)-\vec{\tau}_g-\vec{\tau}_Q
\end{equation}
Noting that $z_1=\Gamma_1\vec{q}_e-\vec{\omega}_b$ and using the quaternion rate's vector definition, Eq:\ref{eq:quat-derivative}, the IBC torque law reduces:
\begin{equation}
=\underbrace{J_b(u)\Big((\Gamma_1\Gamma_2+1)\vec{q}_e-\Gamma_2\vec{\omega}_b+\Gamma_1\dot{\vec{q}}_e \Big)}_{\text{Ideal backstepping}}
+\underbrace{\vec{\omega}_b\times J_b(u)\vec{\omega}_b+\hat{\tau}_b(u)-\vec{\tau}_g-\vec{\tau}_Q}_{\text{Compenstation}}~~~~\in\mathcal{F}^{b}
\end{equation}
\end{subequations}
With $\Gamma_2$ being another positive-definite symmetric coefficient matrix. Then with the control law $\vec{\tau}_{_{IBC}}$ introduced into the LCF derivative, $\dot{V}_2$ simplifies to negative definite:
\begin{subequations}
\begin{multline}
\dot{V}_2=-\vec{q}_e\text{}^{T}\Gamma_1\vec{q}_e+z_1\text{}^T\bigg(\vec{q}_e-\frac{\Gamma_1}{2}\big([\vec{q}_e]_\times+q_0\mathbb{I}_{3\times 3}\big)\vec{\omega}_b
\\
-J_b^{-1}(u)\big(J_b(u)(\Gamma_1\Gamma_2+1)\vec{q}_e-J_b(u)\Gamma_2\vec{\omega}_b+\Gamma_1\dot{\vec{q}}_e\big)\bigg)
\end{multline}
\vspace{-12pt}
\begin{equation}
\therefore\dot{V}_2=-\vec{q}_e\text{}^T\Gamma_1\vec{q}_e+z_1\text{}^T\big(\Gamma_1\Gamma_2\vec{q}_e-\Gamma_2\vec{\omega}_b\big)
\end{equation}
\vspace{-10pt}
\begin{equation}
=-\vec{q}_e\text{}^T\Gamma_1\vec{q}_e-z_1\text{}^T\Gamma_2z_1~~<0,~\forall~(\vec{q}_e,z_1),~\exists(\Gamma_1,\Gamma_2)>0
\end{equation}
\end{subequations}
As such $\vec{q}_e\rightarrow 0$ \& $q_0\rightarrow 1$ as $t\rightarrow\infty$. Similarly $z_1\rightarrow 0$, which leads to the limit:
\begin{equation}
\underset{t\rightarrow\infty}{\lim}(\Gamma_1\vec{q}_e-\vec{\omega}_b)=\vec{0}
\end{equation} 
Because the quaternion error vector already tends to $0$;$~\vec{q}_e\rightarrow \vec{0}$, it follows that $\vec{\omega}_b\rightarrow \vec{0}$ as well. It can also be said that, from the definition of $\vec{\omega}_e$, that the angular velocity error stabilizes too. There is a distinct similarity in the structure of $\vec{\tau}_{_{IBC}}$ from Eq:\ref{eq:control-ibc} and that of the auxiliary PD controller presented in Eq:\ref{eq:control-aux-pd}. Expanding $\vec{\tau}_{_{XPD}}$ into state terms using the definitions of each auxiliary plant, $\widetilde{\Omega}$ and $\dot{\bar{\Omega}}$:
\begin{equation}\label{eq:simplified-auxpd}
\vec{\tau}_{_{XPD}}=\big(\Gamma_1\Gamma_2+\Gamma_3\big)\vec{q}_e-\Gamma_2\vec{\omega}_b-\frac{\Gamma_1J_b(u)}{2}([\vec{q}_e]_\times+q_0\mathbb{I}_{3\times 3})\vec{\omega}_b
\end{equation}
Furthermore, using the same reasoning from Eq:\ref{eq:xpd-ibc}, the exponential stability proof is proposed in the sequel. Recalling Eq:\ref{eq:4.34}:
\begin{subequations}
\begin{equation}
q_0-1\leq 1 - q_0^{2} = \norm{\vec{q_e}}^2
\end{equation}
\vspace{-15pt}
\begin{equation}
V_{_{IBC}}\leq V_2=2\norm{\vec{q}_e}\text{}^2+\frac{1}{2}\norm{z_1}\text{}^2
\end{equation}
\vspace{-15pt}
\begin{equation}
\dot{V}_{_{IBC}}\leq\dot{V}_2=-\Gamma_1\norm{\vec{q}_e}\text{}^2-\Gamma_2\norm{z_1}\text{}^2
\end{equation}
\end{subequations}
Then both the energy function and its derivative are bounded respectively by the following Eigen value limits:
\begin{subequations}
\begin{equation}
V_{_{IBC}}\leq \bigg\{2,~\frac{1}{2}\bigg\}(\norm{\vec{q}_e}\text{}^2+\norm{z_1}\text{}^2)
\end{equation}
\vspace{-10pt}
\begin{equation}
\dot{V}_{_{IBC}}\leq -min\big\{\lambda_{min}(\Gamma_1),~\lambda_{min}(\Gamma_2)\big\}(\norm{\vec{q}_e}\text{}^2+\norm{z_1}\text{}^2)
\end{equation}
\end{subequations}
Which then leads to a similar exponential stability trajectory boundedness such that:
\begin{subequations}
\begin{equation}
\dot{V}_{_{IBC}} \leq -\alpha V_{_{IBC}}
\end{equation}
\vspace{-15pt}
\begin{equation}
\therefore V\big(\norm{\vec{q}_e(t)},\norm{z_1(t)}\big)\leq Me^{-\alpha t/2}V\big(\norm{\vec{q}_e(0)},\norm{z_1(0)}\big)
\end{equation}
\end{subequations}
Whilst stabilizing, the IBC controller is not globaly stable. The plant conditions require exact plant matching and account for no unmodelled disturbances or measurement uncertainty. In practice the introduction of some disturbance torque $\vec{L}$ could potentially drive Eq:\ref{eq:ibc-deriv} away from negative-definite such that stability is lost.
%====================================================
\subsubsection{Adaptive Backstepping Controller}
\label{subsubsec:control.attitude.nonlinear.adaptivebackstep}
%====================================================
A lot of work has been done on the statistical nature of disturbance approximation and how best to adapt a nonlinear control system to the influence of unwanted disturbances; \cite{nonlineardisturbance,disturbanceadaptivebackstepping,robusttrackingcontrol}. Considering only a lumped uncertainty/disturbance term for the adaptive case and assuming state dependent uncertainty can be included in such a term.A lumped $\vec{L}_b$., in the body frame, is then added into the angular acceleration dynamics:
\begin{equation}
\dot{\vec{\omega}}_b=J_b\text{}^{-1}(u)\big(-\vec{\omega}_b\times J_b(u)\vec{\omega}_b-\hat{\tau}_b(u)+\vec{\tau}_g+\vec{\tau}_Q+\vec{L}_b+\vec{\tau}_{_{ABC}}\big)~~~~\in\mathcal{F}^{b}
\end{equation}
Unmodelled disturbances act as external torques on the Lagrangian in Eq:\ref{eq:newton.b}. Plant modelling errors and disturbances could simply be compensated for in the control law; $-\vec{L}_b$. It is, however, practically difficult to estimate disturbances without any \emph{apriori} knowledge about of its properties. Noise compensation in sensors can be done easily due to the known frequency bandwidth within which that noise occurs; the same cannot be said for wind disturbances or payload variations\ldots
\par
An approximate disturbance observer, $\hat{L}$, is used for that compensation in the designed control torque $\vec{\tau}_{_{ABC}}$. Each estimate will have its own error deviating from the physical $\vec{L}_b$ acting on the vehicle:
\begin{equation}\label{eq:estimate-error}
\vec{L}_\Delta=\vec{L}_b-\hat{L}
\end{equation}
Adaptive backstepping control introduces that observer's estimate error into an LFC to develop a derivative term for $\dot{\hat{L}}$, or a \emph{disturbance update law}, to asymptotically stabilize the estimate error. Typically, disturbance update rules are the primary contribution for satellite and generalized attitude control research papers, the statistical nature of disturbance approximation is a subject for another project. That estimate error, $\vec{L}_\Delta$ is then introduced to an LFC derived from the IBC case (previously in Eq:\ref{eq:ibc-lfc}):
\begin{subequations}
\begin{equation}
V_{_{ABC}}(\vec{q}_e,z_1,\vec{L}_\Delta)=V_{_{IBC}}(\vec{q}_e,z_1)+\frac{1}{2}\vec{L}_\Delta^{~T}\Gamma_L^{-1}\vec{L}_\Delta
\end{equation}
\vspace{-10pt}
\begin{equation}
=\vec{q}_e\text{}^T\vec{q}_e+(1-q_0)^2+\frac{1}{2}z_1\text{}^Tz_1+\frac{1}{2}\vec{L}_\Delta^{~T}\Gamma_L^{-1}\vec{L}_\Delta~~>0,~\forall(\vec{q}_e,z_1,\vec{L}_\Delta)
\end{equation}
\end{subequations}
Where $\Gamma_L$ is the positive $3\times 3$ positive adaptation gain matrix. That gain determines the rate at which the system \emph{adapts} to disturbances. The stability proof starts with the LFC rate $\dot{V}_{_{ABC}}$:
\begin{equation}
\dot{V}_{_{ABC}}(\vec{q}_e,z_1,\vec{L}_\Delta)=\dot{V}_{_{IBC}}(\vec{q}_e,~z_1)+\frac{1}{2}\dot{\vec{L}}_\Delta^{~T}\Gamma_L^{-1}\vec{L}_\Delta+\frac{1}{2}\vec{L}_\Delta^{~T}\Gamma_L^{-1}\dot{\vec{L}}_\Delta
\end{equation}
Recalling the definition of $\vec{L}_\Delta$ from Eq:\ref{eq:estimate-error}, for its derivative $\dot{\vec{L}}_\Delta$ it is reasonable to assume the rate at which the physical disturbance $\vec{L}$ changes is significantly slower than that of the control system, or that $\dot{\vec{L}}_b<<\dot{\hat{L}}$. The following approximation is then asserted:
\begin{equation}
\dot{\vec{L}}_\Delta=\dot{\vec{L}}_b-\dot{\hat{L}}\approx\vec{0}-\dot{\hat{L}}=-\dot{\hat{L}}\Big|_{\dot{\vec{L}}\approx\vec{0}}
\end{equation}
Substituting that estimation error rate back into the LFC derivative $\dot{V}_{_{ABC}}$ yields:
\begin{subequations}
\begin{multline}\label{eq:abc-lfc-deriv}
\dot{V}_{_{ABC}}=-\vec{q}_e\text{}^T(\Gamma_1\vec{q}_e-z_1)+z_1\text{}^T\bigg(-\frac{\Gamma_1}{2}\big([\vec{q}_e]_\times+q_0\mathbb{I}_{3\times 3}\big)\vec{\omega}_b
\\
-J_b^{-1}(u)\big(-\vec{\omega}_b\times J_b(u)\vec{\omega}_b-\hat{\tau}_b(u)+\vec{\tau}_g+\vec{\tau}_Q+\vec{L}_b+\vec{\tau}_{_{ABC}}\big)\bigg)-\vec{L}_\Delta^{~T}\Gamma_L^{-1}\dot{\hat{L}}
\end{multline}
Note that the physical disturbance term, $\vec{L}_b$, is included in  Eq:\ref{eq:abc-lfc-deriv}. Extending the ideal back stepping control law, $\vec{\tau}_{_{IBC}}$ from Eq:\ref{eq:control-ibc}, to include a disturbance \emph{estimate} $\hat{L}$ compensation term:
\begin{multline}
\vec{\tau}_{_{ABC}}=J_b(u)\Big(\big(\Gamma_1\Gamma_2+1\big)\vec{q}_e-\Gamma_2\vec{\omega}_b+\Gamma_1\dot{\vec{q}}_e\Big)+\vec{\omega}_b\times J_b(u)\vec{\omega}_b+\hat{\tau}_b(u)-\vec{\tau}_g-\vec{\tau}_Q-\hat{L}~~~~\in\mathcal{F}^b
\end{multline}
Turbulence torques, $\vec{L}_b$, act in the body frame and so require no reference frame transformation. Noting the control law compensates for disturbances with an estimate term. The energy function's derivative, $\dot{V}_{_{ABC}}$, then reduces to:
\begin{equation}
\dot{V}_{_{ABC}}=\dot{V}_{_{IBC}}-z_1\text{}^TJ_b^{-1}(u)\Big(\vec{L}_b-\hat{L}\Big)-\vec{L}_\Delta^{~T}\Gamma_L^{-1}\dot{\hat{L}}
\end{equation}
\vspace{-10pt}
\begin{equation}
=-\vec{q}_e\text{}^T\Gamma_1\vec{q}_e-z_1\text{}^T\Gamma_2z_1-z_1\text{}^TJ_b^{-1}(u)\vec{L}_\Delta-\vec{L}_\Delta^{~T}\Gamma_L^{-1}\dot{\hat{L}}
\end{equation}
\vspace{-6pt}
\begin{equation}
=-\vec{q}_e\text{}^T\Gamma_1\vec{q}_e-z_1\text{}^T\Gamma_2z_1-
\vec{L}_\Delta^{~T}\Gamma_L^{-1}\Big(\dot{\hat{L}}+\Gamma_LJ_b^{-1}(u)z_1\Big)
\end{equation}
\end{subequations}
The decision then needs to be made as to how the disturbance estimate is going to be updated such that asymptotic stability is assured; $\dot{V}_{_{ABC}}<0$. The obvious choice for $\dot{\hat{L}}$ would be to exactly compensate for $\Gamma_LJ_b^{-1}(u)z_1$ in the LFC:
\begin{equation}\label{eq:asymptotic-disturbance}
\dot{\hat{L}}\triangleq\Gamma_L J_b^{-1}(u)z_1=-\Gamma_L J_b^{-1}(u)\big(\Gamma_1\vec{q}_e-\vec{\omega_b}\big)\Big|_{z_1=\Gamma_1\vec{q}_e-\vec{\omega}_b}
\end{equation}
The disturbance is therefore compensated for and the estimate error is ensured to have asymptotic stability seeing that $V_{_{ABC}}$ is positive definite.
\begin{equation}\label{eq:abc-asymptotic}
\dot{V}_{_{ABC}}=-\vec{q}_e\text{}^T\Gamma_1\vec{q}_e-z_1\text{}^T\Gamma_2z_1~~<0,~\forall~(\vec{q}_e,z_1,\vec{L}_\Delta),~\exists(\Gamma_1,\Gamma_2,\Gamma_L)>0
\end{equation}
The typical stabilizing limits exists from Eq:\ref{eq:abc-asymptotic} can be drawn but most importantly the disturbance observer estimation error is stabilized:
\begin{subequations}
\begin{equation}
\underset{t\rightarrow\infty}{\lim}\vec{L}_{\Delta}\rightarrow \vec{0}
\end{equation}
\vspace{-12pt}
\begin{equation}
\therefore\underset{t\rightarrow\infty}{\lim}\hat{L}\rightarrow\vec{L}_b
\end{equation}
\end{subequations}
Fig:\ref{fig:disturbance_L} shows how the disturbance observer $\hat{L}$ approximates a (single axis) torque turbulence acting the vehicle in \emph{steady state}. A moderate damping manifests on the estimate in relationto the physical disturbance; resulting in an error shown Fig:\ref{fig:error_LR}. The example shown in \ref{fig:example_L} contains no attitude steps or trajectory changes. The torque turbulence, the observer and the adaptive controller's performance is detailed later in Sec:\ref{sec:simulation.disturbnace}.
\begin{figure}[hbtp]
\vspace{-6pt}
\centering
\begin{subfigure}{\textwidth}
\centering
\includegraphics[width=0.8\textwidth]{graphs/disturbance_L}
\vspace{-8pt}
\caption{Torque disturbance observer}
\label{fig:disturbance_L}
\end{subfigure}
\begin{subfigure}{\textwidth}
\centering
\includegraphics[width=0.8\textwidth]{graphs/error_LR}
\vspace{-8pt}
\caption{Torque disturbance error deviation $\vec{L}_\Delta$}
\label{fig:error_LR}
\end{subfigure}
\vspace{-6pt}
\caption{Adaptive disturbance observer example}
\label{fig:example_L}
\vspace{-25pt}
\end{figure}
%====================================================
\section{Position Control}
\label{sec:control.position}
%====================================================
Only two plant dependent position control laws are derived here, attitude control is the primary focus. The attitude control loop is stabilized independently from the position loop (Eq:\ref{eq:quaternion-states-angular} and Eq:\ref{eq:quaternion-states-acceleration}) but the Coriolis cross-coupling, from Eq:\ref{eq:states.b}, means the position loop first needs a stable attitude before being stabilized itself. A simple Proportional-Derivative structure is presented first as the reference case. Thereafter an ideal backstepping control which is extended to an adaptive control law is derived. Recalling the dynamics for translational acceleration from Eq:\ref{eq:quaternion-states-acceleration}:
\begin{equation}\label{eq:position-deriv}
\dot{\vec{v}}_b=m_b^{-1}\big(-\vec{\omega}_b\times m_b\vec{v}_b+m_b\vec{G}_b+\vec{F}_\mu(u)\big)~~~~[\text{m.s}^{-2}],~~\in\mathcal{F}^b
\end{equation}
Reiterating that the Coriolis acceleration term $-\vec{\omega_b}\times m_b\vec{v}_b$ is what couples the position loop to the attitude plant. Noting that $\vec{G}_b$ is the gravitational acceleration transformed to the body frame. Furthermore most texts assume that under standard operating conditions (App:\ref{app:equations.standard}) angular velocity is small if not negligible; $\vec{\omega}_b\approx\vec{0}$. Such an approximation makes the coupled Coriolis term assumed to be insignificant; $\vec{\omega}_b\times m\vec{v}_b\approx 0$. 
\par
If the plant's state is known, or atleast estimated with a relative degree of certainty, it is easy to compensate for those dynamics rather than making assumptions about their influence on the system. Such an introduced plant dependency can be compensated for in the designed control force $\vec{F}_\mu(u)$. The translational velocity, $\vec{v}_b$, defined in the body frame is related to the inertial position rates through a quaternion transformation:
\begin{equation}
\dot{\vec{\mathcal{E}}}_b=Q_b\otimes\vec{v}_b\otimes Q_b^*~~~~[\text{m.s}^{-1}],~~\in\mathcal{F}^I
\end{equation}
The difference in reference frames is an important distinction between the position and attitude state equations. Position error is calculated purely as a subtractive term from a particular setpoint $\vec{\mathcal{E}}_d$:
\begin{equation}
\vec{\mathcal{E}}_e=\vec{\mathcal{E}}_d-\vec{\mathcal{E}}_b~~~~[\text{m}],~~\in\mathcal{F}^I
\end{equation}
The translational position rate error $\dot{\vec{\mathcal{E}}}_b(t)$, \emph{not velocity error}, can be similarly calculated but, in the same way. In this case both position rate and velocity setpoints are zero, $\dot{\vec{\mathcal{E}}}_d=\vec{v}_d=\vec{0}$.
\begin{subequations}
\begin{equation}\label{eq:4.85a}
\dot{\vec{\mathcal{E}}}_e=\dot{\vec{\mathcal{E}}}_d-\dot{\vec{\mathcal{E}}}_b=-\dot{\vec{\mathcal{E}}}_b\Big|_{\dot{\vec{\mathcal{E}}}_d=\vec{0}}~~~~\in\mathcal{F}^{I}
\end{equation}
\vspace{-12pt}
\begin{equation}
\therefore \vec{v}_e = Q_b^*\otimes\big(\dot{\vec{\mathcal{E}}}_d-\dot{\vec{\mathcal{E}}}_b\big)\otimes Q_b = -\vec{v}_b~~~~\in\mathcal{F}^{b}
\end{equation}
\end{subequations}
Position setpoint aims is to produce a stabilizing control law $g(\vec{\mathbf{x}}_e,t)$ that ensures the position tracking error asymptotically tends to $\vec{0}$. Or more formally that:
\begin{subequations}
\begin{equation}
\vec{F}_\mu(u)=g\Big(\vec{\mathcal{E}}_d,\dot{\vec{\mathcal{E}}}_d,\vec{\mathcal{E}}_b,\dot{\vec{\mathcal{E}}}_b,t\Big)=g(\vec{\mathcal{E}}_e,\dot{\vec{\mathcal{E}}}_e,t)~~~~[\text{N}],~~\in\mathcal{F}^b
\end{equation}
\vspace{-12pt}
\begin{equation}
\text{Such that:}~~\underset{t\rightarrow\infty}{\lim}\vec{\mathcal{E}}_e\rightarrow\vec{0}
\end{equation}
\end{subequations}
%====================================================
\subsection{PD Controller}
\label{subsec:control.position.pd}
%==================================================== 
Starting with a simple Proportional-Derivative controller to be used for the reference case. Plant dependent control designs the net force proportional to both the position error and the first derivative velocity error:
\begin{subequations}\label{eq:position-pd}
\begin{equation}
\vec{F}_{_{PD}}=K_p\vec{\mathcal{E}}_e+K_d\dot{\vec{\mathcal{E}}}_e+\vec{\omega}_b\times m_b\vec{v}_b-m_b\vec{G}_b~~~~\in\mathcal{F}^b
\end{equation}
\vspace{-12pt}
\begin{equation}
=K_p\big(\vec{\mathcal{E}}_d-\vec{\mathcal{E}}_b\big)-K_d\big(\dot{\vec{\mathcal{E}}}_b\big)+\vec{\omega}_b\times m_b\vec{v}_b-m_b\vec{G}_b\Big|_{\dot{\vec{\mathcal{E}}}_e=-\dot{\vec{\mathcal{E}}}_b}
\end{equation}
\end{subequations}
The stability proof requires that error states are transformed to the body frame $\mathcal{F}^b$, such that the control input and error states all act in a common frame. Defining a position error state $\vec{X}_e$ transformed to the body frame:
\begin{subequations}
\begin{equation}\label{eq:4.80a}
\vec{X}_e\triangleq Q_b\otimes(\vec{\mathcal{E}}_d-\vec{\mathcal{E}}_b)\otimes Q_b^*=\vec{X}_d-\vec{X}_b~~~~\in\mathcal{F}^{b}
\end{equation}
Recalling the difference between position rates and translational velocity in Eq:\ref{eq:quaternion-states-velocity}, position rates are then:
\begin{equation}\label{eq:4.80b}
\dot{\vec{X}}_e\triangleq Q_b\otimes(\dot{\vec{\mathcal{E}}}_d-\dot{\vec{\mathcal{E}}}_b)\otimes Q_b^*=-Q_b\otimes\dot{\vec{\mathcal{E}}}_b\otimes Q_b^* = -\vec{v}_b\Big|_{\dot{\vec{\mathcal{E}}}_d=\vec{0}}
\end{equation}
\end{subequations}
The control law from Eq:\ref{eq:position-pd}, despite being $\in\mathcal{F}^b$ has arguments $\vec{\mathcal{E}}_e,\dot{\vec{\mathcal{E}}}_e\in\mathcal{F}^I$, which are substituted with the transformed position error $\vec{X}e$:
\begin{subequations}
\begin{equation}
\vec{F}_{_{PD}}=K_p\vec{X}_e + K_d\dot{\vec{X}}_e + \vec{\omega}_b\times m_b\vec{v}_b-m_b\vec{G}_b~~~~\in\mathcal{F}^{b}
\end{equation}
\vspace{-15pt}
\begin{equation}
=K_p\vec{X}_e-K_d\vec{v}_b+\vec{\omega}_b\times m_b\vec{v}_b-m_b\vec{G}_b
\end{equation}
\end{subequations}
Then proposing a positive definite Lyapunov function candidate:
\begin{subequations}
\begin{equation}
V_{_{PD}}(\vec{X}_e,\dot{\vec{X}}_e)=\frac{1}{2}\vec{X}_e\text{}^TK_p\vec{X}_e+\frac{1}{2}\dot{\vec{X}}_e\text{}^Tm_b\dot{\vec{X}}_e>0~~~\forall(\vec{X}_e,\dot{\vec{X}}_e)
\end{equation}
\vspace{-12pt}
\begin{equation}
=\frac{1}{2}\vec{X}_e\text{}^TK_p\vec{X}_e+\frac{1}{2}\vec{v}_b\text{}^Tm_b\vec{v}_b\Big|_{\dot{\vec{X}}_e=-\vec{v}_b}
\end{equation}
\end{subequations}
Calculating that LFC's derivative $\dot{V}_{_{PD}}$ with the PD control law substituted:
\begin{subequations}
\begin{equation}
\dot{V}_{_{PD}}(\vec{X}_e,\dot{\vec{X}}_e)=\vec{X}_e\text{}^TK_p\dot{\vec{X}}_e+\vec{v}_b\text{}^Tm_b\dot{\vec{v}}_b
\end{equation}
\vspace{-10pt}
\begin{equation}
=-\vec{X}_e\text{}^TK_p\vec{v}_b+\vec{v}_b\text{}^Tm_b\dot{\vec{v}}_b
\end{equation}
\vspace{-8pt}
\begin{equation}
=-\vec{X}_e\text{}^TK_p\vec{v}_b+\vec{v}_b\text{}^T\big(-\vec{\omega}_b\times m_b\vec{v}_b+m_b\vec{G}_b+\vec{F}_{_{PD}}\big)
\end{equation}
\vspace{-8pt}
\begin{equation}
=-\vec{X}_e\text{}^TK_p \vec{v}_b+\vec{v}_b\text{}^T\big(K_p \vec{X}_e-K_d\vec{v}_b\big)
\end{equation}
\vspace{-8pt}
\begin{equation}\label{eq:position-pd-stability}
\therefore\dot{V}_{_{PD}}=-\vec{v}_b\text{}^TK_d\vec{v}_b~~<0,~\forall(\vec{X}_e,\dot{\vec{X}}_e),~\exists(K_d,K_p)>0
\end{equation}
\end{subequations}
The global stability asserted in Eq:\ref{eq:position-pd-stability} holds for $\forall(\vec{\mathcal{E}}_e,\dot{\vec{\mathcal{E}}}_e)$, irrespective of the transformation applied in Eq:\ref{eq:4.80a} and Eq:\ref{eq:4.80b}. Global asymptotically stabilizing limits then follow:
\begin{subequations}
\begin{equation}
\underset{t\rightarrow\infty}{\lim}\vec{X}_e=Q_b\otimes(\vec{\mathcal{E}}_d-\vec{\mathcal{E}}_b)\otimes Q_b^*\rightarrow\vec{0}
\end{equation}
\vspace{-10pt}
\begin{equation}
\therefore\underset{t\rightarrow\infty}{\lim}\vec{\mathcal{E}}_b\rightarrow\vec{\mathcal{E}}_d
\end{equation}
\vspace{-6pt}
\begin{equation}
\underset{t\rightarrow\infty}{\lim}\dot{X}_e=Q_b^*\otimes(\dot{\vec{\mathcal{E}}}_d-\dot{\vec{\mathcal{E}}}_b)\otimes Q_b=-\vec{v}_b\rightarrow\vec{0}\Big|_{\dot{\vec{\mathcal{E}}}_e=0}
\end{equation}
\end{subequations}
%====================================================
\subsection{Adaptive Backstepping Controller}
\label{subsec:control.position.bacstepping}
%====================================================
An adaptive backstepping algorithm, analogue to the adaptive controller previously in Sec:\ref{subsubsec:control.attitude.nonlinear.adaptivebackstep}, is now applied to position control. The disturbance term, $\vec{D}\in\mathcal{F}^b$, introduced to the position state differential Eq:\ref{eq:position-deriv}, represents any unmodelled lumped drag \emph{and} wind forces encountered by the vehicle in flight. Backstepping iterations for the position control loop first need to stabilize the position error and only thereafter compensate those disturbances (solving for \emph{IBC} then adding adaptivity).
\begin{equation}
\dot{\vec{v}}_b=m_b^{-1}\big(-\vec{\omega}_b\times m_b\vec{v}_b+m_b\vec{G}_b+\vec{D}_b+\vec{F}_{_{ABC}}\big)~~~~\in\mathcal{F}^b
\end{equation}
The compensation for $\vec{D}$ is obviously an approximation for that physical disturbance term; $\hat{D}$. Beginning the backstepping process for position with a position state tracking error:
\begin{equation}
z_1\triangleq\vec{\mathcal{E}}_d-\vec{\mathcal{E}}_b=\vec{\mathcal{E}}_e~~~~\in\mathcal{F}^{I}
\end{equation}
Which then has its own derivative:
\begin{subequations}
\begin{equation}
\dot{z}_1=\dot{\vec{\mathcal{E}}}_e=\dot{\vec{\mathcal{E}}}_d-\dot{\vec{\mathcal{E}}}_b
\end{equation}
\vspace{-12pt}
\begin{equation}
=Q_b^*\otimes \big(\vec{v}_d-\vec{v}_b\big)\otimes Q_b = - Q_b^*\otimes \vec{v}_b\otimes Q_b\Big|_{\vec{v}_d=\vec{0}}
\end{equation}
\end{subequations}
Transforming that error $z_1$ to the body frame $\mathcal{F}^b$ in a similar fashion to Eq:\ref{eq:4.80a} makes the stability proof more concise. The reference frame transformation does not affect the Lie derivative as the energy function's gradient depends on its partial derivative w.r.t it's positional trajectory only, namely $\mathcal{E}_e(t)$.
\begin{subequations}
\begin{equation}
\hat{z}_1\triangleq Q_b\otimes z_1 \otimes Q_b^* =Q_b\otimes\big(\vec{\mathcal{E}}_d-\vec{\mathcal{E}}_b\big)\otimes Q_b^*=\vec{X}_e~~~~\in\mathcal{F}^{b}
\end{equation}
\vspace{-10pt}
\begin{equation}
\therefore \dot{\hat{z}}_1=Q_b\otimes\dot{z}_1\otimes Q_b^* = Q_b\otimes\big(\dot{\vec{\mathcal{E}}}_d-\dot{\vec{\mathcal{E}}}_b\big)\otimes Q_b^* = -\vec{v}_b
\end{equation}
\end{subequations}
Proposing the first LFC, $V_1(\hat{z}_1)$, in terms of that tracking error with a derivative $\dot{V}_1$:
\begin{subequations}
\begin{equation}
V_1(\hat{z}_1)=\frac{1}{2}\hat{z}_1\text{}^{T}\hat{z}_1~~>0,~\forall(\hat{z}_1)
\end{equation}
\vspace{-12pt}
\begin{equation}
\Rightarrow\dot{V}_1(\hat{z}_1)=\hat{z}_1\text{}^T\dot{\hat{z}}_1=-\hat{z}_1\text{}^T\vec{v}_b
\end{equation}
\end{subequations}
The first stabilizing velocity function, $\gamma_d$, and its associated error, $\hat{z}_2$, are defined as:
\begin{subequations}
\begin{equation}
\vec{v}_b\Rightarrow\gamma_d = \Gamma_1 \hat{z}_1
\end{equation}
\vspace{-15pt}
\begin{equation}
\hat{z}_2 \triangleq \gamma_d - \vec{v}_b = \Gamma_1\hat{z}_1-\vec{v}_b
\end{equation}
\vspace{-15pt}
\begin{equation}\label{eq:4.90c}
\therefore \vec{v}_b=\Gamma_1\hat{z}_1-\hat{z}_2
\end{equation}
\end{subequations}
Changing that first LFC with variable substitution such that:
\begin{equation}
V_1=-\hat{z}_1^T\vec{v}_b=-\hat{z}_1^T\Gamma_1\hat{z}_1+\hat{z}_1^T\hat{z}_2
\end{equation}
So that second error state $\hat{z}_2$ has a derivative:
\begin{subequations}
\begin{equation}
\dot{\hat{z}}_2=\dot{\gamma}_d-\dot{\vec{v}}_b=\Gamma_1\dot{\hat{z}}_1-m_b^{-1}\big(-\vec{\omega}_b\times m_b\vec{v}_b+m_b\vec{G}_b+\vec{D}_b+\vec{F}_{_{ABC}}\big)
\end{equation}
\vspace{-12pt}
\begin{equation}\label{eq:102b}
=-\Gamma_1\vec{v}_b-m_b^{-1}\big(-\vec{\omega}_b\times m_b\vec{v}_b+m_b\vec{G}_b+\vec{D}_b+\vec{F}_{_{ABC}}\big)
\end{equation}
\end{subequations}
Introducing that second error $\hat{z}_2$ into a new LFC $V_2$:
\begin{subequations}
\begin{equation}
V_2(\hat{z}_1,\hat{z}_2)=V_1(\hat{z}_1)+\frac{1}{2}\hat{z}_2\text{}^T\hat{z}_2
\end{equation}
\vspace{-12pt}
\begin{equation}
=\frac{1}{2}\hat{z}_1\text{}^T\hat{z}_1+\frac{1}{2}\hat{z}_2\text{}^T\hat{z}_2~~>0,~\forall(\hat{z}_1,\hat{z}_2)
\end{equation}
\end{subequations}
Which has a derivative, with $\dot{\hat{z}}_2$ substituted from Eq:\ref{eq:102b}:
\begin{subequations}
\begin{equation}
\dot{V}_2(\hat{z}_1,\hat{z}_2)=\dot{V}_1(\hat{z}_1)+\hat{z}_2\text{}^T\dot{\hat{z}}_2=\hat{z}_1\text{}^T\dot{\hat{z}}_1+\hat{z}_2\text{}^T\dot{\hat{z}}_2
\end{equation}
\vspace{-10pt}
\begin{equation}
=-\hat{z}_1^T\Gamma_1\hat{z}_1+\hat{z}_1\text{}^T\hat{z}_2+\hat{z}_2\text{}^T\dot{\hat{z}}_2
\end{equation}
\vspace{-10pt}
\begin{equation}
=-\hat{z}_1\text{}^T\Gamma_1\hat{z}_1+\hat{z}_2\text{}^T\bigg(\hat{z
}_1-\Gamma_1\vec{v}_b-m_b^{-1}\big(-\vec{\omega}_b\times m_b\vec{v}_b+m\vec{G}_b+\vec{D}_b+\vec{F}_{_{ABC}}\big)\bigg)
\end{equation}
\end{subequations}
An ideal backstepping control law, with the assumption that $\vec{D}_b$ is precisely known, is then:
\begin{subequations}
\begin{equation}
\vec{F}_{_{IBC}}=m_b\big(\hat{z}_1-\Gamma_1\vec{v}_b+\Gamma_2\hat{z}_2\big)+\vec{\omega}_b\times m_b\vec{v}_b-m_b\vec{G}_b-\vec{D}_b~~~~\in\mathcal{F}^{b}
\end{equation}
\vspace{-12pt}
\begin{equation}
=m_b\Big(\big(1+\Gamma_1\Gamma_2\big)\hat{z}_1-\big(\Gamma_1+\Gamma_2\big)\vec{v}_b\big)\Big)+\vec{\omega}_b\times m_b\vec{v}_b-m_b\vec{G}_b-\vec{D}_b
\end{equation}
Making $\dot{V}_2$ negative definite:
\begin{equation}
\Rightarrow \dot{V}_{_{IBC}}=\dot{V}_2=-\hat{z}_1\text{}^T\Gamma_1\hat{z}_1-\hat{z}_2\text{}^T\Gamma_2\hat{z}_2~~<0,~\forall(\hat{z}_1,\hat{z}_2),~\exists(\Gamma_1,\Gamma_2)>0
\end{equation}
\end{subequations}
Which leads to global asymptotic stability, assuming that the disturbance term $\vec{D}_b$ is known and can be compensated for. In the controller both $\Gamma_1$ and $\Gamma_2$ are positive symmetric control coefficient matrices to be optimized. Extending the backstepping rule and proposed LFC to incorporate an adaptive disturbance approximator $\hat{D}$, similar to the attitude controller in Sec:\ref{subsubsec:control.attitude.nonlinear.adaptivebackstep}. The approximation leads to an estimate error $\vec{D}_\Delta$, assuming that physical disturbances $\dot{\vec{D}}_b$ are far slower than the control dynamics; $\dot{\vec{D}}_b<<\dot{\hat{D}}$.
\begin{subequations}
\begin{equation}
\vec{D}_\Delta=\vec{D}_b-\hat{D}~~~~\in\mathcal{F}^b
\end{equation}
\vspace{-16pt}
\begin{equation}
\therefore\dot{\vec{D}}_\Delta=\dot{\vec{D}}_b-\dot{\hat{D}}\approx\vec{0}-\dot{\hat{D}}=-\dot{\hat{D}}\Big|_{\dot{\vec{D}}_b\approx\vec{0}}
\end{equation}
The control law then designs a force, using that disturbance observer $\hat{D}$:
\begin{equation}
\vec{F}_{_{ABC}}=m_b\big(\hat{z}_1-\Gamma_1\vec{v}_b+\Gamma_2\hat{z}_2\big)+\vec{\omega}_b\times m_b\vec{v}_b-m_b\vec{G}_b-\hat{D}~~~~\in\mathcal{F}^b
\end{equation}
\end{subequations}
Proposing an LFC extended from the IBC case which includes that disturbance estimate error $\vec{D}_\Delta$ and finding it's derivative:
\begin{subequations}
\begin{equation}
V_{_{ABC}}(\hat{z}_1,\hat{z}_2,\vec{D}_\Delta)= V_{_{IBC}}(\hat{z}_1,\hat{z}_2)+\frac{1}{2}\vec{D}_\Delta^{~T}\Gamma_D^{-1}\vec{D}_\Delta
\end{equation}
\vspace{-14pt}
\begin{equation}
=\frac{1}{2}\hat{z}_1\text{}^T\hat{z}_1+\frac{1}{2}\hat{z}_2\text{}^T\hat{z}_2+\frac{1}{2}\vec{D}_\Delta^{~T}\Gamma_D^{-1}\vec{D}_\Delta~~>0,~\forall(\hat{z}_1,\hat{z}_2,\vec{D}_\Delta)
\end{equation}
\vspace{-10pt}
\begin{equation}
\Rightarrow\dot{V}_{_{ABC}}=\hat{z}_1\text{}^T\dot{\hat{z}}_1+\hat{z}_2\text{}^T\dot{\hat{z}}_2+\vec{D}_\Delta^{~T}\Gamma_D^{-1}\dot{\vec{D}}_\Delta
\end{equation}
Then substituting derivatives for $\dot{\hat{z}}_2$ and $\dot{\vec{D}}_\Delta$:
\begin{equation}
=-\hat{z}_1^T\Gamma_1\hat{z}_1+\hat{z}_2^T\bigg(\hat{z}_1-\Gamma_1\vec{v}_b-m_b^{-1}\big(-\vec{\omega}_b\times m_b\vec{v}_b+m_b\vec{G}_b+\vec{D}_b+\vec{F}_{_{ABC}}\big)\bigg)-\vec{D}_\Delta^{~T}\Gamma_D^{-1}\dot{\hat{D}}
\end{equation}
\vspace{-10pt}
\begin{equation}
=-\hat{z}_1\text{}^T\Gamma_1\hat{z}_1+\hat{z}_2\text{}^T\bigg(-\Gamma_2\hat{z}_2-m_b^{-1}\big(\vec{D}_b-\hat{D}\big)\bigg)-\vec{D}_\Delta^{~T}\Gamma_D^{-1}\dot{\hat{D}}
\end{equation}
\vspace{-8pt}
\begin{equation}
=-\hat{z}_1\text{}^T\Gamma_1\hat{z}_1-\hat{z}_2\text{}^T\Gamma_2\hat{z}_2-m_b^{-1}\hat{z}_2^T\vec{D}_\Delta-\vec{D}_\Delta^{~T}\Gamma_D^{-1}\dot{\hat{D}}
\end{equation}
\vspace{-10pt}
\begin{equation}\label{eq:4.105g}
=-\hat{z}_1\text{}^T\Gamma_1\hat{z}_1-\hat{z}_2\text{}^T\Gamma_2\hat{z}_2-m_b^{-1}\vec{D}_\Delta^{~T}\Gamma_D^{-1}\big(\Gamma_D\hat{z}_2+\dot{\hat{D}}\big)
\end{equation}
\end{subequations}
Then, a self-evident choice for the disturbance update law would be; $\dot{\hat{D}}=-m^{-1}\Gamma_D\hat{z}_2$, which ensures asymptotic stability. Substituting that into the LFC derivative Eq:\ref{eq:4.105g} produces:
\begin{subequations}
\begin{equation}\label{eq:abc-asymptotic-position}
\dot{\hat{D}}=-m_b^{-1}\Gamma_D\hat{z}_2=-m_b^{-1}\Gamma_D\Big(\Gamma_1\hat{z}_1-\vec{v}_b\Big)
\end{equation}
\vspace{-12pt}
\begin{equation}
\therefore\dot{V}_{_{ABC}}=-\hat{z}_1\text{}^T\Gamma_1\hat{z}-\hat{z}_2\text{}^T\Gamma_2\hat{z}_2~~<0,~\forall(\hat{z}_1,\hat{z}_2,\vec{D}_\Delta),~\exists(\Gamma_1,\Gamma_2,\Gamma_\Delta)
\end{equation}
\end{subequations}
\par
\begin{figure}[hbtp]
\vspace{-14pt}
\centering
\begin{subfigure}{\textwidth}
\centering
\includegraphics[width=0.8\textwidth]{graphs/disturbance_D}
\vspace{-8pt}
\caption{Force disturbance observer}
\label{fig:disturbance_D}
\end{subfigure}
\vspace{-26pt}
\end{figure}
\newpage
\begin{figure}[htbp]\ContinuedFloat
\begin{subfigure}{\textwidth}
\centering
\includegraphics[width=0.8\textwidth]{graphs/error_LR}
\vspace{-8pt}
\caption{Force disturbance error deviation $\vec{D}_\Delta$}
\label{fig:error_DR}
\end{subfigure}
\vspace{-6pt}
\caption{Adaptive disturbance observer example}
\label{fig:example_D}
\vspace{-10pt}
\end{figure}
\par
The disturbance observer tracks a general single axis directional force disturbance as illustrated in Fig:\ref{fig:disturbance_D}. The disturbance is a combined fluctuating wind force and vector field; the model of which is later described in Sec:\ref{subsec:simulation.disturbance.force}. Note that Fig:\ref{fig:example_D} tracks an \emph{open loop} disturbance on a vehicle stabilized steady state. An estimation error for the deviation from the physical disturbance is plotted in Fig:\ref{fig:error_DR}. Again there is a damping between the physical and approximated forces; no new state information is used to estimate signals in both Fig:\ref{fig:disturbance_L} and Fig:\ref{fig:disturbance_D} for attitude and position disturbances respectively. Adaptive observers in Eq:\ref{eq:abc-asymptotic} and Eq:\ref{eq:abc-asymptotic-position} simply introduce additional free parameters to the control loop\ldots
