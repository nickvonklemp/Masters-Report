%====================================================
%	CHAPTER 4 - Control
%====================================================
\chapter{Controller Development}
\label{ch:control}
%====================================================
\section{Control Loop}
%====================================================
The control problem for this dissertation is, as outlined in Chater:\ref{ch:intro}; to achieve dynamic (\emph{attitude}) set point tracking on a quadrotor by solving the problem of its inherent underactuation. For the purposes of the subsequent controller development, the plant is described in non-linear state space form:
\begin{subequations}
\begin{equation}
\dot{\mathbf{x}}=f(\mathbf{x},t)+g(\mathbf{x},t)\vec{\nu}
\end{equation}
\vspace{-15pt}
\begin{equation}
y = c(\mathbf{x},t)+d(\mathbf{x},t)\vec{\nu}
\end{equation}
\end{subequations}
Where the plant dynamics are governed by $f(\mathbf{x},t)$ and its input response by $g(\mathbf{x},t)\vec{\nu}$, for a given input $\vec{\nu}$. The latter is not necessarily a multiplicative relationship and could take the form $g(\mathbf{x},\vec{\nu},t)$. The objective for setpoint tracking is for the output to track the state $y = \mathbf{x}$. As such, the control problem is to design a control law for an error state $\mathbf{x}_e$:
\\
\vspace{-5pt}
\begin{equation}
\vec{\nu}_d=h(\mathbf{x}_e,t)
\end{equation}
Such that the control plant is globally asymptotically stabilizing or that $\lim_{t\rightarrow\infty}\mathbf{x}_e=0$. It is possible to combine attitude and position states\footnote{Ignoring how error states are formulated for the time being\ldots} into a common trajectory state such that:
\\
\vspace{-5pt}
\begin{equation}
\mathbf{x}=\begin{bmatrix}
\vec{\mathcal{E}}\\
Q_b
\end{bmatrix}
\end{equation}
The body's trajectory is then fully described by $\mathbf{x}(t)$. Independent controllers are developed for attitude and position control and hence attitude and position states aren't combined. However for the purposes of detailing the control plant, a single major loop is considered. The designed control input, $\vec{\nu}_d$, is then implemented by actuator suite $u\in\mathbb{U}$ through its effectiveness function:
\\
\vspace{-5pt}
\begin{equation}
\nu_c=B(\mathbf{x},u,t)
\end{equation}
The exact relationship of the virtual control input and commanded input, $\nu_c\rightarrow\nu_d$, is governed by the allocation algorithm. That allocation function, $B^\dagger$, can be \emph{approximately} referred to as the effectiveness inverse. The actuator positions are then solved as:
\begin{equation}
u=B^{\dagger}(\mathbf{x},\nu_d,t)
\end{equation}
More on the control allocation is discussed subsequently in Section:\ref{sec:control.inputs}. Multiple attitude controllers are developed and compared in the context of an over actuated quadrotor plant. Similarly a series of allocation schemes are compared too. Those comparisons and their details are presented next in Chapter:\ref{ch:simulation}. 
\par
The control loop is then broken into 

Previously, in the Design Chapter:\ref{.}, it was shown that the uniqueness of the prototype considered here stems from its over-actuation. There are 12 actuator positions which are required to effect 6 degrees of freedom. Owing to the complexities which arise as a result of control plant allocation, the control algorithm is first abstracted to virtual control inputs $\tau_net$ and $F_net$. Those victual inputs are then distributed via the allocation algorithm. The control block is completely independent from the allocation block and as such the two can be altered independently.
%====================================================
\section{Control Plant Inputs}
\label{sec:control.inputs}
%====================================================
Until now, the control plant inputs for state equations Eq:\ref{.} have been described with net forces and torques. Previously in Section:\ref{.} the relationship between servo positions and thrust vector directions were quantified. In Section:\ref{.} that thrust vector magnitude was found to be \emph{approximately} quadratically dependent on the propellers rotational speed. 
\par
Appropriate control allocation techniques are required to manipulate all 12 actuators to effect the 6 net forces and torques. The control plant is split into two stages, namely a higher level set point tracking controller which applies 

For the control plant development, the plant inputs are abstracted to a virtual plant force and torque inputs. A secondary control layer applies the 
\subsection*{Model Dependent \& Independent Controllers}
%****************************************************

%****************************************************
\section{Attitude Control}
\label{sec:control.attitude}
%****************************************************
\subsection{The Attitude Control Problem}
\label{subsec:control.attitude.problem}
%****************************************************
\subsection{Quaternion Based Error States}
\label{subsec:control.attitude.quaternion}
%****************************************************
\subsubsection{PD Controller}
%****************************************************
\subsubsection{Auxilliary Plant Controller}
%****************************************************
\subsubsection{PID Controller}
%****************************************************
\subsection{Non-linear Controllers}
\label{subsec:control.attitude.nonlinear}
%****************************************************
\subsubsection{Ideal Back-stepping Controller}
%****************************************************
\subsubsection{Adaptive Back-stepping Controller}
\label{subsubsec:control.attitude.nonlinear.backstep}
\subsubsection*{Disturbance Update Law}
%****************************************************
\subsubsection{Lyupanov Derived Ideal Controller}
%Laselles Theorem
%****************************************************

%****************************************************
\section{Position Control}
\label{sec:control.position}
%****************************************************
\subsection{Backstepping Position Controller}
\label{subsec:control.position.bacstepping}
%****************************************************

%****************************************************
\section{Controller Allocation}
\label{sec:control.allocation}
%****************************************************
\subsection{Non-linear Plant Control Allocation}
\label{subsec:control.allocation.allocators}
%****************************************************
\subsection{Pseudo Inverse Allocator}
%****************************************************
\subsection{Weighted Pseudo Inverse Allocator}
%****************************************************
\subsection{Priority Norm Inverse Allocator}
%****************************************************
\subsection{Online Optimized Secondary Goal Allocator}
%****************************************************
