%====================================================
%	CHAPTER 4 - Control
%====================================================
\chapter{Control Treatment}
\label{ch:control}
%====================================================
\section*{Control Plant \& Discussion}
%====================================================
Previously, in the Design Chapter:\ref{.}, it was shown that the uniqueness of the prototype considered here stems from its over-actuation. There are 12 actuator positions which are required to effect 6 degrees of freedom. Owing to the complexities which arise as a result of control plant allocation, the control algorithm is first abstracted to virtual control inputs $\tau_net$ and $F_net$. Those victual inputs are then distributed via the allocation algorithm. The control block is completely independent from the allocation block and as such the two can be altered independently.
%====================================================
\subsection*{Control Plant Inputs}
%====================================================
Until now, the control plant inputs for state equations Eq:\ref{.} have been described with net forces and torques. Previously in Section:\ref{.} the relationship between servo positions and thrust vector directions were quantified. In Section:\ref{.} that thrust vector magnitude was found to be \emph{approximately} quadratically dependent on the propellers rotational speed. 
\par
Appropriate control allocation techniques are required to manipulate all 12 actuators to effect the 6 net forces and torques. The control plant is split into two stages, namely a higher level set point tracking controller which applies 

For the control plant development, the plant inputs are abstracted to a virtual plant force and torque inputs. A secondary control layer applies the 
\subsection*{Model Dependent \& Independent Controllers}
%****************************************************

%****************************************************
\section{Attitude Control}
\label{sec:control.attitude}
%****************************************************
\subsection{The Attitude Control Problem}
\label{subsec:control.attitude.problem}
%****************************************************
\subsection{Quaternion Based Error States}
\label{subsec:control.attitude.quaternion}
%****************************************************
\subsubsection{PD Controller}
%****************************************************
\subsubsection{Auxilliary Plant Controller}
%****************************************************
\subsubsection{PID Controller}
%****************************************************
\subsection{Non-linear Controllers}
\label{subsec:control.attitude.nonlinear}
%****************************************************
\subsubsection{Ideal Back-stepping Controller}
%****************************************************
\subsubsection{Adaptive Back-stepping Controller}
\subsubsection*{Disturbance Update Law}
%****************************************************
\subsubsection{Lyupanov Derived Ideal Controller}
%Laselles Theorem
%****************************************************

%****************************************************
\section{Position Control}
\label{sec:control.position}
%****************************************************
\subsection{Backstepping Position Controller}
\label{subsec:control.position.bacstepping}
%****************************************************

%****************************************************
\section{Controller Allocation}
\label{sec:control.allocation}
%****************************************************
\subsection{Non-linear Plant Control Allocation}
\label{subsec:control.allocation.allocators}
%****************************************************
\subsection{Pseudo Inverse Allocator}
%****************************************************
\subsection{Weighted Pseudo Inverse Allocator}
%****************************************************
\subsection{Priority Norm Inverse Allocator}
%****************************************************
\subsection{Online Optimized Secondary Goal Allocator}
%****************************************************
