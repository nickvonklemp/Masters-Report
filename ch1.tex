%****************************************************
%	CHAPTER 1 - INTRODUCTION
%****************************************************
\chapter{Introduction}
\label{ch:ch1}
%====================================================
\section{A Brief Background to the Study}
\label{sec:ch1.study}
%====================================================
Currently the most popular topic for control and automation research is the quadrotor UAV, specifically the control thereof. Much work has been done on quadrotors and their attitude control, specifically control around a stable trim point adjacent the inertial frames origin, to which the control algorithm always tends to. The highly coupled non-linear dynamics for a bodies linear and angular motions arise as a result of gyroscopic torques and Coriolis accelerations. Such affects are elegantly linearized around the origin when they can be approximated to 0, decoupling the system and allowing for traditional SISO control techniques to be applied.
\par
As every quadrotor based research paper will tell you, the current interest in them is as a result of the recent spur in availability of MEMS systems and low-cost ARM based microprocessors to perform complicated control calculations and state estimation. This led to development and expansion in the field and introduction of a large range of hobbyist solutions, from professionally made units to DIY kits with room for modification. A rapidly growing enthusiast community was borne out of this progression which was no longer open only to those willing spend lots of money on their pass times.
\par
The avenues of potential applications for both fixed wing and VTOL UAVs is expansive and the quadrotor configuration provides a mechanically simple and low cost platform on which to test advanced aerospace control algorithms. Considering that commercial drone usage is such an emerging sector; especially in Southern Africa following the revision of aviation laws which have legalized the use of UAVs for commercial application, any research into a non-trivial aspect of the field is extremely valuable. 
\par
Large scale quadrotor, hexrotor and even octorotor UAVs are a popular intermediate choice for aerial cinematography.  Whilst still expensive, the cost of a commercial drone like the SteadiDrone Maverik \cite{steadidrone} is far less than the cost of chartering a helicopter to achieve the same panoramic aerial scenes. Another interesting application for UAVs is in the agricultural sector, introducing crop dusting drones instead of the traditional bi-planes which perform the same job. One difficulty which hinders the progress of the commercial drone sector is that of inertia, specifically when scaling up any vehicle, its performance is adversely affected, due to the increased mass inertial effect.
%====================================================
\section{Research Questions \& Hypotheses}
\label{sec:ch1.hypotheses}
%====================================================
The difficulty with a quadrotors' control is that fundamentally it's under-actuated, having only 4 controllable inputs (each propellers rotational speed and hence lift force) available to manipulate all 6 degrees of freedom (linear X-Y-Z position and angular Pitch, $\phi$, Roll, $\theta$ and Yaw, $\psi$ rotations). The resulting solution is to control the perpendicular heave thrust, $\vec{T}$, and angular torques about each axis, $[\tau_\phi\;\tau_\theta\;\tau_\psi]^T$. So the attitude control problem of a quadrotor is a zero set point problem as any other attempt to track attitude is inherently unstable.
%====================================================
\section{Significance of Study}
\label{sec:ch1.significance}
%====================================================

%====================================================
\section{Other Applications of Proposed Investigation}
\label{sec:ch1.applications}
%====================================================

%====================================================
\section{Scope and Limitations}
\label{sec:ch1.scope}
%====================================================

\subsection{Subsection}
\label{subsec:ch1.section1.subsec1}
%----------------------------------------------------

%****************************************************
% END
%****************************************************
