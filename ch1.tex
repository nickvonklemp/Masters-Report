%****************************************************
%	CHAPTER 1 - INTRODUCTION
%****************************************************
\chapter{Introduction}
\label{ch:intro}
%====================================================
\section{Foreword}
\label{sec:intro.foreword}
%====================================================
\subsection{A Brief Background to the Study}
\label{subsec:intro.foreword.background}
%====================================================
A popular topic for current control and automation research is that of quadrotor UAVs. Attitude control of a quadrotor poses a unique 6-DOF control problem, to be solved with an under-actuated 4-DOF system. As a result the pitch, $\phi$, and roll, $\theta$, plants are not directly controllable. The attitude plant is often simplified around a stable operating point. The trimmed operating region is always at the inertial frame's origin; resulting in a zero-set point tracking problem. The highly coupled non-linear dynamics of a rigid body's translational and angular motions arise from gyroscopic torques and Coriolis accelerations (Sec:~\ref{subsec:dynamics.nonlinearities.gyrotorques}). Such effects are mostly negligible around the origin, hence the origin trim point decouples the system's nonlinearities. The control system can therefore reduce each state variable, $\vec{\mathbf{x}}_b=\big[x~y~z~\phi~\theta~\psi\big]^T$, to individual SISO plants. Those simplifications are derived in the Appendix:\ref{app:equations.standard}.
\par
As almost every quadrotor research paper mentions, the recent interest in the platform is due to increased availability of MEMS and low-cost microprocessor systems. These technical advancements accomodate onboard state estimation and control algorithm processes in real time. Developmental progress in quadrotors and, to a lesser extent UAVs in general, has led to rapidly growing enthusiast communities. For example; HobbyKing\cite{hobbyking} is now a name synonymous with providing custom DIY hobbyist quadrotor assembly kits and frames, no longer retailting only prebuilt commercial products like DJI Phantom\cite{phantom} or ParrotAR\cite{parrotar} drones.
\par
The avenue for potential application of both fixed wing and VTOL UAVs is expansive; supporting civil\cite{civilquadcopter}, agricultural\cite{agriculturequadcopter} and security\cite{videosurveillancequadcopter} industries and not just recreational hobbyists. The quadrotor design provides a mechanically simple platform on which to test advanced aerospace control algorithms. Commercial drone usage in industry is already emerging as a prolific sector; especially in Southern Africa. Subsequently following the $8^{th}$ amendment of civil aviation laws \cite{dronelaw}, commercial use of UAVs has been both legalized and regulated. Research into any non-trivial aspect of the field will therefore be to extremely valuable to the field as a whole. Large scale quadrotor, hexrotor and even octorotor UAVs are popular intermediate choices for aerial cinematography and other high payload capacity applications. The cost of a commercial drones such as the SteadiDrone Maverik \cite{steadidrone} are significantly less than a chartered helicopter, used to achieve the same panoramic aerial scenes or on-site inspections. One foreseeable issue which may hinder commercial drone progress in the agricultural and civil sectors is the consequential inertial effects from scaling up the aerial structures. When increasing the size of any vehicle, its performance is adversely affected if actuation rates aren't proportionately increased.
%====================================================
\subsection{Research Questions \& Hypotheses}
\label{subsec:intro.foreword.hypotheses}
%====================================================
The difficulty with quadrotor control is that fundamentally, from their uncertainty and underactuation, they're ill-posed for 6-DOF setpoint tracking. A quadrotor inherently has only four controllable inputs; each propeller's rotational speed, $\Omega_{1,2,3,4}$, which are then abstracted to a net virtual control input net torque, $\vec{\tau}_\mu=[\tau_{\phi}~\tau_{\theta}~\tau_{\psi}]^T$, and a perpendicular heave thrust $\vec{T}_\mu=\sum_{i=1}^{4}~T(\Omega_i)$ in the $\hat{z}_b$ direction. Those four inputs are then used to effect both the translational X-Y-Z positions, $\vec{\mathcal{E}}=[x~y~z]^T$, and angular pitch, roll and yaw attitude rotations, $\vec{\eta}=[\phi~\theta~\psi]^T$. Pitch and roll torques, $\tau_{\phi}$ and $\tau_{\theta}$ respectively, are produced from differential thrusts of each opposing propellers. Yawing torque, $\tau_{\psi}$, is induced only by the sum of aerodynamic torques about each propeller's rotational axis. Aerodynamic responses are highly non-linear (detailed later in Sec:\ref{subsec:dynamics.aero.bem}) and difficult to approximate as sources of control torque. As a result the body's yaw channel control is depreciated. Stemming from the system's under-actuation, the attitude control problem becomes a zero set point problem because any other attempt to track attitude is ill-posed.
\par
The aim of this dissertation is to implement attitude and position dynamic set point tracking on a quadrotor UAV by solving the problem of its inherent under-actuation. Inspired by Boeing/Bell Helicopter's V22 Osprey and the tilting articulation of its propellers, the prototype design proposed here (detail in Sec:\ref{sec:proto.design}) introduces two additional actuators for each of the quadrotor's four lift propellers. Specifically, adding rotations about the $\hat{x}$ and $\hat{y}$ axes for each motor/propeller pair, the resultant are found separately articulated 3-dimensional thrust vectors instead of a bound perpendicular net heave force. The control problem is then posed as the design and allocation of net forces, $\vec{F}_{net} = [F_x\;F_y\;F_z]^T$, and torques, $\vec{\tau}_{net} = [\tau_{\phi}~\tau_{\theta}~\tau_{\psi}]^T$, for a general 6-DOF body such that for any given trajectory, $\vec{\mathbf{x}}_d(t)=[x~y~z~\psi~\theta~\phi]^T$, the error state $\vec{\mathbf{x}}_e(t) = \vec{\mathbf{x}}_d(t) - \vec{\mathbf{x}}_b(t)$ asymptotically tends to $\vec{0}$.
\begin{equation} \label{eq:trajectoryerror}
\lim_{t \rightarrow \infty} \vec{\mathbf{x}}_e(t) = \vec{0}~~~~\forall \vec{\mathbf{x}} \in \mathbb{R}^n
\end{equation}
Where $n$ is number of the degrees of freedom the system has, typically a 6-DOF plant for rigid bodies. The over-actuation brings about the need for a control allocation scheme which distributes the 6 commanded system inputs (net torques and forces) among the actuator set (12 actuators) in order to optimize some objective function secondary to that of Eq:\ref{eq:trajectoryerror}. The potential improvement(s) for exploiting those over-actuated elements is the most novel outcome which the project could yield. A cost function aimed at optimizing some aspect unique to aerospace bodies is going to be a completely unique contribution.
\par
Part of the control research question is the multivariable dynamic modeling of the system; making as few assumptions as possible to the non-linear dynamics involved in the quadrotor's motion and its operational conditions. Common linearizations often applied to the quadrotor's control plant will not hold true for the more aggressive maneuvers; they are dependent on small angle approximations and neglect 2$^{nd}$ or higher order effects. To produce a stabilizing control law solution there first needs to be a dynamic model that incorporates both multibody and actuator dynamics, against which the controller efficacy can be tested. The final key outcomes for the project are; the prototype design, its  dynamic model and simulation analysis, the resultant control law produced and finally conclusions drawn on all of the above.
\par
For a rigidly connected multibody system with rotational joints between sub-bodies, the induced relative motion between those sub-bodies will produce a lot of unwanted dynamics like inertial and gyroscopic responses, amongst others\ldots A rotating propeller will respond to pitching or rolling much like a Control Moment Gyroscope \cite{cmg} or a flywheel, producing a precipitating torque cross product. A less trivial aspect which is occasionally considered are the aerodynamic effects produced from the propeller's aerofoil profile. Such induced responses manifest normal to the propeller's rotational axis. Those aspects are not typically compensated for due to a quadrotor's fundamental co-planar propeller counter-rotating pairs which negate such effects. A plant dependent control law is needed for dynamic compensation which reduces uncertainty associated with the subsequent stability proof. 
%====================================================
\subsection{Significance of Study}
\label{subsec:intro.foreword.significance}
%====================================================
Owing to the huge popularity of quadrotor platforms as research tools (i.e \cite{x4flyercontrol,intelligentbackstep,fullquadcoptercontrol}, etc\ldots), any work that builds on UAV \& quadrotor fundamentals will prove to be valuable. With that being said, there is already a plethora of research on the subject of linear and non-linear control techniques for quadrotor platforms (surveyed in Table:\ref{tab:controllers}). Attitude control loops are the most common topic for research, requiring an ingenious under-actuated solution and mostly linearized around the origin (Appendix:\ref{app:equations.standard}). Far less common is the application of optimal flight path and trajectory planning to a quadrotor's (\emph{augmented}) autopilot system. The difficulty and ill-posed aspect of a quadrotor's attitude control does not hold true for its position plant, so standard techniques can be applied for waypoint and trajectory planning once the attitude control problem has been addressed.
\par
The most significant aspect of this project is the attitude control, discussed later in Sec:\ref{sec:control.attitude}. The over-actuation of the proposed design and, more critically, the manner in which the controller's commanded (virtual) output is distributed among those control effectors would, at the time of writing, appear to be the first of its kind. Otherwise known as control allocation, the requirements of the distribution algorithm(s) are outlined in Sec:\ref{sec:control.allocation}. Dynamic setpoint attitude control for aerospace bodies is not a subject heavily researched outside the field of satellite attitude control. Even papers that propose similarly complicated mechanical over-actuation (expanded upon in next in the literature review, Sec:\ref{sec:intro.litreview}) hardly broach the topic of tracking attitude set points away from the origin.
\par
The control plant presented here (developed in Chapter:\ref{ch:control}) does indeed close both the position and attitude control loops. There is, however, no consideration of trajectory generation nor flight path planning, such topics are well discussed elsewhere. Once closed loop position and attitude control have been achieved, the control algorithms can be adjusted to incorporate higher order state derivative (acceleration, jerk and jounce) tracking needed for nodal waypoint planning. The heuristics involved with flight path planning are well documented\cite{trajectorygeneration,modelingquadcopter,trajectorytracking} and their application is an easily implemented task.
\par
Where possible, the system identification and control (\emph{design} and \emph{allocation}) for this project is kept both modular and generally applicable. The intention is that its pertinence falls not only within the UAV field but also to any aerospace attitude control, rigid or otherwise. Hopefully the investigation can be expanded upon with more focused research on one of the subsystems without compromising the stability of the whole plant. Provisionally, an obvious outcome which the project could yield is improved yaw control of a quadcopter's attitude. However, if the express purpose was just to improve yaw control, it could be done with a dramatically less complicated design\ldots
\par
Moreover, this dissertation could provide greater insight into higher bandwidth actuation and hence faster control responses for larger aerospace bodies. Any standard quadrotor uses differential thrust to develop a torque about its body. Such actuation suffers a second order inertial response when the propellers accelerate or decelerate; $\vec{\tau}_{p}=\mathbb{I}_p\dot{\Omega}_i$ for $i\in[1:4]$. Prioritizing pitching the propeller's principle axis of rotation in lieu of changing the rotational speed could potentially improve the actuator plant rate response. This is entirely dependent on how the allocator block is prioritized (presented in Sec:\ref{sec:control.allocation}). The exact effects of different actuator prioritization and distribution in the context of aerospace control are, at the time of writing, unique to this research.
%====================================================
\subsection{Scope and Limitations}
\label{subsec:intro.foreword.scopeandlim}
%====================================================
\subsubsection{Scope}
\label{subsubsec:intro.foreword.scope}
%====================================================
Critical to this project is the conceptualized design and prototyping of a novel actuation suite to be used on a quadrotor platform. The control research question is to apply dynamic attitude setpoint control to the quadrotor platform. Stemming from this is an investigation into the kinematics that are potentially influenced by the design \emph{changes} and the structure's relative motion. In order to apply correct control theory to achieve the attitude tracking on the physical prototype, plant dynamics must first be identified for the controller to be optimized correctly. Aspects of the mechanical design are detailed in the next chapter, Ch:\ref{sec:proto.design}. There is no scope beyond the cursory investigation for materials analysis or stress testing of the design. This dissertation's scope focuses mainly on the control derivation and embedded systems design/implementation and not the structural integrity of a proposed frame given the forces it may undergo. Physical measurements are only made for critical kinematics such as inertial measurements for the second order gyroscopic and inertial dynamic responses.
\par
As mentioned in the antecedent Sec:\ref{subsec:intro.foreword.significance}; trajectory and flight path planning are not ubiquitous with this dissertation. Derivations for the differential equations of motion which dictate a 6-DOF body's movement are applicable to any aerospace body, rigid or otherwise. Some particular standards are used, like Z-Y-X Euler Aerospace rotational sequence, all of which are covered in Sec:\ref{sec:proto.conventions}. The control plant is stabilized with non-linear state-space control techniques in the time domain, aided and justified by Lyapunov stability theorem\cite{noteonlyapunov,nonlinearsystems}. Alternative solutions using Model Predictive Control or Quantitative Feedback Theory could provide more refined or effective controllers, however they are not presented and remain open to further investigation. Quadrotor attitude control is commonly stabilized with feedback linearizations, decoupling the plant around a trim point so that SISO techniques can be applied. A derivation of such a linearization is included in Appendix:\ref{app:equations.standard} but beyond that there are no further discussions. Any comparisons between non-zero and zero set-point attitude controller efficacy for quadrotors are difficult as the fundamental objectives are in stark contrast with one another.
\par
Arguably the most important and potentially novel aspect of this project is the control allocation. The system has 12 plant inputs and 6 output variables to be controlled. There is then an entire set of compatible actuator solutions, $u\in\mathbb{U}\in\mathbb{R}^{12}$, which exist for each commanded input. Such a plant is classified as over-actuated. Ergo, there must be some logical process as to how those 12 inputs are mixed to achieve the desired 6 control plant inputs, specifically net force $\vec{F}_{net}$ and net torque $\vec{\tau}_{net}$ acting on the system. Appropriate techniques are first derived in Sec:\ref{sec:control.allocation} then simulated and compared before a final solution is implemented in Section:\ref{sec:simulation.comparison}. It is not a comprehensive survey of every possible allocation scheme but rather an analysis of the sub-set of problems and design of what is regarded as a logical and pertinent approach.
\par
With regards to the prototype design, in Sec\ref{sec:proto.design}, it is assumed that certain aspects are readily available and require no design/development. Particularly the state estimation, updated through a 4-camera positioning system fused with a 6-axis IMU through Kalman Filtering (Sec:\ref{sec:simulation.state}), is assumed to precise and readily disposable at a consistent 50 Hz. Hence state estimation is included but is bereft of intricate detail, this is another topic which remains open to further investigation.
%====================================================
\subsubsection{Limitations}
\label{subsubsec:intro.foreword.limits}
%====================================================
The biggest constraint faced by the design is the net weight of the assembled frame. Lift thrusts which are required to keep an aircraft aloft and oppose the net gravitational force are obviously dependent on the body's net weight. The steady state actuator rates ought to be far less than saturation conditions to ensure sufficient actuator headroom to implement control actuations. Conversely the structure's net weight is mostly dependent on the lift motors, often being the heaviest part of the vehicle (\emph{batteries included}). A trade-off between net weight and actuator effectiveness makes designing the prototype a balancing act of compromise; added actuation is needed to produce the desired thrust vectoring. That added actuation is going to increase the weight which then requires more thrust force to ensure the vehicle remains airborne. Larger motors then need stronger actuators to effect the relative motion and overcome the bodies inertial response. It's a compromise between the weight of the body and the strength/quality of the actuation.
\par
To forego the deliberation detailed above, reducing the possibility of unbounded scope creep, a design limitation is self-imposed on the prototype design. Restricting the propeller diameter, and hence maximum thrust/frame size, will provide a constraint upon which all other design considerations must adhere to. Smaller propellers require far greater rotational speeds to produce similar levels of thrust that their larger diameter counterparts could provide. Electing to use 3 bladed $6\times 4.5$ inch small diameter propellers constrained the maximal overall dimensions of the prototype, but as a consequence required very high RPM motors. Specifically a set of four Cobra-2208/2000KV\cite{cobramotor} brushless DC motors are used for lift actuation (Fig:\ref{fig:cobra}). 
\par
A direct consequence of that decision is, provisionally based on thrust tests, the net thrust actuation disposable to the controlk loop is limited to around $950g\approx 9.5 N$, per motor at 14.1V. That thrust test data is provided from the offical Cobra Motors website, \cite{cobramotor}, included in Appendix:\ref{app:cobra-test} and verified independently through testing in Sec:\ref{subsec:dynamics.aero.bem}. It is critical to ensure the control block doesn't induce over-saturation of those BLDC motors, so the frame weight needs to be under 50\% of the maximum available thrust, or roughly below 2 kg. Saturation conditions are detailed later in Sec: \ref{sec:control.allocation}.
\par
Another aspect of limitations produced by design decisions made, mostly to reduce prototype costs and weight, is to use of 180\textdegree ~rotation servo motors. Here Corona DS-339MG metal gear digital servos (Fig:\ref{fig:corona}) are used. The servos are for each individual motor's $\hat{x}_{M_i}$ and $\hat{y}_{M_i}$ axial pitch and roll actuations respectively. Servos act in place of either BLDC gimbal or stepper motors with closed loop position control. The latter pair would both accommodate for continuous  ($>2\pi$) rotation but would need their own control design. Continuous rotation (velocity controlled) servos could otherwise be used but would similarly require angular rotation feedback. Any rotations beyond 2$\pi$ ~would similarly require slip rings to transmit power throughout rotational movement. Implementing such a design whilst maintaining an acceptable weight would prove too costly nor would it provide insight attained from testing. The servo rotational limitations effect can be evaluated in simulation and if it proves to be significant, continuous rotation can be implemented\ldots
\begin{figure}[htbp]
\begin{subfigure}{0.5\textwidth}
\centering
\includegraphics[width=0.9\textwidth]{figs/cobra-motor}
\caption{Cobra CM2208/2000KV BLDC motor~\cite{cobraimage}}
\label{fig:cobra}
\end{subfigure}
\begin{subfigure}{0.5\textwidth}
\centering
\includegraphics[width=0.9\textwidth]{figs/corona-servo}
\caption{Corona DS-339MG digital servo~\cite{hobbyking}}
\label{fig:corona}
\end{subfigure}
\caption{Mechanical actuators}
\vspace{-20pt}
\end{figure}
\par
Discrete elements for the whole system could potentially limit performance but are mitigated where possible. For example analogue servos have an associated $1$ ms dead time from their $50$ Hz refresh rate. That can be addressed by using faster, albeit more expensive, digital servos which samples at $330 Hz$. The prototype's flight controller needs to provide 12 PWM output compare channels for the 8 servos and 4 BLDC speed controllers. State updates from a ground control station and a fail safe 6CH RC receiver module also needs to be processed by the $\mu$C system. Particular attention is paid to the embedded system design and layout in Sec:\ref{sec:proto.layout}.%====================================================
\section{Literature Review}
\label{sec:intro.litreview}
%====================================================
\subsection{Existing \& Related Work}
\label{subsec:intro.lit.related}
%====================================================
The field of transformable aerospace frames is not new, with many commercial examples seeing successes over their operational life span. The most notable tilting-rotor vehicle is the Boeing/Bell V22 Osprey\cite{osprey} aircraft. First introduced into the field in 2007, the Osprey has the ability to pitch its two lift propellers forward to aid translational flight after vertically taking off or landing. In addition to this there have been many papers published on similar tilting bi-rotor UAVs for research purposes.
\subsubsection*{Birotors}
\begin{figure}[hbtp]
\centering
\includegraphics[width=0.8\textwidth]{figs/dualaxistilt}
\caption{General structure for opposed tilting platform, taken from\cite{2007}}
\label{fig:dualaxistilt}
\end{figure}
Research into birotor vehicles (Fig:\ref{fig:dualaxistilt}) with ancilliary lift propeller actuation is oft termed \emph{Opposed Active Tilting} or \emph{OAT}. Such a rotorcraft's mechanical design applies either a single \emph{oblique} 45\textdegree ~tilting axis relative to the body; \cite{smalltwotilting,obliquepitch,passiveobliquetilting}, or a \emph{lateral} tilting axis, adjacent to the body; \cite{tiltrotorUAV,adaptivebackstep,tiltrotorcontrol,tpheonix}. Leading research is currently focussed on applying doubly actuated tilting axes to birotor UAVs. \emph{Dual} axis \emph{Opposed Active Tilting} or \emph{dOAT} introduces vectored thrust with independent propeller pitch and roll motions to further expand the actuation suite, \cite{gres2007,opposedlateraldualaxis}. A birotor is sometimes considered preferable to higher degree of freedom multirotor platforms due to their reduced controller effort. However the controller plant derivation, typically requiring feedback linearization and virtual plant abstraction, often detracts from the quality and effectiveness of its stability solution as a result of the birotor's underactuation. 
\par
Birotor attitude control mostly introduced plant independent PD \cite{obliquepitch} and PID \cite{tiltrotorUAV} stabilizing controller schemes. Sometimes more computationally intensive and plant dependent \emph{ideal} or \emph{adaptive} backstepping controllers are implemented, presented in \cite{smalltwotilting,tpheonix} and \cite{adaptivebackstep} respectively. The gyroscopic response of a birotor vehicle's attitude system is more pronounced than that of a quadrotor, derived in Sec:\ref{sec:dynamics.nonlinearities}, and so feedback linearisation is almost always used. In an interesting progression from the norm, \cite{autopilotPSO} proposed a PID co-efficient selection algorithm for a bi-rotor control block. Using a \emph{particle swarm optimization} techinque, similar to \cite{adaptivepso}, the coefficients were globally optimized around a given performance metric. However their performance criterion is a standard integral time-weighted additive error (ITAE) term and nothing more appropriate involving effects unique to flight systems was used. \emph{PSO} algorithms iteratively search for a globally optimized solution and offer independent, gradient free based optimization. In subsequent chapters, controller coefficients are optimized using PSO algorithms, shown later in Sec:\ref{sec:simulation.tuning}.
\par
\subsubsection*{Quadrotors}
Expanding on bi-rotor vehicles, the quadrotor UAV is a popular and well researched multirotor platform due to its mechanical simplicity. The recent popularity in quadrotors as research platforms began in 2002, with a control algrithm implemented on what is now known as the X4-Flyer quadrotor \cite{x4flyer,x4flyercontrol}. Alternative iterations the followed, like the Microraptor\cite{microraptor} and STARMAC\cite{starmac} quadcopters which have subsequently been built and tested. A multitude of literature exists around quadrotor kinematics \& control \cite{dynamicmodelling2013, dynamicmodelling2009, modelingquadcopter, quaddynamics, fullquadcoptercontrol}, however dedicated rigid body 6-DOF dynamic papers \cite{rigidbodylecture,eulerrigidbody} offer better explanations of the kinematics. Often the plant's dynamics are simplified around an origin trim point and assumed to reduce into 6 SISO plants for each degree of freedom (Appendix:\ref{app:equations.standard}). Lately research projects have begun to incorporate non-linear aerodynamic effects like drag and propeller blade-element momentum (BEM) theory into the plant model\cite{lowreynolds,bem,starmac}. The higher fidelity models for thrust and propeller responses offer more precision by making fewer linearisations and assumptions;\cite{nonlineardynamics,starmac}.
\par
\begin{figure}[hbtp]
\centering
\begin{subfigure}{.5\textwidth}
\centering
\includegraphics[width=\textwidth]{figs/dji-inspire1}
\caption{Inspire1 articulated upwards}
\label{fig:inspireup}
\end{subfigure}%
\begin{subfigure}{.5\textwidth}
\centering
\includegraphics[width=\textwidth]{figs/dji-inspire2}
\caption{Inspire1 articulated downwards}
\label{fig:inspiredown}
\end{subfigure}
\caption{DJI Inspire1, the notations are with regards to the DJI patent}
\label{fig:inspire1}
\end{figure}
At the time of writing, the only commercial UAV multirotor capable of structural transformation is the DJI Inspire1 quadrotor\cite{inspire}, manufactured by Shenzen DJI Technologies. DJI are better known for their hugely successful DJI Phantom commercial drone\cite{phantom}. The Inspire1 can articulate its supporting arms up and down as shown in Fig:\ref{fig:inspire1}, both images were sourced from the drone's patent; held by SZ DJI Tech Co\cite{djinspire}. The purpose of such transformations is to both alter the center of gravity and to further expose a belly mounted camera gimbal for panoramic viewing angles. This changes the bodies inertial tensor about its center of gravity, affecting the second order inertial response opposed to changes in angular velocity; $\tau=J\dot{\vec{\omega}}_b$. That variable inertial matrix is a detrimental effect which makes researchers apprehensive of transformable aerospace frames. The range of transformations which the Inspire1 frame can undergo is limited to just articulating its arms up and down.
\par
In a similar fashion to the progression seen in birotor state-of-the-art, quadrotor research is engaging the topics of single and dual axis propeller module tilting articulations. The extra actuation scheme(s) was first conceptualized and implemented on a prototype related to an ongoing project covered in two reports; \cite{tiltpropellercontrol,tiltpropellerflight}. Those authors modified and tested a QuadroXL four rotor helicopter, propduced by MikroKopter \cite{mikrokopter}, to actuate a single axis of tilting aligned with the frame's arms (Fig:\ref{fig:tiltpropellercontrol1}). Their proposed control solution, detailed next in Sec:\ref{subsec:intro.lit.control}, assumes no nominal linearised conditions around hover flight, unlike a similar single axis tilting quadrotor prototype designed by Nemati in \cite{singleaxistilting}. The latter is \emph{simulated} but remains as yet untested.
\par
One approach to improving quadrotor flight response is to alter the manner in which the thrust is mechanically actuated, potentially improving actuator bandwidth (demonstrated in \cite{tiltgasco,tiltrihani}). Drawing from helicopter design, \cite{napsholm} purported a novel quadrotor UAV prototype that used swashplates for varying the propeller pitch and generating torque moments. The aim was a design which was not dependent on rotational speed controlling power electronics (\emph{ESCs}) to actuate variable thrust forces. Petrol motors were intended for use in place of BLDC motors. Furthermore, the design proposed a single axis of tilt actuation to each of the four motor modules. Whilst mechanically complex, Napsholm made use of existing off-the-shelf hobbyist helicopter components to design a rotor actuation bracket (Fig:\ref{fig:tiltrotor-napsholm}). The cyclic-pitch swashplates\cite{autonomousrobotspitch} used could apply pitching and rolling torques, $\tau_{\phi}$ and $\tau_{\theta}$, about each propeller's hub, its \emph{principle axis of rotation}. The torques were induced by altering the blades angle of attack throughout the propeller's rotational cycle. The actuation rate of such a configuration is far greater than that of a differential torque produced rolling/pitching motion.
\begin{figure}[htbp]
\centering
\begin{subfigure}{.5\textwidth}
\includegraphics[width=\textwidth]{figs/tiltpropellercontrol1}
\caption{Single aligned tilting axis, proposed in~\cite{tiltpropellercontrol}}
\label{fig:tiltpropellercontrol1}
\end{subfigure}%
\begin{subfigure}{.5\textwidth}
\includegraphics[width=\textwidth]{figs/napsholm-mech}
\caption{Cyclic-pitch \& swashplate mechanism}
\label{fig:tiltrotor-napsholm}
\end{subfigure}
\caption{Tilt-rotor mechanisms}
\label{fig:tiltprop}
\end{figure}
\par
Irrespective of the strong initial design in the early stages of his project, it would appear that Napsholm's research suffered due to time constraints. The introductory derivation on aerodynamic effects and deliberation over the design provide clear insight into the projects goals. However the control solution and system architecture, electronic and software, are severely lacking. A brief introductory proposal of an MPC attitude control system detracted from the comprehensive dynamics discussed. The project ended before testing, simulation or results could be obtained. Unfortunately, despite the novel over-actuated design, there was no discussion given on how that actuator allocation, being the most unique aspect of the project, would be achieved.
\par
Finally, the most crucial research to mention is a project completed by Pau Segui Gasco in \cite{tiltgasco}, which was a dual presented MSc project with Yazan Al-Rihani whose respective research was presented in \cite{tiltrihani}. At the time of writing, this would appear to be the only project published pertaining to \emph{over-actuation} in aerospace bodies implemented and tested on a quadrotor platform. The research was split between the two authors who completed the electronic/control design and the mechanical design for their respective MSc dissertations. Shown in Fig:\ref{fig:tiltrotor-gasco}, the dual-axis articulation is achieved using an RC helicopter tail bracket and servo push-rod mechanism; reducing the mass of the articulated components but limiting the range of its possible actuation. Considering the propellers as energy storing flywheels, the induced gyroscopic response was then treated as an additional controllable actuator plant. Their commanded virtual control is distributed by weighted inversion amongst the actuator set, Sec:\ref{subsec:intro.lit.control}. The whole project justifies the extra actuation as fault tolerance redundancy (\emph{FTC}) but doesn't necessarily prove how such a redundancy could be beneficial.
\begin{figure}[htbp]
\centering
\includegraphics[width=0.7\textwidth]{figs/gasco-mech}
\caption{Dual-axis tilt-rotor mechanism used in \cite{tiltgasco}}
\label{fig:tiltrotor-gasco}
\end{figure}
%====================================================
\subsection{Notable Quadrotor Control Implementations}
\label{subsec:intro.lit.control}
%====================================================
\subsubsection*{Quadcopter Attitude Control}
%====================================================
Attitude control of a 6-DOF aerospace body, quadrotor or otherwise, is best described by \emph{The Attitude Control Problem},  \cite{attitudecontrolproblem}. For a rigid body that has an instantaneous attitude state $\vec{\eta}_b$ and a desired state $\vec{\eta}_d$, the problem is to then find a stabilizing torque control $\vec{\tau}_\mu$. The control law is dependent on some feedback error state $\vec{\eta}_e$. Quaternion attitude states later replace Euler angles for attitude representation, $\vec{\eta}_b\Rightarrow Q_b$. The general control law is defined as:
\begin{equation} \label{eq:2}
\vec{\tau}_\mu = h(\vec{\eta}_d,\dot{\vec{\eta}}_d,\vec{\eta}_b,\dot{\vec{\eta}}_b,t)~~~~\in\mathcal{F}^b
\end{equation}
Where the control law designs a net torque such that both the angular position and rates are stabilized with the bounded limits $\lim\vec{\eta}_b \rightarrow \vec{\eta}_d$ and $\lim\dot{\vec{\eta}}_b \rightarrow \dot{\vec{\eta}}_d$ respectively as $t \rightarrow \infty$. A distinction must be made between euler angular rate vector, $\dot{\vec{\eta}}_b=[\dot{\phi}~\dot{\theta}~\dot{\psi}]^T$ and the angular velocity vector $\vec{\omega}_b=[p~q~r]^T$. Depending on how the attitude is posed; with rotation matrices \cite{rigidbodylecture,eulerrigidbody,rotationsequences}, quaternions \cite{quaterniondynamics, rotationsequences, spacecraftattitutdequaternions,fullquaternion} or otherwise (Direct Cosine Matrix etc \ldots) the error sate $\vec{\eta}_e= \vec{\eta}_d - \vec{\eta}_b$ could then differ to a (Hamilton) multiplicative relationship. \cite{attitudecontrolproblem} describes these conventionally different error states.
\\
\emph{\color{Gray}Note that here $\vec{\eta}$ is not necessarily an Euler angle set but any attitude representative state variable.}
\par
Simulation and modelling papers often rely on Euler angle based rotation matrices for attitude representation, \cite{adaptivedisturbancecontrol, optimizedpidquadcopter, singleaxistilting, backsteppingquadcoptercontrol, fullquadcoptercontrol} without addressing the inherent singularity associated with such an attitude representation (known as gimbal lock, \cite{euleranglesingularity}, Sec:\ref{subsec:dynamics.rigidbody.singularity}). The alternative quaternion attitude representation, first implemented in 2006 on a quadrotor UAV in \cite{attitudestabilization}, is often used in lieu of rotation matrices. Quaternions do have their own caveat of \emph{unwinding} as a result of the dual-coverage in $\mathbb{R}^3$ space, discussed in \cite{unwinding} and derived mathematically later in Sec:\ref{subsec:dynamics.rigidbody.unwinding}. Quaternions are $\in\mathbb{R}^4$ variables for attitude representations in $\mathbb{R}^3$ and so a mapping $\mathbb{R}^4\rightarrow\mathbb{R}^3$ produces an infinite coverage set for each unique attitude state.
\par
Quadrotor plant dynamics, as mentioned previously, are often simplified; especially when represented with a 3-variable Euler angle set, $\vec{\eta} = [\phi ~\theta ~\psi]^T$. The cross-product gyroscopic and Coriolis terms are both neglected when the angular velocity is small, $\vec{\omega}_b \approx 0$, and the inertial matrix $J_b$ is approximately diagonal, $rank(J_b)= x$ for $\in\mathbb{R}^x$. The consequence of such simplifications is the deterioration of both the gyroscopic term, $\vec{\tau}_{gyro}=-\vec{\omega}_b \times J_b\vec{\omega}_b \approx \vec{0}$ and the  Coriolis force term, $\vec{F}_{cor}=-\vec{\omega}_b \times m\vec{v}_b \approx \vec{0}$ in the body's dynamics~~(Ch:\ref{ch:dynamics} for context). Once the coupled cross-product terms are no longer of consequence, the 6 DOF trajectory, $\vec{\mathbf{x}}=[x ~y ~z ~\phi ~\theta ~\psi]^T$, can be treated as a series of independent SISO plants each controlled with an appropriate technique. Quaternion represented attitude plants cannot easily be decomposed into individual single-input-single-output systems (quaternion dynamics in Sec:\ref{subsec:dynamics.rigidbody.quaternion}). So a quaternion combined four variable attitude state-space vector is then used, $Q_b = [q_0 ~\vec{q}\>]^T$, for the major loop trajectory plant of $\vec{\mathbf{x}}(t)$.
\par
\begin{figure}[hbtp]
\centering
\includegraphics[width=0.75\textwidth]{figs/arducopter-pi}
\caption{ArduCopter PI Euler angle attitude control loop, image cited from~\cite{buildyourownquad}}
\label{fig:arducopter-pi}
\end{figure}
\vspace{-10pt}
Opensource and hobbyist flight controller's software (Arducopter\cite{arducoptersite}, Openpilot\cite{openpilotsite} whose firmware stack is now maintained by LibrePilot, CleanFlight\cite{cleanflight}, BetaFlight\cite{betaflight}, etc \ldots) for custom fabricated UAV platforms all apply their own flavour of structured attitude controllers and state estimation algorithms, based on onboard hardware sensor fusion. The article \emph{Build Your Own Quadrotor}\cite{buildyourownquad} summarizes the control structures implemented on a range of popular flight controllers. The most popular of which, ArduCopter, implements a feed-forward PI compensation controller (Fig:\ref{fig:arducopter-pi}).  PI, PD and PID controllers are all popular and effective plant independent control solutions for general attitude plants. Table:\ref{tab:controllers} lists the common attitude control blocks (not exclusively quadrotors UAVs but MAVs too) and which projects they've been implemented in, after which a critique on the more unique adaptations is given.
\begin{table}[h]
\centering
\begin{tabular}{ |c|l|l|c| }
\hline
Controller Type & Independent & Dependent & Total\\ \hline
PI & \cite{attitudecontrolproblem} & \cite{attitudecontrolproblem} & 2\\ \hline
PD & \cite{modelingquadcopter, tiltrihani} & \cite{fullquaternion,singleaxistilting} & 4\\ \hline
PID & \cite{optimizedpidquadcopter, attitudecontrolproblem, quaddynamics, tiltpropellercontrol, pidlqr} & \cite{attitudecontrolproblem, starmac, adaptivedisturbancecontrol} & 8\\ \hline
Lead & \cite{x4flyer, dynamicmodelling2009} & N/A & 2\\ \hline
IBC & \cite{tpheonix, backsteppingquadcoptercontrol}\tablefootnote{\cite{tpheonix} applies an IBC algorithm derived through Hurwitz polynomials, not lyapunov theorem.} & \cite{backsteppingquadcoptercontrol} & 3\\ \hline
ABC & \multicolumn{2}{l|}{\cite{adaptivebackstep, nonlinearadaptive, 6dofbackstep, intelligentbackstep}} & 4\\ \hline
LQR & \cite{pidlqr} & N/A & 1\\ \hline
\end{tabular}
\caption{A breakdown of common attitude controllers}
\label{tab:controllers}
\end{table}
\par
\vspace{-15pt}
In a collection of papers, written by Bouabdallah, et al.[2003,2004,2007]\ldots , arguably the most prolific early quadrotor authors, a range of different control implementations are derived and reviewed. Their last paper, \emph{Full Control of a Quadrotor}[2007]\cite{fullquadcoptercontrol}, derived and pratically tested an Integral Backstepping attitude controller on an OS4 quadrotor platform. It builds on their research from an earlier paper, [2003]\cite{pidlqr}, wherein an analysis of PID vs LQR attitude controllers in the context of quadrotors is posed. LQR controllers aim to optimize the controller effort (with $u\in\mathbb{U}$, controller effort is then $||u||$ or the $L_2$ norm of the plant input). Although, in theory, solving the assocaited Ricatti$^{\dagger}$ cost function may produce an optimial, stable and efficient control law it needs exact plant matching. In reality, exact plant matching is difficult to achieve for a quadcopter or any aerospace body for that matter. The resultant controller in \cite{pidlqr} achieved asymptotic stability but had poor steady state performance due to low accuracy of the identified actuator dynamics and poor confidence inertial measurements.
\par
Adaptive Backstepping Control\cite{backstepping}(any of the examples in Table:\ref{tab:controllers}) builds on nominal IBC fundamentals by introducing an aditional disturbance state term in the LCF used for the backstepping iteration. The drawback with this form of Backstepping approach is that, from the Lyapunov control theorem, a derivative for the estimated disturbance (or an \emph{update law}) is needed. Disturbance approximation has been investigated thoroughly but, for a signal without \emph{apriori} information, some heuristic needs to be adopted with the approximation, which usually involves some compromise.
\newpage
In one example, \cite{nonlinearadaptive}, the authors implemented a statistical $proj(.)$ operator based technique. Which, when used in adaptive control, the projection operator \cite{outputfeedback}, $proj(.)$, ensures a derivative based estimator is bounded for adaptive regression approxmation \cite{nonlinearregression}.
\par
Although the control implementation isn't backstepping based, in \cite{adaptiveslidingmode}, a sliding mode controller was used to compensate for the disturbances in an Unmanned Submersible Vehicle atttiude plant. The underwater current disturbances were approximated using a fuzzy logic system, specifically a \emph{zero-order TSK} fuzzy controller. The TSK system has been proven to act in the same way as an Artificial Neural Network approximator\cite{zeroTSK}; where the fuzzy TSK system is more comprehensible than the latter. Statistical analysis and investigation of approximators without \emph{apriori} knowledge of a system are well beyond the scope of this research but are worth mentioning.
%====================================================
\subsubsection*{Single/Dual Axis Control \& Allocation}
\label{subsubsec:intro.lit.control.allocation}
%====================================================
The extra actuation introduced with single and dual axis articulation provides room for more control goals to be achieved as the order of actuation increases. Of the few papers published on tilting-axis quadrotors, PD controllers (Nemati et al.[2014]\cite{singleaxistilting} and again in Gasco [2012]\cite{tiltgasco} \& Rihani [2012]\cite{tiltrihani}) and PID controllers (Ryll, et al. [2012,2013]\cite{tiltpropellercontrol,tiltpropellerflight}) are the norm for attitude control blocks. For either of these systems there needs to be an allocation rule to distribute a commanded input amongst the actuator set. In \cite{allocation}, Johansen, et al. [2012] describes\footnote{State variable representations of \cite{allocation} were changed to match this dissertation's conventions.} the control allocation problem for a dynamic plant:
\begin{subequations} 
\begin{equation} \label{eq:3.1}
\dot{\mathbf{x}}=f(\mathbf{x},t)+g(\mathbf{x},t)\vec{\nu}
\end{equation}
\vspace{-15pt}
\begin{equation} \label{eq:3.2}
y=c(\mathbf{x},t)
\end{equation}
\end{subequations}
\emph{\color{Gray} Note in the state space Equation:\ref{eq:3.1}, it's assumed the plant input, $\vec{\nu}$, has a linear multiplicative relationship with the input response, $g(\mathbf{x},t,\vec{\nu})\Rightarrow g(\mathbf{x},t)\vec{\nu}$.}
\\
With a state $\mathbf{x}\in \mathbb{R}^n$ and $f(\mathbf{x},t)$ \& $g(\mathbf{x},t)$ being the plant's dynamics and input response respectively. In set point tracking, the output is then \emph{tracking} the state $y=c(\mathbf{x},t)=\mathbf{x}$, and hence $y \in \mathbb{R}^n$. In an ideal well posed system the number of actuator inputs equals the number of controllable variable outputs; that being $dim(\mathbf{x})=dim(\nu)\in \mathbb{R}^n$. In the case where the control input $\nu \in \mathbb{R}^m$, if $m>n$ the problem is then over-actuated and a level of abstraction is needed; an asymptotically stabilizing virtual control input $\nu_d$ is designed by a control law $\nu_d=h(\mathbf{x}_e,t)$ to affect dynamics. The goal is to then find a function that maps $\mathbb{R}^m \rightarrow \mathbb{R}^n$ for an actuator matrix $u \in \mathbb{U}^m$. An over-actuated plant can be described in state-space as:
\begin{subequations}
\vspace{-10pt}
\begin{equation} \label{eq:3.3}
\dot{x}=f(\mathbf{x},t)+g(\mathbf{x},t)\nu_d~~~~\nu_d \in \mathbb{R}^n
\end{equation}
\vspace{-15pt}
\begin{equation} \label{eq:3.4}
\nu_c=B(\mathbf{x},u,t)\Rightarrow B(\mathbf{x},t)u~~~~u\in\mathbb{U}^m,~\nu_c\in\mathbb{R}^n
\end{equation}
\vspace{-15pt}
\begin{equation}
y=c(\mathbf{x},t)=\mathbf{x}
\end{equation}
\end{subequations}
$B(\mathbf{x},u,t)$ is the effectiveness function which quantifies how the actuator inputs $u$ relate to the virtual commanded input $\nu_c$. $B(\mathbf{x},t,u)$ can be abstracted to a multiplicative relationship $B(\mathbf{x},t)u$ if the plant's dynamics permit it, such that; $B(\mathbf{x},t)\in\mathbb{R}^{n\times m}$. For generic set point tracking the control law will design a desired virtual control input $\nu_d$, the allocation rule then has to solve $u$ for $\nu_c$ such that for some slack variable $s=\nu_c-\nu_d$ is minimized:
\begin{equation}\label{eq:quadraticallocator}
\underset{u \in \mathbb{R}^m ,s \in \mathbb{R}^n}{min}\norm{Q_s} ~\text{subject to} ~B(\mathbf{x},u,t) - h(\mathbf{x}_e,t)=\nu_c-\nu_d=s~~~~u \in \mathbb{U}
\end{equation}
Which ensures the commanded input $\nu_c$ tracks the desired control input $\nu_d$; $\nu_c\rightarrow\nu_d$ as per some cost function of the slack variable $Q_s$. Mostly the L2 norm, $\norm{Q_s}$, is used. In an over-actuated system it then follows that there is a whole set of possible inputs for each $\nu_c$. A unique actuator solution (rather than a family of solutions) to Eq:\ref{eq:quadraticallocator} needs a secondary objective function, $J(\mathbf{x},u,t)$. Eq:\ref{eq:quadraticallocator} then becomes;
\begin{equation} \label{eq:quadraticallocatorcost}
\underset{u \in \mathbb{R}^m ,s \in \mathbb{R}^n}{min}(\norm{Q_s}+J(\mathbf{x},u,t)) ~\text{subject to} ~\nu_c - \nu_d=s~~~~u \in \mathbb{U}
\end{equation}
\par
Those same authors, Johansen and Tj{\o}nn\r{a}s [2004,2005,2008], proposed multiple control allocation solutions to a variety of systems. Following \cite{allocation}; in a subsequent paper [2005]\cite{efficientallocation}, Johansen and Tj{\o}nn\r{a}s introduce a secondary cost function, driving the solution away from the typical quadratic programming direct or weighted inversion solution. Aiming for optimal efficiency and not just actuator saturation. In a followup paper, [2008]\cite{adaptiveallocation}, they propose an online adaptive algorithm approach. Using a Lyapunov energy function the minimization adaptive law always settles to a feasible solution.
\par
Over-actuation is not something often applied to quadrotors and rather than providing a comprehensive literature review of associated papers here (which are all mostly theoretical derivation), the contextual application and solutions are expanded upon later in Sec:\ref{sec:control.allocation}. The only overactuated quadrotor\footnote{Birotor dual-axis tilting makes the system critically actuated and so requires no allocation.} literature which covers allocation of the extra actuators is \cite{tiltgasco,tiltrihani}, where the authors apply a weighted pseudo inverse (sic Moore Penrose Inverse \cite{moorepenrose}) allocation rule. A prerequisite for pseudo inversion is a multiplicative \emph{linear} control effectiveness relationship for Eq:\ref{eq:3.4}. 
\par
Gasco, et al. [2012]\cite{tiltgasco,tiltrihani} applied weighted inversion, relying on some very specific assumptions to achieve that linearity relationship in Eq:\ref{eq:3.4}. For the net torque response the authors assumed the extra actuators pitch and roll angular rates, $\dot{\phi}$~and~$\dot{\theta}$ respectively, were proportionally related as follows:
\begin{equation}
\dot{\phi}\approx\frac{\phi}{t_{rise}}
\end{equation}
In which $t_{rise}$ is the actuators rise time to a set-point. As a result the gyroscopic first order torque $\tau_{gyro}=-\vec{\omega}_b\times\mathbb{I}_b\vec{\omega}_b$ and second order inertial torque $\tau=\mathbb{I}_b\dot{\vec{\omega}}_b$ are functions of position $\phi$ or $\theta$, not their derivatives. The extent of that consequence is contrasted with the allocation solution in Sec:\ref{sec:control.allocation}.
%====================================================
\vspace{-10pt}
\subsubsection*{Satellite Attitude Control}
%====================================================
Unconstrained attitude set-point tracking for 6-DOF bodies, quaternion represented or otherwise, is a topic well covered in the field of satellite attitude control; \cite{axissymmetricspacecraft, satellitebackstepping,lpvbackstepping}. The \emph{status quo} for recent research is on non-linear adaptive attitude backstepping control systems, wherein the adaptive update rule is the novel contribution. Often plant uncertainty affects the inertial tensor of a satellite. In \cite{lpvbackstepping}, the authors Wang Jia, et al. [2010], proposed applying adaptive backstepping to compensate for steady state errors of (asymmetric) inertial estimations. Alternatively, instead of deliberating on costly non-orbital prelaunch inertial measurements Bodrany, et al.[2000]\cite{inertiaestimation} developed an algorithm for estimating the inertial tensor based on controlled single axis perturbations. Such an approach does assume any initial estimates are sufficiently close to true body values such that they will settle and stability can be ensured, irrespective of how unacceptable the transient performance may be.
\par
Satellite actuator suites mostly include additional redundant effectors, to ensure fault tollerance, and thus require control allocation. Often the extra allocators are CMG actuators, flywheels driven by DC motors, to produce rotational torques. Fuel burning can only actuate for a certain period of time and so thrusters are scheduled to have a lower priority. Seen in the paper \cite{satellitebackstepping}; the authors, Kristiansen et al. [2005], address the over-actuation with direct and well-matched inversion before applying quaternion based backstepping for attitude control. A direct inversion solves Eq:\ref{eq:quadraticallocatorcost} such that:
\begin{subequations}\label{eq:pseudoinv}
\begin{equation}\label{eq:pseudoinva}
u=B^{\dagger}({\vec{\tau}_a}^{\hspace{3pt}b}-D{\vec{\omega}_{ib}}^{\hspace{3pt}b})
\end{equation}
\vspace{-15pt}
\begin{equation}\label{eq:pseudoinvb}
B^\dagger=B^T(BB^T)^{-1}
\end{equation}
\end{subequations}
Where $B$ is the effectiveness matrix and $B^{\dagger}$ is such that $BB^{\dagger}=\mathbb{I}$. Specifically $B^{\dagger}$ is the general \emph{pseudo} inverse of $B$ (more on inversions in Sec:\ref{sec:control.allocation}). It's assumed there's a linear multiplicative relationship between the input, $u\in\mathbb{U}$, and the input effectiveness matrix in Eq:\ref{eq:3.4}. The controller designed actuator torque ${\vec{\tau}_a}^{\hspace{3pt}b}$ then dictates the input $u$ as in Eq:\ref{eq:pseudoinva}. Much like the over-actuation previously discussed W.R.T quadcopters; the pseudo inversion method of actuator distribution applies quadratic optimization to the allocation slack cost function, Eq:\ref{eq:quadraticallocator}. 