%****************************************************
%	CHAPTER 1 - INTRODUCTION
%****************************************************
\chapter{Introduction}
\label{ch:intro}
%====================================================
\section{Foreword}
\label{sec:intro.foreword}
%====================================================
\subsection{A Brief Background to the Study}
\label{subsec:intro.foreword.background}
%====================================================
Currently the most popular topic for control and automation research is the around quadrotor UAV, specifically its attitude control. A wire range of work has been done on quadrotors and their attitude control, mostly designing control systems around a stable trim point adjacent to the inertial frames origin, to which the control algorithm always tends to. The highly coupled non-linear dynamics for a rigid bodies linear and angular motions arise as a result of gyroscopic torques [ \ref{subsec:dynamics.nonlinearities.gyrotorques} ] and Coriolis accelerations [ \ref{subsec:dynamics.nonlinearities.coriolis} ]. Such affects are can be linearised around the origin when they can be approximated to $\approx \vec{0}$ , thus decoupling the system and allowing for traditional SISO control techniques to be applied.
\par
As every quadrotor based research paper will tell you, the current interest in them is as a result of the recent emergence in availability of MEMS systems and low-cost ARM microprocessor sectors, allowing the on-board flight computer to perform complicated control calculations and state estimation in real time. As a result this led to development and expansion in the field and introduction of a large range of hobbyist solutions, from professionally made units to DIY kits, with large room for modification, depending on how much your wallet can spare. A rapidly growing enthusiast community was borne from this progression, meaning the environment was no longer open only to those willing spend lots of money.
\par
The avenues of potential applications for both fixed wing and VTOL UAVs is expansive and the quadrotor configuration provides a mechanically simple and low cost platform on which to test advanced aerospace control algorithms. Considering that commercial drone usage is such an emerging sector; especially in Southern Africa following the revision of aviation laws \cite{safedrone} which have legalized the use of UAVs for commercial application, any research into a non-trivial aspect of the field is extremely valuable. 
\par
Large scale quadrotor, hexrotor and even octorotor UAVs are a popular intermediate choice for aerial cinematography.  Whilst still expensive, the cost of a commercial drone like the SteadiDrone Maverik \cite{steadidrone} is far less exorbitant than the cost of chartering a helicopter to achieve the same panoramic aerial scenes or on-site inspections. Another interesting application for UAVs is in the agricultural sector, introducing crop dusting drones instead of the traditional bi-planes which perform the same job. One problem which hinders the progress of the commercial drone sector is that of inertia, specifically when scaling up any vehicle, its performance is adversely affected, due to the increased mass inertial effect.
%====================================================
\subsection{Research Questions \& Hypotheses}
\label{subsec:intro.foreword.hypotheses}
%====================================================
The difficulty with a quadrotors' control is that fundamentally it's unstable and under-actuated, having only 4 controllable inputs (each propellers rotational speed and hence net lift force) available to manipulate all 6 degrees of freedom (linear X-Y-Z position and angular Pitch, $\phi$, Roll, $\theta$ and Yaw, $\psi$ rotations). The resulting solution, whose derivation is explored in Appendix \ref{app:stddynamics}, is to control the perpendicular heave thrust, $\vec{T}$, and angular torques about each axis, $[\tau_\phi\;\tau_\theta\;\tau_\psi]^T$. So the attitude control problem of a quadrotor is a zero set point problem as any other attempt to track attitude can't be achieved.
\par
The aim of this project is to then implement dynamic set point tracking of a quadrotors' attitude and position by solving the problem of its inherent under-actuation. Inspired by Boeing/Bell Helicopters' V22 Osprey and the tilting articulation of its propellers, the prototype on which this paper is focused introduces two additional actuators for each of the four quadrotors' lift propellers. Specifically, adding rotations about the X and Y axes for each of the propellers. The resultant is a vectored thrust force which exists in 3-Dimensions with respect to the body frame, unlike a traditional quadrotor helicopter which has a bound perpendicular lift force. The control problem is then posed as the design of net forces, $\vec{F}_{net} = [F_x\;F_y\;F_z]^T$, and torques, $\vec{\tau}_{net} = [\tau_{\phi}\;\tau_{\theta}\;\tau_{\psi}]^T$, such that for any given trajectory, $X_d$, the error state, $X_e = X_d - X$, asymptotically tends to $\vec{0}$.
\begin{equation} \label{eq:1}
lim_{t \rightarrow \infty} X_e = \vec{0}\;\forall X \in \mathbb{R}^n
\end{equation}
Where $n$ is the degrees of freedom. The over-actuation brings about the need for a control allocation scheme which distributes the 6 commanded system inputs (net torques and forces) among the actuator set (12 actuators) in order to optimize some objective function secondary to that of Eq:\ref{eq:1}.
\par
Part of the control research question is the multivariate treatment of the system without making any simplifications to the non-linear dynamics involved in the quadrotors motion or making any assumptions about its operational conditions. The standard linearisations usually applied to the quadrotors control plant won't hold true for the more aggressive angular maneuvers envisioned for this prototype. Inherent to this is the expansion and simulation of existing kinematic models describing an aerial body and applying them to a quadrotor vehicles' motion. Thereafter, design, development and control of this new actuator suite to be implemented on such a quadrotor platform. The final key outcomes for the project are the simulation analysis and prototype construction of the proposed design.
\par
Introducing relative motion within an unconstrained body is going to produce a lot of unwanted dynamics. The obvious consequences of which are the inertial and gyroscopic responses. Pitching a rotating propeller is going to react much like a Control Moment Gyroscope, \cite{cmg}. A less trivial result is the aerodynamic torque produced from the propellers aerofoil profile. Such induced responses occur in obscure planes, normal to whatever the propellers thrust direction is. These aspects are normally cancelled out because a regular quadrotors' propellers all have the same plane of rotation. Because of these factors, a plant dependent control solution needs to be used to compensate for these dynamics, which if left unaccounted for would cause instability. 
%====================================================
\subsection{Significance of Study}
\label{subsec:intro.foreword.significance}
%====================================================
Due to the huge popularity of quadrotor platforms as research tools, any work which expands on the general body of knowledge relating to UAVs' \& quadrotors is going to be valuable to the community as a whole. With that being said, there already is a vast amount of existing research on linear and non-linear control techniques for regular quadrotor platforms. The attitude loop is the most common topic for control research, requiring an under-actuated solution and mostly linearised around the origin (See Appendix:\ref{app:stddynamics}). Far less common is the application of optimal flight path and trajectory planning to quadrotor control. The uniqueness and difficulty of the quadrotors attitude control does not hold true for its position control. Standard techniques can be used for way point planning and the like once the attitude control problem has been answered.
\par
The most significant aspect of this project is the attitude control, discussed later in \ref{sec:control.attitude}. The over-actuation of the proposed design and, more importantly, the manner in which the controllers' (virtual) output is distributed amongst those control effectors would appear to be the first of its kind. Otherwise known as control allocation, the requirements of the distribution algorithm(s) are outlined in Section: \ref{sec:control.allocation}. Dynamic set point attitude control for aerospace vehicles is not a subject heavily researched outside the satellite attitude control field. Even papers which propose similarly complex mechanical over-actuation (investigated in Section:\ref{sec:intro.litreview}) barely elude to the possibility of attitude set points away from $\vec{0}$.
\par
Whilst the control plant (developed in Chapter:\ref{ch:control}) does indeed close both the position  and attitudes control loops, there is no discussion of trajectory or flight path planning. Such topics are well discussed elsewhere in a far more concise and deliberate way than this project could ever hope to achieve. Once closed loop position and attitude controls have been achieved, the control algorithms can be adjusted to account for velocity and acceleration set point tracking to be used with nodal way point planning easily. The heuristics involved with flight path planning are well documented elsewhere and implementation of them is an academic task.
\par
For the proposed systems' identification and control treatment (design and allocation), a generic and modular approach is adopted. The intention is that applicability here falls not only within the UAV and quadrotor sections but to any other aerospace and freely rotating bodies needing attitude control, such as orbital satellites or underwater vehicles. Hopefully the investigation here can be built upon with more research focusing on one of the system subsets without compromising the functionality of the remainder of the system. A possible improvement which the investigation could yield is a higher actuator bandwidth and thus a faster control response for larger aerospace bodies. A standard quadrotor uses differential thrust to develop a torque about its body which suffers a slow inertial deceleration when changes speeds. Prioritizing pitching the propellers principle plane of rotation away rather than changes the propellers speed could improve response. This depends on what or how the allocator block is prioritized (presented in Section:\ref{sec:control.allocation}).
%====================================================
\subsection{Scope and Limitations}
\label{subsec:intro.foreword.scopeandlim}
%====================================================
\subsubsection{Scope}
\label{subsubsec:intro.foreword.scope}
%====================================================
This project includes the conceptualized design and implementation of a novel actuation suite to be used on a quadrotor platform. The express purpose of which is to apply set point attitude tracking control to the body. Stemming from this is an investigation of the associated kinematics which are influenced by the design and its relative motions. In order to apply control theory to achieve the attitude tracking goal, a sound model of the plant dynamics is first needed so that the plants' dynamics can be analysed.
\par
Aspects of the mechanical design are covered in Section \ref{sec:proto.design} but, beyond the cursory consideration, there is no scope for materials analysis or stress testing of the design. To the detriment of the project, the design will either produce an over-engineered or catastrophically under-engineered solution. The focus is rather on the control application and embedded systems design, not the structural integrity of a proposed frame. The only physical measurements made are ones which pertain to the critical kinematics like inertial measurements for the second order gyroscopic and inertial dynamic responses.
\par
As mentioned in the antecedent , Section: \ref{subsec:intro.foreword.significance}, trajectory \& flight path planning are not ubiquitous with this investigation. The kinematic derivation for a 6-DOF body is wholly applicable to any dynamic (rigid or otherwise) aerospace body, although some particular standards are used [sic ZYX Euler Aerospace Sequence, \ref{subsec:proto.conventions}]. Similarly the control treatment of the plant is that of a non-linear multivariate control, aided and justified by Lyupanov theorem. Whilst alternative solutions through Model Predictive Control or Quantitative Feedback Theory could provide a more refined or effective controller, they aren't presented and remain open to further investigation. The standard approach for quadrotor attitude control is feedback linearisation of the plant around a trim point to decouple the non-linear dynamics and apply SISO techniques. A derivation of such a linearisation is presented in \ref{app:stddynamics} but there are no further discussions beyond that. Comparison between attitude set point tracking proposed here and normal zero-set point attitude control of fixed rotor quads' is difficult as the fundamental objectives are in stark contrast with each other.
\par
Arguably the most important and indeed novel aspect of this project is the control allocation. Seeing as the system has 12 controllable inputs and 6 possible responses to that input, hence the system is classified as over-actuated. Ergo, there needs to be some logical process as to how those 12 inputs are articulated to achieve the desired 6 movements. Appropriate techniques are first investigated in section \ref{sec:control.allocation} and compared before a final solution is implemented in \ref{ch:ch5}. It is by no means a comprehensive investigation of all solutions available but rather an analysis of the sub-set of problems and design of what is regarded as a logical and appropriate solution.
\par
With regards to the actual prototype design, in Section \ref{sec:proto.design}, it's assumed that certain aspects are a given certainty. Particularly the state estimation, updated through a 4-camera positioning system fused with a 6-axis IMU through Kalman Filtering, is assumed to precise and readily disposable at a consistent 50 Hz. Hence state estimation is presented but is bereft of intricate detail, this is another topic which has been well documented elsewhere.
%====================================================
\subsubsection{Limitations}
\label{subsubsec:intro.foreword.limits}
%====================================================
By far the biggest constraint of the design is the net weight of the assembled frame. The lift required to keep the body aloft is obviously dependent on the all up weight. Thrust forces disposable to the controller then need to be such that there is clear headroom below actuator saturation during hover flight. The controller effort increases with the magnitude of change for desired state, so steady state actuation conditions must be just a fraction of the maximum actuator outputs. Conversely the all up weight is mostly dependent on the lift motors, being the heaviest part of the vehicle, and their associated power electronics. A trade-off between these two factors makes designing the prototype a balancing act of compromise; added actuation is needed to produce the desired thrust vectoring. That added actuation is going to increase the weight which will then need more thrust force to ensure the vehicle remains airborne. Bigger motors then require stronger actuators to effect the relative motion and overcome the bodies inertial response. It's a compromise between the weight of the body and the strength/quality of the actuation.
\par
To forego the deliberation detailed above, a self imposed limitation applied to the design is to only make use of a particular predetermined motor, namely a set of four Turnigy DST-700 motors. The \dept ~at \uni ~has a surplus of these from previous projects so it saves on new motor costs. A direct consequence of this decision is that the net thrust disposable for actuation is limited to around 700g, $\approx$ 6.9 N, per motor (see Section: \ref{subsec:dynamics.aero.bem}, later in Chapter \ref{ch:dynamics}). This means that all other aspects of the prototype need to adhere to this weight limitation. It is crucial to ensure the control algorithm doesn't induce over-saturation of the motor actuations, so the frame weight needs to be around 40-50\% of the maximum available thrust. These saturation conditions are expanded upon later in Section: \ref{sec:control.allocation} in more detail.
\par
Another aspect of the design limitations resulting from decisions taken, mainly to reduce the costs of construction and complexity, is the use of 180\textdegree ~rotatable servos. The servos are for the individual motors' pitch and roll actuation and act in lieu of continuous rotation DC (brushless or stepper) motors. Any rotation beyond 360\textdegree would require both closed loop position control of the actuator, unlike a servo, and slip rings for power transmission so that no wiring would impede the bodies relative rotation. However the logistics of implementing such a design whilst maintaining an acceptable weight is almost impossible without dramatically scaling up the size of the prototype to accommodate for the massive weight increases.
Commercial camera stabilizing gimbals already make use of similar configurations but the I/O requirements from the flight controller $\mu$C already constricts the amount of expansion at hand.
\par
Some of the discretionary elements for  the whole system will limit performance but are mitigated where possible. For example analogue servos have an associated 1 ms deadband from their 20 Hz refresh rate which can be addressed by using faster, albeit more expensive, digital servos which sample at 333 Hz. The on-board flight control system, see \ref{sec:proto.layout}, needs to apply PWM outputs to 12 different actuators as well as receiving command updates from a ground control station so the I/O capability of most embedded systems are going to be at capacity. Sub-systems will have to be divided and relative communications adopted for various comms and on-board logging. All of these things are addressed in the following Chapter \ref{ch:proto}.
%====================================================
\section{Literature Review}
\label{sec:intro.litreview}
%====================================================
\subsection{Existing \& Related Work}
\label{subsec:intro.lit.related}
%====================================================
The field of transformable aerospace frames is not necessarily a new one, with many commercial examples having seen a lot of success over their operational life span. The most notable tilting-rotor application is that of the Boeing/Bell V22 Osprey aircraft. First introduced in the field in 2007, the Osprey has the ability to pitch its two lift propellers forward to aid in translational flight after a VTOL manoeuvre has been completed. In addition to this there have been a handful of papers published on similar tilting bi-rotor UAVs' (Fig: \ref{fig:dualaxistilt}\footnote{Image used from G. Gress: \cite{gres2007}}) for research purposes.
\begin{figure}[hbtp]
\centering
\includegraphics[width=0.7\textwidth]{figs/dualaxistilt}
\caption{General Structure for Opposed Tilting Platform}
\label{fig:dualaxistilt}
\end{figure}
\subsubsection*{Birotors}
Research into birotor vehicles (Fig: \ref{fig:dualaxistilt}) with ancilliary lift propeller actuation is often termed Opposed Active Tilting, \emph{OAT}. Such a rotorcrafts' mechanical design applies either a single \emph{oblique} 45\textdegree ~tilting axis relative to the body; \cite{smalltwotilting,obliquepitch,passiveobliquetilting}, or a \emph{lateral} tilting axis, adjacent to the body; \cite{tiltrotorUAV,adaptivebackstep,tiltrotorcontrol,tpheonix}. Leading research is currently focussed on applying doubly actuated tilting axes to birotor UAVs. Dual axis Opposed Active Tilting or \emph{dOAT} introduces vectored thrust with propeller pitch and roll motions to further expand the actuation suite, \cite{gres2007,opposedlateraldualaxis}. A birotor is sometimes considered preferable to the multirotor platform due to its reduced controller effort. However the controller plant abstraction often detracts from the quality and effectiveness of its treatment as a result of its' underactuation. 
\par
Birotor attitude control incorporates the typical plant independent PD \cite{obliquepitch} and PID \cite{tiltrotorUAV} controller schemes but often more computationally expensive and plant dependent Ideal and Adaptive backstepping controllers are investigated, presented in \cite{smalltwotilting,tpheonix} and \cite{adaptivebackstep} respectively. The coupling of a birotor vehicles' attitude plant is more prominent than a quadrotor, derived in Section: \ref{sec:dynamics.nonlinearities}, and so feedback linearisation is almost always used. In an interesting progression from the norm, Lee et al,  \cite{autopilotPSO}, proposed a PID co-efficient selection algorithm using a Particle Swarm Optimization techinque, similar to \cite{adaptivepso}. However their performance metric criterion was a basic ITAE term and not anything more unique involving effects specific to flight. \emph{PSO} algorithms iteratively search for a globally optimized solution and offer independent, derivative free optimization. This project report exploits PSO optimization for non-linear controller coefficient selection, shown in Section:\ref{sec:control.tuning}.
\par
\subsubsection*{Quadrotors}
Expanding on multirotor vehicles, the quadrotor UAV is a popular and well covered research platform due to its relative mechanical simplicity. What would appear to be one of the first quadrotor research implementations, in 2002, is the X4-Flyer, \cite{x4flyer,x4flyercontrol}. Subsequently alternative iterations like the Microraptor, \cite{microraptor}, \& STARMAC, \cite{starmac}, quadcopters have been built and tested. A plethora of literature exists around basic quadrotor kinematics \& control \cite{dynamicmodelling2013, dynamicmodelling2009, modellingquadcopter, quaddynamics, fullquadcoptercontrol}.  however dedicated 6-DOF rigid body dynamic papers \cite{rigidbodylecture,eulerrigidbody} are far better at providing insight into the appropriate kinematics and apply no simplifying assumptions about the bodies dynamics. Advanced aerodynamic effects \cite{} and non-linear models have sometimes been applied to quadrotors



Their kinematic models are often intuitively represented with Rotation Matrix dynamics (Section: \ref{subsec:dynamics.rigidbody.singularity});


 and simulation of a 6-DOF quadrotor is another well covered topic;  however the 6-DOF dynamics are applicable in more general terms 


The dynamics of a 6-DOF quadrotor, modelling and simulation thereof is another well covered topic. rotation matrix kinematics (Section: \ref{subsec:dynamics.rigidbody.singularity}) are given in 
 There are many more examples of simple quadrotor designs all with similar content \& conclusions. Some research has been done on transformable quadrotor frames which, like the single and dual axis \emph{OAT} birotors, can articulate their motors. Single axis tilting is the most common
\par
The dynamic 6-DOF model is a well covered concept, \cite{dynamicmodelling2013,dynamicmodelling2009}


%================UPTOHERE============================
. Some hobbyists and enthusaists introduce an angled bracket to pitch the motors in order to improve forward translational speeds. This however adds no extra actuation and reliance is put on the autopilot flight controller to account for the actuator changes. An expensive commercial quadrotor which has some transformation capability is the Inspire 1 \cite{inspire}, made by Shenzen DJI Technologies who are more commonly known for their Phantom drone.
\par
\begin{figure}[htbp]
\centering
\begin{subfigure}{.5\textwidth}
\centering
\includegraphics[width=\textwidth]{figs/dji-inspire1}
\caption{Inspire 1 articulated upwards}
\label{fig:inspireup}
\end{subfigure}%
\begin{subfigure}{.5\textwidth}
\centering
\includegraphics[width=\textwidth]{figs/dji-inspire2}
\caption{Inspire 1 articulated downwards}
\label{fig:inspiredown}
\end{subfigure}
\caption{DJI Inspire 1}
\label{fig:inspire1}
\end{figure}
The Inspire can articulate its supporting arms up and down [Fig: \ref{fig:inspire1} \footnote{Both images were sourced from the drones patent, held by SZ DJI Tech Co\cite{djinspire}}]. The purpose of such movements is to both alter the center of gravity and further expose the belly mounted camera gimbal to achieve panoramic sequences. The change in the center of gravity affects the bodies inertial response to angular acceleration (reducing the magnitude of the mass moment of inertia about each axis). However the range of "transformation" the frame can undergo is just limited to articulating the arms up and down.

\par
It is worth mentioning that  hardly any research has been done into aerospace frames that can dynamically transform their structure during flight. Rigid body dynamics are mostly used to solve the control problem so introducing relative motion within the body frame during flight complicates matters. In two reports, \cite{tiltpropellercontrol},\cite{tiltpropellerflight}, M. Ryll et al adapted and tested a QuadroXL from MikroKopter \cite{mikrokopter} which has a single axis of tilt aligned with the X-axis to change the direction of the propellers. Their dynamic equations neglect any second order aerodynamic responses and the \underline{gyroscopic and inertial} effects. The simplified solution also fails to address the inherent singularity associated with their Rotation Matrix solution. A similar implementation of single axis tilting was done by Nemati et al \cite{singleaxistilting} but had a far more brief kinematic derivation and only featured a simulation as proof of concept.
\par
Lastly, the most advanced implementation of over actuation for quadrotor control is that of Pau Segui Gasco \cite{tiltgasco}, which was a dual presented MSc project with Yazan Al-Rihani \cite{tiltrihani}. At the time of writing, this would appear to be the only published project which bears semblance to the proposed concept of this paper. The work was split between the two authors who completed the control/electronic design and the mechanical platform design for their respect MSc projects. Shown in Fig:\ref{fig:tiltrotor-gasco} \footnote{Development of a Dual Axis Tilt Rotorcraft UAV: Modelling, Simulation and Control \cite{tiltgasco}}, their dual-axis articulation is achieved with an adapted helicopter tail bracket, reducing the mass of the articulated component but limiting the range of rotation. Their justification for adding extra actuations is to ensure control reliability even in the event of losing up to 2 rotors.
\begin{figure}[hbtp]
\centering
\includegraphics[width=0.7\textwidth]{figs/gasco-mech}
\caption{Dual-axis tilt-rotor mechanism}
\label{fig:tiltrotor-gasco}
\end{figure}
\par
In the control and dynamics derivation, Gesco et al provides an excellent model for 6-DOF motion. As is standard fare with quadrotor papers, the dynamic equations are linearised around a trim point parallel to the inertial frame to allow for SISO control analysis of the system. It is important to note the difference between this papers' prototype and the Bi-Directional Tilting Quadrotor Prototype developed in Chapter: \ref{ch:design} ,the added actuation in \cite{tiltgasco} is not used to vector thrust produced but rather to leverage induced gyroscopic torques as an actuation input. The control allocation technique developed is wholly unique, fusing differential torque and torques induced from the relative motion with a (simplified) weighted pseudo inverse method. This all resulted in a control plant with a far higher control bandwidth. 
\par
In practice the gyroscopic response (see Section \ref{ch3:gyroscpic torque}), or rather regarding the spinning propellers as control moment gyroscopes \cite{cmg}, has a response two orders of magnitude smaller than that of the inertial response from pitching the mass of the motor \& propeller combination. And so such an inertial response is far more important to account for, weather it be compensation or exploitation, that the gyroscopic torques induced.
\par



Another interesting project is that by G.Gres \cite{gres2007} which attempts to use oblique active tilting of a bi-rotor helicopter to induce gyroscopic torques. Whilst the eVader prototype referred to in the paper has dual axis tilting, the actuators are coupled together making the tilt axis 45\textdegree to the bodies' frame. The paper discusses in depth the mechanical system identification aspects of the prototype but, however, gives no insight into attitude control. The author instead discusses the dynamic equations of motion applicable in different flight modes. Whilst the later is irrelevant to quadrotor attitude control, the derivation of induced gyroscopic torque responses as a result of pitching the propellers from their principle plane of rotation is highly pertinent and appears to be the most accurate and well thought out of its kind.

\subsection{Notable Quadrotor Control Implementations}
\label{subsec:intro.lit.control}
%----------------------------------------------------
The majority of papers based on Quadrotor research, \cite{quaddynamics},\cite{optimizedPID}, \cite{fourrotorrobot} all make the assumption that the coupled non-linear dynamics can be linearised. This assumptions holds true as long as the angular rate, $\vec{\Omega}$ is small and the inertial matrix, $\mathbb{I}$ is a diagonal matrix. As a consequence the gyroscopic term, (see \ref{ch:ch3}) which manifests itself as: $\tau _{gyro} = \vec{\Omega} \times \mathbb{I} \vec{\Omega} \approx \vec{0}$. This decouples the angular equations of motion and similarly the coriolis acceleration term becomes negligible; $-\vec{a} \times \vec{\Omega} \approx \vec{0}$.
\par
A PID strucuture for attitude controllers are the norm, with \cite{optimizedPID},\cite{quaddynamics},\cite{tiltpropellerflight} all implementing standard PID controllers and even \cite{singleaxistilting} using only a PD controller despite having an over-actuated platofrm. Even commercial hardware flight controllers like Arducopter\cite{arducopter} and OpenPilot \cite{openpilot}(whose firmware source code is available at \cite{openpilotgit}) all use PID structures with some manner of feedforward or feedback elements, \cite{buildyourownquad}.

However, in \cite{optimizedPID} the controller coefficients were selected through a learning social algorithm, a particle swarm optimization, instead of the regular "tuning" by hand. 
\par
As a result of the inherent singularity with using Rotation Matrices to represent attitude, 
%****************************************************

At the time of writing, there appears to be only two other projects which have been published that bear some similarity. Discussion is given later in Section:\ref{subsec:intro.lit.related} where comparison is made to justify how they are different and why this project is still unique and perhaps even novel. The concepts developed here are unique to the application of quadrotor control, mostly having been developed in the late 90s for satellite control. Similarly, the non-linearity with which the control solution is developed is uncommon with respect to UAV control.