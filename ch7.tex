%====================================================
%	CHAPTER 7 - Conclusions
%====================================================
\chapter{Conclusions and Recommendations}
\label{ch:conclusion}
%====================================================
The original objective for the project was to design, simulate and physically test the prototype outlined in Sec:\ref{sec:proto.design}. Modeling the responses of the multibody prototype to obtain the dynamic equations of motion, derived in Sec:\ref{sec:dynamics.nonlinearities}, proved to be dramatically more complex than initially anticipated. 
Time varying moments of inertia (generalized in Eq:\ref{eq:inertial-rate-def}) introduced to the Lagrangian kinematics resulted in an unique problem formulation. Each actuated motor module's response to the net vehicle's dynamics (Eq:\ref{eq:torque-induced-inner},\ref{eq:torque-induced-middle} and \ref{eq:module-response}) required multiple revisions. Each step in the derivation was tested in simulation to ensure the mathematics applied were sound. The difficulties accompanying those derivations pushed back the project's time-line significantly. As a result, it was decided to cancel inclusion of physical flight tests. Physical implementation of the proposed control laws therefore remains open to further work.  With the above being considered, the dynamic model for the system is a significant contribution of this work. The uniqueness of the multibody structure made solving for the differential equations of motion a sizeable task. A consequence of the complex dynamics was an extremely high degree of stiffness in the system which adversely affected the simulation times. Alternatively, relative coordinates could have been implemented in lieu of the used Cartesian coordinates which describe the vehicle and its configuration. Relative descriptions of each state variable could reduce the complexity in calculating instantaneous moments of inertia at each simulation interval. Moreover, implicit Euler integration could have been applied to the simulation, both changes could potentially yield simulation improvements. The cost of such changes would be to reconstruct the entire simulation environment.
\par
The physical tests which corroborated aspects of the dynamic model (Sec:\ref{subsec:dynamics.nonlinearities.torque-tests}) would ideally be extended to physical flight tests. However, considering the complexity of the system and modelling thereof, verification of the dynamics is a useful result. The time varying, non-diagonal inertias of each body  in the multibody system are consequences of the design process and the cost constraint applied to the prototype. In practice, if the rigid component of the frame ($J_y$ from Eq:\ref{eq:inertia.body.c}) was sufficiently larger than that of the actuated (\emph{rotating}) bodies, the relative effect of the multibody interaction responses ($\vec{\tau}_b(\hat{u})$ from Eq:\ref{eq:net-body-response}) on the dynamics would be diminished.
\par
One of the original justifications for the increased platform complexity was the improved actuator bandwidth that would accompany thrust vectoring. The hypothesis was that pitching or rolling a thrust vector would have a faster response than changing the propeller's rotational speed (see actuator transfer functions in Sec:\ref{subsec:proto.design.transfer}). The firmware changes made to the ESCs improved the brushless DC motor's transfer function's time constant significantly. The original firmware which the ESCs used by default produced an exponentially approaching speed curve rather than the standard linear relationship (illustrated in Fig:\ref{fig:rpm-sensor}). Step tests comparing changes before and after the ESC firmware was changed were not be performed. The overall constraint encountered by the actuator plant was rate (\emph{current}) limiting imposed on the rotational servos as a result of their electrical design. That constraint limited the performance of the more aggressive Ideal Backstepping attitude controller, Fig:\ref{fig:IBC_controller_result}. The servo rate limits prevented the motor modules from actuating fast enough, seen in the difference between controller designed and physically commanded inputs in Fig:\ref{fig:IBC_Torque}.
\par
The control solutions presented in this dissertation all stabilize the plant. Respective results for attitude and position controller steps (Sec:\ref{sec:simulation.attitude} and Sec:\ref{sec:simulation.position}) demonstrate the improvement exponential stability yields on a controller plant. All of the control laws proposed were able to track the applied chirp trajectory for low trajectory rates. Each controller's optimization was an ITAE optimization, prioritizing settling times and overshoot errors over aggression or input magnitude. Alternatively, the optimization could apply a penalty to a proposed set of controller coefficients based on energy expenditure or induced torque response, emphasizing stability and smooth transitions over settling times. A particle swarm optimization in Sec:\ref{subsec:simulation.tuning.pso} was chosen due to its simplicity and lack of an explicitly defined gradient function. More complex optimization paradigms could have potentially produced more efficient optimizations.
\par
Certain constraints or assumptions were applied to the model in simulation. It was shown in Sec:\ref{sec:simulation.saturation} that applying rotational limits to the actuation servo broke down the overall setpoint tracking of the control loop. Extending the actuators to accommodate for continuous rotation requires an alteration of the mechanical design and drastically improves the range of motion. The only significant assumption made on the plant's aerodynamics was neglecting to account for any propeller's down-wash becoming incident flow into other propeller. This would have a sizeable impact on the thrust plant model, requiring a complicated fluid dynamics solution to approximate for such effects. The decision to apply nonlinear state space control to the plant prevented the use of Model Predictive control. An MPC control law could potentially better compensate for the vehicle's non-linearities, which were otherwise relegated to feedback compensation.
\par
In conclusion, the non-zero state setpoint tracking goal was achieved by each of the control laws proposed. The control allocation rules applied did not have a notable effect on the plant's performance because of the structure applied in Sec:\ref{sec:allocation.inversion}. Finally, the dynamic model's complexity and the difficulties involved in verification of that model outweighed the control improvements shown. The thrust vectoring accomodated for unique 6-DOF trajectory tracking to be performed, however the same could have been achieved with only a single axis of rotational tilt applied to each lift propeller (similar to related projects described in Sec:\ref{sec:intro.litreview}). The same dynamic complexities led to catastrophic failures with earlier versions of Osprey \cite{ospreywired}, the inspiration for this project. Those complexities led to the subsequent redesign of the Osprey's successor, the V-280 Valor, which has significantly smaller actuator inertias due to fewer moving parts (at the cost of more frequent maintenance).