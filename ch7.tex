%****************************************************
%	CHAPTER 7 - Conclusions
%****************************************************
\chapter{Conclusion}
\label{ch:conclusion}
%****************************************************
- Lagrange dynmaics for multibody system could have produced a more concise model etc \ldots\\
- Particule multibody dynamics with internactions could provide a more verbose simulation environment rather than the very newtonian simulation loop constructed\\
- Implicit equation dynamics in simulation may improve optimization loops in PSO algorithm\\
- Firmware changes to ESC drastically improved transfer function time constant, made assumption that servos would improve actuator response times redundant.\\
-Difficulty with non-linear multibody in Sec:\ref{sec:dynamics.nonlinearities} causes stiffness in control optimization. Same troubles with time varying inertias caused Osprey inspiration issues too...\\
- non-linear multibody dynamics required multiple revisions, took longer than expected\\
- suffered time constraints as a result\\
- same multibody dynamics which caused issues with the original osprey testing\cite{}\\
- track angular momentum state, not angular position state. Could potentially remove the complixites of calculating explicit inertial values at discrete simulation intervals however 'unwinding' analogue could be detremental. Given the high degree of freedom the system has, each angular momentum state probably has an entire set of solutions.
- complexities from non-zero $\dot{J}(t)$ for inertial equations and lagrange mechanics in appropriate chapter. At design stage it was elected to design around a smaller frame. If the rigid component of the frame, $J_y$ in Ch:\ref{}, was sufficiently greater than the inertias of actuated components, the complexities from $\tau_b$ could be simplified greater. Moreover the inherent inertial damping would compensate for a lot of the torque spikes shown in Eq:\ref{} from Ch:\ref{}.
- Model for $\tau_b$, Eq:\ref{} from Ch:\ref{}, is made such that an alternative model could easily be incorporated.
- Probably should've used principle axes for inertias, would have made calculating derivatives easier but there are just too many bodies to keep track of, making the inertia transformations an easier choice...
-Model assumes no downwash effects
%****************************************************
