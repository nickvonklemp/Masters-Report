%====================================================
%	CHAPTER 7 - Conclusions
%====================================================
\chapter{Conclusions and Recommendations}
\label{ch:conclusion}
%====================================================
The initial objective for the project was to design, simulate and physically test the prototype outlined in Sec:\ref{sec:proto.design}. Modelling the responses which the multibody prototype had to the dynamic equations of motion, derived in Sec:\ref{sec:dynamics.nonlinearities}, proved to be dramatically more complex than initially anticipated. 
Non-zero inertial time rates (generalized in Eq:\ref{eq:inertial-rate-def}) introduced to the Lagrangian kinematics resulted in an entirely unique problem formulation. Each actuated motor module's response to the net vehicle's dynamics (Eq:\ref{eq:torque-induced-inner},\ref{eq:torque-induced-middle} and \ref{eq:module-response}) required multiple revisions. Each step in the derivation was tested in simulation to ensure the mathematics applied were sound. The difficulties accompanying those derivations pushed back the project's time line significantly. As a result the final flight tests, which were initially envisioned were, unfortunately not completed. Physical implementation of the proposed control laws then remains open to further research.
\par
With the above being considered; the produced dynamic model for the system would appear to be a significant contribution for this work. The uniqueness of the multibody structure made solving for the differential equations of motion a sizeable task. A consequence of the complex dynamics was an extremely high degree of stiffness in the system which adversely affected the simulation times. Alternatively, relative coordinates could have been implemented in lieu of the used Cartesian coordinates which describe the vehicle and its configuration. Moreover implicit Euler integration could have been applied to the simulation, both changes could potentially yield simulation improvements. The cost of such changes would be to construct a dedicated simulation program. Matlab's Simulink does not support implicit Euler integration for simulations.
\par
The physical tests which corroporated aspects of the dynamic model, in Sec:\ref{subsec:dynamics.nonlinearities.torque-tests}, would ideally be extended to physical flight tests. However considering the peculiarities of the system under consideration, the dynamics' verification (to a relative degree) was a significant result. The non-symmetrical inertias of each body within the entire multibody system was a consequence of the design process and the cost constraint applied to the prototype. In practice, if the rigid component of the frame ($J_y$ from Eq:\ref{eq:inertia.body.c}) was sufficiently greater than the inertial components of the actuated bodies, the complexities of the multibody interaction responses ($\vec{\tau}_b(\hat{u})$ from Eq:\ref{eq:net-body-response}) would be diminished. The magnitude of those responses are actually inconsequential at slow or steady state operation (Fig:\ref{fig:tau-body}), only spiking to significant values with large steps, Sec:\ref{subsubsec:dynamicmodel}.
\par
One of the original justifications for the project and its increased platform complexity was the improved actuator bandwidth accompanying the thrust vectoring. Asserting that pitching or rolling a thrust vector has a faster response than changing the propeller's rotational speed (see actuator transfer functions in Sec:\ref{subsec:proto.design.transfer}. The firmware changes made to the ESCs improved the brushless DC motor's transfer functions so much that the initial conjecture was rendered moot. Unfortunately the basic ESC's had exponential speed curves which were unsatisfactory for the control, so step tests comparing the before and after cases could not be performed.
\par
The control solution(s) presented here each stabilize the trajectory in their own right. Results for controller attitude steps in Sec:\ref{sec:simulation.attitude} and Sec:\ref{sec:simulation.position} demonstrate the dramatic improvement exponential stability yields on a controller plant. None of the control laws proposed (attitude or position) were unable to track the applied trajectory, albeit a simple one. Each controller's optimization was a time error optimization which was shown in their respective performance. The controllers prioritized settling times and overshoot errors over aggression or input magnitude. Alternatively, the optimization apply penalty based on energy expenditure or induced torque response, prioritizing stability and smooth transitions over settling times. 
A particle swarm optimization in Sec:\ref{subsec:simulation.tuning.pso} was chosen due to its simplicity and lack of an explicitly defined gradient function. Perhaps more complicated optimizers could produce stronger control responses in less time and the cost of computational complexity.
\par
Certain constraints or assumptions were applied to the model in simulation. It was shown in Sec:\ref{sec:simulation.saturation} that applying rotational limits to the actuation servo's dramatically hampered the over-all setpoint tracking ability of the control loop. Extending the actuators to accommodate for continuous rotation requires only a simple alteration to the mechanical design. The only significant assumption made on the plant's aerodynamics was neglecting to account for propeller's down wash being incident flow into other propeller modules. This would definitely have a sizeable impact on the thrust plant model, requiring a complicated fluid dynamics solution to approximate for such effects. Lastly, the decision to apply nonlinear state space control to the plant prevented the use of Model Predictive control. An MPC control law could potentially better compensate for or exploit the vehicle's non-linearities which were otherwise relegated to feedback compensation.
\par
In conclusion, the non-zero state setpoint tracking goal was achieved by each of the control laws proposed. The control allocation rules applied did not have a notable effect on the plant's performance because of the structure applied in Sec:\ref{sec:allocation.inversion}. Finally there is a good reason why little to no work exists on reconfigurable aerospace vehicles, the dynamic complexity always out weighs the perceivable control improvements. Ironically those same dynamic complexities led catastrophic failures with earlier versions of Osprey \cite{ospreywired}, the inspiration for this project. Those complexities led to subsequent redesign of the Osprey's successor, the V-280 Valor, which has significantly smaller actuator inertias due to less moving parts (at the cost of higher maintenance fees)\ldots