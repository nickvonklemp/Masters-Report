%=====================================================================
%	UCT MECHATRONICS GROUP TEMPLATE
%---------------------------------------------------------------------
%	Template made by:		R.A. Verrinder
% 	Written  by:	N.R. Von Klemperer
%	Date Modified:	June 2016
%---------------------------------------------------------------------
%	Compile with:	TexMaker 
%=====================================================================
%	Document properties:
%---------------------------------------------------------------------
%	Paper size:			A4
%	Margins:			1 inch margins(top,bottom,left,right)
%	Printing:			Double sided (twoside book)
%	Base font size:		11pt
%	Line spacing:		
%	Paragraph spacing:	10pt
%	Paragraph indent:	0pt
%=====================================================================

\documentclass[a4paper, 11pt, oneside, openright, parskip=full]{book}

%---------------------------------------------------------------------
%	PACKAGES
%---------------------------------------------------------------------
\usepackage{afterpage}
\usepackage[margin = 2cm]{geometry}		
\usepackage{lscape}
\usepackage{rotating}
\usepackage{hyphenat}
\usepackage{setspace}
\usepackage{hyperref}
\usepackage{titlesec}
\usepackage{lastpage}
\usepackage{fancyref}
\usepackage{fancyvrb}
\usepackage{cite}
\usepackage[usenames,dvipsnames]{xcolor}
\usepackage{graphicx}					% Can include graphics
\usepackage{float}						% Can place figures at exact places in text
\usepackage{wrapfig}					% Can wrap text around figures
\usepackage{import}

\usepackage{array}						% Can set column widths in tables
\usepackage{mdframed}					% Can draw boxes around text etc.
\usepackage{subcaption}					% Can place figures side by side
\usepackage{tikz}
\usetikzlibrary{positioning}		
\usetikzlibrary{shapes.geometric, arrows} % For block diagrams
\usetikzlibrary{circuits.logic.US} 
\usetikzlibrary{calc}
\usepackage[siunitx]{circuitikz}		% Can draw circuits
\usepackage{tikz-timing}				% Can draw timing diagrams
\usepackage{enumerate}
\usepackage{enumitem}
\usepackage{xspace}						% For box spaces
\usepackage{caption}					% Captions and subcaptions
\usepackage{tabularx}
\usepackage{fix-cm}						% Can set font size
\usepackage{color}						% Can change font color

\usepackage[scaled=0.9]{helvet}			% Helvetica font scaled \phv
\usepackage{courier}					% Courier font \pcr
%\usepackage{mathptmx}					% Times New Roman font
%\usepackage{cmbright}					% Sans serif font
%\normalfont 							% in case the EC fonts aren't available
%\usepackage[T1]{fontenc}				% Used to switch to type 1 font encoding
\usepackage{textcomp}
\usepackage{amsmath, amsthm, amssymb}
\usepackage{datetime}					% Can use dates and times
\usepackage{gensymb}				
\usepackage{ragged2e}
\usepackage{multirow}
\usepackage{hhline}
\usepackage{amsmath}
\usepackage{wrapfig}
\usepackage{mathtools}
\usepackage{tablefootnote}
%\usepackage[all]{nowidow}
%---------------------------------------------------------------------
% 	PAGE FORMATTING
%---------------------------------------------------------------------
% Include page formatting here. 
%---------------------------------------------------------------------
\parskip	= 10pt						% Paragraph spacing
\parindent 	= 0pt						% No para. indent

% Caption margins 80% of text
\captionsetup{width=0.8\textwidth,font=small,labelfont=bf}		

%---------------------------------------------------------------------
%	DOCUMENT PROPERTIES
%---------------------------------------------------------------------
\newcommand{\auth}{Nicholas Von Klemperer}								% Author
\newcommand{\titl}{Dual-Axis Tilting Quadrotor Aircraft}						% Thesis title
\newcommand{\subtitle}{Dynamic modelling and control thereof}						% Thesis subtitle
\newcommand{\dept}{Department of Electrical Engineering}% Department 
\newcommand{\uni}{University of Cape Town}				% University
\newcommand{\city}{Rondebosch, Cape Town}				% City
\newcommand{\country}{South Africa}						% Country
\newcommand{\degre}{Masters of Science}				% Degree (Doctor of Philosophy 
														% or Master of Science)
\newcommand{\degreeabv}{MSc}							% Degree abbreviation (PhD or MSc.)			
\newcommand{\aemail}{Nicholas.VonKlemperer@alumni.uct.ac.za}			% Author email address
\newcommand{\supervisor}{Supervisor}					% Supervisor
\newcommand{\hod}{Prof. E. Boje}						% Head of Department

\newcommand{\logo}										% UCT Logo
{	
	\includegraphics[scale = 0.3]{figs/uctLogo}
}

\newdateformat{mydate}{\monthname[\THEMONTH] \THEYEAR} % Month Year date
\newcommand{\key}{Non-linear, control, allocation, quadrotor, UAV}					   % Keywords (add logical keywords)
\newcommand{\norm}[1]{\left\lVert#1\right\rVert}
%---------------------------------------------------------------------
%	MACROS
%---------------------------------------------------------------------
% Signature box
%---------------------------------------------------------------------
\newcommand*{\signature}[1]
{
	\par\noindent\makebox [5cm]{\hrulefill}
	\par\noindent\makebox [5cm][1]{#1}
}
\newtheorem{theorem}{Theorem}[section]
\newtheorem{corollary}{Corollary}[theorem]
\newtheorem{lemma}[theorem]{Lemma}
%---------------------------------------------------------------------
% Macro to select font style
% phv = helvetica (sans serif)
% pcr = courier	  (typewritter font)
%---------------------------------------------------------------------
\def\nfont#1#2
{
	{\fontfamily{#1}\selectfont #2}
}					
%---------------------------------------------------------------------
% Creates an block for a block diagram
%---------------------------------------------------------------------
\tikzstyle{arrow}=[->,shorten <=1pt,>=stealth',semithick]
%---------------------------------------------------------------------
%block 1
%---------------------------------------------------------------------
\tikzstyle{block}=[rectangle,
				   draw=black,
				   fill=gray!10,
				   thick, 
				   minimum height = 2cm, 
				   text width=0.15*\columnwidth,]
\tikzstyle{input}  		= [coordinate]
\tikzstyle{output} 		= [coordinate]
\tikzstyle{pinstyle} 	= [pin edge={to-,thin,black}]
\tikzstyle{sum} 		= [draw, fill=blue!20, circle, node distance=1cm]
%---------------------------------------------------------------------
%block 2
%---------------------------------------------------------------------
\tikzstyle{block2}=[rectangle,
				   draw=black,
				   fill=gray!10,
				   thick, 
				   minimum height = 1.8cm, 
				   text width=0.1*\columnwidth,]
\tikzstyle{input}  		= [coordinate]
\tikzstyle{output} 		= [coordinate]
\tikzstyle{pinstyle} 	= [pin edge={to-,thin,black}]
\tikzstyle{sum} 		= [draw, fill=blue!20, circle, node distance=1cm]
%---------------------------------------------------------------------
%flow charts
%---------------------------------------------------------------------
\tikzstyle{state}     = 	[rectangle, 
							 text width=0.15*\columnwidth=2cm, 
							 minimum  height=1.5cm, 
							 text centered, 
							 draw=black]
							 
\tikzstyle{statement}  = 	[rectangle, 
							 text width=0.15*\columnwidth=2cm, 
							 minimum  height=0.5cm, 
							 text centered, 
							 draw=black]							 
							 
\tikzstyle{condition}  = 	[rectangle, 
							 text width=0.15*\columnwidth=2cm, 
							 minimum  height=1cm, 
							 rounded corners =10pt,
							 text centered, 
							 draw=black]
\tikzstyle{decision}  = 	[diamond, 
			                 minimum width=2cm, 
			                 minimum height=1cm, 
			                 text centered, 
			                 draw=black]
\tikzstyle{line} = [draw, -latex']

\definecolor{purp}{HTML}{6C42AA}
\definecolor{brn}{HTML}{939125}
%=====================================================================
%	Start of the document
%---------------------------------------------------------------------
\begin{document}
\setcounter{MaxMatrixCols}{20}
%=====================================================================
%	TITLE PAGE
%=====================================================================
\begin{titlepage}
	\centering
	\vspace*{1cm}

%---------------------------------------------------------------------
% Title
%---------------------------------------------------------------------	
	\begin{Huge}					
		\bfseries\titl\par
		\vskip 5mm
		%\hrule
		%\vskip 1.2cm
	\end{Huge}
%---------------------------------------------------------------------
% Subtitle	(optional)
%---------------------------------------------------------------------
	\begin{Large}					
		\subtitle\\*
		\vskip 3cm
	\end{Large}
%---------------------------------------------------------------------
% UCT logo
%---------------------------------------------------------------------	
	\logo							
	\vskip 1.5cm	
%---------------------------------------------------------------------
% Author	
%---------------------------------------------------------------------	
	\begin{Large}					
		\bfseries\auth\\
	\end{Large}
%---------------------------------------------------------------------
% Author's address	
%---------------------------------------------------------------------		
	\begin{normalsize}				
		\vskip 2mm	
		\dept\\*
		\uni\\*
		\city\\*
		\country\\*
		
		\vskip 15mm	
	\end{normalsize}
%---------------------------------------------------------------------
% Supervisor (optional)	
%---------------------------------------------------------------------		
%	\begin{large}					
%		{\itshape Supervisor: \\*}
%		\supervisor\\
%		\vskip 5mm
%	\end{large}
%---------------------------------------------------------------------
% Date	
%---------------------------------------------------------------------
	\begin{Large}					
		{\bfseries \mydate\today}
		\vskip 15mm	
	\end{Large}
%---------------------------------------------------------------------
% Degree	
%---------------------------------------------------------------------		
		\degreeabv\ thesis submitted in fulfillment of the requirements for the degree of \degre\ in the \dept\ at the \uni
		
		\vskip 15mm	
%---------------------------------------------------------------------
% Keywords	
%---------------------------------------------------------------------		
	\begin{normalsize}				
		{\itshape Keywords:}
		\key			
	\end{normalsize}	
\end{titlepage}
\afterpage{\null\newpage}
%=====================================================================
%	FRONT MATTER
%=====================================================================
\frontmatter
\begin{center}
\begin{minipage}{0.7\textwidth}
\centering
``\emph{We're gonna have a superconductor turned up full blast and pointed at you for the duration of this next test. I'll be honest, we're throwing science at the wall here to see what sticks. No idea what it'll do. Probably nothing. Best-case scenario, you might get some superpowers\ldots}"
\\
Cave Johnson -Founder \& CEO of Aperture Science
\end{minipage}
\end{center}
%---------------------------------------------------------------------
%	Declaration
%---------------------------------------------------------------------
\chapter{Declaration}				
\label{ch:decl}
% Change name
I, \auth,  hereby:

\begin{enumerate}
	\item		grant the \uni\  free license to reproduce the above thesis in whole or in part, for the purpose of research only;
	\item		declare that:
	
	\begin{enumerate}
		\item		This thesis is my own unaided work, both in concept and execution, and apart from the normal guidance from my supervisor, I have received no assistance except as stated below:
				% Note any exceptions
		\item		Neither the substance nor any part of the above thesis has been submitted in the past, or is being, or is to be submitted for a degree at this University or at any other university, except as stated below.
		\item		Unless otherwise stated or cited, any and all illustrations or diagrams demonstrated in this work are my own productions.
				% Note any exceptions
		\item 		All the content used to compile this report and complete the investigation revolving around the whole project is collectively hosted on the following GIT repositories:
		\begin{itemize}
			\item \LaTeX report: \url{https://github.com/nickvonklemp/Masters-Report}
			\item STM32F303 projects: \url{https://github.com/nickvonklemp/Code}
			\item Hardware Schematics: \url{https://github.com/nickvonklemp/visio} \&
			\item EagleCad Schematics \url{https://github.com/nickvonklemp/Eagle}
			\item MatLab Simulink Code: \url{https://github.com/nickvonklemp/Simulink}
			\item Results \& Simulation Data: \url{https://github.com/nickvonklemp/results}
			\item All CAD design files \& assemblies: \url{https://grabcad.com/nick.vk-1}
		\end{itemize}
	\end{enumerate}	
\end{enumerate}

\begin{flushright}	% Signature field
	\vskip 4cm
	\noindent \signature{x}
	\noindent \auth \\*
	\vskip 2mm
	\noindent \dept \\*
	\noindent \uni \\*
	\vskip 2mm
	\noindent \today
\end{flushright}
%---------------------------------------------------------------------
%	Abstract
%---------------------------------------------------------------------
\chapter{Abstract}				
\label{ch:abs}
%---------------------------------------------------------------------
\begin{center}
	\textbf{\Large \titl}\\
			\vskip 0.2cm
			\auth\\
			\vskip 0.2cm
	\textit{\footnotesize\today}
			\vskip 1cm
\end{center}
%---------------------------------------------------------------------
This dissertation aims to apply non-zero attitude and position setpoint tracking to a quadrotor aircraft; achieved by solving the problem of a quadrotor's inherent under actuation. The introduction of extra actuation intends to mechanically accommodate for stable tracking of non-zero state trajectories. The requirement of the project is then to design, model, simulate and control a novel quadrotor platform which can articulate all six degrees of rotational and translational freedom (6-\emph{DOF}) by redirecting each propeller's individual resultant thrust vector. 
\par
Considering the extended articulation, the proposal is to add an additional two axes (degrees) of actuation to each propeller on a traditional quadrotor helicopter. Each lift propeller can be independently pitched and rolled relative to the body frame. Such an adaptation, to what is an otherwise well understood aircraft, produces an over-actuated control problem. Being first and foremost a control engineering project, the focus of this work is plant model identification and control solution of the proposed aircraft design. A higher level setpoint tracking control loop designs a generalized plant input (net forces and torques) to act on the vehicle. An allocation rule then distributes that \emph{virtual} input in solving for explicit actuator servo positions and rotational propeller speeds.
\par
The dissertation is structured as follows; first a schedule of relevant existing works is reviewed in Chapter:\ref{ch:intro} following an introduction to the project. Thereafter the prototype's design is detailed in Chapter:\ref{ch:proto}; only the final outcome of the design stage is presented. Following that, kinematics associated with generalized rigid body motion are derived in Chapter:\ref{ch:dynamics} and subsequently expanded to incorporate any aerodynamic and multibody non-linearities which may arise as a result of the aircraft's configuration (changes). Higher level state tracking control design is applied in Chapter:\ref{ch:control} whilst lower level control allocation rules are then proposed in Chapter:\ref{ch:allocation}. Next a comprehensive simulation is constructed in Chapter:\ref{ch:simulation}; built upon the plant dynamics derived in order to test and compare the proposed controller techniques. Finally a conclusion on the design(s) proposed and results achieved is presented in Chapter:\ref{ch:conclusion}\ldots
\par
Throughout the research, physical tests and simulations are used to corroborate proposed models or theorems. Final flight tests of the platform remain open to further investigation. The subsequent embedded systems design stemming from the proposed control plant, however, is outlined in the latter of Chapter:\ref{ch:proto}, Sec:\ref{sec:proto.layout}. Implementations of which are not investigated here but design proposals are suggested. The primary outcome of the investigation is ascertaining the practicality and feasibility for such a design, most importantly whether the complexity of the mechanical design is an acceptable compromise for the additional degrees of control actuation introduced. Control derivations and the prototype design presented here are by no means optimal nor the most exhaustive solutions, focus is placed on the system and not just a single aspect of it.
\par
%---------------------------------------------------------------------
%	Acknowledgements
%---------------------------------------------------------------------
\chapter{Acknowledgements}		
\label{ch:ack}
%---------------------------------------------------------------------



%---------------------------------------------------------------------
%	Nomenclature
%---------------------------------------------------------------------
\chapter{Nomenclature}
\label{ch:nom}
%---------------------------------------------------------------------
\underline{In order of appearance:}
\par
DOF - Degree of Freedom(s)\\
$\mu$C - micro-controller\\
UAV - Unmanned aerial vehicle\\
SISO - Single input single output, control loop\\
MEMS - Micro-electromechanical system\\
DIY - Do it yourself\\
VTOL - Vertical takeoff/landing\\
IMU - Inertial measurement unit\\
BLDC - Brushless direct current, motor type\\
KV - Kilo-volt, BLDC motor rating\\
$\mu$C - Micro-controller shorthand\\
PWM - Pulse width modulation\\
CH - Channel, radio control \& PWM signals typically\\
RC - Radio control\\
OAT - Opposed active tilting\\
dOAT - Dual axis opposed active tilting\\
PD - Proportional derivative, control law\\
PID - Proportional integral derivative, control law\\
IBC - Ideal backstepping control\\
ABC - Adaptive backstepping control\\
PSO - Particle swarm optimization, gradient free genetic algorithm\\
BEM - Blade element theory\\
ESC - Electronic speed controller\\
MPC - Model predictive control\\
LQR - Linear quadratic regulator\\
LCF - Lyupanov candidate function\\
ITAE - Integral time additive error\\
TSK - Takagi-Sugeno-kang\\
I/O - Input/Output\\
RPM - Revolution Per Minute\\
RPS - Revolution Per Second\\
W.R.T - With respect to\\
LCF - Lyupanov Candidate Function\\
\emph{iff} - If and only if\\
P.D - Positive definite, NOT proportional derivative\\
S.T - such that\\
FTC - Fault Tolerant Control

%---------------------------------------------------------------------
%	Symbols
%---------------------------------------------------------------------
\chapter{Symbols}
\label{ch:symbol}
%---------------------------------------------------------------------
Propeller Rotational Speed: $\Omega_i~~[rpm]~~\text{for motors:}~~i\in\begin{bmatrix}
1,&2,&3,&4
\end{bmatrix}$
\\
\emph{\color{Gray} Rotational speeds in [RPS] are used for Blade Element Theory Calculations in Chapter:\ref{ch:dynamics}}
\\
Net body torque: $\mu\vec{\tau}=\begin{bmatrix}
\tau_\phi & \tau_\theta & \tau_\psi
\end{bmatrix}\text{}^T~~~~\in\mathcal{F}^b$
\\
Net body thrust: $\mu\vec{T}=\begin{bmatrix}
T_x & T_y & T_z
\end{bmatrix}\text{}^T~~~~\in\mathcal{F}^b$
\\
Body Position: $\vec{\mathcal{E}}=\begin{bmatrix}
x & y & z
\end{bmatrix}^T~~~~\in\mathcal{F}^I$
\\
Euler Angles: $\vec{\mathcal{E}}=\begin{bmatrix}
\phi & \theta & \psi
\end{bmatrix}^T~~~~\in\mathcal{F}^{I,v1,b}$
\\
Servo 1 Position: $\lambda_i~~[rad]$
\\
Servo 2 Position: $\alpha_i~~[rad]$
\\
Motor module actuator positions: $\begin{bmatrix}
\Omega_i & \lambda_i & \alpha_i
\end{bmatrix}^T~~~~\in\mathcal{F}^{M_i}$
\\
Actuator matrix: $u=\begin{bmatrix}
M_1 & \ldots & M_4
\end{bmatrix}~~~~\in\mathbb{U}^{12}$
\\
Motor module displacement arm: $\vec{L}_{arm}=195.16~~[mm]$
\\
Euler Rates: $\frac{d}{dt}\vec{\eta}=\dot{\vec{\eta}}=\Phi(\eta)\dot{\omega}_b=\begin{bmatrix}
\dot{\phi} & \dot{\theta} & \dot{\psi}
\end{bmatrix}^T~~~~\in\mathcal{F}^{v1,v2,I}$
\\
Angular Velocity: $\omega=\begin{bmatrix}
p & q & r\\
\end{bmatrix}^T~\in\mathcal{F}^b$
\\
Linear Velocity: $\nu=\begin{bmatrix}
u & v & w
\end{bmatrix}^T~\in\mathcal{F}^b$
%---------------------------------------------------------------------
%	Dedication (optional)
%---------------------------------------------------------------------
%\begin{flushright}
%	\vspace*{10cm}
%	{\itshape dedication .....}
%\end{flushright}
%---------------------------------------------------------------------
%	Table of Contents
%---------------------------------------------------------------------
\tableofcontents

%---------------------------------------------------------------------
%	List of Figures
%---------------------------------------------------------------------
\listoffigures

%---------------------------------------------------------------------
%	List of Tables
%---------------------------------------------------------------------
\listoftables

%---------------------------------------------------------------------
%	List of Symbols/Document conventions (optional)
%---------------------------------------------------------------------

%---------------------------------------------------------------------
%	Glossary (optional)
%---------------------------------------------------------------------

%=====================================================================
%	MAIN MATTER
%=====================================================================
% Create a separate file for each chapter. You may chose to rename these
% chapters as needed.
%---------------------------------------------------------------------
\mainmatter
%****************************************************
%	CHAPTER 1 - INTRODUCTION
%****************************************************
\chapter{Introduction}
\label{ch:intro}
%====================================================
\section{Foreword}
\label{sec:intro.foreword}
%====================================================
\subsection{A Brief Background to the Study}
\label{subsec:intro.foreword.background}
%====================================================
A popular topic for current control and automation research is that of quadrotor UAVs. Attitude control of a quadrotor poses a unique 6-DOF control problem, to be solved with an under-actuated 4-DOF system. As a result the $\phi$ pitch and $\theta$ roll plants aren't directly controllable. The attitude plant is often simplified around a stable operating point. The trimmed operating region is always at the inertial frame's origin; resulting in a zero-set point tracking problem. The highly coupled non-linear dynamics of a rigid body's transnational and angular motions arise from gyroscopic torques [Section:~\ref{subsec:dynamics.nonlinearities.gyrotorques}] and Coriolis accelerations [Section:~\ref{subsec:dynamics.nonlinearities.coriolis}]. These effects are negligible around the origin\footnote{Expanded upon in Appendix:\ref{app:stddynamics}}, hence the origin trim point removes the system's nonlinearities. The control system can then reduce each state variable, $\vec{X}_b=\big[\phi~\theta~\psi~x~y~z\big]^T$, to individual SISO plants.
\par
As almost every recent quadrotor research paper mentions, the late interest in the platform is from recent emergences in availability of MEMS systems and low-cost microprocessors. Such advancements accomodate onboard state estimation and control algorithm processing in real time. Developmental progress in quadrotors and, to a lesser extent, UAVs in general has led to rapidly growing enthusiast communities. HobbyKing\cite{hobbyking} is now synonymous with providing custom DIY hobbyist quadrotor kits, not just prebuilt commercial products like the DJI Phantom\cite{phantom}.
\par
The avenue for potential application of both fixed wing and VTOL UAVs is expansive, supporting civil\cite{civilquadcopter}, agricultural\cite{agriculturequadcopter} and security\cite{videosurveillancequadcopter} industries. The quadrotor platform provides a mechanically simple platform on which to test advanced aerospace control algorithms. Commercial drone use in industry is already emerging as a prolific sector; especially in Southern Africa. Subsequently following the $8^{th}$ amendment of civil aviation laws \cite{dronelaw}, commercial use of UAVs is now both legal and regulated. Research into any non-trivial aspect of the field will therefore be to extremely valuable to the field as a whole. 
\par
Large scale quadrotor, hexrotor and even octorotor UAVs are popular intermediate choices for aerial cinematography due to their high payload capacity.  The cost of a commercial drone like the SteadiDrone Maverik \cite{steadidrone} is far less than a chartered helicopter used for the same panoramic aerial scenes or on-site inspections. One foreseeable issue which may hinder commercial drone progress in the agricultural and civil sectors is the consequential inertial effects from scaling up the aerospace bodies. When scaling up any vehicle, its performance is adversely affected if actuation rates aren't proportionately scaled.
%====================================================
\subsection{Research Questions \& Hypotheses}
\label{subsec:intro.foreword.hypotheses}
%====================================================
The difficulty with quadrotor control is that fundamentally it's unstable and under-actuated, \emph{empirically proven later with Layupanov Theorem in Chapter:\ref{ch:control}}. A quadrotor only has four controllable inputs, namely propeller rotational speeds, $\Omega_{1,2,3,4}$, which are then abstracted\footnote{The abstraction of which is explored in Appendix:\ref{app:stddynamics}} to virtual control inputs net torque, $\vec{\tau}_{net}=[\tau_{\phi}~\tau_{\theta}~\tau_{\psi}]^T$, and a perpendicular heave thrust $\vec{T}_{net}=\sum_{i=1}^{4}~T(\Omega_i)$. Those four inputs have to affect both the linear X-Y-Z positions, $P=[x~y~z]^T$, and angular pitch, roll and yaw rotations, $\mathcal{E}=[\phi~\theta~\psi]^T$. Pitch and roll torques, $\tau_{\phi}$ \& $\tau_{\theta}$, are induced from differentials thrusts of each opposing propeller. Yaw torque, $\tau_{\psi}$, is dependent on net aerodynamic torque about the rotational axes of each propeller (See Section:\ref{subsec:dynamics.aero.bem}). Aerodynamic responses are non-linear and fluctuating sources of control torques and as such the body's yaw control is depreciated. A result of the under-actuation is that the attitude control problem then becomes a zero set point problem, any other attempt to track attitude cannot be achieved.
\par
The aim of this project is to implement quadrotor attitude and position set point tracking by solving the problem of its inherent under-actuation. Inspired by Boeing/Bell Helicopter's V22 Osprey and the tilting articulation of its propellers, the prototype design proposed here introduces two additional actuators for each of the quadrotor's lift propellers. Specifically, adding rotations about the X and Y axes for each motor/propeller pair. The result is a vectored 3 dimensional thrust force rather than a bound perpendicular heave thrust. The control problem is then posed as the design of net forces, $\vec{F}_{net} = [F_x\;F_y\;F_z]^T$, and torques, $\vec{\tau}_{net} = [\tau_{\phi}~\tau_{\theta}~\tau_{\psi}]^T$, for a general 6-DOF body such that for any given trajectory, $X_d=[x~y~z~\psi~\theta~\phi]^T$, the error state $X_e = X_d - X_b$ asymptotically tends to $\vec{0}$.
\begin{equation} \label{eq:trajectoryerror}
lim_{t \rightarrow \infty} X_e = \vec{0}\hspace{10pt}\forall X \in \mathbb{R}^n
\end{equation}
Where $n$ is the degrees of freedom. The over-actuation brings about the need for a control allocation scheme which distributes the 6 commanded system inputs (net torques and forces) among the actuator set (12 actuators) in order to optimize some objective function secondary to that of Eq:\ref{eq:trajectoryerror}.
\par
Part of the control research question is the multivariable treatment of the system, making no assumptions or simplifications to the non-linear dynamics involved in the quadrotors motion and its operational conditions. Standard linearizations applied to the quadrotor's control plant won't hold true for the more aggressive manoeuvres; they're dependent on small angle approximations and negligible 2$^nd$ order effects. Stable control law design will need to expand and simulate the existing kinematic model of an aerial body and apply it to a quadrotor's motion. Following this there must be design, development and control of the new actuator suite which is to be implemented on a quadrotor platform. Final key outcomes for the project are the simulation analysis and prototype construction for the proposed design and the conclusion drawn thereon.
\par
Introducing relative motion within an unconstrained body will produce a lot of unwanted dynamics like inertial and gyroscopic responses, amongst others. A rotating propeller will respond to pitching much like a Control Moment Gyroscope \cite{cmg} or a flywheel and produce a precipitating torque. A less trivial aspect to consider is the aerodynamic torque produced from the propeller's aerofoil profile. Such induced responses occur in planes perpendicular to whatever the propeller's rotation exists in. These aspects are normally compensated for due to a quadrotor's fundamental co-planar propeller rotation. It's anticipated that a plant dependent control solution will have to compensate for these dynamics, which if left unaccounted for could potentially cause instability.
%====================================================
\subsection{Significance of Study}
\label{subsec:intro.foreword.significance}
%====================================================
Due to the huge popularity of quadrotor platforms as research tools, any work that improves the UAV \& quadrotor general body of knowledge will prove to be valuable. With that being said, there is already a vast amount of existing research on linear and non-linear control techniques for regular quadrotor platforms. The attitude loop is the most common topic for control research, requiring an under-actuated solution and mostly linearized around the origin (See Appendix:\ref{app:stddynamics}). Far less common is the application of optimal flight path and trajectory planning to quadrotor control. The uniqueness and difficulty of the quadrotor attitude control does not hold true for its position control, so standard techniques can be used for way point planning and the like once the attitude control problem has been solved.
\par
The most significant aspect of this project is the attitude control, discussed later in Section:\ref{sec:control.attitude}. The over-actuation of the proposed design and, more critically, the manner in which the controller's (virtual) output is distributed among those control effectors would appear to be the first of its kind. Otherwise known as control allocation, the requirements of the distribution algorithm(s) are outlined in Section:\ref{sec:control.allocation}. Dynamic set point attitude control for aerospace bodies is not a subject heavily researched outside the field of satellite attitude control. Even papers which propose similarly complex mechanical over-actuation (expanded upon in next in the literature review, Section:\ref{sec:intro.litreview}) hardly broach the topic of tracking attitude set points away from the origin.
\par
Whilst the control plant (developed in Chapter:\ref{ch:control}) does indeed close both the position and attitude control loops, there is no consideration of trajectory generation nor flight path planning. Such topics are well discussed elsewhere in a far more concise and deliberate way than this project could ever hope to achieve. Once closed loop position and attitude control has been achieved, the control algorithms can be adjusted to account for higher order state derivative (acceleration, jerk and jounce) tracking needed for nodal way point planning. The heuristics involved with flight path planning are well documented and their implementation is an academic task.
\par
Where possible the system identification and control (design and allocation) for this project is kept both modular and generically applicable. The intention here is that its pertinence falls not only within the UAV field but to any aerospace or free body attitude control. Hopefully this investigation can be expanded upon with more in-depth research on one of the subsystems without compromising the stability of the remainder of the whole plant.
\par
Provisionally, an obvious outcome which the investigation could yield is improved yaw control of a quadcopter's attitude. However, if the express purpose was just to improve yaw control, it could be done with a dramatically more simple design. Furthermore, the project could provide greater insight into high bandwidth actuation and thus a faster control response for larger aerospace bodies. Any standard quadrotor uses differential thrust to develop a torque about its body. Such actuation suffers a second order inertial response when the propellers accelerate or decelerate, $\tau_{simplified}=\mathbb{I}_f~\dot{\omega}_{i}$. Prioritizing pitching the propeller's principle axis of rotation rather than changes to the propeller's speed could potentially improve the virtual control response. This is entirely dependent on how the allocator block is prioritized (presented in Section:\ref{sec:control.allocation}).
%====================================================
\subsection{Scope and Limitations}
\label{subsec:intro.foreword.scopeandlim}
%====================================================
\subsubsection{Scope}
\label{subsubsec:intro.foreword.scope}
%====================================================
Critical to this project is the conceptualized design and prototyping of a novel actuation suite to be used on a quadrotor platform. The express purpose of which is to apply set point attitude tracking control to the body. Stemming from this is an investigation into the kinematics that are potentially influenced by the design and the structure's relative motion. In order to apply correct control theory to achieve the attitude tracking on a physical prototype, the plant dynamics must first be identified for input responses to be approximated with confidence. Aspects of the mechanical design are covered next in Section:\ref{sec:proto.design} but, beyond the cursory investigation, there is no scope for materials analysis or stress testing of the design. To the detriment of the project, the design will either produce an over-engineered or catastrophically under-engineered solution. The scope focuses mainly on the control application and embedded systems design, not the structural integrity of a proposed frame given the forces it may undergo. Physical measurements are only made for critical kinematics, such as inertial measurements for the second order gyroscopic and inertial dynamic responses.
\par
As mentioned in the antecedent Section: \ref{subsec:intro.foreword.significance}, trajectory \& flight path planning are not ubiquitous with this dissertation. Derivations for the differential equations which dictate a 6-DOF body's movement are wholly applicable to any dynamic (rigid or otherwise) aerospace body, although some particular standards are used [sic Z-Y-X Euler Aerospace Sequence, Section:\ref{sec:proto.conventions}]. Similarly the control plant is stabilized with non-linear state space control techniques, aided and justified by Lyupanov theorem. Alternative solutions through Model Predictive Control or Quantitative Feedback Theory could provide more refined or effective controllers, they aren't presented and remain open to further investigation. Quadrotor attitude control is commonly stabilized with feedback linearizations, decoupling plant around a trim point so that SISO techniques can be applied. A derivation of such a linearization is included in Appendix:\ref{app:stddynamics} but beyond that there are no further discussions. Any comparison between non-zero and zero-set point attitude control of quadrotor is difficult as the fundamental objectives are in stark contrast with one another.
\par
Arguably the most important and indeed novel aspect of this project is the control allocation. The system has 12 plant inputs and 6 output variables to be controlled. There is then a family of actuator set $u\in\mathbb{U}$ solutions that exist for each commanded input. Such a plant is classified as over-actuated. Ergo, there must be some logical process as to how those 12 inputs are articulated to achieve the desired 6 movements. Appropriate techniques are first investigated in Section:\ref{sec:control.allocation} and compared before a final solution is implemented in Section:\ref{sec:simulation.comparison}. It is by no means a comprehensive investigation of every possible allocation scheme but rather an analysis of the sub-set of problems and design of what is regarded as a logical and pertinent approach.
\par
With regards to the actual prototype design, in Section \ref{sec:proto.design}, it's assumed that certain aspects are a given certainty. Particularly the state estimation, updated through a 4-camera positioning system fused with a 6-axis IMU through Kalman Filtering, is assumed to precise and readily disposable at a consistent 50 Hz. Hence state estimation is included but is bereft of intricate detail, this is another topic which remains open to further investigation.
%====================================================
\subsubsection{Limitations}
\label{subsubsec:intro.foreword.limits}
%====================================================
The biggest constraint faced by the design is the net weight of the assembled frame. Lift forces required to keep the body aloft are obviously dependent on the all up weight. Conventional wisdom has it that steady state actuator rates ought to be far less that saturation conditions. For stability to be guaranteed at all feasible operating conditions, the actuators must have sufficient headroom to still effect the desired control inputs. Conversely the structure's net weight is mostly dependent on the lift motors, often being the heaviest part of the vehicle (\emph{batteries too}). A trade-off between net weight and actuator efficacy makes designing the prototype a balancing act of compromise; added actuation is needed to produce the desired thrust vectoring. That added actuation is going to increase the weight which then requires more thrust force to ensure the vehicle remains airborne. Larger motors then need stronger actuators to effect the relative motion and overcome the bodies inertial response. It's a compromise between the weight of the body and the strength/quality of the actuation.
\par
To forego the deliberation detailed above, reducing the possibility of unbounded scope creep, a limitation is self-imposed on the prototype design. Restricting the propeller diameter, and hence maximum thrust, will provide a constraint upon which all other design considerations must conform. Smaller propellers require a far greater rotational to produce a similar level of thrust as their larger diameter counterparts. Electing to use 3 blade $6X4.5$ inch small diameter propellers is going to reduce the overall dimensions of the prototype, but as a consequence will require very high RPM motor. Specifically a set of four Cobra-2208/2000KV\cite{cobramotor} Brushless DC motors are be used for lift actuation. A direct consequence of this decision is that, provisionally based upon test data\footnote{Official test data from\cite{cobramotor} included in Appendix:\ref{app:cobra-test} and tested independently in Section:\ref{subsec:dynamics.aero.bem}}, the net thrust disposable for actuation is limited to around 950g, $\approx$ 9 N, per motor (see Section:\ref{subsec:dynamics.aero.bem}). It's critical to ensure the control block doesn't induce over-saturation of the motor actuation, so the frame weight needs to be around 50-60\% of the maximum available thrust, or roughly 2 Kg. Saturation conditions are detailed later in Section: \ref{sec:control.allocation}.
\par
Another aspect of limitations produced by design decisions made, mostly to reduce prototype costs and weight, is to use of 180\textdegree ~rotation servo motors. The servos are for individual motor's $\vec{X}_{M_i}$ and $\vec{Y}_{M_i}$ axial pitch and  roll actuations respectively. The servos act in lieu of either continuous BLDC or stepper motors. Any non-servo rotations beyond 360\textdegree~ will require closed loop position control and, unlike servos, would need slip rings to transmit power throughout rotational movement. However the logistics of implementing such a design whilst maintaining an acceptable weight is almost impossible. Such an implementation is going to dramatically scale up the size of the prototype to accommodate for weight increases. Commercial camera stabilizing gimbals already make use of similar configurations but the I/O requirements from the flight controller $\mu$C already constricts the amount of expansion available.
\par
\begin{figure}[htbp]
\begin{subfigure}{0.5\textwidth}
\centering
\includegraphics[width=\textwidth]{figs/cobra-motor}
\caption{Cobra CM2208/2000KV BLDC Motor}
\end{subfigure}
\begin{subfigure}{0.5\textwidth}
\centering
\includegraphics[width=\textwidth]{figs/corona-servo}
\caption{Corona DS-339MG Digital Servo}
\end{subfigure}
\caption{}
\end{figure}
Discrete elements for the whole system can potentially limit performance but are going to be mitigated if possible. For example analogue servos have an associated $1 ms$ deadband from their $20 Hz$ refresh rate. That can be addressed by using faster, albeit more expensive, digital servos which samples at $330 Hz$. The prototype's flight controller has to provide 12 PWM output compare channels for the 8 servos and 4 BLDC speed controllers. State updates from a ground control station and a fail safe 6Ch RC receiver module also needs to be processed by the $\mu$controller system. Particular attention is paid to the embedded system layout in Section:\ref{sec:proto.layout}.
%====================================================
\newpage
%====================================================
\section{Literature Review}
\label{sec:intro.litreview}
%====================================================
\subsection{Existing \& Related Work}
\label{subsec:intro.lit.related}
%====================================================
The field of transformable aerospace frames is not necessarily a new one, with many commercial examples having seen a lot of success over their operational life span. The most notable tilting-rotor vehicle is that of the Boeing/Bell V22 Osprey\cite{} aircraft. First introduced in the field in 2007, the Osprey has the ability to pitch its two lift propellers forward to aid translational flight after vertically taking off or landing. In addition to this there have been a few papers published on similar tilting bi-rotor UAVs for research purposes.
\subsubsection*{Birotors}
\begin{figure}[hbtp]
\centering
\includegraphics[width=0.8\textwidth]{figs/dualaxistilt}
\caption{General Structure for Opposed Tilting Platform}
\label{fig:dualaxistilt}
\end{figure}
Research into birotor vehicles (Fig:\ref{fig:dualaxistilt})\footnote{Image from G. Gress:\cite{gres2007}} with ancilliary lift propeller actuation is oft termed Opposed Active Tilting or \emph{OAT}. Such a rotorcraft's mechanical design applies either a single \emph{oblique} 45\textdegree ~tilting axis relative to the body; \cite{smalltwotilting,obliquepitch,passiveobliquetilting}, or a \emph{lateral} tilting axis, adjacent to the body; \cite{tiltrotorUAV,adaptivebackstep,tiltrotorcontrol,tpheonix}. Leading research is currently focussed on applying doubly actuated tilting axes to birotor UAVs. Dual axis Opposed Active Tilting or \emph{dOAT} introduces vectored thrust with propeller pitch and roll motions to further expand the actuation suite, \cite{gres2007,opposedlateraldualaxis}. A birotor is sometimes considered preferable to the multirotor platform due to its reduced controller effort. However the controller plant abstraction often detracts from the quality and effectiveness of its stability solution as a result of the birotor's underactuation. 
\par
Birotor attitude control typically incorporates plant independent PD \cite{obliquepitch} and PID \cite{tiltrotorUAV} controller schemes. Occasionally more computationally exhaustive and plant dependent Ideal and Adaptive backstepping controllers (\emph{IBC} or \emph{ABC}) are exploited, presented in \cite{smalltwotilting,tpheonix} and \cite{adaptivebackstep} respectively. The cross-coupling of a birotor vehicle's attitude system is more pronounced than that of a quadrotor, derived in Section:\ref{sec:dynamics.nonlinearities}, and so feedback linearisation is almost always used. In an interesting progression from the norm, Lee et al,  \cite{autopilotPSO}, proposed a PID co-efficient selection algorithm for a bi-rotor control block. Using a Particle Swarm Optimization techinque, similar to \cite{adaptivepso}, the coefficients were globally optimized around a given performance metric. However their performance criterion is a basic ITAE$^\dagger$ term and nothing more appropriate involving effects unique to flight systems. \emph{PSO} algorithms iteratively search for a globally optimized solution and offer independent, derivative free optimization. Later on non-linear controller coefficient are also optimized here using a PSO algorithm, shown in Section:\ref{sec:simulation.tuning}.
\par
\subsubsection*{Quadrotors}
Expanding on multirotor vehicles, the quadrotor UAV is a popular and well researched platform due to its mechanical simplicity. What would appear to be one of the first quadrotor research implementations, in 2002, is the X4-Flyer quadrotor, \cite{x4flyer,x4flyercontrol}. Alternative iterations like the Microraptor\cite{microraptor} and STARMAC\cite{starmac} quadcopters have subsequently been built and tested. A plethora of literature exists around quadrotor kinematics \& control \cite{dynamicmodelling2013, dynamicmodelling2009, modelingquadcopter, quaddynamics, fullquadcoptercontrol}, however dedicated rigid body 6-DOF dynamic papers \cite{rigidbodylecture,eulerrigidbody} provide better explanations of the kinematics. Often the plant's dynamics are simplified around an origin trim point and assumed to reduce into 6 SISO plants for each degree of freedom (Appendix:\ref{app:stddynamics}). Lately research projects have begun to incorporate aerodynamic effects like drag and propeller BEM theory into the plant model\cite{lowreynolds,bem,starmac}. Although mostly negligible under standard opperating conditions, the higher fidelity models offer more precision without linearisations or assumptions,\cite{nonlineardynamics,starmac}.
\par
\begin{figure}[hbtp]
\centering
\begin{subfigure}{.5\textwidth}
\centering
\includegraphics[width=\textwidth]{figs/dji-inspire1}
\caption{Inspire 1 articulated upwards}
\label{fig:inspireup}
\end{subfigure}%
\begin{subfigure}{.5\textwidth}
\centering
\includegraphics[width=\textwidth]{figs/dji-inspire2}
\caption{Inspire 1 articulated downwards}
\label{fig:inspiredown}
\end{subfigure}
\caption{DJI Inspire 1}
\label{fig:inspire1}
\end{figure}
At the time of writing, the only commercial example of a transforming quadrotor is the DJI Inspire1 \cite{inspire}, made by Shenzen DJI Technologies (better known for the hugely successful DJI Phantom drone\cite{phantom}). The Inspire can articulate its supporting arms up and down as shown in Fig:\ref{fig:inspire1} \footnote{Both images were sourced from the drone's patent, held by SZ DJI Tech Co\cite{djinspire}}. The aim of such movements is to both alter the center of gravity and further expose a belly mounted camera gimbal for panoramic viewing angles. This transformation changes the moment of inertia about the body's center of gravity, in turn changing the inertial torque response induced by angular movements, an otherwise detrimental effect which makes researchers apprehensive of transformable aerospace frames. The range of transformations which the frame can undergo is limited to just articulating the arms up and down.
\par
In a similar fashion to the progression seen in birotor state-of-the-art, quadrotor research is engaging the topics of single and dual axis tilting articulations. First conceptualized and implemented on a prototype related to an ongoing project covered in two reports, \cite{tiltpropellercontrol,tiltpropellerflight}. The authors M. Ryll et al.(2012, 2013) modified and tested a QuadroXL four rotor helicopter, propduced by MikroKopter \cite{mikrokopter}, to actuate a single axis of tilting aligned with the frame's arms (Fig:\ref{fig:tiltpropellercontrol1})\footnote{Image sourced from Modelling and Control of a Quadrotor UAV with tilting propellers, \cite{tiltpropellercontrol}}. Their proposed control solution, discussed in detail next in Section:\ref{subsec:intro.lit.control}, assumes no nominal linearised conditions around hover flight, unlike a similar single axis tilting quadrotor prototype designed by Nemati, et al. (2012)\cite{singleaxistilting}. The latter remains  \underline{simulated} but as yet untested.
\begin{figure}[htbp]
\centering
\begin{subfigure}{.5\textwidth}
\includegraphics[width=\textwidth]{figs/tiltpropellercontrol1}
\caption{Single rotation axis aligned with the frames arm}
\label{fig:tiltpropellercontrol1}
\end{subfigure}%
\begin{subfigure}{.5\textwidth}
\includegraphics[width=\textwidth]{figs/napsholm-mech}
\caption{Cyclic-pitch \& swashplate mechanism}
\label{fig:tiltrotor-napsholm}
\end{subfigure}
\caption{}
\label{fig:tiltprop}
\end{figure}
\par
One approach to improving quadrotor flight response is to alter the manner in which the thrust is mechanically actuated, potentially improving the actuator bandwidth. Drawing from helicopter design, a project by Napsholm, (2013)\cite{napsholm}, purported a quadrotor UAV prototype that used swashplates for varying the propeller pitch and generating torque moments. The aim was a design which wasn't dependent on speed control (\emph{ESC}) power electronics to actuate variable thrust forces. Petrol motors were intended for use in place of BLDC motors. Furthermore, the design proposed a single axis of tilt actuation to each of the four motor modules. Whilst mechanically complex, Napsholm made use of existing RC helicopter components to design a rotor actuation bracket (Fig:\ref{fig:tiltrotor-napsholm}). The cyclic-pitch swashplate used \cite{autonomousrobotspitch} could apply torques, $\tau_{\phi}$ and $\tau_{\theta}$, about the propeller's hub, \emph{principle axis of rotation}, by altering the blades angle of attack throughout its rotational cycle. The actuation rate of such a configuration is far faster than that of a differential torque produced rolling/pitching motion.
\par
Irrespective of the strong initial design in the early stages of his project, it would appear that Napsholm's research suffered due to time constraints. The introductory derivation on aerodynamic effects and deliberation over the design provide clear insight into the projects goals. However the control solution and system architecture, electronic and software, are significantly lacking. An introductory proposal of an MPC attitude control system detracted from the comprehensive dynamics discussed. The project ended before testing, simulation and results could be obtained. Unfortunately, despite the novel over-actuated design, there was no discussion given on how the allocation, being the most unique aspect, would be performed.
\par
Finally, the most crucial research to mention is a project completed by Pau Segui Gasco \cite{tiltgasco}, which was a dual presented MSc project with Yazan Al-Rihani \cite{tiltrihani}. At the time of writing, this would appear to be the only project published pertaining to \emph{over-actuation} in aerospace bodies implemented on a quadrotor platform. The research was split between the two authors who completed the control/electronic design and the mechanical design for their respective MSc dissertations. Shown in Fig:\ref{fig:tiltrotor-gasco}\footnote{Image from Development of a Dual Axis Tilt Rotorcraft UAV: Modelling, Simulation and Control \cite{tiltgasco}}, the dual-axis articulation is achieved using an RC helicopter tail bracket and servo push-rod mechanism; reducing the mass of the articulated component but limiting the range of actuation. Considering the propellers as a spinning flywheel, the induced gyroscopic response can then be treated as an actuator plant. The commanded virtual control is then distributed by weighted inversion among the actuator set, Section: \ref{subsec:intro.lit.control}. The whole project justifies the extra actuation as redundancy but doesn't necessarily prove how such a redundancy could be beneficial.
\begin{figure}[htbp]
\centering
\includegraphics[width=0.7\textwidth]{figs/gasco-mech}
\caption{Dual-axis tilt-rotor mechanism}
\label{fig:tiltrotor-gasco}
\end{figure}
%====================================================
\subsection{Notable Quadrotor Control Implementations}
\label{subsec:intro.lit.control}
%====================================================
\subsubsection*{Quadcopter Attitude Control}
%====================================================
Attitude control of a 6-DOF body is best described by \emph{The Attitude Control Problem} \cite{attitudecontrolproblem}. A rigid body that currently has an attitude state\footnote{Quaternion attitude states will replace Euler angles} $\vec{\mathcal{E}}_s$ and a desired state $\vec{\mathcal{E}}_d$, the problem is to then find a torque control law:
\begin{equation} \label{eq:2}
\mu\tau = h(\vec{\mathcal{E}}_s,\vec{\mathcal{E}}_d,\dot{\vec{\mathcal{E}}}_s,\dot{\vec{\mathcal{E}}}_d)
\end{equation}
Such that both the angular position $lim~\vec{\mathcal{E}}_s \rightarrow \vec{\mathcal{E}}_d$ and that angular rates $lim~\dot{\vec{\mathcal{E}}}_s \rightarrow \dot{\vec{\mathcal{E}}}_d$ asymptotically stabilize as $t \rightarrow \infty$. A distinction must be made between angular rate vector, $\dot{\vec{\mathcal{E}}}=[\dot{\phi}~\dot{\theta}~\dot{\psi}]^T$ and the angular velocity vector $\vec{\omega}_b=[p~q~r]^T$. Depending on how the attitude is posed; with rotation matrices \cite{rigidbodylecture,eulerrigidbody,rotationsequences}, quaternions \cite{quaterniondynamics, rotationsequences, spacecraftattitutdequaternions,fullquaternion} or otherwise (Direct Cosine Matrix etc \ldots) the error sate\footnote{\emph{The Attitude Control} \cite{attitudecontrolproblem} describes these conventionally different error states} $\Delta\vec{\mathcal{E}}= \vec{\mathcal{E}}_d - \vec{\mathcal{E}}_s$ could then differ to a (hamilton) multiplicative relationship. Note that here $\vec{E}$ is not necessarily an Euler set but any attitude representative state variable. Simulation and modelling papers often rely on Euler angle based rotation matrices for attitude representation, \cite{adaptivedisturbancecontrol, optimizedpidquadcopter, singleaxistilting, backsteppingquadcoptercontrol, fullquadcoptercontrol} without addressing the inherent singularity associated with such an attitude representation (sic Gimbal Lock, \cite{euleranglesingularity}, Section:\ref{subsec:dynamics.rigidbody.singularity}). The alternative quaternion attitude representation, first implemented on a quadrotor UAV in 2006 \cite{attitudestabilization}, is often used in lieu of rotation matrices but has its own caveat of \emph{unwinding}, (Section:\ref{subsec:dynamics.rigidbody.unwinding}), as a result of quaternions dual-coverage \cite{unwinding} in $\mathbb{R}^3$ space.
\par
Quadrotor plant dynamics, as mentioned previously, are often simplified; especially when represented with a 3-variable Euler angle set, $\vec{\mathcal{E}} = [\phi ~\theta ~\psi]^T$. The coupled gyroscopic and Coriolis responses are both neglected when the angular velocity rate is small, $\vec{\omega}_b \approx 0$, and the inertial matrix is diagonal, $rk(\mathbb{I}_f)= x$ for $\in\mathbb{R}^x$. The consequence of which is the ineffectual deterioration of both the gyroscopic term, $\vec{\tau}_{gyro}=-\vec{\omega}_b \times \mathbb{I}_b\vec{\omega}_b \approx 0$ and the  Coriolis force term, $\vec{F}_{cor}=-\vec{\omega}_b \times \vec{a_b} \approx 0$ in the bodies dynamics~(Chapter:\ref{ch:dynamics} for context). Once the coupled cross-product terms are no longer of consequence, the 6 degrees of freedom, $[x ~y ~z ~\phi ~\theta ~\psi]^T$, can each be treated as an individual SISO plant controlled with an appropriate technique. Quaternion represented attitude plants cannot easily be decomposed into individual single-input-single-output systems (quaternion dynamics in Section:\ref{subsec:dynamics.rigidbody.quaternion}). So a quaternion (combined four variable attitude state vector) is then used, $Q_b = [q_0 ~\vec{q}\>]^T$ for the abstracted major loop plant.
\par
\begin{figure}[hbtp]
\centering
\includegraphics[width=0.8\textwidth]{figs/arducopter-pi}
\caption{ArduCopter PI Euler Angle Attitude Control loop}
\label{fig:arducopter-pi}
\end{figure}
Commercial flight controllers (Arducopter\cite{arducoptersite}, Openpilot\cite{openpilotsite}\footnote{\underline{NOTE:} OpenPilot's firmware stack is now maintained by LibrePilot}, BetaFlight\cite{betaflight}, etc \ldots) for custom fabricated UAV platforms all apply their own flavour of structured attitude controllers and state estimation algorithms, based on onboard hardware sensor fusion. The article \emph{Build Your Own Quadrotor}\cite{buildyourownquad} summarizes the control structures implemented on a range of common flight controllers. The most popular of which, ArduCopter, implements a feed-forward PI compensation controller (Fig:\ref{fig:arducopter-pi})\footnote{Image sourced from \emph{Build your own Quadrotor}\cite{buildyourownquad}}.  PI, PD and PID controllers are all easy and effective plant independent control solutions for general attitude plants. Table:\ref{tab:controllers} collectively lists the common attitude control blocks (not exclusively quadrotors UAVs but MAVs too) and which projects they've been implemented in, after which a critique on the more unique adaptations is given.
\begin{table}[h]
\centering
\begin{tabular}{ |c|l|l|c| }
\hline
Controller Type & Independent & Dependent & Total\\ \hline
PI & \cite{attitudecontrolproblem} & \cite{attitudecontrolproblem} & 2\\ \hline
PD & \cite{modelingquadcopter, tiltrihani} & \cite{fullquaternion,singleaxistilting} & 4\\ \hline
PID & \cite{optimizedpidquadcopter, attitudecontrolproblem, quaddynamics, tiltpropellercontrol, pidlqr} & \cite{attitudecontrolproblem, starmac, adaptivedisturbancecontrol} & 8\\ \hline
Lead & \cite{x4flyer, dynamicmodelling2009} & lead & 2\\ \hline
IBC & \cite{tpheonix, backsteppingquadcoptercontrol}\footnote{\cite{tpheonix} applied an IBC algorithm derived through Hurwitz polynomials, not lyupanov theorem.} & \cite{backsteppingquadcoptercontrol} & 3\\ \hline
ABC & \multicolumn{2}{l|}{\cite{adaptivebackstep, nonlinearadaptive, 6dofbackstep, intelligentbackstep}} & 4\\ \hline
LQR & \cite{pidlqr} & LQR & 1\\ \hline
\end{tabular}
\caption{A Breakdown of common Attitude Controllers}
\label{tab:controllers}
\end{table}
\par
In a collection of papers, written by Bouabdallah et al \ldots (2003,2004,2007) arguably the most prolific early quadrotor authors, a range of different control implementations are derived and reviewed. Their last paper (2007)\cite{fullquadcoptercontrol} derived and pratically tested an Integral Backstepping attitude controller on an OS4 quadrotor. It builds on their research from an earlier paper (2003)\cite{pidlqr} wherein an analysis of PID vs LQR attitude controllers in the context of quadrotors is posed. LQR controllers aim to optimize the controller effort (with $u\in\mathbb{U}$, controller effort is then $||u||$ or the $L_2$ norm of the plant input). Although, in theory, solving the assocaited Ricatti cost function may produce an optimial, stable and efficient control law it needs exact plant matching. In practice, exact plant matching is difficult to achieve for a quadcopter or any aerospace body for that matter. The resultant controller in \cite{pidlqr} achieved asymptotic stability but had poor steady state performance due to low confidence of the identified actuator dynamics and poor inertial measurements.
\par
Adaptive Backstepping Control\cite{backstepping}(any of the examples in Table:\ref{tab:controllers}) expands on nominal IBC fundamentals by introducing an added disturbance state term in the LCF used for the backstepping iteration. The drawback with this form of Backstepping approach is that, from the Lyupanov control theorem, a time derivative for the estimated disturbance (or an \emph{update law}) is needed. Disturbance approximation has been investigated thoroughly but, for a signal without \emph{apriori} information, some heuristic needs to be adopted with the approximation which usually involves compromise. In one example, \cite{nonlinearadaptive}, the authors implemented a statistical $proj(.)$ operator based technique. When used in adaptive control the projection operator \cite{outputfeedback}, $proj(.)$, ensures a derivative based estimator is bounded for adaptive regression approxmation \cite{nonlinearregression}.
\par
Although the control implementation isn't backstepping based, in \cite{adaptiveslidingmode}, a sliding mode controller was used to compensate for the disturbances in an Unmanned Submersible Vehicle atttiude plant. The underwater current disturbances are approximated using a fuzzy logic system, specifically a \emph{zero-order TSK} fuzzy controller. The TSK system has been proven to act in the same way as an Artificial Neural Network approximator\cite{zeroTSK}; where the TSK system is more comprehensible than the latter. Statistical analysis and investigation of approximators without \emph{apriori} knowledge of a system are well beyond the scope of what this project but are worth mentioning.
%====================================================
\subsubsection*{Single/Dual Axis Control \& Allocation}
%====================================================
The extra actuation introduced with single and dual axis articulation provides room for more control goals to be achieved as the order of actuation increases. Of the few papers published on tilting-axis quadrotors, PD controllers (Nemati et al.[2014]\cite{singleaxistilting} and again in Gasco \& Rihani \cite{tiltgasco,tiltrihani}) and PID controllers (Ryll et al.[2012,2013]\cite{tiltpropellercontrol,tiltpropellerflight}) are the norm for control blocks. For either of these systems there needs to be an allocation rule to distribute a commanded input amongst the actuator set. In \cite{allocation}, Johansen et al.[2012] describes the control allocation problem for a dynamic plant:\\
\emph{\color{Gray} Note in state space Equation:\ref{eq:3.1}, it's assumed the plant input, $\tau$, has a multiplicative relationship with the response, $g(x,t,\tau)\iff g(x,t)\tau$.}
\begin{subequations} 
\begin{equation} \label{eq:3.1}
\dot{x}=f(x,t)+g(x,t)\tau
\end{equation}
\vspace{-20pt}
\begin{equation} \label{eq:3.2}
y=l(x,t)
\end{equation}
\end{subequations}
With a state $x\in \mathbb{R}^n$ and $f(x,t)$ \& $g(x,t)$ being the plants' dynamics and input responses respectively. In set point tracking, the output is then \emph{tracking} the state $y = x$, and hence $y \in \mathbb{R}^n$. In an ideal well posed system the number of actuator inputs equals the number of controllable variable outputs; $dim(x)=dim(\tau)\in \mathbb{R}^n$. In the case where the input $\tau \in \mathbb{R}^m$, if $m>n$ the problem is then overactuated and a level of abstraction is needed; a virtual control input $\nu_d$ is designed by a control law $\nu_d=h(x_e,t)$ to affect dynamics. The goal is to then find a function that maps $\mathbb{R}^m \rightarrow \mathbb{R}^n$ for an actuator matrix $u \in \mathbb{U}^m$. An overactuated plant can be described as:
\begin{subequations}
\begin{equation} \label{eq:3.3}
\dot{x}=f(x,t)+g(x,t)\nu_d,~\nu_d \in \mathbb{R}^n
\end{equation}
\vspace{-20pt}
\begin{equation} \label{eq:3.4}
\nu_c=B(x,t,u)\approx B(x,t)u, ~u\in \mathbb{U}^m,~\nu_c\in\mathbb{R}^n
\end{equation}
\vspace{-20pt}
\begin{equation}
y=x
\end{equation}
\end{subequations}
$B(x,t,u)$ is the effectiveness function which formulates how the actuator inputs $u$ relate to the virtual commanded input $\nu_c$. $B(x,t,u)$ can be abstracted to a multiplicative relationship $B(x,t)u$ if the plants' dynamics permit it, such that; $B(x,t)\in\mathbb{R}^{n\times m}$. For setpoint tracking the control law will design a desired virtual control input $\nu_d$, the allocation rule then has to solve $u$ for $\nu_c$ such that a slack variable $s=\nu_c-\nu_d$ is minimized:
\begin{equation}\label{eq:quadraticallocator}
\underset{u \in \mathbb{R}^m ,s \in \mathbb{R}^n}{min}\norm{Q_s} ~\text{subject to} ~\nu_c - h(x_e,t)=\nu_c-\nu_d=s,~u \in \mathbb{U}
\end{equation}
Which ensures the commanded input $\nu_c$ tracks the desired control input $\nu_d$; $\nu_c\rightarrow\nu_d$ as per some cost function of the slack variable $Q_s$. Mostly the L2 norm, $\norm{Q_s}$, is used. In an overactuated system it then follows that there is a set of possible inputs for each $\nu_c$. A unique actuator solution (rather than a family solution set) to Eq:\ref{eq:quadraticallocator} needs a secondary objective function, $J(x,t,u)$. Eq:\ref{eq:quadraticallocator} then becomes;
\begin{equation} \label{eq:quadraticallocatorcost}
\underset{u \in \mathbb{R}^m ,s \in \mathbb{R}^n}{min}(\norm{Q_s}+J(x,t,u)) ~\text{subject to} ~\nu_c - h(x_e,t)=s, u \in \mathbb{U}
\end{equation}
\par
Over-actuation is not something often applied to quadrotors and as a result rather than providing a comprehensive literature review of associated papers here (which are all mostly theoretical derivation), the contextual application and solutions to the above posed problems are expanded later in Section: \ref{subsec:control.allocation.allocators}. The only overactuated quadrotor (birotor dual-axis tilting makes the system critically actuated and so requires no allocation) literature which covers allocation of the given actuators is \cite{tiltgasco,tiltrihani}, where the authors apply a weighted pseudo inverse (sic Moore Penrose Inverse \cite{moorepenrose}) allocation rule. A prerequisite for pseudo inversion is a multiplicative (\emph{linear}) control effectiveness relationship for Eq:\ref{eq:3.4}. 
\par
Segui et al. [2012] applied weighted inversion that relies on some very specific assumptions. For the net torque response, the authors assumed the extra actuators pitch and roll angular rates, $\dot{\phi}$~and~$\dot{\theta}$ respectively, were proportionally related as follows:
\begin{equation}
\dot{\phi}\approx\frac{\phi}{t_{rise}}
\end{equation}
In which $t_{rise}$ is the actuators rise time to a set-point. As a result the gyroscopic first order torque $\tau_{gyro}=-\omega\times\mathbb{I}_f\omega$ and second order inertial torque $\tau=\mathbb{I}\dot{\omega}$ are then functions of position $\phi$ or $\theta$ and not their derivatives. The extent of that consequence is contrasted with the allocation solution in Section:\ref{sec:control.allocation}.
%====================================================
\subsubsection*{Satellite Attitude Control}
%====================================================
Unconstrained attitude set-point tracking for 6-DOF bodies, quaternion based or otherwise, is a topic well covered in the field of satellite attitude control; \cite{axissymmetricspacecraft, satellitebackstepping,lpvbackstepping}. The \emph{status quo} for recent research is on non-linear adaptive attitude back-stepping control systems, wherein the adaptive update rule is the novel focus. Often plant uncertainty affects the inertia tensor of a satellite. In \cite{lpvbackstepping}, the authors Wang Jia, et al. [2010], proposed applying adaptive back-stepping to compensate for steady state (asymmetric) inertial estimation errors. Alternatively, instead of deliberating on costly non-orbital prelaunch inertial measurements, \cite{inertiaestimation} developed an algorithm for estimating the inertia tensor based on single axis controlled perturbations. However that does assume any initial estimates are sufficiently close to true body values such that they will settle and stability can be ensured.
\par
Satellite actuator suites mostly include additional redundant effectors, to ensure fault tollerance and reliability, and hence require control allocation. Often the extra allocators are CMG actuators, driven by DC motors, to produce rotational torques. Fuel burning can only actuate for a certain period of time and so thrusters are scheduled to have a lower priority. Seen in the paper \cite{satellitebackstepping}; the authors, Kristiansen et al. [2005], addresses the over-actuation with direct and well-matched inversion before applying quaternion based back-stepping for attitude control. A quadratic pseudo inverse solution to Eq:\ref{eq:quadraticallocatorcost} is:
\begin{subequations}\label{eq:pseudoinv}
\begin{equation}\label{eq:pseudoinva}
u=B^{\dagger}({\tau_a}^{b}-D{\omega_{ib}}^b)
\end{equation}
\vspace{-20pt}
\begin{equation}\label{eq:pseudoinvb}
B^\dagger=B^T(BB^T)^{-1}
\end{equation}
\end{subequations}
Where $B$ is the control effectiveness matrix and $B^{\dagger}$ is such that $BB^{\dagger}=\mathbb{I}$. Specifically $B^{\dagger}$ is the general \emph{pseudo} inverse of $B$ (more on inversions in Sec:\ref{sec:control.allocation}). It's assumed there's a multiplicative relationship between the input, $u\in\mathbb{U}$, and the input effectiveness matrix in Eq:\ref{eq:3.4}. The controller designed actuator torque ${\tau_a}^b$ then dictates the input $u$ as in Eq:\ref{eq:pseudoinva}. Much like the over-actuation previously discussed W.R.T quadcopters; the pseudo inversion method of control distribution applies quadratic optimization to the allocation slack cost function, Eq:\ref{eq:quadraticallocator}. 
%****************************************************
%	CHAPTER 2 - Prototype Design
%****************************************************
\chapter{Prototype Design}
\label{ch:proto}
%====================================================
\section{Conventions Used}
\label{sec:proto.conventions}
Here the conventions adopted for the later derived dynamic equations are discussed. Often these aspects relating to conventions are omitted or assumed commonplace. It's important to clearly and unambiguously define a standard set frames so that there can be no uncertainty later in the kinematics.
%====================================================
\subsection{Reference Frames Convention}
\label{subsec:proto.conventions.frames}
%====================================================
\begin{figure}[htbp]
\centering
\includegraphics[width=0.6\textwidth]{figs/reference_frame}
\caption{Inertial and Body Reference Frames}
\label{fig:ref_frame}
\end{figure}
Regular aerospace (Euler,\cite{rigidbodylecture}) frames are used for principle inertial and body directions. Shown in Fig: \ref{fig:ref_frame}, the inertial frame,~$\mathcal{F}^i$, is aligned such that the $\vec{X}_i$ axis is in the $\hat{N}$orth direction, $\vec{Y}_i$ is in the $\hat{E}$ast direction and $\vec{Z}_i$ is  in the $\hat{D}$ownward direction\footnote{In orbital sequences this will be toward the Earths' center.}. The body frame, $\mathcal{F}^b$, then has both $\vec{X}_b$ and $\vec{Y}_b$ aligned with two perpendicular arms of the quadrotors' body and finally the $\vec{Z}_b$ axis pointing in the body's normal direction. The relative angular displacement between the two frames is commonly defined by the three angle Euler set, $[\phi ~\theta ~\psi]^T$. An inherent singularity does exists with such attitude representations. Whilst Quaternions are used later in Sec: \ref{subsec:dynamics.rigidbody.quaternion} in lieu of Euler angles, they are easily understood and well suited to illustrate the distinction between rotation and transformation angles which needs to be made.
\par

\subsection{Motor Axis Layout}
\label{subsec:proto.conventions.motoraxis}
%****************************************************

%****************************************************
\section{Design}
\label{sec:proto.design}
%****************************************************
\subsection{Gimbal Articulation}
\label{subsec:proto.design.actuation}
%****************************************************
\subsection{Inertial Matrix Function}
\label{subsec:proto.design.inertia}
%****************************************************
\subsection{Overall Aspects}
\label{subsec:proto.design.aspects}
%****************************************************
\subsubsection{Vibration Damping}
%****************************************************
\subsubsection{Duct}
%****************************************************
\subsubsection{Landing Skids}
%****************************************************
\subsubsection{Motors \& ESCs}
%****************************************************

%****************************************************
\section{System Layout}
\label{sec:proto.layout}
%****************************************************

%****************************************************
%	CHAPTER 3 - Dynamics
%****************************************************
\chapter{Kinematics \& Dynamics}
\label{ch:dynamics}
%====================================================
The body's dynamics are first solved as rigid, with appropriate equations derived for generic 6-DOF motion. There after, non-linear aerodynamic and inertial effects, unique to multi-body relative rotations, are presented and introduced into the plant's model. Finally a consolidated, quaternion based plant model is presented which is used for the later control plant development next in Chapter:\ref{ch:control}.
%====================================================
\section{Rigid Body Dynamics}
\label{sec:dynamics.rigidbody}
%====================================================
\subsection{Lagrange Derivation}
\label{subsec:dynamics.rigidbody.lagrange}
%====================================================
Fundamentally any body, rigid or otherwise, can undergo two kinds of movements, namely rotational and translation motions. Often a Lagrangian\cite{classicaldynamics,rotationrigidbody} approach for combined angular and translational movements is used to derive the differential equations of motion for each degree of freedom. The Lagrangian principle ensures that translational and rotational kinematic energies and potential energy are conserved all throughout the system's trajectory progression. When combined with Euler-Rotational equations, the Euler-Lagrangian\cite{lagrange-formalism} formulation fully defines the aerospace 6-DOF equation set.
\par
Lagrangian formalism is regarded as especially useful in non-cartesian (\emph{spherical etc\ldots}) co-ordinate frames and multi-body systems. With that being said, a cartesian co-ordinate system was already defined in Section:\ref{subsec:proto.conventions.motoraxis}, rigid body dynamics in a cartesian co-ordinate frame do lend themselves to Newtonian mechanics. The Newtonain-Euler or Euler-Lagrange formulations stipulate the same resultant equations. The Lagrangian operator, $\mathcal{L}$, is a term made up of the difference between kinetic and potential energies, $T$ and $U$ respectively. Considering some generalized path co-ordinates $\mathbf{r}(t)$, for both linear $\mathcal{E}$ and angular $\eta$ relative positions;
\begin{equation}\label{eq:generalpath}
\mathbf{r}(t)=\begin{bmatrix}
\mathcal{E} & \eta
\end{bmatrix}^T
\end{equation}
The co-ordinates in Eq:\ref{eq:generalpath} are generalized here, despite being symbols commonly used to represent linear and angular positions. The generalized co-ordinates are later be refined as Cartesian body co-ordinates with respect to the inertial frame. The Lagrangian, by definition, is then:
\begin{subequations}
\begin{equation}\label{eq:lagrangian.a}
\mathcal{L}(\mathbf{r},\dot{\mathbf{r}},t)=T(\mathbf{r},\dot{\mathbf{r}})-U(\mathbf{r},\dot{\mathbf{r}})
\end{equation}
Introducing generic kinetic (angular \& linear) and potential energies, the latter being only gravitational potential energy in this case;
\begin{equation}\label{eq:lagrangian.b}
\mathcal{L}=\frac{1}{2}\dot{\mathcal{E}}^{T}(m)\dot{\mathcal{E}}+\frac{1}{2}\dot{\eta}^T(\mathbb{I})\dot{\eta}-mgz
\end{equation}
\end{subequations}
\newpage
Noting that $\mathbb{I}$ is the inertial tensor aligned w.r.t to whichever generalized coordinates are being used. The Euler-Lagrange formulation equates partial derivatives of the Lagrangian to any generalized forces, $\mathbf{V}$, acting on the system. In this case the generalized forces or a net force, $F~~[N]$, and a net torque, $\tau~~[N.m]$.
\begin{equation}\label{eq:euler-lagrange}
\frac{d}{dt}\bigg(\frac{\delta L}{\delta \dot{\mathbf{r}}}\bigg)-\frac{\delta L}{\delta \mathbf{r}} = \mathbf{V} = \begin{bmatrix}
F\\
\tau
\end{bmatrix}
\end{equation}
Taking the partial derivatives of Eq:\ref{eq:lagrangian.b} with respect to its path co-ordinates $\mathbf{r}$:
\begin{subequations}
\begin{equation}\label{eq:partial.a}
\frac{\delta L}{\delta \mathbf{r}}=\begin{bmatrix}
mG\\
0
\end{bmatrix}
\end{equation}
\vspace{-5pt}
\begin{equation}\label{eq:partial.b}
\frac{d}{dt}\bigg(\frac{\delta L}{\delta \dot{\mathbf{r}}}\bigg)=\bigg[
m\frac{d}{dt}\dot{\mathcal{E}} ~~~ \mathbb{I}\frac{d}{dt}\dot{\eta}\bigg]^T
\end{equation}
\end{subequations}
In any generalized coordinate system a rotating vector's time derivative, according to the Reynolds Transportation Theorem\cite{reynolds,conservationequations}, is given by:
\begin{equation}\label{eq:reynolds}
\frac{d\vec{f}}{dt_a}=\frac{d\vec{f}}{dt_b}+\vec{\omega}_{a/b}\times\vec{f}
\end{equation}
So applying that theorem (Eq:\ref{eq:reynolds}) to the partial derivatives in Eq:\ref{eq:partial.b} and further defining the generalized co-ordinates as cartesian body coordinates with respect to an inertial origin. Noting that in Eq:\ref{eq:partial.b} the place holders used for linear and angular positions are in a common shared frame\footnote{In this case $\eta\not=\phi~\theta~\psi]^T$ seeing that it's defined in a common frame}, and hence $[ \dot{\mathcal{E}}~~\dot{\eta} ]^T\equiv [ \nu ~~ \omega]^T\in \mathcal{F}^b$. It then follows that Lagrangian will change:
\begin{subequations}
\begin{equation}
\mathcal{L}=\frac{1}{2}\nu^{T}(m)\nu+\frac{1}{2}\omega^T(\mathbb{I})\omega -mG_b z
\end{equation}
\vspace{-5pt}
\begin{equation}
\frac{d}{dt}\bigg(\frac{\delta L}{\delta \dot{\mathbf{r}}}\bigg)=\bigg[
m\frac{d}{dt}\nu ~~~ \mathbb{I}\frac{d}{dt}\omega\bigg]^T
\end{equation}
\vspace{-5pt}
\begin{equation}
\rightarrow m\frac{d}{dt}\nu_b=m\dot{\nu}+\vec{\omega}_{I/b}\times\nu
\end{equation}
\vspace{-5pt}
\begin{equation}
\rightarrow \mathbb{I}_b \frac{d}{dt}\omega_b=\mathbb{I}_b\dot{\omega}+\omega_{I/b}\times\mathbb{I}_b\omega
\end{equation}
\end{subequations}
Which, when reintroduced to the Euler-Langrage formulation Eq:\ref{eq:euler-lagrange}, results in the familiar Newton-Euler equations for linear and angular movements, both in the body frame;
\begin{subequations}\label{eq:newton}
\begin{equation}\label{eq:newton.a}
F=m\dot{\nu}+\omega_b\times m \nu - m\mathbb{R}_I^b(-\eta) G
\end{equation}
\vspace{-15pt}
\begin{equation}\label{eq:newton.b}
\tau=\mathbb{I}_b\dot{\omega}_b+\omega_b\times\mathbb{I}_b\omega
\end{equation}
\end{subequations}
It's important to recall that $\omega_b\not= \dot{\eta}$ where $\eta=[\phi~\theta~\psi]^T$, seeing that Euler Angles are defined in sequentially rotated reference frames. So the four differential equations often used to completely describe the entire state derivatives are:
\begin{subequations}\label{eq:states}
\begin{equation}
\dot{\mathcal{E}}=\mathbb{R}_b^I(-\eta)\nu~~~~\in\mathcal{F}^I
\end{equation}
\vspace{-15pt}
\begin{equation}
F=m\dot{\nu}+\omega_b\times m\nu -m \mathbb{R}_I^b(-\eta) G ~~~~\in\mathcal{F}^b
\end{equation}
\vspace{-10pt}
\begin{equation}
\dot{\eta}=\Psi(\eta)\omega_b~~~~\in\mathcal{F}^{v2},\mathcal{F}^{v1},\mathcal{F}^I
\end{equation}
\vspace{-10pt}
\begin{equation}
\tau=\mathbb{I}_b\omega_b+\omega_b\times\mathbb{I}_b\omega~~~~\in\mathcal{F}^b
\end{equation}
\end{subequations}
The set of state differentials in Eq:\ref{eq:states} can be reduced to a set of two equations, defined only in their respective reference frames of the state variables which they describe. The non-linear form of those equations substitutes\footnote{Originally introduced in Eq:\ref{eq:angular-rates.e}} $\dot{\eta}=\Phi(\eta)\omega_b$ in the Lagrangian derivative, Eq:\ref{eq:partial.b}.
\begin{equation}
\frac{d}{dt}\bigg(\frac{\delta L}{\delta \dot{\mathbf{r}}}\bigg)=\bigg[m\frac{d}{dt}\nu~~~\mathbb{I}\frac{d}{dt}\dot{\eta}\bigg]=\bigg[m\frac{d}{dt}\nu~~~\mathbb{I}\frac{d}{dt}\Phi(\eta)\omega_b\bigg]
\end{equation}
This only changes the angular component, and so applying the differential chain rule:
\begin{equation}
\mathbb{I}\frac{d}{dt}\Phi(\eta)\dot{\omega}_b=\mathbb{I}\big(\Phi\dot{(\eta)}\omega_b+\Phi(\eta)\dot{\omega}_b \big)
\end{equation}
Drawing from \cite{autonomousrobotseuler} and recognizing that $\mathbb{I}$ must be transformed to common axes, $\mathbb{J}=\Psi(\eta)^T\mathbb{I}\Psi(\eta)$. The controllable differential equations in Eq:\ref{eq:newton}, then in the inertial frame for force and intermediate Euler frames for each angle becomes\footnote{The relationship $\dot{\Phi}=\Phi\dot{\Psi}\Phi$ was used to simplify Eq:\ref{eq:nonlinear}, the singularity in $\Phi$ still remains\ldots}:
\begin{subequations}\label{eq:nonlinear}
\begin{equation}
M(\eta)\ddot{\eta}+C(\eta,\dot{\eta})\dot{\eta}=\Psi(\eta)\tau
\end{equation}
\vspace{-10pt}
\begin{equation}
M(\eta)=\Psi(\eta)^T\mathbb{I}\Psi(\eta)
\end{equation}
\vspace{-10pt}
\begin{equation}
C(\eta,\dot{\eta})=-\Psi(\eta)\mathbb{I}\Psi\dot{(\eta)}+\Psi(\eta)^T sk(\Psi(\eta)\dot{\eta})\mathbb{I}\Psi(\eta)
\end{equation}
\end{subequations}
Equation \ref{eq:nonlinear} fully describes the state derivative $\ddot{\eta}$ in its own frame(s). The two differential equations which describe the entire bodies motion are:
\begin{subequations}
\begin{equation}
F=m\dot{\mathcal{E}}+\mathbb{R}_b^I(-\eta)\omega_b \times m \dot{\mathcal{E}}-mG~~~~\in\mathcal{F}^I
\end{equation}
\vspace{-10pt}
\begin{equation}
\Psi(\eta)\tau=M(\eta)\ddot{\eta}+C(\eta,\dot{\eta})~~~~\in\mathcal{F}^{v2},\mathcal{F}^{v1},\mathcal{F}^I
\end{equation}
\end{subequations}
\par
The generalized forces effecting the system, $F$ and $\tau$, are the system's controllable inputs and are going to be affected directly the systems actuators and their associated effectiveness function. In the general case, which is expanded upon in Section:\ref{sec:dynamics.aero}, the control inputs are typically as follows:
\begin{subequations}
The net force acting on the system is just the sum of all thrust vectors produced by rotating propellers, $T(\Omega_i)$.
\begin{equation}
\mu F = \sum \vec{T}_i
\end{equation}
Secondly the net torque is the sum of all torque arms produced from those propeller thrust vectors.
\begin{equation}
\mu \tau = \sum \vec{l}_i \times \vec{T}_i
\end{equation}
\end{subequations}
Where $\vec{T}_i$ is the $i^{th}$ motor's thrust vector, not necessarily in 3 dimensions and $\vec{l}_i$ is that motor's torque arm. The above equations are still applicable to any 6 DOF body, although aspects unique to the quadrotor and tilting quadrotor platforms will now be introduced\ldots
%====================================================
\subsection{Rotation Matrix Singularity}\label{subsec:dynamics.rigidbody.singularity}
%====================================================
The Euler Angle singularity is often noted but infrequently proven mathematically, far less common is the demonstration of exactly how that singularity \emph{mathematically} manifests itself. By definition, a singularity occurs when a loss of differentiability is encountered. In the case of an affixed 3-axis gimbal, when the an intermediate axis, for example the rolling angle $\theta$, is at $\pi$ then the remaining two axes become co-linear. That being a pitch $\phi$ or yaw $\psi$ rotations will subsequently have the same effect. Such a situation results in the deterioration of a degree of freedom.
\begin{figure}
\begin{subfigure}{0.5\textwidth}
\centering
\includegraphics[width=\textwidth]{figs/gimbal-lock-a}
\caption{3-Axis gimbal}
\end{subfigure}
\begin{subfigure}{0.5\textwidth}
\centering
\includegraphics[width=\textwidth]{figs/gimbal-lock-b}
\caption{Locked gimbal with loss of DOF}
\end{subfigure}
\end{figure} 
\par
What is obvious in a physical system is not necessarily as clear mathematically. A clear loss of differentiability is manifested in the Euler Matrix $\Psi(\eta)$, Eq:\ref{eq:angular-rates.e} from Section:\ref{subsec:proto.conventions.frames}. The relation between angular velocity, in the inertial frame or inversely in the body frame, and the angular rates of the Euler Angles.
\begin{equation}\label{eq:euler-derivative}
\begin{bmatrix}
\dot{\phi}\\
\dot{\theta}\\
\dot{\psi}
\end{bmatrix}
=\begin{bmatrix}
1 & sin(\phi)tan(\theta) & cos(\phi)tan(\theta)\\
0 & cos(\phi) & -sin(\phi)\\
0 & sin(\phi)sec(\theta) & cos(\phi)sec(\theta)\\
\end{bmatrix}
\begin{bmatrix}
p\\
q\\
r
\end{bmatrix}
=\Phi(\eta)\omega_b
\end{equation}
\begin{equation}
\text{As}~\underset{{\theta \rightarrow \pi /2}}{lim}~sec(\theta)\rightarrow \infty
\end{equation}
Or that $\Phi(\eta)$ is undefined at $\theta=\pi/2$. 
It's clear to see that in Eq:\ref{eq:euler-derivative} there exists an undefined singularity as $\theta\rightarrow\pi/2$. The physical consequence of this is the loss of a degree of freedom. More specifically, if one looks at how the rotation (or transformation) matrices are formulated:
\begin{subequations}
\begin{equation}
\mathbb{R}_I^b = \mathbb{R}_z\mathbb{R}_y\mathbb{R}_x=\begin{bmatrix}
c_\psi & -s_\psi & 0\\
s_\psi & c_\psi & 0\\
0 & 0 & 1
\end{bmatrix}
\begin{bmatrix}
c_\theta & 0 & s_\theta\\
0 & 1 & 0\\
-s_\theta & 0 & c_\theta
\end{bmatrix}
\begin{bmatrix}
1 & 0 & 0\\
c_\phi & -s_\phi & 0\\
s_\phi & c_\phi & 0
\end{bmatrix}
\end{equation}
\begin{equation}
\mathbb{R}_I^b=\begin{bmatrix}
c_\psi c_\theta & c_\psi s_\theta s_\phi - s_\psi c_\phi & c_\psi s_\theta c_\phi + s_\psi s_\phi\\
s_\psi c_\theta & s_\psi s_\theta s_\phi + c_\psi c_\phi & s_\psi s_\theta  c_\phi - c_\psi s_\phi\\
-s_\theta & c_\theta s_\phi & c_\phi c_\theta\\
\end{bmatrix}
\end{equation}
In the case where $\theta=\pi/2$;
\begin{equation}
=\begin{bmatrix}
0 & c_\psi s_\phi - s_\psi c_\phi & c_\psi c_\phi + s_\psi s_\phi\\
0 & s_\psi s_\phi + c_\psi c_\phi & s_\psi c_\phi - c_\psi s_\phi\\
-1 & 0 & 0\\
\end{bmatrix}
=
\begin{bmatrix}
0 & s(\phi - \psi) & c(\phi - \psi)\\
0 & c(\phi - \psi) & s(\phi - \psi)\\
-1 & 0 & 0
\end{bmatrix}
\end{equation}
\end{subequations}
Where the second term in Eq: represents an x-axis rotation in a new intermeiate frame 
%====================================================
\subsection{Quaternion Dynamics}
\label{subsec:dynamics.rigidbody.quaternion}
%====================================================
\subsection{The Unwinding Problem}
\label{subsec:dynamics.rigidbody.unwinding}
%====================================================

%****************************************************
\section{Non-linearities}
\label{sec:dynamics.nonlinearities}
%****************************************************
\subsection{Gyroscopic Torques}
\label{subsec:dynamics.nonlinearities.gyrotorques}
%****************************************************
\subsection{Coriolis Acceleration}
\label{subsec:dynamics.nonlinearities.coriolis}
%****************************************************
\subsection{Inertial Matrix}
\label{subsec:dynamics.nonlinearities.inertia}
%****************************************************

%****************************************************
\section{Aerodynamics}
\label{sec:dynamics.aero}
%****************************************************
\subsection{Thrust Forces \& Propeller Torques}
\label{subsec:dynamics.aero.bem}
%****************************************************
\subsection{Drag}
\label{subsec:dynamics.aero.drag}
%****************************************************
\subsection{Conning \& Flapping}
\label{subsec:dynamics.aero.flap}
%****************************************************
\subsection{Vortex Ring State}
\label{subsec:dynamics.aero.vrs}
%****************************************************

%****************************************************
\section{Consolidated Model}
\label{sec:dynamics.model}
%====================================================
%	CHAPTER 4 - Control
%====================================================
\chapter{Controller Development}
\label{ch:control}
%====================================================
\section{Control Loop}
\label{sec:control.loop}
%====================================================
The control problem here is, as outlined in Ch:\ref{ch:intro}, to achieve non-zero set-point tracking (\emph{attitude and position} states) on a quadrotor by solving the problem of its inherent under-actuation. For the purposes of the subsequent controller development, the plant is described in the following typical non-linear state space form:
\begin{subequations}\label{eq:control-loop-states}
\begin{equation}
\frac{d}{dt}{\vec{\mathbf{x}}}=f(\vec{\mathbf{x}},t)+g(\vec{\mathbf{x}},\vec{\nu},t)
\end{equation}
\vspace{-12pt}
\begin{equation}
\vec{y} = c(\vec{\mathbf{x}},t)+d(\vec{\mathbf{x}},\vec{\nu},t)
\end{equation}
\end{subequations}
Where the plant's dynamics are governed by state progression $f(\vec{\mathbf{x}},t)$ and the plant's input response $g(\vec{\mathbf{x}},\vec{\nu},t)$ for a given control input $\vec{\nu}$. The latter could take the affine form; $g(\mathbf{x},t)\vec{\nu}$. Set-point tracking aims for the output to track the plant's state; namely $\vec{y} = c(\vec{\mathbf{x}},t)=\vec{\mathbf{x}}$. As such, the control problem is to design a stabilizing control law $h$ for some error state $\mathbf{x}_e$:
\begin{equation}
\vec{\nu}_d=h(\mathbf{x}_e,t)=h(\mathbf{x}_b,\dot{\mathbf{x}}_b,\mathbf{x}_d,\dot{\mathbf{x}}_d,t)
\end{equation}
Such that the controlled plant is asymptotically stabilizing or that $\lim_{t\rightarrow\infty}\mathbf{x}_e=0$. Trajectory stability conditions are further defined next in Sec:\ref{sec:control.stability}. Note that it is possible to combine attitude and position states into a single common trajectory state such that:
\\
\vspace{-5pt}
\begin{equation}
\vec{\mathbf{x}}=\begin{bmatrix}\vec{\mathcal{E}}&Q_b\end{bmatrix}^T
\end{equation}
The body's trajectory is then fully described by $\vec{\mathbf{x}}(t)$. Separate control laws are developed for attitude and position tracking and hence those states are not combined in the context of this control project. However for the purposes of describing the control plant, a single major loop is considered.
\par
Because of the plant's overactuatedness the control loop is split into two blocks; first a higher level \emph{set-point tracking} controller designs a virtual control input $\vec{\nu}_d$. That being net forces $\vec{F}_d$ and torques $\vec{\tau}_d$ to act on the body. A lower level \emph{allocator} then solves for explicit actuator positions from $\vec{\nu}_d$ to physically actuate that \emph{virtual} control input. The actuator set then implements a commanded control input $\vec{\nu}_c$ through its effectiveness function (Eq:\ref{eq:dynamic-plant-inputs})
\begin{equation}
\vec{\nu}_c=B(\vec{\mathbf{x}},u,t)
\end{equation}
The allocator solves for actuator values $u\in\mathbb{U}$ such that $\vec{\nu}_c\rightarrow\vec{\nu}_d$. That allocation function, $B^\dagger$, can be \emph{roughly} referred to as the effectiveness inverse:
\begin{equation}
\underset{\in\mathbb{U}}{u}=B^{\dagger}(\vec{\mathbf{x}},\vec{\nu}_d,t)
\end{equation}
This chapter derives higher level controllers for $\vec{\nu}_d=h(\vec{\mathbf{x}}_e,t)$; allocation rules are discussed next in Ch:\ref{ch:allocation}. A collection of attitude and position controllers are presented here whose stability is proven with Lyapunov theorem \cite{bojelayupanov,lyapunovstabilitytheorem,noteonlyapunov}. Each controller is compared in the context of an over actuated quadrotor plant. Similarly a series of proposed allocation schemes are presented. Controller comparisons, their details and efficacy are evaluated is subsequently in Ch:\ref{ch:simulation}. 
\par
A generalized over-actuated control loop consists of a series of cascaded control blocks (Fig:\ref{fig:control-loop}). From the trajectory's error state $\vec{\mathbf{x}}_e$, a control law designs a virtual control input $\vec{\nu}_d$ which is applied to the allocation block. The allocation law $B^{\dagger}(\mathbf{x},\vec{\nu}_d,t)$ solves for physical actuator positions $u\in\mathbb{U}$. Actuator positions command a physical input $\vec{\nu}_c=B(\mathbf{x},u,t)$ which is an input to the state's dynamics, Eq:\ref{eq:control-loop-states}. Finally the output tracking state is estimated with some filter $\hat{\mathbf{x}}=A(\mathbf{x},t)$ which is used to calculate the error state (Sec:\ref{sec:simulation.state}).
\begin{figure}[htbp]
\centering
\includegraphics[width=\textwidth]{figs/control-loop}
\caption{Generalized control loop with allocation}
\vspace{-12pt}
\label{fig:control-loop}
\end{figure}
\par
Fig:\ref{fig:control-loop} is a generalized illustration of the control loop's structure; the plant's dynamics from Eq:\ref{eq:quaternion-states} include state derivative feedback. Moreover aspects of the state transfer function includes multi-body non-linearities dependent on actuator positions and rates as detailed in Sec:\ref{sec:dynamics.nonlinearities}. That generalized case is now refined in the context of an over-actuated quadcopter.
%====================================================
\section{Control Plant Inputs}
\label{sec:control.inputs}
%====================================================
Control inputs for the differential state equations, from Eq:\ref{eq:quaternion-states}, have mostly been described with net forces and torques; $\vec{F}_\mu(u)$ and $\vec{\tau}_\mu(u)$. The relationship (\emph{effectiveness function}) between each propeller's rotational speed and servo positions with the produced thrust vector is calculated from Eq:\ref{eq:quaternion-inputs}.
\begin{subequations}\label{eq:control-input}
\begin{equation}
\vec{\nu}_c\triangleq\begin{bmatrix}
\vec{F}_\mu(u) & \vec{\tau}_\mu(u)
\end{bmatrix}^T=B(\vec{\mathbf{x}},u,t)~~~~\in\mathbb{R}^6,~u\in\mathbb{U}
\end{equation}
\vspace{-10pt}
\begin{equation}
\vec{F}_\mu(u)=\sum_{i=1}^4 Q_{M_i}^*(\lambda_i,\alpha_i)\otimes T(\Omega_i)\otimes Q_{M_i}(\lambda_i,\alpha_i)~~~~\in\mathcal{F}^b
\end{equation}
\vspace{-4pt}
\begin{equation}
\vec{\tau}_\mu(u)=\sum_{i=1}^4 \vec{L}_i\times\big(Q_{M_i}^*(\lambda_i,\alpha_i)\otimes T(\Omega_i)\otimes Q_{M_i}(\lambda_i,\alpha_i)\big)~~~~\in\mathcal{F}^b
\end{equation}
\end{subequations}
As mentioned previously, a higher level controller designs desired net plant inputs $\vec{\nu}_d=\begin{bmatrix}\vec{F}_d&\vec{\tau}_d\end{bmatrix}^T$ whilst a lower level allocator commands physical inputs $u=B^{\dagger}(\vec{\mathbf{x}},\vec{\nu}_d,t)$. This allows for independent comparison of proposed control and allocation laws. However, typical allocation rules like pseudo-inversion require an affine relationship between plant and control inputs (Sec:\ref{subsubsec:intro.lit.control.allocation}). The relationship in Eq:\ref{eq:control-input} is not reducible to a single multiplicative relationship with the actuator matrix $u\in\mathbb{U}$. So the effectiveness function needs an extra layer of abstraction to incorporate a multiplicative relationship. Rather than calculating explicit actuator positions directly from $\vec{\nu}_d$; a set of four 3-D thrust vectors $\vec{T}_{1\rightarrow 4}$ for each motor module are first calculated.
\begin{subequations}\label{eq:4.7}
\begin{equation}
\vec{\nu}_c=\begin{bmatrix}
\vec{F}_\mu(u)\\
\vec{\tau}_\mu(u)
\end{bmatrix}
= 
\begin{bmatrix}
1 & 1 & 1 & 1\\
[\vec{L}_1]_\times & [\vec{L}_2]_\times & [\vec{L}_3]_\times & [\vec{L}_4]_\times
\end{bmatrix}
\begin{bmatrix}
\vec{T}_1&
\vec{T}_2&
\vec{T}_3&
\vec{T}_4
\end{bmatrix}^T
\end{equation}
\begin{equation}
\rightarrow\vec{\nu}_f=B'(\vec{\mathbf{x}},t)\begin{bmatrix}
\vec{T}_1&
\vec{T}_2&
\vec{T}_3&
\vec{T}_4
\end{bmatrix}^T
\end{equation}
\end{subequations}
Where $[\vec{L}_i]_\times$ is the cross product vector of the $i^{th}$ torque arm. Explicit actuator positions for each module, $[\Omega_i,\lambda_i,\alpha_i]^T$, can then be solved from those thrust vectors $\vec{T}_i$ for $i\in[1:4]$ with some trigonometry, "unwinding" transformations applied in Eq:\ref{eq:control-input}. That trigonometric inversion is detailed later in Sec:\ref{sec:allocation.slack} but is described as the function $R^\dagger$:
\begin{equation}\label{eq:4.8}
[\Omega_i,~\lambda_i,~\alpha_i]^T=R^\dagger(\mathbf{x},\vec{T}_i,t)~~~~\text{for}~i\in[1:4]
\end{equation}
The generalized control loop illustrated in Fig:\ref{fig:control-loop} is extended to include the abstracted allocation blocks of Eq:\ref{eq:4.7} and Eq:\ref{eq:4.8}, shown in Fig:\ref{fig:control-block}. The net control block still solves for the same actuator matrix $u\in\mathbb{U}$. The entire loop accommodates for comparison of various $B^\dagger(\mathbf{x},\vec{\nu}_d,t)$ allocation rules without having to redesign the remainder of the loop's structure.
\begin{figure}[htbp]
\vspace{-8pt}
\centering
\includegraphics[width=0.95\textwidth]{figs/control-block}
\caption{Extended control loop with over-actuation}
\label{fig:control-block}
\vspace{-16pt}
\end{figure}
\par
In summary; each controller designs a net force and torque to act on the body. Allocation rules decompose that virtual input into four separate 3-D thrust vectors, or 12 directional components. The force components are an abstracted allocation layer in place of explicit actuator positions, which are subsequently solved for\ldots
\begin{equation}
B^{\dagger}(\mathbf{x},\vec{\nu}_d,t)=\big[ T_{1x},~T_{1y},~T_{1z},~\ldots~T_{4x},~T_{4y},~T_{4z}\big]^T
\end{equation}
Each control law is co-dependent on an accompanying allocation algorithm. Traditional control loops (under-actuated or well matched) typically have a unity allocation rule and as such require no consideration so they're mostly disregarded. Separate control laws for attitude ad position control are presented in Section:\ref{sec:control.attitude} and \ref{sec:control.position} respectively. Thereafter a series of allocation rules are proposed in Ch:\ref{ch:allocation}. Although presented independently, the controller and allocation laws are mutually inclusive. The stability of each controller is proven objectively but explicit controller coefficients are optimized in the subsequent Ch:\ref{ch:simulation}, in Sec:\ref{sec:simulation.tuning}.
\newpage
%====================================================
\section{Stability}
\label{sec:control.stability}
%====================================================
Before undertaking the control plant derivations, it is worth outlining definitions of what the control objectives are. The research question aims to achieve non-zero set-point tracking of state's trajectory. A control loop then aims to \emph{stabilize} the dynamics described previously in Sec:\ref{sec:dynamics.model} whilst tracking particular trajectories for attitude and position, $\mathbf{x}_d(t)=[\vec{\mathcal{E}}_d(t)~Q_d(t)]^T$. 
\par
The entire system's control loop was previously detailed in Sec:\ref{sec:control.loop}. Stability in the context of trajectory tracking must be defined. Generalized trajectory stability definitions are not uncommon in the context of energy based control design, or Lyapunov theorem (Sec:\ref{sec:control.lyapunov}). Stability definitions pertinent to Lyapunov's stability theorem are briefly presented here; the following is adapted from \cite{bojelayupanov,lyapunovstabilitytheorem}. In general for some autonomous trajectory $\vec{\mathbf{x}}(t)$, an equilibrium point $\vec{\mathbf{x}}(t_0)$ is said to be stable (\textbf{S}) if and only if (\emph{iff}) the following is true:
\begin{subequations}\label{eq:basic-stability}
\begin{equation}
\forall\varepsilon>0,~\exists~\delta_0(t_0,\varepsilon):~\norm{\vec{\mathbf{x}}(t_0)}<\delta_0(t_0,\varepsilon)
\end{equation}
\vspace{-20pt}
\begin{equation}
\Rightarrow\norm{\vec{\mathbf{x}}(t)}<\varepsilon,~~\forall t\geq t_0
\end{equation}
\end{subequations}
The implication of which is that if, for some initial condition $\vec{\mathbf{x}}(t_0)$ whose magnitude is bound by the manifold $\delta_0(t_0,\varepsilon)$, the entire subsequent trajectory of $\vec{\mathbf{x}}(t)$ is bound from above by some other manifold $\varepsilon$. Basic stability is illustrated in Fig:\ref{fig:basic-stability} for a 2-D trajectory.
\begin{figure}[hbtp]
\vspace{-6pt}
\centering
\begin{subfigure}{0.49\textwidth}
\centering
\includegraphics[width=\textwidth]{figs/basic-stability}
\vspace{-16pt}
\caption{Basic stability}
\label{fig:basic-stability}
\end{subfigure}
\begin{subfigure}{0.49\textwidth}
\centering
\includegraphics[width=\textwidth]{figs/uniform-stability}
\vspace{-16pt}
\caption{Uniform stability}
\label{fig:uniform-stability}
\end{subfigure}
\vspace{-8pt}
\caption{Trajectory illustrations for $\mathbf{S}$ and $\mathbf{US}$}
\vspace{-18pt}
\end{figure}
\par
An equilibrium point is further said to be uniformly stable (\textbf{US}) \emph{iff} for the time $t\in[t_0,\infty)$ the following criteria, being an extension of basic stability, is met:
\begin{subequations}\label{eq:uniform-stability}
\begin{equation}
\forall\varepsilon>0,~\exists~\delta_0(\varepsilon)>0:~\norm{\vec{\mathbf{x}}(t_1)}<\delta_0(\varepsilon),~~t_1>t_0
\end{equation}
\vspace{-18pt}
\begin{equation}
\Rightarrow\norm{\vec{\mathbf{x}}(t)}<\varepsilon,~~\forall
t\geq t_1
\end{equation}
\end{subequations}
\textbf{US} similarly bounds a trajectory from above by $\varepsilon$ if the trajectory originates from within $\delta_0(\varepsilon)$. The difference is that the principle trajectory region $\delta_0(\varepsilon)$ is independent of $t_0$ in the case of \textbf{US}. The two surfaces are effectively non-concentric; a $\mathbf{US}$ trajectory is illustrated in Fig:\ref{fig:uniform-stability}. Uniform stability is a subset of general stability, $\mathbf{US}\subset\mathbf{S}$, however the converse is not true. Furthermore \textbf{US} is a stronger assertion of stability. 
\par
Extending stability definitions to include settling; an equilibrium point is said to be asymptotically stable (\textbf{AS}) \emph{iff} conditions for \textbf{S} are met (Eq:\ref{eq:basic-stability}) and that the following holds true:
\begin{subequations}\label{eq:asymptotic-stability}
\begin{equation}
\exists~\delta_1(t_0,\varepsilon) >0:~\norm{\vec{\mathbf{x}}(t_0)}<\delta_1(t_0,\varepsilon)
\end{equation}
\vspace{-18pt}
\begin{equation}
\Rightarrow \lim_{t\rightarrow\infty}\norm{\vec{\mathbf{x}}(t)}\rightarrow 0
\end{equation}
\end{subequations}
This asserts that trajectories originating within some finer region $\delta_1(t_0,\varepsilon)$, being a subset of $\delta_0(t_0,\varepsilon)$, the trajectory tends to and \emph{asymptotically} settles at the origin. In the case of \textbf{AS} the origin is both \underline{stable} and \underline{attractive} (shown in Fig:\ref{fig:asymptotic-stability}). Asymptotic stability is typically the first requirement for any control law, being a stronger stability than both \textbf{US} and \textbf{S}\ldots
\begin{figure}[hbtp]
\centering
\begin{subfigure}{0.49\textwidth}
\centering
\includegraphics[width=\textwidth]{figs/asymptotic-stability}
\vspace{-8pt}
\caption{Asymptotic stability}
\label{fig:asymptotic-stability}
\end{subfigure}
\begin{subfigure}{0.49\textwidth}
\centering
\includegraphics[width=\textwidth]{figs/uniform-asymptotic-stability}
\vspace{-8pt}
\caption{Uniform asymptotic stability}
\label{fig:uniform-asymptotic-stability}
\end{subfigure}
\vspace{-4pt}
\caption{Trajectory illustrations for $\mathbf{AS}$ and $\mathbf{UAS}$}
\vspace{-14pt}
\end{figure}
\par
Uniform asymptotic stability (\textbf{UAS}), an extension of uniform stability, occurs when the asymptotically stable bound region $\delta_1(\epsilon)$ is independent of the principle starting $t_0$. An equilibrium point is \textbf{UAS} \emph{iff} conditions for \textbf{S} are met and that:
\begin{subequations}\label{eq:uniform-asymptotic-stability}
\begin{equation}
\exists~\delta_1(\varepsilon)>0:~\norm{\vec{\mathbf{x}}(t_1)}<\delta_1(\varepsilon),~~t_1\geq t_0
\end{equation} 
\vspace{-16pt}
\begin{equation}
\Rightarrow \lim_{t\rightarrow\infty}\norm{\vec{\mathbf{x}}(t)}\rightarrow 0
\end{equation}
\end{subequations}
A uniformly asymptotic equilibrium point implies stability from a non-concentric ball of attraction; settling to the origin (illustrated in Fig:\ref{fig:uniform-asymptotic-stability}). 
\par
An equilibrium point is regarded as exponentially stable (\textbf{UES}) if conditions for \textbf{UAS} are met and that there exist $\exists~a,b,r$ that bound the settling of the trajectory such that:
\begin{equation}\label{eq:exponential-stability}
\norm{\vec{\mathbf{x}}(t,t_0,\vec{\mathbf{x}}_0)}\leq a\norm{\vec{\mathbf{x}}_0}e^{-bt},~~\forall\norm{\vec{\mathbf{x}}_0}\leq r
\end{equation}
The term $a\norm{\vec{\mathbf{x}}_0}e^{-bt}$ bounds the rate at which the trajectory settles to the origin, illustrated in Fig:\ref{fig:exponential-stability}. The initial point of the trajectory, $\vec{\mathbf{x}}_0$ is bound from above by some $r\triangleq \delta_1(\varepsilon)$. Moreover uniform stability is implied with exponential stability.
\begin{figure}[hbtp]
\centering
\includegraphics[width=0.5\textwidth]{figs/exponential-stability}
\vspace{-4pt}
\caption{Exponential stability, $\mathbf{UES}$}
\label{fig:exponential-stability}
\vspace{-6pt}
\end{figure}
\par
The above definitions of stability are only locally defined, and so the stabilities hold true only for local trajectories, only in the case of $\vec{\mathbf{x}}(t_0)\leq\varepsilon$. Extending \textbf{UAS} to global uniform asymptotic stability (\textbf{GUAS}); the origin's equilibrium point is \textbf{GUAS} \emph{iff} conditions for \textbf{UAS} are first met, the origin is only the equilibrium point and the asymptotic approach can be extended such that:
\begin{subequations}
\begin{equation}
\exists~\delta_1(\varepsilon)>0:~\norm{\vec{\mathbf{x}}(t_1)}<\delta_1(\varepsilon),~~t_1\geq t_0
\end{equation}
\vspace{-16pt}
\begin{equation}
\Rightarrow \lim_{t\rightarrow\infty}\norm{\vec{\mathbf{x}}(t)}\rightarrow \vec{0},~~\forall \vec{\mathbf{x}}(t_0)
\end{equation}
\end{subequations}
Similarly exponential stability can extend to the global case \emph{iff} \textbf{UES} conditions are first met. In the global case, the origin can be the only equilibrium point. Stability from Eq:\ref{eq:exponential-stability} is then globally:
\begin{equation}\label{eq:global-exponential-stability}
\norm{\vec{\mathbf{x}}(t,t_0)}\leq a\norm{\vec{\mathbf{x}}_0}e^{-bt},~~\forall\norm{\vec{\mathbf{x}}_0}
\end{equation}
Initial equilibrium point conditions are dropped in Eq:\ref{eq:global-exponential-stability}. It follows that, irrespective of the starting point for the trajectory, the system \underline{always} settles to the origin. \textbf{GUES} is the strongest sense of stability and further provides insight into the rate at which the trajectory stabilizes. The most desirable control design outcome is a controller which applies globally uniform exponential stability to a plant.
%====================================================
\section{Lyapunov Stability Theorem}
\label{sec:control.lyapunov}
%====================================================
Lyapunov's stability theory is an important aspect of non-linear control design. An abundance of literature exists on the subject, included in almost every meritable textbook or control paper. If the reader is unfamiliar with Lyapunov's theorem,  \cite{noteonlyapunov,nonlinearsystems,bojelayupanov} all explain in detail the concept. The following is adapted from \cite{bojelayupanov} and \cite{lyapunovstabilitytheorem} and briefly outlines how Lyapunov's stability theorem is used to prove (\emph{global}) asymptotic stability for continuous time invariant systems, linear or otherwise.
\par
The theory analyses a generalized energy function of a system's autonomous trajectory. If the trajectory has a negative energy derivative that implies the system's energy will always dissipate towards a stable equilibrium point. 
\par
Lyapunov analysis is a powerful method for stability verification, the system's trajectory itself need not be explicitly defined for stability to be ascertained. Proof of Lyapunov's theorem is done with a contradiction disproof and, as such, the theoretical underpinning is somewhat cumbersome.
\par
It's worth reiterating its fundamentals given that backstepping controllers are proposed later in Sec:\ref{subsec:control.attitude.nonlinear} for attitude control. A backstepping controller iteratively enforces Lyapunov stability criterion onto the system through the control structure, \cite{backstepping,adaptivebackstep,intelligentbackstep}. In general, given a non-linear time invariant system that follows some continually differentiable trajectory $\mathbf{x}(t)$, typically the trajectory is going to progress subject to some rule:
\begin{equation}
\dot{\mathbf{x}}(t)=f\big(\vec{\mathbf{x}}(t),u\big)
\end{equation}
Then, constructing a generalized positive-definite function (generalized energy or \emph{Lyapunov function candidate}) $V(\mathbf{x})$ for a trajectory $\mathbf{x}(t)$. A positive definite matrix, $M$, is defined such that:
\begin{equation}
\mathbf{z}^TM\mathbf{z}\geq 0~~~\forall \mathbf{z}
\end{equation}
As such an LFC typically, but not exclusively, has the quadratic and positive-definite form:
\begin{equation}
V(\mathbf{x})=\mathbf{x}^TP\mathbf{x}
\end{equation}
An LFC could simply be positive semi-definite over the trajectory's bound, the quadratic form is convenient for the use of backstepping. From its definition the trajectory is continually differentiable; there is then a gradient matrix for each component of $V(x)$ in the form:
\begin{equation}
\nabla V(\vec{\mathbf{x}})\triangleq\bigg[\frac{\partial V(\vec{\mathbf{x}})}{\partial x_1}~\frac{\partial V(\vec{\mathbf{x}})}{\partial x_2}~\ldots~\frac{\partial V(\vec{\mathbf{x}})}{\partial x_n}\bigg]~~~~\vec{\mathbf{x}}\in\mathbb{R}^n
\end{equation}
The energy function's derivative, otherwise referred to as the \emph{Lie derivative}\cite{noteonlyapunov,nonlinearsystems} is calculated as follows:
\begin{equation}
\dot{V}(\vec{\mathbf{x}})\triangleq\nabla V(\vec{\mathbf{x}})^T\frac{d}{dt}f(\vec{\mathbf{x}})=\frac{\delta V(x)}{\delta x_1}\frac{df_1(x)}{dt}+\frac{\delta V(x)}{\delta x_2}\frac{df_2(x_2)}{dt}+~\ldots~+\frac{\delta V(x)}{\delta x_n}\frac{df_n(x)}{dt}
\end{equation}
Lyapunov's theorem states that \emph{iff} the candidate function $V(\vec{\mathbf{x}})$ is positive definite with $V(\vec{0})=0$ and its derivative is strictly negative; $\dot{V}(\vec{\mathbf{x}})< 0~~\forall \vec{\mathbf{x}}(t) \not= 0$, the system is then asymptotically stable ($\mathbf{AS}$ from Eq:\ref{eq:asymptotic-stability}). Mathematically that means, for any $\vec{\mathbf{x}}(t)$ with $t\geq t_0$:
\begin{equation}
V\big(\vec{\mathbf{x}}(t)\big)=V\big(\vec{\mathbf{x}}(t_0)\big)+\int_{t_0}^t \dot{V}\big(\mathbf{x}(t)\big).dt \leq V\big(\mathbf{x}(t_0)\big)
\end{equation}
Which can be physically interpreted as the system's generalized energy function dissipating, irrespective of the trajectory path taken. With a strictly decreasing energy function, the system will stabilize to an equilibrium point. 
\begin{equation}
\lim_{t\rightarrow\infty}\norm{V\big(\vec{\mathbf{x}}(t)\big)}\rightarrow 0
\end{equation}
The trajectory's asymptotic stability can be extended to exponential stability boundedness, such that \emph{iff} the same conditions are met and there exists some positive coefficient $\alpha>0$ such that $\dot{V}(\vec{\mathbf{x}})\leq-\alpha V(\vec{\mathbf{x}})$. That implies the system is globally exponentially stable as is bound in such a way that:
\begin{equation}
\norm{V\big(\vec{\mathbf{x}}(t)\big)}\leq Me^{-\alpha t/2}\norm{V\big(\vec{\mathbf{x}}(t_0)\big)}
\end{equation}
%====================================================
\section{Model Dependent \& Independent Controllers}
%====================================================
Two classes of controllers are included for a full state trajectory tracking control loop; both attitude and position control laws. Attitude set-point tracking is the primary focus of this research project (Sec:\ref{subsec:control.attitude.problem}) and incorporates a more detailed schedule of controller design and evaluation. 
\par
The allocation law combines both virtual control inputs from attitude and position controllers, $\vec{\nu}_d=[\vec{F}_\mu(u)~\vec{\tau}_\mu(u)]^T$, to solve for explicit actuator positions. Controller dependency on the plant's state is as a consequence of the actuator responses and complex inertial dynamics, as derived previously in Sec:\ref{subsec:dynamics.nonlinearities.gyrotorques}. Whilst not a prerequisite for stability, plant dependent compensation obviously improves controller performances. Independent and dependent cases are only considered for one type of controller; the most basic case PD controller in Section:\ref{subsubsec:control.attitude.controllers.pd} and tested in Sec:\ref{subsec:simulation.attitude.pd}. All other control laws compensate for unwanted plant dynamics in a feedback configuration.
\par
The plant dependency makes backstepping controllers an effective controller choice for this dissertation's context. The proposed plant dependent control laws compensate for undesirable dynamics their design, basic PD and PID control structures (\emph{and the like}) will not. The first and most basic control solution, used as a reference case, is a PD controller for attitude and position with direct-inversion (Pseudo or Moore-Penrose inversion) allocation. 
%====================================================
\section{Attitude Control}
\label{sec:control.attitude}
%====================================================
\subsection{The Attitude Control Problem}
\label{subsec:control.attitude.problem}
%====================================================
The set-point tracking control problem (\cite{attitudecontrolproblem}) for the attitude plant is to design a stabilizing control torque $\vec{\tau}_d=h(\mathbf{x}_e,t)$ such that for any desired attitude quaternion, $\forall~Q_d\in\mathbb{Q}$, and an instantaneous attitude body quaternion, $Q_b(t)\in\mathbb{Q}$, the error state asymptotically stabilizes to the origin; $Q_e\rightarrow[1~\vec{0}~]^T$. Or that:
\begin{equation}
\vec{\tau}_d=h(Q_d,~\dot{Q}_d,~Q_b(t),~\dot{Q}_b(t))~~\text{such that}~~\underset{t\rightarrow\infty}{\lim}Q_b(t)\rightarrow Q_d
\end{equation}
Quaternion attitude error states are defined as the Hamilton product (\emph{difference}) between the desired and instantaneous quaternion attitude states. Quaternion error states are multiplicative, in contrast with the subtractive relationship for Euler angle error states. The attitude error state is calculated as:
\begin{equation}\label{eq:quaternion-error}
Q_e\triangleq Q_b^*(t)\otimes Q_d
\end{equation}
The relative angular velocity error between the body frame, $\mathcal{F}^b$, and the trajectory's desired frame, $\mathcal{F}^d$, is given as $\vec{\omega}_e$. The body angular velocity, $\vec{\omega}_b$ is subject to the differential Eq:\ref{eq:quaternion-states-angular}. As such there is an angular rate error:
\begin{subequations}
\begin{equation}\label{eq:angular-error}
\vec{\omega}_e\triangleq\vec{\omega}_d-\vec{\omega}_b(t)
\end{equation}
The desired angular velocity is taken with respect to the desired angular attitude frame, and so it must be transformed back onto the existing body frame.
\begin{equation}
\vec{\omega}_e=Q_e^*\otimes\vec{\omega}_d\otimes Q_e-\vec{\omega}_b(t)
\end{equation}
Typically for the trajectories generated here the desired angular velocity is zero; $\vec{\omega}_d=\vec{0}$. It follows that the angular rate error is then simply the negative body angular velocity. It would be easy to incorporate a non-zero angular velocity setpoint to accommodate for higher order state derivative tracking trajectories.
\begin{equation}
\vec{\omega}_e=-\vec{\omega}_b(t)\Big|_{\vec{\omega}_d=\vec{0}}
\end{equation}
\end{subequations}
The time derivative of the quaternion error state is calculated from the quaternion derivative definition Eq:\ref{eq:quaternion-deriv}. The derivative $\dot{Q}_e$ is then dependent on the angular velocity error and calculated as follows:
\begin{equation}
\dot{Q}_e=\frac{1}{2}Q_e\otimes\vec{\omega}_e=-\frac{1}{2}Q_e\otimes\vec{\omega}_b(t)\Big|_{\vec{\omega}_d=\vec{0}}
\end{equation}
%====================================================
\subsection{Linear Controllers}
\label{subsec:control.attitude.controllers}
%====================================================
\subsubsection{PD Controller}
\label{subsubsec:control.attitude.controllers.pd}
%====================================================
The following control law is used as a reference case for comparison of the remaining controllers derived. It is a simple Proportional-Derivative attitude controller, adapted from \cite{fullquaternion} and following a stability proof similar to the one derived in \cite{attitudecontrolproblem}. An attitude PD controller is proportional only to the vector quaternion error. Such that the error is the same dimension as the angular velocity error; $\in\mathbb{R}^3$. A PD controller designs the control torque as:
\begin{equation}\label{eq:independent-pd}
\vec{\tau}_{_{PD}}=K_d\vec{\omega}_e+K_p\vec{q}_e~~~~[Nm],~\in\mathcal{F}^b
\end{equation}
Where both $K_d$ and $K_p$ are positive definite symmetrical $3\times 3$ coefficient matrices to be determined later. Eq:\ref{eq:independent-pd} neglects the quaternion scalar error and is therefore susceptible to unwinding. Using a positive-definite Lyapunov function candidate $V_{_{PD}}$ for the attitude trajectory:
\begin{equation}\label{eq:lyapunov-pd}
V_{_{PD}}(\vec{q}_e,\vec{\omega}_e)=K_p\vec{q}_e\text{}^T\vec{q}_e+K_p(|q_0|-1)^2+\frac{1}{2}\vec{\omega}_e\text{}^TJ_b(u)\vec{\omega}_e
\end{equation}
Recalling the angular velocity differential equation from Eq:\ref{eq:quaternion-states-angular}, $\dot{\vec{\omega}}_b$ is:
\begin{equation}
\dot{\vec{\omega}}_b=J_b\text{}^{-1}(u)\big(-\vec{\omega}_b\times J_b(u)\vec{\omega}_b-\hat{\tau}_b(u)+\vec{\tau}_g+\vec{\tau}_Q+\vec{\tau}_\mu(u)\big)~~~~\in\mathcal{F}^b
\end{equation}
Where $\vec{\tau}_Q$ is a simplified representation of the net aerodynamic torque experienced by the body from the rotating propellers, drawn from Eq:\ref{eq:aerodynamic-torque}. Then, exploiting a unit quaternion's inherent property, it follows that:
\begin{equation}\label{eq:4.17}
\norm{Q}=\vec{q}\text{}^{\hspace{3pt}T}\vec{q}+q_0\text{}^2=\vec{q}\text{}^{\hspace{3pt}2}+q_0\text{}^2=1
\end{equation}
Substituting the angular velocity error state, $\vec{\omega}_e=-\vec{\omega}_b$, the proportional derivative LFC in Eq:\ref{eq:lyapunov-pd} is simplified:
\begin{subequations}\label{eq:4.18}
\begin{equation}
V_{_{PD}}=K_p\vec{q}_e\text{}^2+K_p\big(|q_0|\text{}^2 -2|q_0|\big) + 1 +\frac{1}{2}\vec{\omega}_e\text{}^TJ_b(u)\vec{\omega}_e
\end{equation}
\vspace{-10pt}
\begin{equation}
=2K_p(1-|q_0|)+\frac{1}{2}\vec{\omega}_b\text{}^TJ_b(u)\vec{\omega}_b
\end{equation}
\end{subequations}
Similarly, using the fact that a quaternion's derivative is defined as:
\begin{equation}\label{eq:quat-derivative}
\dot{Q}=\begin{bmatrix}
-\frac{1}{2}\vec{q}^{\hspace{3pt}T}\vec{\omega}\\
\frac{1}{2}\big([\vec{q}]_\times+q_0\mathbb{I}_{3\times 3}\big)\vec{\omega}
\end{bmatrix}
\end{equation}
Substituting the above into the LFC derivative, $\dot{V}_{_{PD}}$, yields:
\begin{subequations}
\begin{equation}
\dot{V}_{_{PD}}=2K_p\frac{1}{2}\vec{q}_e\text{}^T\vec{\omega}_e+\frac{1}{2}\dot{\vec{\omega}}_b\text{}^TJ_b(u)\vec{\omega}_b+\frac{1}{2}\vec{\omega}_bJ_b(u)\dot{\vec{\omega}}_b\text{}^T
\end{equation}
\vspace{-12pt}
\begin{equation}
=-K_p\vec{q}_e\text{}^T\vec{\omega}_b+\vec{\omega}_b\text{}^TJ_b(u)\dot{\vec{\omega}}_b
\end{equation}
\end{subequations}
Expanding the angular acceleration $\dot{\vec{\omega}}_b$ and introducing the PD control law from Eq:\ref{eq:independent-pd}, $\vec{\tau}_{_{PD}}$ substituted back into the LFC derivative:
\begin{subequations}
\begin{equation}
\rightarrow\dot{V}_{_{PD}}=-K_p\vec{q}_e\text{}^T\vec{\omega}_b+\vec{\omega}_b\text{}^T\big(-\vec{\omega}_b\times J_b(u)\vec{\omega}_b-\hat{\tau}_b+\vec{\tau}_g+\vec{\tau}_Q-K_d\vec{\omega}_b+K_p\vec{q}_e\big)
\end{equation}
\vspace{-10pt}
\begin{equation}
=-K_p\vec{q}_e\text{}^T\vec{\omega}_b+K_p\vec{\omega}_b\text{}^T\vec{q}_e-\vec{\omega}_b\text{}^TK_d\vec{\omega}_b+\vec{\omega}_b\text{}^T\big(-\vec{\omega}_b\times J_b(u)\vec{\omega}_b-\hat{\tau}_b+\vec{\tau}_g+\vec{\tau}_Q\big)
\end{equation}
It follows that the transpose term $\vec{q}_e\text{}^T\vec{\omega}_b\iff\vec{\omega}_b\text{}^T\vec{q}_e$ is interchangeable as its resultant product is the same. The LCF derivative then simplifies:
\begin{equation}
\therefore\dot{V}_{_{PD}}=-\vec{\omega}_b\text{}^TK_d\vec{\omega}_b+\vec{\omega}_b\text{}^T\big(-\vec{\omega}_b\times J_b(u)\vec{\omega}_b-\hat{\tau}_b+\vec{\tau}_g+\vec{\tau}_Q\big)
\end{equation}
\end{subequations}
Then, as long as $\big(-\vec{\omega}_b\times J_b(u)\vec{\omega}_b-\hat{\tau}_b+\vec{\tau}_g+\vec{\tau}_Q\big)\leq \vec{0}$, some basic stability is guaranteed. Under specific circumstances the following assumptions can be made to ensure the asymptotic stability proof can be applied. The stability obviously breaks down if any of the assumptions fail, as such the stability is not global\ldots
\vspace{-10pt}
\begin{enumerate}[itemsep=0em]
\item The inertial matrix, $J_b(u)$, is approximately diagonal which, given the inertia ranges from Eq:\ref{eq:inertia-max} and Eq:\ref{eq:inertia-min}, is reasonable. Similarly that the angular rate can be made small with appropriately slow trajectory updates such that the torque gyroscopic cross-product is negligible:
\begin{center}
\vspace{-10pt}
$\vec{\omega}_b^{~T}\big(\vec{\omega}_b\times J_b(u)\vec{\omega}_b\big)\approx\vec{0}$
\vspace{-8pt}
\end{center}
\item The actuator rate torque response, $\hat{\tau}_b(u)$, is a second order effect dependent on $\dot{u}$. Typically the actuator rates are going to be kept small, Fig:\ref{fig:tau-body} shows torques $\hat{\tau}_b(u)$ for a typical trajectory. For slow attitude steps those position changes are small enough to be considered negligible. The approximation is made:
\begin{center}
\vspace{-10pt}
$\hat{\tau}_b(u)\approx\vec{0}$
\vspace{-8pt}
\end{center}
\item Finally, for the sake of the stability proof, the eccentric gravitational torque arm is neglected, $\vec{\tau}_g\approx\vec{0}$. Such a situation only holds true if $u\approx\vec{0}$ or that servo actuator positions are close to their zero positions.
\end{enumerate}
{\emph{\color{Gray}All of the above assumptions are made under extraneous circumstances and can not be assumed for almost all of the prototype's flight envelope. The plant independent case is considered and simulated in Sec\ref{subsubsec:simulation.atttiude.pd.independent} purely for contrition; mainly to demonstrate the need for plant dependent compensation.}
\par
If each of the assumptions made hold true, then the Lyapunov function's derivative is approximately negative definite. The stability proof for a very local trajectory is then:
\begin{subequations}
\begin{equation}
\dot{V}_{_{PD}}\approx-K_p\vec{q}_e\text{}^T\vec{\omega}_b+\vec{\omega}_b\text{}^T\big(-K_d\vec{\omega}_b+K_p\vec{q}_e\big)
\end{equation}
\vspace{-14pt}
\begin{equation}\label{eq:pd-local-stability}
\rightarrow\dot{V}_{_{PD}}=-\vec{\omega}_b\text{}^TK_d\vec{\omega}_b=-K_d\norm{\vec{\omega}_b}\text{}^2<0
\end{equation}
\end{subequations}
From Lyapunov stability theorem there then exists the limits for \emph{local} asymptotic stability: 
\begin{subequations}
\begin{equation}
\lim_{t\rightarrow\infty}\vec{\omega}_e\rightarrow\vec{0}~~\therefore~~\lim_{t\rightarrow\infty}\vec{\omega}_b\rightarrow\vec{0}^{-}
\end{equation}
\vspace{-14pt}
\begin{equation}
\lim_{t\rightarrow\infty}\vec{q}_e\rightarrow \vec{0}~~\text{and}~~\lim_{t\rightarrow\infty}(1-q_0)\rightarrow 0
\end{equation}
\end{subequations}
Hence the quaternion error stabilizes $Q_e\rightarrow[1~\vec{0}\hspace{3pt}]^{T}$ as $t\rightarrow\infty$. The stability shown in Eq:\ref{eq:pd-local-stability} is only local; introducing plant dependent compensation to the PD control law in Eq:\ref{eq:independent-pd} alleviates the stringent requirements on assumptions 1 through 3. Obviously compensation terms are \emph{state estimates} and so a small degree of uncertainty exists (stability is discussed in Sec:\ref{subsec:simulation.comparison.attitude}):
\begin{equation}\label{eq:dependent-pd}
\vec{\tau}_{_{PD}}=\vec{\omega}_b\times J_b(u)\vec{\omega}_b+\hat{\tau}_b(u)-\vec{\tau}_g-\vec{\tau}_Q+K_d\vec{\omega}_e+K_p\vec{q}_e
\end{equation}
The resultant stability proof for the plant dependent case Eq:\ref{eq:dependent-pd} is much the same as that for the independent controller, Eq:\ref{eq:independent-pd}. The same LCF from Eq:\ref{eq:lyapunov-pd} shows that Eq:\ref{eq:pd-local-stability} holds for all global trajectories:
\begin{equation}\label{eq:dependent-global-stability}
\rightarrow\dot{V}_{_{PD}}=-\vec{\omega}_b\text{}^TK_d\vec{\omega}_b=-K_d\norm{\vec{\omega}_b}\text{}^2<\vec{0}~~\forall(Q_e,\vec{\omega}_b)
\end{equation}
A plant dependent controller is not reliant on the very limiting assumptions needed for independent asymptotic stability to be achieved. The dynamic compensation in Eq:\ref{eq:dependent-pd} is simple to implement; especially considering the form of unwanted dynamics which have already quantified and corroborated previously in Sec:\ref{subsec:dynamics.nonlinearities.torque-tests}.
%====================================================
\subsubsection{Auxiliary Plant Controller}
\label{subsubsec:control.attitude.controllers.auxpd}
%====================================================
Expanding on what has, in practice (Table:\ref{tab:controllers}, Sec:\ref{subsec:intro.lit.related}), proven to be a popular and effective controller for attitude stabilization, \cite{attitudestabilization} proposed an auxiliary plant term to a P-D attitude controller. Most significantly, the altered PD controller adds auxiliary terms proportional to the quaternion rate error (Eq:\ref{eq:quaternion-deriv}). Moreover part of the auxiliary plant is proportional to the \emph{quaternion scalar} $q_0$, a term that is otherwise neglected in the previous PD control law (Sec:\ref{subsubsec:control.attitude.controllers.pd}) and prevents unwinding if incorporated. The \emph{auxilliarly} PD control torque is a function of errors states:
\begin{equation}\label{eq:control-aux-pd}
\vec{\tau}_{_{XPD}}=\underbrace{-\Gamma_2{\widetilde{\Omega}}-\Gamma_3\vec{q}_e+J_b(u)\dot{\bar{\Omega}}}_{Independent}+\underbrace{\vec{\omega}_b\times J_b(u)\vec{\omega}_b+\hat{\tau}_b-\vec{\tau}_g-\vec{\tau}_Q}_{Compensation}
\end{equation}
Wherein the coefficients $\Gamma_2$ and $\Gamma_3$ are both diagonal positive definite coefficient matrices whilst $\Gamma_1$, used next in Eq:\ref{eq:aux-pd-1}, is a symmetrical matrix. Each coefficient matrix is explicitly determined later. Auxiliary plants $\widetilde{\Omega}$ and $\dot{\bar{\Omega}}$ are defined as follows. The first auxiliary plant $\bar{\Omega}$ is proportional to the quaternion error and hence its derivative $\dot{\bar{\Omega}}$ is a quaternion rate:
\begin{subequations}\label{eq:aux-pd-1}
\begin{equation}
\bar{\Omega}\triangleq-\Gamma_1\vec{q}_e~~\therefore~~\dot{\bar{\Omega}}=-\Gamma_1\dot{\vec{q}}_e
\end{equation}
\vspace{-15pt}
\begin{equation}
\rightarrow\dot{\bar{\Omega}}=-\frac{1}{2}\Gamma_1\big(q_0\mathbb{I}_{3X3}+[\vec{q}_e]_{\times}\big)\vec{\omega}_e
\end{equation}
\vspace{-10pt}
\begin{equation}
=\frac{1}{2}\Gamma_1\big(q_0\mathbb{I}_{3X3}+[\vec{q}_e]_{\times}\big)\vec{\omega}_b\Big|_{\vec{\omega}_e=-\vec{\omega}_b}
\end{equation}
\end{subequations}
The second auxiliary plant, $\widetilde{\Omega}$, is proportional to both quaternion vector and angular velocity errors.
\begin{subequations}\label{eq:aux-pd-2}
\begin{equation}
\widetilde{\Omega}\triangleq\vec{\omega}_e-\bar{\Omega}=\vec{\omega}_e+\Gamma_1\vec{q}_e
\end{equation}
\vspace{-15pt}
\begin{equation}
=-\vec{\omega}_b+\Gamma_1\vec{q}_e\Big|_{\vec{\omega_e}=-\vec{\omega}_b}
\end{equation}
\end{subequations}
Using an LFC similar to the basic $V_{_{PD}}$ function candidate from Eq:\ref{eq:lyapunov-pd}, but substituting an auxiliary term $\widetilde{\Omega}$ for the body's angular velocity $\vec{\omega}_b$ into the LFC $V_{_{XPD}}$:
\begin{equation}\label{eq:lyapunov-xpd}
V_{_{XPD}}\big(\vec{q}_e,~\widetilde{\Omega}\big)=\vec{q}_e\text{}^T\vec{q}_e+\big(|q_0|-1\big)^2+\frac{1}{2}\widetilde{\Omega}\text{}^{\hspace{1pt}T}\big(\Gamma_3^{-1}J_b(u)\big)\widetilde{\Omega}
\end{equation}
Again using the simplification from a quaternion's inherent properties in Eq:\ref{eq:4.17}, the LFC from Eq:\ref{eq:lyapunov-xpd} then simplifies with the following derivative:
\begin{subequations}
\vspace{-5pt}
\begin{equation}
V_{_{XPD}}=2(1-|q_0|)+\frac{1}{2}\widetilde{\Omega}\text{}^{\hspace{1pt}T}\big(\Gamma_3^{-1}J_b(u)\big)\widetilde{\Omega}
\end{equation}
\vspace{-11pt}
\begin{equation}
\dot{V}_{_{XPD}}=2\frac{1}{2}\vec{q}_e\text{}^T\vec{\omega}_e+\frac{1}{2}\dot{\widetilde{\Omega}}\text{}^{\hspace{1pt}T}\big(\Gamma_3^{-1}J_b(u)\big)\widetilde{\Omega}+\frac{1}{2}\widetilde{\Omega}\text{}^{\hspace{1pt}T}\big(\Gamma_3^{-1}J_b(u)\big)\dot{\widetilde{\Omega}}
\end{equation}
\vspace{-8pt}
\begin{equation}\label{eq:4.34c}
\dot{V}_{_{XPD}}=-\vec{q}_e\text{}^T\vec{\omega}_b+\frac{1}{2}\dot{\widetilde{\Omega}}\text{}^{\hspace{1pt}T}\big(\Gamma_3^{-1}J_b(u)\big)\widetilde{\Omega}+\frac{1}{2}\widetilde{\Omega}\text{}^{\hspace{1pt}T}\big(\Gamma_3^{-1}J_b(u)\big)\dot{\widetilde{\Omega}}\Big|_{\vec{\omega}_e=-\vec{\omega}_b}
\end{equation}
\end{subequations}
It then follows, substituting $\dot{\vec{\omega}}_b$ from Eq:\ref{eq:aux-pd-2}, the auxiliary plant's derivative $\dot{\widetilde{\Omega}}$ is:
\begin{subequations}
\vspace{-6pt}
\begin{equation}
\dot{\widetilde{\Omega}}=-\dot{\vec{\omega}}_b+\Gamma_1\dot{\bar{\Omega}}
\end{equation}
\vspace{-15pt}
\begin{equation}
\rightarrow\dot{\vec{\omega}}_b=J_b^{-1}(u)\big(-\vec{\omega}_b\times J_b(u)\vec{\omega}_b-\hat{\tau}_b+\vec{\tau}_g+\vec{\tau}_Q+\vec{\tau}_{_{XPD}}\big)
\end{equation}
\vspace{-10pt}
\begin{equation}
\therefore\dot{\widetilde{\Omega}}=-J_b^{-1}(u)\big(-\vec{\omega}_b\times J_b(u)\vec{\omega}_b-\hat{\tau}_b+\vec{\tau}_g+\vec{\tau}_Q+\vec{\tau}_{_{XPD}}\big)-\Gamma_1\dot{\bar{\Omega}}
\end{equation}
Substituting the auxiliary PD control law, $\vec{\tau}_{_{XPD}}$ from Eq:\ref{eq:control-aux-pd}, into the auxiliary derivative $\dot{\widetilde{\Omega}}$:
\begin{equation}
\rightarrow\dot{\widetilde{\Omega}}=J_b^{-1}(u)\big(J_b(u)\dot{\bar{\Omega}}-\Gamma_2\widetilde{\Omega}-\Gamma_3\vec{q}_e\big)-\dot{\bar{\Omega}}
\end{equation}
\vspace{-15pt}
\begin{equation}
=J_b^{-1}(u)\big(-\Gamma_2\widetilde{\Omega}-\Gamma_3\vec{q}_e\big)
\end{equation}
\end{subequations}
From the \emph{approximately} diagonal inertial matrix $J_b(u)$ and the positive symmetric (or \emph{diagonal}) properties of the coefficient matrices $\Gamma_1$,$\Gamma_2$ and $\Gamma_3$; the auxiliary plant $\dot{\widetilde{\Omega}}$ has a transpose:
\begin{equation}
\dot{\widetilde{\Omega}}\text{}^{\hspace{1pt}T}=J_b^{-1}\big(-\Gamma_2\widetilde{\Omega}^{\hspace{1pt}T}-\Gamma_3\vec{q}_e\text{}^T\big)
\end{equation}
The PD auxiliary plant component(s) in the LFC derivative $\dot{V}_{_{XPD}}$ in Eq:\ref{eq:lyapunov-xpd} simplifies:
\begin{subequations}
\begin{equation}
\frac{1}{2}\dot{\widetilde{\Omega}}\text{}^{\hspace{1pt}T}\big(\Gamma_3^{-1}J_b(u)\big)\widetilde{\Omega}=\frac{1}{2}\big(-\Gamma_2\widetilde{\Omega}\text{}^{\hspace{1pt}T}-\Gamma_3\vec{q}_e\text{}^T\big)\Gamma_3^{-1}\widetilde{\Omega}
\end{equation}
\vspace{-12pt}
\begin{equation}\label{eq:4.48b}
=\frac{1}{2}\big(-\widetilde{\Omega}\text{}^{\hspace{1pt}T}\Gamma_2\Gamma_3^{-1}\widetilde{\Omega}-\vec{q}_e\text{}^{T}\widetilde{\Omega}\big)
\end{equation}
Substituting Eq:\ref{eq:aux-pd-2} for $\vec{q}_e\text{}^T\widetilde{\Omega}$ into Eq:\ref{eq:4.48b}:
\begin{equation}\label{eq:4.48c}
\rightarrow\frac{1}{2}\dot{\widetilde{\Omega}}^T\big(\Gamma_3^{-}J_b(u)\big)\widetilde{\Omega}=\frac{1}{2}\big(-\widetilde{\Omega}\text{}^{\hspace{1pt}T}\Gamma_2\Gamma_3^{-1}\widetilde{\Omega}+\vec{q}_e\text{}^T\vec{\omega}_b-\vec{q}_e\text{}^T\Gamma_1\vec{q}_e\big)\Big|_{\vec{q}_e\text{}^{T}\widetilde{\Omega}=-\vec{q}_e\text{}^{T}\vec{\omega}_b+\Gamma_1\vec{q}_e\text{}^{T}}
\end{equation}
Similarly, for the transposed counterpart of Eq:\ref{eq:4.48c} in Eq:\ref{eq:4.34c}:
\begin{equation}
\frac{1}{2}\widetilde{\Omega}\text{}^{\hspace{1pt}T}\big(\Gamma_3^{-1}J_b(u)\big)\dot{\widetilde{\Omega}}=\frac{1}{2}\big(-\widetilde{\Omega}\Gamma_2\Gamma_3^{-1}\widetilde{\Omega}\text{}^{\hspace{1pt}T}+\vec{q}_e\vec{\omega}_b\text{}^T-\vec{q}_e\Gamma_1\vec{q}_e\text{}^T\big)
\end{equation}
Which, when substituted back into Eq:\ref{eq:4.34c}, then simplifies the LFC derivative to negative definite:
\end{subequations}
\begin{equation}
\Rightarrow\dot{V}_{_{XPD}}=-\vec{q}_e\text{}^T\Gamma_1\vec{q}_e-\widetilde{\Omega}\Gamma_2\Gamma_3^{-1}\widetilde{\Omega}\text{}^{\hspace{1pt}T}<0~~~~\forall~(\vec{q}_e,\widetilde{\Omega})
\end{equation}
As such, the control law $\vec{\tau}_{_{XPD}}$ asymptomatically stabilizes the attitude plant globally. Both $\widetilde{\Omega}$ and $\vec{q}_e$ tend to $\vec{0}$, or more specifically the following global stability limits exist:
\begin{subequations}
\begin{equation}
\underset{t\rightarrow\infty}{\lim}\vec{q}_e=0~~\text{and}~~\underset{t\rightarrow\infty}{\lim}\widetilde{\Omega}=0
\end{equation}
Then, from the auxiliary plant definition(s) in Eq:\ref{eq:aux-pd-2}, the extended limits present themselves;
\begin{equation}
\underset{t\rightarrow\infty}{\lim}\vec{\omega}_b=\vec{0}\Big|_{\vec{\omega}_d=\vec{0}}~~~\text{and}~~~\underset{t\rightarrow\infty}{\lim}\bar{\Omega}=\vec{0}
\end{equation}
\end{subequations}
Whilst global asymptotic stability is indeed satisfactory, faster exponential stability is obviously more desirable. The stability proof for $V_{_{XPD}}$ can be extended to a stable exponentially bounded trajectory. From a quaternion's inherent definition it follows that $0\leq |q_0| \leq 1$. It can then be stated that:
\begin{equation}\label{eq:4.34}
1-|q_0|\leq 1-q_0^2=\norm{\vec{q}_e}^2
\end{equation}
Exponential stability is a maximum boundedness proof; the relationship Eq:\ref{eq:4.34} can then replace the quaternion scalar term $2(1-|q_0|)$ in $V_{_{XPD}}$ as an upper bound. The LFC is then rewritten in terms of its component's norm(s) to produce a bounding inequality:
\begin{subequations}\label{eq:xpd-ibc}
\begin{equation}
V_{_{XPD}}=\vec{q}_e\text{}^T\vec{q}_e+(|q_0|-1)^2+\frac{1}{2}\widetilde{\Omega}\text{}^{\hspace{1pt}T}\big(\Gamma_3^{-1}J_b(u)\big)\widetilde{\Omega}
\end{equation}
\vspace{-10pt}
\begin{equation}
\rightarrow V_{_{XPD}}\leq 2\norm{\vec{q}_e}\text{}^2+\frac{1}{2}\Gamma_3^{-1}J_b(u)||\widetilde{\Omega}||\text{}^2
\end{equation}
Similarly the LFC's derivative can be written in terms of its norms as:
\begin{equation}
\dot{V}_{_{XPD}}\leq-\Gamma_2\Gamma_3^{-1}||\widetilde{\Omega}||\text{}^2-\Gamma_1\norm{\vec{q}_e}\text{}^2
\end{equation}
\end{subequations}
The LFC, $V_{_{XPD}}$, has a maximum such that:
\begin{equation}
V_{_{XPD}}\leq max \bigg\{ 2,\frac{\lambda_{max}(\Gamma_3^{-1}J_b(u))}{2}\bigg\}\big(\norm{\vec{q}_e}\text{}^2+||\widetilde{\Omega}||\text{}^2\big)
\end{equation}
Where the function $\lambda_{max}$ represents the maximum eigenvalue of its argument; in this case $\Gamma_3^{-1}J_b(u)$. Similarly the \emph{negative definite} LCF derivative is bound by the minimum:
\begin{equation}
\dot{V}_{_{XPD}} \leq -min \big\{ \lambda_{min}(\Gamma_1),\lambda_{min}(\Gamma_2\Gamma_3\text{}^{-1})\big\}\big(\norm{\vec{q}_e}\text{}^2+||\widetilde{\Omega}||^2 \big)
\end{equation}
Therefore there exists some ratio $\alpha>0$ that satisfies the relationship requirement between the LCF and its derivative; $\dot{V}_{_{XPD}}\leq -\alpha V_{_{XPD}}$, where $\alpha$ is defined as the ratio:
\begin{equation}
\alpha=\frac{min\big\{\lambda_{min}(\Gamma_1),\lambda_{min}(\Gamma_2\Gamma_3\text{}^{-1})\big\}}{max\big\{2,\frac{\lambda_{max}(\Gamma_3\text{}^{-1}J_b(u))}{2}\big\}}
\end{equation}
The attitude trajectory $\big(\vec{q}_e(t),\widetilde{\Omega}(t)\big)$ is then exponentially bounded by:
\begin{equation}\label{eq:exponential-pd}
\big(\norm{\vec{q}_e(t)},||\widetilde{\Omega}(t)||\big)\leq Me^{-\alpha t/2}\big(\norm{\vec{q}_e(0)},||\widetilde{\Omega}(0)||\big)
\end{equation}
The bounding exponential coefficient $\alpha$ can be found using maximum Eigen values for the maximum inertia $J_b(u_{\Lambda})$ from Eq:\ref{eq:inertia-max}. Using the relationship in Eq:\ref{eq:exponential-pd} and testing proposed controller coefficients for $\Gamma_1,~\Gamma_2,~\Gamma_3$ the settling rate can be optimized.
\par
\emph{\color{Gray}The above stability proof for the auxiliary attitude controller was expanded upon and derived from \cite{attitudestabilization}, adapted to fit attitude setpoint tracking. Introduction of the quaternion error, which is dependent on the quaternion scalar, dramatically improves controller performance. The exponential stability notably improves settling times and overshoot errors, demonstrated in Sec:\ref{subsec:simulation.attitude.xpd}.}
\newpage
Interestingly a previously \cite{robustattitude} was the precursor for PD based attitude plants with asymptotic exponential stability. That proposed control law first did not make use of any defined \emph{auxiliary plants}, unlike Eq:\ref{eq:control-aux-pd}; however equivalent terms were effectively incorporated. The control law was developed for spacecraft attitude tracking and proposed a very similar exponentially stabilizing control scheme to that of $\vec{\tau}_{_{XPD}}$. That controller, when changed to the notational convention used above, designs body torque as:
\begin{equation}\label{eq:control-auxp-pd}
\vec{\tau}^{\hspace{3pt}'}_{_{XPD}}=-\frac{1}{2}\Big[\big([\vec{q}_e]_\times+q_0\mathbb{I}_{3\times 3}\big)\Gamma_1+\alpha\big(1-q_0\mathbb{I}_{3\times 3}\big)\Big]\vec{q}_e-\Gamma_2\vec{\omega}_b\in\mathcal{F}^b
\end{equation} 
Eq:\ref{eq:control-auxp-pd} could easily incorporate plant dependent compensation to accomodate for unwanted non-linear dynamics. Both exponentially stabilizing PD controllers, from Eq:\ref{eq:control-aux-pd} and above Eq:\ref{eq:control-auxp-pd}, bear a striking similarity to the ideal backstepping controllers derived next in Eq:\ref{eq:control-ibc}. 
%====================================================
\subsection{Non-linear Controllers}
\label{subsec:control.attitude.nonlinear}
%====================================================
Backstepping controllers(\cite{satellitebackstepping,intelligentbackstep,backstepslidingmode},etc\ldots) are a popular choice for non-linear attitude control plants. The process, through iterative design, enforces Lyapunov stability criteria to ensure asymptotic stability. A report, \cite{backstepping}, surveys the fundamentals of backstepping procedure. Ideal backstepping control (\emph{IBC}) is a precise control solution which requires exact plant matching, something that is difficult to achieve in practice. Considering that some state estimate, $\hat{\mathbf{x}}(t)$ is used for feedback control. The caveat of IBC control is poor robust stability performance; being especially susceptible to state dependent uncertainty. Unmodelled disturbances could easily drive the energy function away from stability conditions. The ideal backstepping algorithm can then be extended to incorporate such uncertainties. Adatively including disturbance and \emph{estimate} uncertainty into the LFC energy function improves the stability's robustness (Adaptive backstepping control, \emph{ABC}). By Lyapunov's theorem the respective estimation error terms are stabilized.
%====================================================
\subsubsection{Ideal Backstepping Controller}
\label{subsubsec:control.attitude.nonlinear.idealbackstep}
%====================================================
Starting with the ideal case for the first proposed backstepping controller, similar to \cite{satellitebackstepping}; it is assumed the attitude plant described in Eq:\ref{eq:quaternion-states-angular} from the consolidated model in Sec:\ref{sec:dynamics.model} exactly matches the dynamics of the physical prototype. The ideal backstepping controller aims to compensate for the plant's dynamic response to trajectory inputs perfectly. Neglecting uncertainties associated with the dynamic model, the aim here is to apply a stabilizing torque design. Recalling the quaternion tracking error from Eq:\ref{eq:quaternion-error}; $Q_e=Q_b^*\otimes Q_e$, considering the first LFC proposal:
\begin{equation}\label{eq:ibc-lfc-1}
V_1(\vec{q}_e)=\vec{q}_e\text{}^T\vec{q}_e+(|q_0|-1)^2
\end{equation}
The absolute $|q_0|$ quaternion scalar is applied to ensure global stability. After substituting in the quaternion derivatives and \emph{without} using the quaternion simplification in Eq:\ref{eq:4.18}, has a Lie derivative:
\begin{subequations}
\begin{equation}
\dot{V}_1=2\vec{q}_e\text{}^T\frac{1}{2}\big([\vec{q}_e]_\times+q_0\mathbb{I}_{3X3}\big)\vec{\omega}_e-2\big(q_0-1\big)\frac{1}{2}\vec{q}_e\text{}^T\vec{\omega}_e
\end{equation}
\vspace{-5pt}
\begin{equation}
=\vec{q}_e\text{}^T\big([\vec{q}_e]_\times+q_0\mathbb{I}_{3X3}\big)\vec{\omega}_e-q_0\vec{q}_e\text{}^T\vec{\omega}_e+\vec{q}_e\text{}^T\vec{\omega}_e
\end{equation}
Simplifying and then substituting $\vec{\omega}_e=\vec{\omega}_d-\vec{\omega}_b=\vec{0}-\vec{\omega}-b$:
\begin{equation}
=\vec{q}_e\text{}^T[\vec{q}_e]_\times\vec{\omega}_e+\vec{q}_e\text{}^T\vec{\omega}_e
\end{equation}
\vspace{-10pt}
\begin{equation}
=-\vec{q}_e\text{}^T[\vec{q}_e]_\times\vec{\omega}_b-\vec{q}_e\text{}^T\vec{\omega}_b\Big|_{\vec{\omega}_e=-\vec{\omega}_b}
\end{equation}
\end{subequations}
Then choosing the first virtual backstepping control input $\gamma_d$. Note that $\gamma_d$ is used here to \emph{differentiate the backstepping design value} from the trajectory commanded $\vec{\omega}_d$, Eq:\ref{eq:angular-error}. Choosing $\gamma_d$ such that the first LFC Eq:\ref{eq:ibc-lfc-1} is negative definite, $\dot{V}_1<0$:
\begin{equation}
\vec{\omega}_b\Rightarrow\gamma_d=\Gamma_1\vec{q}_e
\end{equation}
Where $\Gamma_1$ is the first symmetric positive definite gain matrix, a fact that is important to stress due to positive definite matrix's invertability. That backstepping input simplifies the LFC derivative $\dot{V}_1$ to the negative definite term:
\begin{subequations}
\begin{equation}
\dot{V}_1=-\vec{q}_e\text{}^T[\vec{q}_e]_\times\gamma_d-\vec{q}_e\text{}^T\gamma_d
\end{equation}
\vspace{-14pt}
\begin{equation}
=-\vec{q}_e\text{}^T[\vec{q}_e]_\times\Gamma_1\vec{q}_e-\vec{q}_e\text{}^T\Gamma_1\vec{q}_e
\end{equation}
And considering a vector cross product with itself has a zero resultant, $\vec{q}_e\text{}^T[\vec{q}_e]_\times=\vec{0}$, $\dot{V}_1$ then reduces:
\begin{equation}
=-\vec{q}_e\text{}^T\Gamma_1\vec{q}_e<0
\end{equation}
\end{subequations}
However, that backstepping input $\gamma_d$ has its own associated error. A stabilizing law $z_1$ needs to control that error:
\begin{subequations}
\begin{equation}
z_1\triangleq\vec{\omega}_b-\gamma_d=\vec{\omega}_b-\Gamma_1\vec{q}_e\Big|_{\gamma_d=\Gamma_1\vec{q}_e}
\end{equation}
\vspace{-8pt}
\begin{equation}
\rightarrow\vec{\omega}_b=z_1+\Gamma_1\vec{q}_e
\end{equation}
\vspace{-10pt}
\begin{equation}
\therefore\dot{V}_1=-\vec{q}_e\text{}^{T}\Gamma_1\vec{q}_e-\vec{q}_e\text{}^{T}z_1
\end{equation}
\end{subequations}
Introducing that error $z_1$ into a second LCF, which expands the first proposed $V_1$. And exploiting the fact that $\Gamma_1$ is symmetrical:
\begin{subequations}
\begin{equation}
V_2(\vec{q}_e,z_1)=V_1(\vec{q}_e)+\frac{1}{2}z_1\text{}^T\Gamma_1^{-1}z_1
\end{equation}
\vspace{-6pt}
\begin{equation}
=\vec{q}_e\text{}^T\vec{q}_e+(|q_0|-1)^2+\frac{1}{2}z_1\text{}^T\Gamma_1^{-1}z_1
\end{equation}
\end{subequations}
That first error $z_1$ has its own time derivative, and recalling the body's angular acceleration $\dot{\vec{\omega}}_b$ from earlier with an undefined input $\vec{\tau}_{IBC}$, which still has plant dependency compensation.
\begin{subequations}
\begin{equation}
\dot{z}_1=\dot{\vec{\omega}}_b-\Gamma_1\dot{\vec{q}}_e
\end{equation}
\vspace{-12pt}
\begin{equation}
=\dot{\vec{\omega}}_b-\frac{\Gamma_1}{2}\big([\vec{q}_e]_\times+q_0\mathbb{I}_{3X3}\big)\vec{\omega}_e
\end{equation}
\vspace{-10pt}
\begin{equation}\label{eq:z-deriv}
=J_b(u)^{-1}\big(-\vec{\omega}_b\times J_b(u)\vec{\omega}_b-\hat{\tau}_b(u)+\vec{\tau}_g+\vec{\tau}_Q+\vec{\tau}_{IBC}\big)+\frac{\Gamma_1}{2}\big([\vec{q}_e]_\times+q_0\mathbb{I}_{3X3}\big)\vec{\omega}_b\Big|_{\vec{\omega}_e=-\vec{\omega}_b}
\end{equation}
\end{subequations}
So then, following from Eq:\ref{eq:z-deriv}, finding the derivative of $\dot{V}_2$ with $\dot{V}_1=-\vec{q}_e\text{}^T(\Gamma_1\vec{q}_e+z_1)$:
\begin{subequations}\label{eq:ibc-deriv}
\begin{multline}
\dot{V}_2=-\vec{q}_e\text{}^T\big(\Gamma_1\vec{q}_e+z_1\big)+z_1\text{}^T\Gamma_1^{-1}\bigg(J_b^{-1}(u)\big(-\vec{\omega}_b\times J_b(u)\vec{\omega}_b-\hat{\tau}_b(u)+\vec{\tau}_g+\vec{\tau}_Q+\vec{\tau}_{IBC}\big)\\+\frac{\Gamma_1}{2}\big([\vec{q}_e]_\times+q_0\mathbb{I}_{3X3}\big)\vec{\omega}_b\bigg)
\end{multline}
\vspace{-25pt}
\begin{multline}
=-\vec{q}_e\text{}^T\Gamma_1\vec{q}_e+z_1\text{}^T\Gamma_1^{-1}\bigg(J_b^{-1}(u)\big(-\vec{\omega}_b\times J_b(u)\vec{\omega}_b-\hat{\tau}_b(u)+\vec{\tau}_g+\vec{\tau}_Q+\vec{\tau}_{IBC}\big)\\-\Gamma_1\vec{q}_e+\frac{\Gamma_1}{2}\big([\vec{q}_e]_\times+q_0\mathbb{I}_{3X3}\big)\vec{\omega}_b\bigg)
\end{multline}
\end{subequations}
So then proposing the compensated stabilizing backstepping control law:
\begin{subequations}\label{eq:control-ibc}
\begin{equation}
\vec{\tau}_{_{IBC}}=J_b(u)\Gamma_1\vec{q}_e-\frac{J_b(u)\Gamma_1}{2}\big([\vec{q}_e]_\times+q_0\mathbb{I}_{3X3}\big)\vec{\omega}_b-J_b(u)\Gamma_2z_1+\vec{\omega}_b\times J_b(u)\vec{\omega}_b+\hat{\tau}_b(u)-\vec{\tau}_g-\vec{\tau}_Q
\end{equation}
Noting that $z_1=\vec{\omega}_b-\Gamma_1\vec{q}_e$, the torque law simplifies:
\begin{multline}
=\underbrace{J_b(u)\Bigg(\Gamma_1(1+\Gamma_2)\vec{q}_e-\frac{\Gamma_1}{2}\big([\vec{q}_e]_\times+q_0\mathbb{I}_{3X3}\big)\vec{\omega}_b-\Gamma_2\vec{\omega}_b\Bigg)}_{Ideal backstepping}
\\
+\underbrace{\vec{\omega}_b\times J_b(u)\vec{\omega}_b+\hat{\tau}_b(u)-\vec{\tau}_g-\vec{\tau}_Q}_{Compenstation}\in\mathcal{F}^{b}
\end{multline}
\end{subequations}
%============================================================================================
With $\Gamma_2$ being another positive definite symmetric coefficient matrix. Then with the control law $\vec{\tau}_{_{IBC}}$ introduced into the LCF derivative $\dot{V}_2$ simplifies to negative definite:
\begin{equation}
\dot{V}_2=-\vec{q}_e\text{}^T\Gamma_1\vec{q}_e-z_1\text{}^T\Gamma_2z_1< 0~~~~\forall~(\vec{q}_e,z_1)
\end{equation}
As such $\vec{q}_e\rightarrow 0$ \& $q_0\rightarrow 1$ as $t\rightarrow\infty$. Similarly $z_1\rightarrow 0$, which leads to the limit:
\begin{equation}
\underset{t\rightarrow\infty}{\lim}(\vec{\omega}_b-\Gamma_1\vec{q}_e)=0
\end{equation} 
Because the quaternion error vector already tends to $0$; $\vec{q}_e\rightarrow 0$, it follows that $\vec{\omega}_b\rightarrow 0$. It can also be said that, from the definition of $\vec{\omega}_e$, that the angular velocity error stabilizes too. There is a distinct similarity in the structure of $\mu\vec{\tau}_{_{IBC}}$ from Eq:\ref{eq:control-ibc} and that of the auxiliary PD controller presented in Eq:\ref{eq:control-aux-pd}. Furthermore, using the same reasoning from Eq:\ref{eq:xpd-ibc}, the exponential stability proof then follows:
\begin{subequations}
\begin{equation}
V_{_{IBC}}\leq V_2=2\norm{\vec{q}_e}\text{}^2+\frac{\Gamma_1^{-1}}{2}\norm{z_1}\text{}^2
\end{equation}
\vspace{-15pt}
\begin{equation}
\dot{V}_{_{IBC}}\leq\dot{V}_2=-\Gamma_1\norm{\vec{q}_e}\text{}^2-\Gamma_2\norm{z_1}\text{}^2
\end{equation}
\end{subequations}
Then both the energy function and its derivative are bound respectively by:
\begin{subequations}
\begin{equation}
V_{_{IBC}}\leq max\bigg\{2,~\frac{\lambda_{max}(\Gamma_1\text{}^{-1})}{2}\bigg\}(\norm{\vec{q}_e}\text{}^2+\norm{z_1}\text{}^2)
\end{equation}
\vspace{-10pt}
\begin{equation}
\dot{V}_{_{IBC}}\leq min\big\{\lambda_{min}(\Gamma_1),~\lambda_{min}(\Gamma_2)\big\}(\norm{\vec{q}_e}\text{}^2+\norm{z_1}\text{}^2)
\end{equation}
\end{subequations}
Which then leads to a similar exponential stability trajectory boundedness such that:
\begin{subequations}
\begin{equation}
\dot{V}_{_{IBC}} \leq \alpha V_{_{IBC}}
\end{equation}
\vspace{-15pt}
\begin{equation}
\therefore \big(\norm{\vec{q}_e(t)},\norm{z_1(t)}\big)\leq Me^{-\alpha t/2}\big(\norm{\vec{q}_e(0)},\norm{z_1(0)}\big)
\end{equation}
\end{subequations}
%====================================================
\subsubsection{Adaptive Backstepping Controller}
\label{subsubsec:control.attitude.nonlinear.adaptivebackstep}
%====================================================
As effective as the control law defined above in Section:\ref{subsubsec:control.attitude.nonlinear.idealbackstep} may be, it lacks suitable disturbance rejection properties. Any plant uncertainties or disturbances encountered would adversely affect the controller in a dramatic manner (Sec:\ref{sec:simulation.comparison}). Introducing a term for lumped uncertainty/disturbance torques, $\vec{L}$, into the dynamic equations leads to:
\begin{equation}
\dot{\vec{\omega}}_b=\mathbb{I}_b\text{}^{-1}\big(-\vec{\omega}_b\times\mathbb{I}_b\vec{\omega}_b+\vec{\tau}_Q+\vec{\tau}_g+\vec{Q}+\vec{L}+\mu\vec{\tau}\big)
\end{equation}
It would obviously be easy to simply introduce a compensation term for $-\vec{L}$ into the control law. In practice, however, it is very difficult to approximate a disturbance term without \emph{apriori} knowledge about any of its properties. Noise compensation in sensors can be done easily due to the known frequency bandwidth which that noise occurs in, the same cannot be said for wind disturbances and the like.
\par
An approximate estimation term $\hat{L}$ has to be used for that disturbance compensation in the designed control torque $\mu\vec{\tau}$. That estimate term is then going to have its own error from the physical disturbance affecting the system:
\begin{equation}\label{eq:estimate-error}
\widetilde{L}=\vec{L}-\hat{L}
\end{equation}
The purpose of adaptive backstepping is to introduce that estimate error term into an LCF and develop a derivative term for $\dot{\hat{L}}$, or a disturbance update law, such that even the estimate error asymptotically stabilizes. Typically, that disturbance update rule is the contribution of satellite and general attitude control papers. Similar terms can be introduced for plant uncertainty which can similarly be adapted for but are not included here\ldots
\par
The estimate error is then introduced into the LCF from an ideal backstepping control, in order for it to be dissipated as per Lyapunov theorem.
\begin{subequations}
\begin{equation}
V_{_{ABC}}(\vec{q}_e,z_1,\widetilde{L})=V_{_{IBC}}(\vec{q}_e,z_1)+\frac{1}{2}\widetilde{L}\text{}^T \Gamma_L^{-1}\widetilde{L}
\end{equation}
\vspace{-10pt}
\begin{equation}
=\vec{q}_e\text{}^T\vec{q}_e+(q_0-1)^2+\frac{1}{2}z_1\text{}^T\Gamma_1^{-1}z_1+\frac{1}{2}\widetilde{L}\text{}^T\Gamma_L^{-1}\widetilde{L}
\end{equation}
\end{subequations}
Where the positive symmetric matrix $\Gamma_L\geq 0\in\mathbb{R}^{3X3}$ is termed as the adaptation gain coefficient matrix. Those particular coefficients determine how responsive the system is to disturbances and the rate at which it adapts to compensate for them. Then, to prove stability one starts with the Lie derivative $\dot{V}_{_{ABC}}$:
\begin{equation}
\dot{V}_{_{ABC}}(\vec{q}_e,~z_1,~\widetilde{L})=\dot{V}_{_{IBC}}(\vec{q}_e,~z_1)+\frac{1}{2}\dot{\widetilde{L}}\text{}^T\Gamma_L\text{}^{-1}\widetilde{L}+\frac{1}{2}\widetilde{L}\text{}^T\Gamma_L\text{}^{-1}\dot{\widetilde{L}}
\end{equation}
Recalling the definition of $\widetilde{L}$ from Eq:\ref{eq:estimate-error}. For its derivative $\dot{\widetilde{L}}$ it's reasonable to assume the dynamics of the physical disturbance $\vec{L}$ are far slower than the time constant of the control system, or that $\dot{\vec{L}}<<\dot{\hat{L}}$. Then it follows:
\begin{equation}
\dot{\widetilde{L}}=\dot{\vec{L}}-\dot{\hat{L}}\approx\vec{0}-\dot{\hat{L}}=-\dot{\hat{L}}
\end{equation}
Substituting that estimation error rate back into the derivative $\dot{V}_{_{ABC}}$, which expands upon Eq:\ref{eq:ibc-deriv}, yields:
\begin{subequations}
\begin{multline}
\dot{V}_{_{ABC}}(\vec{q}_e,z_1,\widetilde{L})=\vec{q}_e\text{}^T(z_1-\Gamma_1\vec{q}_e)+z_1\text{}^T\Gamma_1^{-1}\bigg(\mathbb{I}_b^{-1}\big(-\vec{\omega}_b\times\mathbb{I}_b\vec{\omega}_b+\vec{\tau}_Q+\vec{\tau}_g+\vec{Q}+\vec{L}+\mu\vec{\tau}\big)\\
+\frac{\Gamma_1}{2}\big([\vec{q}_e]_\times+q_0\mathbb{I}_{3X3}\big)\vec{\omega}_b\bigg)-\widetilde{L}\text{}^T\Gamma_L^{-1}\dot{\hat{L}}
\end{multline}
\end{subequations}
And using a similar control law to $\mu\vec{\tau}_{_{IBC}}$, which has a disturbance estimate compensation term:
\begin{subequations}
\begin{equation}
\mu\vec{\tau}_{_{ABC}}=\vec{\omega}_b\times\mathbb{I}_b\vec{\omega}_b-\vec{\tau}_Q-\vec{\tau}_g-\vec{Q}-\hat{L}-\mathbb{I}_b\Gamma_1\vec{q}_e-\frac{\Gamma_1\mathbb{I}_b}{2}\big([\vec{q}_e]_\times+q_0\mathbb{I}_{3X3}\big)\vec{\omega}_b-\mathbb{I}_b\Gamma_2z_1
\end{equation}
Which reduces the energy function's derivative to:
\begin{equation}
\dot{V}_{_{ABC}}=-\vec{q}_e\text{}^T\Gamma_1\vec{q}_e-z_1\text{}^T\Gamma_2z_1+z_1\text{}^T\Gamma_1^{-1}\bigg(\mathbb{I}_b^{-1}\big(\vec{L}-\hat{L}\big)\bigg)-\widetilde{L}\text{}^T\Gamma_L\text{}^{-1}\dot{\hat{L}}
\end{equation}
\vspace{-10pt}
\begin{equation}
=-\vec{q}_e\text{}^T\Gamma_1\vec{q}_e-z_1\text{}^T\Gamma_2z_1+z_1\text{}^T\big(\Gamma_1^{-1}\mathbb{I}_b^{-1}\big)\widetilde{L}-\widetilde{L}\text{}^T\Gamma_L^{-1}\dot{\hat{L}}
\end{equation}
\vspace{-10pt}
\begin{equation}
=-\vec{q}_e\text{}^T\Gamma_1\vec{q}_e-z_1\text{}^T\Gamma_2z_1+\widetilde{L}\text{}^T\Gamma_L^{-1}\big(\Gamma_1^{-1}\Gamma_L\mathbb{I}_b^{-1}z_1-\dot{\hat{L}}\big)
\end{equation}
\end{subequations}
The decision then needs to be made as to how the disturbance estimate is going to be updated, or what $\dot{\hat{L}}$ is defined as. The clear choice would be to compensate for the final term in the LCF, making it purely negative definite:
\begin{equation}
\dot{\hat{L}}=\Gamma_1^{-1}\Gamma_L\mathbb{I}_b^{-1}z_1=\Gamma_1^{-1}\Gamma_L\mathbb{I}_b^{-1}\vec{\omega}_b-\Gamma_L\mathbb{I}_b^{-1}\vec{q}_e
\end{equation}
The disturbance is therefore compensated for and the estimate error is ensured to have asymptotic stability seeing that $V_{_{ABC}}$ is positive definite.
\begin{equation}
\dot{V}_{_{ABC}}=-\vec{q}_e\text{}^T\Gamma_1\vec{q}_e-z_1\text{}^T\Gamma_2z_1<\vec{0}~~~~\forall~(\vec{q}_e,z_1,\widetilde{L})
\end{equation}
\par
Exponential stability for the plant however cannot be proven with the above control and disturbance laws, there is no non-zero estimate error coefficient in the LCF derivative. A lot of work has been done on the statistical nature of disturbance approximation and how best to adapt a non-linear control system to the influence of unwanted disturbances. An interesting approach would be to use the previous disturbance estimate, $\vec{L}=\hat{L}_{n-1}$, such that:
\begin{subequations}
\begin{equation}
\widetilde{L}'=\vec{L}-\hat{L}=(\hat{L}_{n-1}-\hat{L}_n)
\end{equation}
\vspace{-15pt}
\begin{equation}
\dot{\hat{L}}=\Gamma_1^{-1}\Gamma_L\mathbb{I}_b^{-1}z_1+\widetilde{L}'
\end{equation}
\vspace{-10pt}
\begin{equation}
\dot{\hat{L}}=\Gamma_1\text{}^{-1}\Gamma_L\mathbb{I}_b^{-1}\vec{\omega}_b-\Gamma_L\mathbb{I}_b^{-1}\vec{q}_e+(\hat{L}_{n-1}-\hat{L}_n)
\end{equation}
\vspace{-10pt}
\begin{equation}
\therefore\dot{V}_{_{ABC}}'=-\vec{q}_e\text{}^T\Gamma_1\vec{q}_e-z_1\text{}^T\Gamma_2z_1-\widetilde{L}\text{}^T\Gamma_L^{-1}\widetilde{L}'
\end{equation}
\end{subequations}
Given that the starting estimate $\hat{L}_0=\vec{0}$ and that the change of disturbance over a single control cycle is going to be small once the approximator has settled, its fair to assume the following:
\begin{equation}\label{eq:estimator-assumption}
\underset{t\rightarrow\infty}{\lim}\widetilde{L}'=\big(\hat{L}_{n-1}-\hat{L}_n\big)\rightarrow\widetilde{L}
\end{equation}
Then, it leads to the following LCF derivative which can then prove exponential stability. It clear that a coefficient $\dot{V}_{_{ABC}}\leq\alpha V_{_{ABC}}$ exists and can be found:
\begin{equation}
\dot{V}_{_{ABC}}(\vec{q}_e,z_1,\widetilde{L})=-\vec{q}_e\text{}^T\Gamma_1\vec{q}_e-z_1\text{}^T\Gamma_2z_1-\widetilde{L}\text{}^T\Gamma_L^{-1}\widetilde{L}
\end{equation}
The assumption in Eq:\ref{eq:estimator-assumption} is going to need to be tested in simulation later in Chapter:\ref{ch:simulation}; the adaptive gain matrix $\Gamma_L$ is something that will similarly need to be designed. For control coefficients a separate optimization loop will be run, later disturbance will be introduced and the adaptive gain will independently be attained and optimized.
%====================================================
\section{Position Control}
\label{sec:control.position}
%====================================================
Only two control laws for position control are proposed. Due to the nature of Coriolis cross-coupling, an attitude plant can be stabilized independently from the position plant, the converse is however not true. A basic Proportional-Derivative control structure is presented as the reference case, thereafter a more complicated adaptive backstepping control algorithm is derived\ldots
\par
The dynamics for position control, Eq:\ref{eq:quaternion-states-acceleration}, include a coupled angular velocity element.
\begin{equation}\label{eq:position-deriv}
\dot{\vec{v}}_b=m^{-1}\big(-\vec{\omega}_b\times m\vec{v}_b+Q_b^*\otimes m\vec{G}_I\otimes Q_b+\mu\vec{F}\big)~~~~\in\mathcal{F}^b
\end{equation}
Typically, given the standard operating conditions of a quadrotor, it's assumed that $\vec{\omega}_b\approx\vec{0}$. As such the inherent angular velocity coupled dynamics are negligible; $\vec{\omega}_b\times m\vec{v}_b\approx 0$. If the entire state vector, both attitude and position $\mathbf{x}(t)=[\mathcal{E},~Q_b]^T$, of the plant is known then it's easy to compensate for those dynamics rather than making assumptions about their influence on the system given particular operating conditions. That plant dependency can be introduced in the control force $\mu\vec{F}$. 
\par
The translational velocity, $\vec{v}_b$, defined in the body frame is related to the inertial frame through a quaternion transformation:
\begin{equation}
\dot{\mathcal{E}}=Q_b\otimes\vec{v}_b\otimes Q_b^*~~~~\in\mathcal{F}^I
\end{equation}
The difference in reference frames is an important distinction between the position and attitude control loops. Position error is calculated purely as a subtractive term:
\begin{equation}
\mathcal{E}_e=\mathcal{E}_d-\mathcal{E}_b~~~~\in\mathcal{F}^I
\end{equation}
With $\mathcal{E}_d(t)$ being some desired position designed by the trajectory generation block. The translational velocity error can be similarly calculated but, in the same way angular velocity $\vec{\omega}_d=\vec{0}$, the desired translational velocity is zero.
\begin{equation}
\dot{\mathcal{E}}_e=\dot{\mathcal{E}}_d-\dot{\mathcal{E}}_b=-\dot{\mathcal{E}}_b\Big|_{\dot{\mathcal{E}}_d=\vec{0}}
\end{equation}
The objective for position setpoint tracking is analogous to that of the attitude setpoint tracking. In particular the aim is to produce a stabilizing control law that ensures the position tracking error asymptotically tends to $\vec{0}$:
\begin{equation}
\mu\vec{F}=g(\mathcal{E}_e,\dot{\mathcal{E}}_e)~~\text{such that}~~\underset{t\rightarrow\infty}{\lim}\mathcal{E}_e=\vec{0}
\end{equation}
Where $\mu\vec{F}$ is the control force to effect Eq:\ref{eq:position-deriv} $\in\mathcal{F}^b$.
%====================================================
\subsection{PD Controller}
\label{subsec:control.position.pd}
%==================================================== 
Starting with a simple PD structure to use as a reference case. A plant dependent controller designs the net force proportional to both the position error and the first derivative velocity error\footnote{The same P and D coefficient symbols are used for continuity.}.
\begin{equation}\label{eq:position-pd}
\mu\vec{F}_{_{PD}}=K\dot{\mathcal{E}}_e+\alpha\mathcal{E}_e+\vec{\omega}_b\times m\vec{v}_b-m\vec{G}_b~~~~\in\mathcal{F}^b
\end{equation}
For the stability proof the error states must be transformed to the body frame $\mathcal{F}^b$ such that the control input and error states all act in a common frame. So defining an error state in the body frame $X_e$:
\begin{subequations}
\begin{equation}\label{eq:4.80a}
X_e=Q_b^*\otimes(\mathcal{E}_d-\mathcal{E}_b)\otimes Q_b=X_d-X_b
\end{equation}
\vspace{-10pt}
\begin{equation}
\dot{X}_e=Q_b^*\otimes(\dot{\mathcal{E}}_d-\dot{\mathcal{E}}_b)\otimes Q_b=-Q_b^*\otimes\dot{\mathcal{E}}_b\otimes Q_b = -\vec{v}_b\Big|_{\dot{\mathcal{E}}_d=\vec{0}}
\end{equation}
\end{subequations}
As such the control law from Eq:\ref{eq:position-pd}, despite being $\in\mathcal{F}^b$ has arguments $\mathcal{E},\dot{\mathcal{E}}\in\mathcal{F}^I$, which must similarly transform to:
\begin{subequations}
\begin{equation}
\mu\vec{F}_{PD}=K\dot{X}_e+\alpha X_e+\vec{\omega}_b\times m\vec{v}_b-m\vec{G}_b
\end{equation}
\vspace{-10pt}
\begin{equation}
=-K\vec{v}_b+\alpha X_e+\vec{\omega}\times m\vec{v}_b-m\vec{G}_b
\end{equation}
\end{subequations}
Then using a p.d Lyapunov candidate function:
\begin{equation}
V_{_{PD}}(X_e,\dot{X}_e)=\frac{\alpha}{2}X_e\text{}^TX_e+\frac{m}{2}\dot{X}_e\text{}^T\dot{X}_e=\frac{\alpha}{2}X_e\text{}^TX_e+\frac{m}{2}\vec{v}_b\text{}^T\vec{v}_b
\end{equation}
Then calculating the LCF derivative with the PD control law substituted:
\begin{subequations}
\begin{equation}
\dot{V}_{_{PD}}=\alpha X_e\text{}^T\dot{X}_e+\vec{v}_b\text{}^Tm\dot{\vec{v}}_b=-\alpha X_e\text{}^T\vec{v}_b+\vec{v}_b\text{}^Tm\dot{\vec{v}}_b
\end{equation}
\vspace{-10pt}
\begin{equation}
=-\alpha X_e\text{}^T\vec{v}_b+\vec{v}_b\text{}^T\big(-\vec{\omega}_b\times m\vec{v}_b+m\vec{G}_b+\mu\vec{F}_{PD}\big)
\end{equation}
\vspace{-10pt}
\begin{equation}
=-\alpha X_e\text{}^T\vec{v}_b+\vec{v}_b\text{}^T\big(-K\vec{v}_b+\alpha X_e\big)
\end{equation}
\vspace{-10pt}
\begin{equation}
\Rightarrow \dot{V}_{_{PD}}=-K\vec{v}_b\text{}^T\vec{v}_b<\vec{0}~~~~\forall~(X_e,\dot{X}_e)
\end{equation}
\end{subequations}
It then follows that the following global asymptotically\footnote{Not exponentially stabilizing however.} stabilizing limits exist:
\begin{subequations}
\begin{equation}
\underset{t\rightarrow\infty}{\lim}X_e=Q_b^*\otimes(\mathcal{E}_d-\mathcal{E}_b)\otimes Q_b=\vec{0}
\end{equation}
\vspace{-14pt}
\begin{equation}
\therefore\underset{t\rightarrow\infty}{\lim}\mathcal{E}_b=\mathcal{E}_d
\end{equation}
\vspace{-10pt}
\begin{equation}
\underset{t\rightarrow\infty}{\lim}\dot{X}_e=Q_b^*\otimes(\dot{\mathcal{E}}_d-\dot{\mathcal{E}}_b)\otimes Q_b=-\vec{v}_b\Big|_{\dot{\mathcal{E}}_e=0}=0
\end{equation}
\end{subequations}
%====================================================
\subsection{Adaptive Backstepping Controller}
\label{subsec:control.position.bacstepping}
%====================================================
An adaptive backstepping algorithm, similar the attitude controller derived previously in Sec:\ref{subsubsec:control.attitude.nonlinear.adaptivebackstep}, is now applied to position control. The disturbance term, $\vec{D}\in\mathcal{F}^b$, introduced to the differential Eq:\ref{eq:position-deriv} represents any lumped drag and wind forces encountered by the body which weren't quantified numerically in Sec:\ref{subsec:dynamics.aero.drag}. The backstepping iterations of the position control loop first need to stabilize the position error and then compensate for those disturbances\ldots
\begin{equation}
\dot{\vec{v}}_b=m^{-1}\big(-\vec{\omega}_b\times m\vec{v}_b+m\vec{G}_b+\vec{D}_b+\mu\vec{F}\big)~~~~\in\mathcal{F}^b
\end{equation}
Obviously the compensation for $\vec{D}$ is going to be an approximation of that physical disturbance term; $\hat{D}$. Beginning the backstepping process for position with the position state tracking error:
\begin{equation}
z_1=\mathcal{E}_d-\mathcal{E}_b
\end{equation}
Which then has its own derivative:
\begin{equation}
\dot{z}_1=\dot{\mathcal{E}}_d-\dot{\mathcal{E}}_b=Q_b\otimes \big(\vec{0}-\vec{v}_b\big)\otimes Q_b^*= - Q_b\otimes \vec{v}_b\otimes Q_b^*
\end{equation}
Transforming that error, $z_1$, to the body frame $\mathcal{F}^b$, in the same way as Eq:\ref{eq:4.80a}, makes the stability proof more concise. That reference frame transformation doesn't affect the Lie derivative as the energy function's gradient depends on its partial derivative w.r.t it's positional trajectory only, namely $\mathcal{E}_e(t)$.
\begin{subequations}
\begin{equation}
\hat{z}_1=X_e=Q_b^*\otimes z_1 \otimes Q_b =Q_b^*\otimes\big(\mathcal{E}_d-\mathcal{E}_b\big)\otimes Q_b
\end{equation}
\vspace{-12pt}
\begin{equation}
\therefore \dot{\hat{z}}_1=Q_b^*\otimes\dot{z}_1\otimes Q_b = Q_b^*\otimes\big(\dot{\mathcal{E}}_d-\dot{\mathcal{E}}_b\big)\otimes Q_b = -\vec{v}_b
\end{equation}
\end{subequations}
Then proposing the first positive definite LCF, $V_1(\hat{z}_1)$, in terms of that tracking error:
\par
\vspace{-12pt}
\begin{subequations}
\begin{equation}
V_1(\hat{z}_1)=\frac{1}{2}\hat{z}_1\text{}^{T}\hat{z}_1
\end{equation}
\vspace{-10pt}
\begin{equation}
\Rightarrow\dot{V}_1=\hat{z}_1\text{}^T\dot{\hat{z}}_1=-\hat{z}_1\text{}^T\vec{v}_b\Big|_{\dot{\mathcal{E}}_d=\vec{0}}
\end{equation}
\end{subequations}
The first stabilizing velocity function\footnote{Using $\Omega_d$ to differentiate from $\vec{v}_d$ which would otherwise be the translational velocity produced from the desired trajectory\ldots}, $\Omega_d$, and its associated error, $\hat{z}_2$, can be defined as:
\begin{subequations}
\begin{equation}
\vec{v}_b\Rightarrow\Omega_d = \Gamma_1 \hat{z}_1
\end{equation}
\vspace{-15pt}
\begin{equation}
\hat{z}_2 = \Omega_d - \vec{v}_b = \Gamma_1\hat{z}_1-\vec{v}_b
\end{equation}
\vspace{-15pt}
\begin{equation}\label{eq:4.90c}
\therefore \vec{v}_b=\Gamma_1\hat{z}_1-\hat{z}_2
\end{equation}
\end{subequations}
So that second error state $\hat{z}_2$ has a derivative:
\begin{subequations}
\begin{equation}
\dot{\hat{z}}_2=\dot{\Omega}_d-\dot{\vec{v}}_b=\Gamma_1\dot{\hat{z}}_1-m^{-1}\big(-\vec{\omega}_b\times m\vec{v}_b+m\vec{G}_b+\vec{D}_b+\mu\vec{F}\big)
\end{equation}
\vspace{-15pt}
\begin{equation}
=-\Gamma_1\vec{v}_b-m^{-1}\big(-\vec{\omega}_b\times m\vec{v}_b+m\vec{G}_b+\vec{D}_b+\mu\vec{F}\big)
\end{equation}
\end{subequations}
Introducing that second error $\hat{z}_2$ into a new LCF $V_2$:
\begin{equation}
V_2(\hat{z}_1,\hat{z}_2)=V_1(\hat{z}_1)+\frac{1}{2}\hat{z}_2\text{}^T\hat{z}_2=\frac{1}{2}\hat{z}_1\text{}^T\hat{z}_1+\frac{1}{2}\hat{z}_2\text{}^T\hat{z}_2
\end{equation}
Which has a derivative:
\begin{subequations}
\begin{equation}
\dot{V}_2=\hat{z}_1\text{}^T\dot{\hat{z}}_1+\hat{z}_2\text{}^T\dot{\hat{z}}_2=-\hat{z}_1\text{}^T\vec{v}_b+\hat{z}_2\text{}^T\dot{\hat{z}}_2
\end{equation}
\vspace{-10pt}
\begin{equation}
=-\hat{z}_1\text{}^T\vec{v}_b+\hat{z}_2\text{}^T\bigg(-\Gamma_1\vec{v}_b-m^{-1}\big(-\vec{\omega}_b\times m\vec{v}_b+m\vec{G}_b+\vec{D}_b+\mu\vec{F}\big)\bigg)
\end{equation}
And substituting Eq:\ref{eq:4.90c} for $\vec{v}_b$ into only the first energy term of the LCF derivative. Specifically; $-\hat{z}_1\text{}^T\vec{v}_b=-\hat{z}_1\text{}^T\big(\Gamma_1\hat{z}_1-\hat{z}_2\big)$. The remaining terms for $\vec{v}_b$ are left unchanged:
\begin{equation}
=-\hat{z}_1\text{}^T\big(\Gamma_1\hat{z}_1-\hat{z}_2\big)+\hat{z}_2\text{}^T\bigg(-\Gamma_1\vec{v}_b-m^{-1}\big(-\vec{\omega}_b\times m\vec{v}_b+m\vec{G}_b+\vec{D}_b+\mu\vec{F}\big)\bigg)
\end{equation}
\vspace{-5pt}
\begin{equation}
=-\hat{z}_1\text{}^T\Gamma_1\hat{z}_1+\hat{z}_2\text{}^T\bigg(-\hat{z
}_1-\Gamma_1\vec{v}_b-m^{-1}\big(-\vec{\omega}_b\times m\vec{v}_b+m\vec{G}_b+\vec{D}_b+\mu\vec{F}\big)\bigg)
\end{equation}
\end{subequations}
An ideal backstepping control law, with the assumption that $\vec{D}_b$ is precisely known, is then:
\begin{subequations}
\begin{equation}
\mu\vec{F}_{_{IBC}}=\vec{\omega}_b\times m\vec{v}_b-m\vec{G}_b-\vec{D}_b-m\hat{z}_1-m\Gamma_1\vec{v}_b+m\Gamma_2\hat{z}_2
\end{equation}
\vspace{-12pt}
\begin{equation}
=\vec{\omega}_b\times m\vec{v}_b-m\vec{G}_b-\vec{D}_b+\big(\Gamma_1\Gamma_2-m\big)\hat{z}_1-m\big(\Gamma_1+\Gamma_2\big)\vec{v}_b
\end{equation}
\vspace{-10pt}
\begin{equation}
\Rightarrow \dot{V}_{_{IBC}}=\dot{V}_2=-\hat{z}_1\text{}^T\Gamma_1\hat{z}_1-\hat{z}_2\text{}^T\Gamma_2\hat{z}_2<\vec{0}~~~~\forall~(\hat{z}_1,\hat{z}_2)~\text{\&}~\forall~(z_1,\hat{z_2})
\end{equation}
\end{subequations}
Which clearly leads to asymptotic (\emph{extended to exponential next}) stability under the assumption that the disturbance term $\vec{D}_b$ is known and can be compensated for well. In the controller both $\Gamma_1$ \& $\Gamma_2$ are symmetric positive definite control coefficient matrices to be determined later\ldots
\par
Adjusting the backstepping rule and proposed LCF to incorporate an adaptive disturbance approximation term $\hat{D}$, similar to the adaptive backstepping attitude controller previously in Sec:\ref{subsubsec:control.attitude.nonlinear.adaptivebackstep}. That approximation leads to an estimation error $\widetilde{D}$, once again assuming the physical disturbance dynamics $\dot{\vec{D}}_b$ are far slower than the control dynamics; $\dot{\vec{D}}_b<<\dot{\hat{D}}$.
\begin{subequations}
\begin{equation}
\widetilde{D}=\vec{D}_b-\hat{D}~~~~\in\mathcal{F}^b
\end{equation}
\vspace{-12pt}
\begin{equation}
\dot{\widetilde{D}}=\dot{\vec{D}}_b-\dot{\hat{D}}\approx\vec{0}-\dot{\hat{D}}=-\dot{\hat{D}}
\end{equation}
\vspace{-10pt}
\begin{equation}
\rightarrow\mu\vec{F}_{_{ABC}}=\vec{\omega}_b\times m\vec{v}_b-m\vec{G}_b-\hat{D}-m\hat{z}_1-m\Gamma_1\vec{v}_d+m\Gamma_2\hat{z}_2
\end{equation}
\vspace{-10pt}
\begin{equation}
=\vec{\omega}_b\times m\vec{v}_b-m\vec{G}_b-\hat{D}+\big(\Gamma_1\Gamma_2-m\big)\hat{z}_1-m\big(\Gamma_1+\Gamma_2\big)\vec{v}_b
\end{equation}
\end{subequations}
Then proposing an LCF which includes that disturbance estimate error $\widetilde{D}$ and finding its derivative:
\begin{subequations}
\begin{equation}
V_{_{ABC}}=\frac{1}{2}\hat{z}_1\text{}^T\hat{z}_1+\frac{1}{2}\hat{z}_2\text{}^T\hat{z}_2+\frac{1}{2}\widetilde{D}\text{}^T\Gamma_D^{-1}\widetilde{D}
\end{equation}
\vspace{-10pt}
\begin{equation}
\Rightarrow\dot{V}_{_{ABC}}=\hat{z}_1\text{}^T\dot{\hat{z}}_1+\hat{z}_2\text{}^T\dot{\hat{z}}_2+\widetilde{D}\text{}^T\Gamma_D^{-1}\dot{\widetilde{D}}
\end{equation}
\vspace{-20pt}
\begin{equation}
=-\hat{z}_1\text{}^T\Gamma_1\hat{z}_1+\hat{z}_2\text{}^T\bigg(-\hat{z}_1-\Gamma_1\vec{v}_b-m^{-1}\big(-\vec{\omega}_b\times m\vec{v}_b+m\vec{G}_b+\vec{D}_b+\mu\vec{F}_{_{ABC}}\big)\bigg)-\widetilde{D}\Gamma_D^{-1}\dot{\hat{D}}
\end{equation}
\vspace{-5pt}
\begin{equation}
=-\hat{z}_1\text{}^T\Gamma_1\hat{z}_1+\hat{z}_2\text{}^T\bigg(-\Gamma_2\hat{z}_2-m^{-1}\big(\vec{D}_b-\hat{D}\big)\bigg)-\widetilde{D}\text{}^T\Gamma_D^{-1}\dot{\hat{D}}
\end{equation}
\vspace{-5pt}
\begin{equation}
=-\hat{z}_1\text{}^T\Gamma_1\hat{z}_1-\hat{z}_2\text{}^T\Gamma_2\hat{z}_2-\hat{z}_2\text{}^Tm^{-1}\widetilde{D}-\widetilde{D}\text{}^T\Gamma_D\text{}^{-1}\dot{\hat{D}}
\end{equation}
\vspace{-10pt}
\begin{equation}
=-\hat{z}_1\text{}^T\Gamma_1\hat{z}_1-\hat{z}_2\text{}^T\Gamma_2\hat{z}_2-\widetilde{D}^T\Gamma_D\text{}^{-1}\big(m^{-1}\Gamma_D\hat{z}_2+\dot{\hat{D}}\big)
\end{equation}
\end{subequations}
Then, a self-evident choice for the disturbance update law would be; $\dot{\hat{D}}=-m^{-1}\Gamma_D\hat{z}_2$, which would ensure asymptotic stability. Expanding on that, an interesting solution which could potentially enforce exponential stability would be to use $\hat{D}_{n-1}$ as a disturbance estimate:
\begin{subequations}
\begin{equation}
\dot{\hat{D}}=-m^{-1}\Gamma_D\hat{z}_2+(\hat{D}_{n-1}-\hat{D})
\end{equation}
\vspace{-12pt}
\begin{equation}
\therefore\dot{V}_{_{ABC}}=-\hat{z}_1\text{}^T\Gamma_1\hat{z}-\hat{z}_2\text{}^T\Gamma_2\hat{z}_2-\widetilde{D}\text{}^T\Gamma_D\text{}^{-1}\big(\hat{D}_{n-1}-\hat{D}_n\big)
\end{equation}
\vspace{-10pt}
\begin{equation}
\approx-\hat{z}_1\text{}^T\Gamma_1\hat{z}_1-\hat{z}_2\text{}^T\Gamma_2\hat{z}_2-\widetilde{D}\text{}^T\Gamma_D\text{}^{-1}\widetilde{D}<\vec{0}~~~~\forall(\hat{z}_1,\hat{z}_2,\widetilde{D})
\end{equation}
\end{subequations}
Similarly for the suggested exponentially stable adaptive backstepping controller for the attitude plant, using $\hat{D}_{n-1}$ for an existing disturbance estimate will need to be simulated in order to ascertain if it's a suitable conjecture. Note here that adaptive control laws refer to adaptability to unknown disturbances and not plant model uncertainty which would take a different approach to the backstepping algorithm.

%****************************************************
%	CHAPTER 5 - Control & Simulations
%****************************************************
\chapter{Simulations \& Results}
\label{ch:simulation}
%****************************************************
\section{Controller Tuning}
\label{sec:simulation.tuning}
%****************************************************
\subsection{Partical Swarm Based Optimization}
\label{subsec:simulation.tuning.pso}
%****************************************************
\subsection{Performance Metric}
\label{subsec:simulation.tuning.metric}
%****************************************************
\subsection{Global \& Local Minima}
\subsection{Fmincon Differences}

%****************************************************
\section{Simulation Block}
\label{sec:simulation.block}
%****************************************************
\section{State Estimation}
\label{sec:simulation.state}
%****************************************************
\section{Optimized Controller Comparisons}
\label{sec:simulation.comparison}
%****************************************************
\subsection{Allocator Performance}
\subsection{Attitude Control Results}
\subsection{Autopilot Outcome}
%****************************************************

%====================================================
%	CHAPTER 6 - Control & Simulations
%====================================================
\chapter{Simulations and Results}
\label{ch:simulation}
%====================================================
\section{Simulator description}
\label{sec:simulation.block}
%====================================================
The proposed attitude and position control laws, together with the system's equations of motion including each actuator's transfer function, were all tested in simulation to determine a particular controller's efficacy. The rigid-body equations of motion from Sec:\ref{subsec:dynamics.rigidbody.lagrange}, with nonlinearities from Sec:\ref{sec:dynamics.aero} and multibody responses from Sec:\ref{sec:dynamics.nonlinearities}, were incorporated into a high fidelity simulation environment. Closely matching the dynamics of the physical quadrotor prototype proposed in Sec:\ref{sec:proto.design}; where measurement data produced by tests in Sec:\ref{subsec:dynamics.nonlinearities.torque-tests} provide a degree of confidence in the simulation's accuracy. The consolidated quaternion dynamics in Sec:\ref{sec:dynamics.model} formed the basis of the simulation, building a loop extended from the control structure in Fig:\ref{fig:control-block}. Each control law is optimized first without the effect of the servo's $180$\textdegree ~saturation limit. Limiting the servos was a conscious design decision, and so its effects are investigated in Sec:\ref{sec:simulation.saturation}. For now, the servos are treated as continuous rotational actuators without saturation limits.
\par
\begin{figure}[htbp]
\centering
\includegraphics[width=\textwidth]{figs/simulation-block}
\vspace{-10pt}
\caption{Simulation loop}
\label{fig:simulation-block}
\end{figure}
\par
An abstracted simulation loop is illustrated in Fig:\ref{fig:simulation-block}, incorporating both attitude and position control loops together with the additive nonlinearities. Certain feedback elements were omitted to retain clarity in the diagram, both Coriolis and Gyroscopic nonlinear cross-products were included to highlight the inherent coupling between attitude and position. Not shown, but implied, is some form of state-estimation or discretization between the state-tracking output $y=\vec{\mathbf{x}}=[\vec{\mathcal{E}}_I~Q_b\hspace{2pt}]^T$ and the feedback state estimate $\hat{\mathbf{x}}$ used for setpoint tracking. Discretized effects of state-estimators are discussed later in Sec:\ref{sec:simulation.state}. Initial conditions for each state's integrator, both position $\vec{\mathcal{E}}_I(0)$ and attitude $Q_b(0)$ origins, for their velocities $\dot{\mathcal{E}}_b$ and $\dot{Q}_b$ (and accelerations $\dot{\vec{v}}_b$ and $\dot{\vec{\omega}}_b$ in the body frame $\mathcal{F}^b$) are not illustrated but implied. Obviously starting conditions are important for each trajectory's simulation but are specifically defined for each simulation in question. Actuator transfer functions from Sec:\ref{subsec:proto.design.transfer} are combined into a bundled $C_{i}(S)$ block, accounting for transfer functions and nonlinear saturation limits of each motor module. Each bundled input $u_{[1:4]}$ is similarly the projected actuator matrix:
\begin{equation}
u_i = \begin{bmatrix}
\Omega_i & \lambda_i & \alpha_i
\end{bmatrix}~~~~\text{for}~i\in[1:4]
\end{equation}
The resultant thrust vector $\vec{T}_i$ produced by each motor module has a net transfer function as a result of the propeller speed and servo rotation dynamics. Lastly, setpoints for both attitude and position states are either stepped or produced from an orbital trajectory. The former is used for controller optimization whilst the latter is used setpoint tracking performance evaluation. To discuss the question of non-zero setpoint tracking an orbital trajectory with an increasing orbital (\emph{chirp}) frequency generates attitude and position setpoints, illustrated in Fig:\ref{fig:trajectory}.
\begin{figure}[htbp]
\vspace{-10pt}
\centering
\includegraphics[width=0.9\textwidth]{figs/trajectory}
\vspace{-12pt}
\caption{Orbital trajectory}
\label{fig:trajectory}
\vspace{-20pt}
\end{figure}
\par
The trajectory generates only first order attitude and position setpoints, furthermore the trajectory setpoint is a body's attitude and inertial position, no actuator values for the aircraft's configuration are commanded by a trajectory. For a central point in the inertial frame $\vec{C}_0\in\mathcal{F}^I$, the trajectory orbits with a \emph{chirpyness} (frequency rate) of $\dot{\omega}~[\text{Hz.s}^{-1}]$ around that center. The orbit is at a height of $\hat{z}_c~[\text{m}]$ and at a radius $R_c~[\text{m}]$ from the center $\vec{C}$. The position setpoint then follows:
\begin{subequations}
\begin{equation}\label{eq:position-trajectory}
\vec{\mathcal{E}}_{d}(t)=\begin{bmatrix}
C_{0x}+R_c\cos\big(\dot{\omega}(t^2)\big)\\
C_{0y}+R_c\sin\big(\dot{\omega}(t^2)\big)\\
\hat{z}_c
\end{bmatrix}~~~~\in\mathcal{F}^I
\end{equation}
{\emph{\color{Gray}The frequency rate $\dot{\omega}$ used in Eq:\ref{eq:position-trajectory} describes the rate at which a chirp generated trajectory increases. It is not the same as an body's angular velocity $\vec{\omega}_b$. Convention has it that a signals frequency is annotated by $\omega$.}}
\par
The time varying trajectory's attitude setpoint is aligned with a normal vector $\hat{n}_d(t)$, banking the vehicle away from the center point $\vec{C}_0$:
\begin{equation}
\hat{n}_d(t)\triangleq\frac{\vec{\mathcal{E}}_d(t)-\vec{C}_0}{\sqrt{\hat{z}_c^2+R_c^2}}
\end{equation}
Using that normal $\hat{n}_d$ to construct a quaternion setpoint $Q_d(t)$ which varies together with the orbital trajectory:
\begin{equation}
Q_d(t)=\begin{bmatrix}
\sin \frac{\theta(t)}{2} & \cos \frac{\theta(t)}{2}\hat{n}_d(t)
\end{bmatrix}^T
\end{equation}
\end{subequations}
Whilst the trajectory equation itself may have a non-zero time derivative, both attitude and position controllers still apply the respective rate setpoints to the state variables $\dot{\vec{\mathcal{E}}}_d(t)=\vec{0}$ and $\dot{Q}_d(t)=\vec{0}$ throughout the entire trajectory, as per Eq:\ref{eq:4.85a} and Eq:\ref{eq:4.27c}. The plot in Fig:\ref{fig:Trajectory_Plot} shows a chirp quaternion and attitude trajectory, $Q_d(t)$ and $\vec{\mathcal{E}}_d(t)$ respectively, over a time period of $240~[\text{s}]$. The trajectory starts at $\vec{\mathcal{E}}_d(t_0)=[6~6~5]^T~[\text{m}]$, which produces a non-zero starting attitude $Q_d(t_0)=[0.38~0.42~0.24~0.7]^T$.\ref{fig:Position_Trajectory}.
\begin{figure}[htbp]
\vspace{-6pt}
\centering
\begin{subfigure}{\textwidth}
\centering
\includegraphics[width=0.8\textwidth]{graphs/Attitude_Trajectory}
\vspace{-6pt}
\caption{Quaternion setpoints}
\label{fig:Attitude_Trajectory}
\end{subfigure}
\begin{subfigure}{\textwidth}
\centering
\includegraphics[width=0.8\textwidth]{graphs/Position_Trajectory}
\vspace{-6pt}
\caption{Position setpoints}
\label{fig:Position_Trajectory}
\end{subfigure}
\vspace{-8pt}
\caption{Chirp trajectory plots}
\label{fig:Trajectory_Plot}
\vspace{-12pt}
\end{figure}
%====================================================
\section{Controller Tuning}
\label{sec:simulation.tuning}
%====================================================
Each proposed control law's proven stability (in Sec:\ref{sec:control.attitude} and Sec:\ref{sec:control.position} for attitude and postition controllers respectively) demonstrates only a control law's setpoint tracking (error stability) for a time $t\rightarrow\infty$. That is the caveat with Lyapunov stability theorem, a trajectory is shown to be stabilizing, however no further insight into the controller coefficient design is provided. Often at the coefficient selection stage a \emph{Monte Carlo} approach is applied to select and optimize the controller coefficients.
%====================================================
\subsection{Particle Swarm Based Optimization}
\label{subsec:simulation.tuning.pso}
%====================================================
Particle swarm based optimization (\emph{PSO}) has been shown in both \cite{adaptivepso} and \cite{autopilotPSO}, amongst others, to be an effective controller coefficient design tool. The algorithm treats a potential set of controller coefficients as a single \emph{particle} which exists within some defined search space. The collection or \emph{swarm} of possible particles explores the search space directed by both the swarm's previous performance as well as the relative performance of the swarm between each particle. In \cite{particletrajectories} the statistical nature of the swarm's trajectory is discussed, however such investigations are beyond the scope of this work.
\par
In general, the PSO algorithm applies a \emph{gradient free} based search of solutions for a given optimization problem. The lack of a specified cost function gradient is an important distinction which differentiates PSO from other algorithms (note the swarm does update as per a pseudo-velocity function). Often a predefined cost function gradient is required to direct the optimization search at each interval, MatLab's fmincon\cite{fmincon} or Interior-Point optimizer\cite{ipopt} algorithms for example. Interval gradient calculations can be computationally exhaustive and reduce the rate of execution for the entire process. An optimizer's performance is directly proportional to the number of complete iterations it executes, if an iteration has a high degree of complexity (simulation \emph{stiffness}) its solution time is then adversely affected. The PSO algorithm is defined as follows: if there exists a set $\vec{x}$ of $k$ variables, $\vec{x}\in\mathbb{R}^{k\times 1}$ to be optimized. The swarm of particles, starting at particle $\vec{x}_0$, has an $n^{\text{th}}$ interval position $\vec{x}_n$ which progresses through the search space as per a velocity $\vec{v}_n$:
\begin{subequations}
\begin{equation}
\vec{x}_{n+1}=\vec{v}_{n}+x_{n}
\end{equation}
\vspace{-18pt}
\begin{equation}\label{eq:swarm-velocity}
\vec{v}_{n+1}\triangleq w\ast \vec{v}_n+c_1\ast r_1\big(\vec{P}_{best}-\vec{x}_n\big)+c_2\ast r_2\big(\vec{G}_{best}-\vec{x}_n\big)
\end{equation}
\end{subequations}
where each $\ast$ operator in Eq:\ref{eq:swarm-velocity} applies an element-by-element matrix coefficient multiplication. Both $\vec{P}_{best}$ and $\vec{G}_{best}$ are previous swarm positions where local and global optima were respectively achieved. Performance of the swarm's current interval is evaluated as per some cost function, responding to a system's dynamics. Finally $r_1$ and $r_2$ are random seeded $\mathbb{R}^{1\times k}$ exploratory matrices which progress the search direction, biased by the two weighting coefficients $c_1$ and $c_2$. The search is prejudiced toward local optima by $c_1$ whilst $c_2$ directs the swarm toward global optima. Fig:\ref{fig:swarm-trajectory} illustrates how positions of both local and global optima influence subsequent velocities.
\begin{figure}[hbtp]
\centering
\includegraphics[width=0.8\textwidth]{figs/swarm-trajectory}
\vspace{-10pt}
\caption{Swarm trajectory's velocity direction}
\label{fig:swarm-trajectory}
\vspace{-8pt}
\end{figure}
\par
The swarm's interval performance is evaluated by the response of some modelled system to the swarm's position, typically an error deviation away from some desired state. Here, the simulation described in Fig:\ref{fig:simulation-block}, is parsed a swarm of controller coefficients as an argument and the plant's setpoint response is simulated over a series of step tests. Particulars with regards to attitude controller optimization is discussed in Sec:\ref{sec:simulation.attitude}, thereafter position controller optimization is detailed in Sec:\ref{sec:simulation.position}. The objective is for zero-error setpoint tracking so each particle's coefficient performance metric calculates an integral-time-absolute-error (\emph{ITAE}) cost function, \cite{ITAE}.
\begin{equation}
\vec{\zeta}\triangleq\int_{t_0}^{t_\infty}t||\vec{e}(t)||_2.dt
\end{equation}
With an error $\vec{e}(t)$ deviating from the plant's given setpoint. The ITAE integral $\vec{\zeta}$ is calculated over the entire simulation time, or an effective $t_\infty$. The time multiplier ensures setpoint error \emph{and} settling time optimality, punishing overshoot and under-damped or oscillatory like behavior. Generally a PSO algorithm progresses following the flow diagram in Fig:\ref{fig:particle-diagram}. Seeing that each controller was empirically proven to be stable irrespective of it's trajectory, the controller will settle irrespective of the proposed interval coefficient values. A consequence of this is that starting conditions $\vec{x}_0$ were chosen to be a rounded set of unity.
\\
\emph{\color{Gray} Note that $\vec{\mathbf{x}}$ is a state variable in the particle swarm flow diagram Fig:\ref{fig:particle-diagram}, $\vec{j}_n$ was chosen to represent the swarm of particles acted by the optimization algorithm in order to differentiate the two.}
\\
\begin{figure}[htbp]
\centering
\includegraphics[width=0.8\textwidth]{figs/particle-diagram}\vspace{-12pt}
\caption{Particle swarm flow diagram}
\label{fig:particle-diagram}
\vspace{-18pt}
\end{figure}
\par
Termination conditions for the iterative optimization loop either limit the number of iteration cycles performed or break from the process once a result is regarded as sufficiently close to optimal. Each optimization loop was terminated only after a limited $tx=1000$ iterations, testing and evaluating one thousand different swarm values for a series of stepped setpoints. With control coefficients it is difficult to quantify how close to optimal a particular proposed set of coefficients are. As the optimizer progressed through iterations it adapted its bias coefficients $c_1$ and $c_2$ from focusing on global optima to local optima using fractions of the iteration number, refining the way in which it searched for potential controller coefficients.
\begin{equation}
\vec{v}_{n+1}=\vec{v}_n+\frac{tx}{1000}\ast r_1(\vec{P}_{best}-\vec{x}_n)+\frac{1000-tx}{1000}\ast r_2(\vec{G}_{best}-\vec{x}_n)~~~~\text{for}~~tx\in[1:1000]
\end{equation}
Each particle's progression was constrained such that it never violated the Lyapunov stability conditions of its respective control laws, mostly ensuring that the coefficient matrices were kept positive definite and symmetrical.
%====================================================
\section{Attitude Controllers}
\label{sec:simulation.attitude}
%====================================================
Attitude controllers derived in Sec:\ref{sec:control.attitude} were optimized first because of their lack of coupling with the position loop. The position control loop was left in open loop with only a constant hovering force condition applied to control input $\vec{F}_d$ for each test. Pseudo inverse allocation, Sec:\ref{subsec:allocation.allocators.inverse}, was applied to the control loop when testing each attitude controller. To evaluate an individual particle's performance a number of step tests were performed. Each attitude setpoint was first defined in the Euler angle parametrization, being conceptually easier to visualize. Thereafter the attitude setpoints were converted to a desired quaternion attitude and applied to the simulation.
\begin{equation}
\vec{\eta}_d(t)\triangleq \begin{bmatrix}
\phi_d(t)&
\theta_d(t)&
\psi_d(t)
\end{bmatrix}^T\underset{Q}{\iff}Q_d(t)
\end{equation}
Each of the three Euler angles were stepped in the range $[-90\text{\textdegree}:+90\text{\textdegree}]$ at intervals of $30\text{\textdegree}$. This resulted in a test of three hundred and forty three possible attitude setpoints, making a test-space sphere as illustrated in Fig:\ref{fig:attitude-setpoint}. Each attitude step was simulated for $t=15~[\text{s}]$ to allow its settling point, with an initial attitude position always set to the origin $Q_b(t_0)=[1~\vec{0}\hspace{2pt}]$, with a \emph{positive} quaternion scalar. The quaternion error's scalar component was limited to $q_0\in[0:1]$, detailed in Eq:\ref{eq:constrained-quaternion-error} of Sec:\ref{subsec:control.attitude.problem}, to ensure positive definite compatibility of the proposed Lyapunov candidate functions in the control proofs. The effect of an unconstrained, negative quaternion error scalar is illustrated in Fig:\ref{fig:PD_Quaternion_Step}. 
\par
Performance for each attitude step test was evaluated by an ITAE integral for the quaternion error vector and angular velocity error. Note that each gain coefficient in a particle has its own local and global error, so the performance metric $\vec{\zeta}_{q}$ is a \emph{vector}, not a scalar quantity:
\begin{equation}\label{eq:attitude-performance}
\vec{\zeta}_{Q}=\int_{t=0}^{15}C_Q*t*q_0*||\vec{q}_e(t)||.dt+\int_{t=0}^{15}C_\omega*t*||\vec{\omega}_e(t)||.dt~~~~\in\mathbb{R}^{3}
\end{equation}
Weighting coefficients $C_Q$ and $Q_\omega$ balance priority of either quaternion or angular velocity tracking, however, tracking both were equally important and so those weights were kept at unit with respective scalar units. The cost integral in Eq:\ref{eq:attitude-performance} was averaged over all three hundred and forty three possible attitude steps to determine the overall performance of a proposed swarm of controller coefficients.
\begin{figure}[htbp]
\vspace{-6pt}
\centering
\includegraphics[width=0.72\textwidth]{figs/attitude-setpoint}
\vspace{-6pt}
\caption{Attitude setpoint working space}
\vspace{-14pt}
\label{fig:attitude-setpoint}
\end{figure}
\par
The integral in Eq:\ref{eq:attitude-performance} produces a $\mathbb{R}^{3}$ vector result. Each coefficient in a particular controller contributes towards a local error in one of the $\hat{X},\hat{Y},\hat{Z}$ components, or in certain cases a pair of axial components if the control coefficient is an off-diagonal element. A global error for the performance of each controller is simply the magnitude $||\vec{\zeta}_Q||$. The same global error is applicable to all controllers. 
\par
To compare the relative performance and effectiveness of each optimized control structure a single attitude step was investigated. That attitude change was chosen to be a sizable step in all three Euler angles to demonstrate the actuator dynamics' effect:
\begin{equation}\label{eq:attitude-step-position}
\vec{\eta}_d\triangleq\begin{bmatrix}
\phi_d\\
\theta_d\\
\psi_d\\
\end{bmatrix}
=
\begin{bmatrix*}[r]
-142\text{\textdegree}\\
167\text{\textdegree}\\
-45\text{\textdegree}
\end{bmatrix*}
\underset{Q}{\iff}
\begin{bmatrix}
-0.3254&
0.2226&
-0.2579&
0.8821
\end{bmatrix}^T
\end{equation}
Then each controller's settling time to $95$\% of its final value $t_{95}$ and its relative angular velocity (the setpoint $\vec{\omega}_d=\vec{0}$) for such a step is calculated. Settling time, overshoot and setpoint error are all factors to consider when discussing a controller's efficacy. Lastly, the commanded (virtual) and applied input torque to the actuator set are discussed too. A feasible controller should not induce torque saturation or unachievable input rate changes.
%====================================================
\subsection{PD}
\label{subsec:simulation.attitude.pd}
%====================================================
The first controller evaluated, the Proportional-Derivative structure, is investigated under three different circumstances. Before discussing each of different scenario, it is worth recalling that control structure from Sec:\ref{subsubsec:control.attitude.controllers.pd}. Control torque is designed by two coefficient matrices $K_p$ and $K_d$:
\begin{equation}\label{eq:simulation-attitde-pd}
\vec{\tau}_{_{PD}}=\underbrace{K_p\vec{q}_e+K_d\vec{\omega}_e}_{Independent}+\underbrace{\hat{\omega}_b\times J_b(\hat{u})\hat{\omega}_b+\vec{\tau}_b(\hat{u})-\vec{\tau}_g-\vec{\tau}_H}_{Compensation}~~~~\in\mathcal{F}^{b}
\end{equation}
The first two tests regard both coefficient matrices as purely diagonal, with no skew elements, testing the effect inclusion of plant dependent compensation has on the controller's performance. Finally a plant dependent compensating PD controller is tested \emph{with} symmetrical coefficient matrices. The diagonal coefficient matrices are defined as follows:
\begin{equation}\label{eq:simulation-attitde-pd-diagonal-coefficients}
K_p\triangleq \begin{bmatrix}
K_p(1) & 0 & 0\\
0 & K_p(2) & 0\\
0 & 0 & K_p(3)
\end{bmatrix}
~~\text{and}~~K_d\triangleq \begin{bmatrix}
K_d(1) & 0 & 0\\
0 & K_d(2) & 0\\
0 & 0 & K_d(3)
\end{bmatrix}
\end{equation}
The proportional coefficient $K_p$ acts on $\vec{q}_e$ whilst the derivative coefficient $K_d$ acts on $\vec{\omega}_e$, so local best positions are determined by elements of the error variable upon which each coefficient acts. A globally best position is tested simply with the magnitude $||\vec{z}_q||$ from Eq:\ref{eq:attitude-performance}. Then local and global best coefficient positions are updated if the minimum (best) result is improved on. 
\par
For the symmetrical coefficient case, each off-diagonal element acts on two components of the error states so then their local best positions depend on two elements of the error variables which they're related to. Then local and global coefficient positions are found when skew elements improve on \emph{two} combined error components. The  controller coefficients are structured:
\begin{equation}\label{eq:simulation-attitde-pd-symmetric-coefficients}
K_p\triangleq \begin{bmatrix}
K_p(1) & K_p(4) & K_p(5)\\
K_p(4) & K_p(2) & K_p(6)\\
K_p(5) & K_p(6) & K_p(3)
\end{bmatrix}
~~\text{and}~~K_d\triangleq \begin{bmatrix}
K_d(1) & K_d(4) & K_d(5)\\
K_d(4) & K_d(2) & K_d(6)\\
K_d(5) & K_d(6) & K_d(3)
\end{bmatrix}
\end{equation}
%====================================================
\subsubsection{Independent Performance}
\label{subsubsec:simulation.atttiude.pd.independent}
%====================================================
For the independent controller case, the same diagonal coefficients are used as those for the plant dependent case. The \emph{attitude} compensation terms in Eq:\ref{eq:simulation-attitde-pd} are neglected to produce a plant independent controller. 
\par
Optimizing the diagonal only PD controller produced the following coefficients:
\begin{equation}\label{eq:optimized-pd-independent}
K_p = \begin{bmatrix}
3.5679 & 0 & 0\\
0 & 5.2698 & 0\\
0 & 0 & 6.0695
\end{bmatrix}
~~\text{and}~~K_d = \begin{bmatrix}
9.0150 & 0 & 0\\
0 & 11.4848 & 0\\
0 & 0 & 20.1827
\end{bmatrix}
\end{equation}
Fig:\ref{fig:PD_Diagonal_Independent_Step} plots the quaternion response to an attitude step, described in Eq:\ref{eq:attitude-step-position}. The uncompensated plant never settles to its setpoint, constant steady state errors manifests due to the uncompensated gravitational and aerodynamic torques. The plant does, however, stabilize to steady state in $t = 3.35~\text{s}$. Fig:\ref{fig:PD_Diagonal_Independent_Torque} compares the controller designed and physically actuated input torques, $\vec{\tau}_d$ and $\vec{\tau}_c$ respectively. Actuator transfer functions produce a lagging response to those input changes. The body's angular velocity $\vec{\omega}_b\in\mathcal{F}^{b}$ is shown in Fig:\ref{fig:PD_Diagonal_Independent_Angular}, which changes as an attitude step is applied. Finally, Fig:\ref{fig:PD_Diagonal_Independent_Input} plots the motor modules' actuator inputs, still with sufficient input headroom from saturation despite such a large attitude step. The actuator servos for redirection of each module's produced thrust vector are, however, rate limited.
\\
\emph{\color{Gray}Note that module 2 and module 4 both anti-clockwise propeller directions, represented by negative speeds in Fig:\ref{fig:PD_Diagonal_Independent_Input}. }
\\
\begin{figure}[htbp]
\vspace{-22pt}
\centering
\begin{subfigure}{\textwidth}
\centering
\includegraphics[width=0.7\textwidth]{graphs/PD_Diagonal_Independent_Step}
\vspace{-6pt}
\caption{Quaternion attitude step}
\label{fig:PD_Diagonal_Independent_Step}
\end{subfigure}
\begin{subfigure}{0.49\textwidth}
\centering
\includegraphics[width=\textwidth]{graphs/PD_Diagonal_Independent_Torque}
\vspace{-20pt}
\caption{Plant input torques}
\label{fig:PD_Diagonal_Independent_Torque}
\end{subfigure}
\begin{subfigure}{0.49\textwidth}
\centering
\includegraphics[width=\textwidth]{graphs/PD_Diagonal_Independent_Angular}
\vspace{-20pt}
\caption{Angular velocity}
\label{fig:PD_Diagonal_Independent_Angular}
\end{subfigure}
\begin{subfigure}{\textwidth}
\centering
\includegraphics[width=0.6\textwidth]{graphs/PD_Diagonal_Independent_Input}
\vspace{-8pt}
\caption{Plant actuator inputs}
\label{fig:PD_Diagonal_Independent_Input}
\end{subfigure}
\vspace{-8pt}
\caption{Independent diagonal PD}
\vspace{-26pt}
\end{figure}
%====================================================
\subsubsection{Dependent Performance}
\label{subsubsec:simulation.atttiude.pd.dependent}
%====================================================
The inclusion of a plant independent PD controller is purely for the sake of comparison, indicating the need for plant dependency to account for steady state tracking errors (best illustrated with a trajectory test, later in Fig:\ref{fig:independent_diagonal_trjaectory}). The same controller coefficients from Eq:\ref{eq:optimized-pd-independent} were used to test the controller dependent case, where the controller accounts for plant dynamics in Eq:\ref{eq:simulation-attitde-pd} with feedback compensation.
\par
The standard quaternion attitude step applied in Fig:\ref{fig:PD_Diagonal_Dependent_Step} with plant dependent compensation. The attitude settles in $t_{95}=3.0764~\text{s}$ with a dynamic response much the same as that of the independent case, Fig:\ref{fig:PD_Diagonal_Independent_Step}. However the dependent controller removes steady state tracking errors. The only difference is that the plant dependent controller commands a non-zero steady state torque, illustrated by small increases in the servo actuator input rotations, shown in Fig:\ref{fig:PD_Diagonal_Dependent_Input}. It is interesting to note that the additional torque input is generated from redirection of the thrust vectors and not increasing or decreasing the propeller speeds.
\par
\begin{figure}[htbp]
\vspace{-10pt}
\centering
\begin{subfigure}{\textwidth}
\centering
\includegraphics[width=0.7\textwidth]{graphs/PD_Diagonal_Dependent_Step}
\vspace{-6pt}
\caption{Quaternion attitude step}
\label{fig:PD_Diagonal_Dependent_Step}
\end{subfigure}
\begin{subfigure}{0.49\textwidth}
\centering
\includegraphics[width=\textwidth]{graphs/PD_Diagonal_Dependent_Torque}
\vspace{-20pt}
\caption{Plant input torques}
\label{fig:PD_Diagonal_Dependent_Torque}
\end{subfigure}
\begin{subfigure}{0.49\textwidth}
\centering
\includegraphics[width=\textwidth]{graphs/PD_Diagonal_Dependent_Angular}
\vspace{-20pt}
\caption{Angular velocity}
\label{fig:PD_Diagonal_Dependent_Angular}
\end{subfigure}
\begin{subfigure}{\textwidth}
\centering
\includegraphics[width=0.6\textwidth]{graphs/PD_Diagonal_Dependent_Input}
\vspace{-8pt}
\caption{Plant actuator inputs}
\label{fig:PD_Diagonal_Dependent_Input}
\end{subfigure}
\vspace{-8pt}
\caption{Dependent diagonal PD}
\vspace{-26pt}
\end{figure}
%====================================================
\subsubsection{Symmetric Controller Performance}
\label{subsubsec:simulation.atttiude.pd.3x3}
%====================================================
The last PD structured attitude controller considers both coefficient matrices with non-zero off-diagonal skew elements. Eq:\ref{eq:simulation-attitde-pd-symmetric-coefficients} shows the structure of both symmetric matrices whose optimized coefficients were found to be:
\begin{equation}\label{eq:optimized-pd-symmetric}
K_p = \begin{bmatrix*}[l]
5.9157 & 0.4165 & 0.4714\\
0.4165 & 7.3141 & 0.4945\\
0.4714 & 0.4945 & 7.3135
\end{bmatrix*}
~~\text{and}~~K_d = \begin{bmatrix*}[l]
17.4318 & 0.45311 & 0.15258\\
0.45311 & 15.3569 & 0.57719\\
0.15258 & 0.57719 & 26.3436
\end{bmatrix*}
\end{equation}
The biggest change the symmetric controller imposes is greater controller gain, $||K_p||_2$ abd $||K_d||_2$ applied to quaternion and angular velocity errors. The increased gain in Eq:\ref{eq:optimized-pd-symmetric} results in larger overshoot and, as a result, slower settling time $t_{95}=3.2993~[\text{s}]$. Neither greater commanded torque, Fig:\ref{fig:PD_3x3_Dependent_Torque}, nor an increased angular velocity spike, Fig:\ref{fig:PD_3x3_Dependent_Angular} are unexpected consequences of a more aggressive control law. More coefficients to be tuned simply meant that optimization intervals to produce Eq:\ref{eq:optimized-pd-symmetric} were perhaps not as effective at reduction of step errors than the diagonal Eq:\ref{eq:optimized-pd-independent}.
\begin{figure}[htbp]
\vspace{-10pt}
\centering
\begin{subfigure}{\textwidth}
\centering
\includegraphics[width=0.7\textwidth]{graphs/PD_3x3_Dependent_Step}
\vspace{-6pt}
\caption{Quaternion attitude step}
\label{fig:PD_3x3_Dependent_Step}
\end{subfigure}
\begin{subfigure}{0.49\textwidth}
\centering
\includegraphics[width=\textwidth]{graphs/PD_3x3_Dependent_Torque}
\vspace{-20pt}
\caption{Plant input torques}
\label{fig:PD_3x3_Dependent_Torque}
\end{subfigure}
\begin{subfigure}{0.49\textwidth}
\centering
\includegraphics[width=\textwidth]{graphs/PD_3x3_Dependent_Angular}
\vspace{-20pt}
\caption{Angular velocity}
\label{fig:PD_3x3_Dependent_Angular}
\end{subfigure}
\begin{subfigure}{\textwidth}
\centering
\includegraphics[width=0.6\textwidth]{graphs/PD_Diagonal_Dependent_Input}
\vspace{-8pt}
\caption{Plant actuator inputs}
\label{fig:PD_3x3_Dependent_Input}
\end{subfigure}
\vspace{-8pt}
\caption{Dependent symmetric PD}
\vspace{-26pt}
\end{figure}
\par
The $[3\times 3]$ symmetric controller's coefficients in Eq:\ref{eq:optimized-pd-symmetric} demonstrate that improving a controller's performance is not as simple as just increasing the controller's gain. To that end, consider the diagonal plant dependent PD controller previously in Sec:\ref{subsubsec:simulation.atttiude.pd.dependent}. If the gain is increased by a scale factor of 2, the settling time decreases to $t_{95}=6.8017~[\text{s}]$ from $t_{95}=3.0764~[\text{s}]$. The attitude step for a PD controller with an increased gain is shown in Fig:\ref{fig:PD_Diagonal_Gain_Step}, with actuator inputs shown in Fig:\ref{fig:PD_Diagonal_Gain_Input}. More gain does not necessarily mean a faster controller.
\begin{figure}[htbp]
\vspace{-10pt}
\centering
\begin{subfigure}{\textwidth}
\centering
\includegraphics[width=0.7\textwidth]{graphs/PD_Diagonal_Gain_Step}
\vspace{-6pt}
\caption{Quaternion attitude step}
\label{fig:PD_Diagonal_Gain_Step}
\end{subfigure}
\begin{subfigure}{\textwidth}
\centering
\includegraphics[width=0.6\textwidth]{graphs/PD_Diagonal_Gain_Input}
\vspace{-8pt}
\caption{Plant actuator inputs}
\label{fig:PD_Diagonal_Gain_Input}
\end{subfigure}
\vspace{-8pt}
\caption{Increased gain PD}
\vspace{-16pt}
\end{figure}
\par
It is worth discussing the constrained quaternion error's scalar limited to $q_0\in[0:1]$. That constraint was imposed during the control stability proof (Sec:\ref{subsec:control.attitude.problem}) where the constrained quaternion ensured that each proposed Lyapunov function candidate was always positive definite. Commanding the same quaternion step from Eq:\ref{eq:attitude-step-position}, however using the negative counterpart of the quaternion error's scalar  $q_0\in[-1:0]$, gives a step response shown in Fig:\ref{fig:PD_Quaternion_Step}. No difference is produced from using $Q_e=[\pm q_0~\vec{q}_e]^T$, despite the requirements of the stability proof. A positive definite LFC \emph{ensures} suitable stability but is by no means the \emph{only condition} at which stability is achieved.
\begin{figure}[hbtp]
\centering
\includegraphics[width=0.7\textwidth]{graphs/PD_Unconstrained_Step}
\vspace{-6pt}
\caption{Unconstrained Error Quaternion attitude step}
\label{fig:PD_Quaternion_Step}
\vspace{-10pt}
\end{figure}
%====================================================
\subsection{Auxiliary Plant Controller}
\label{subsec:simulation.attitude.xpd}
%====================================================
The first of two the exponentially stabilizing controllers is the auxiliary Plant controller from Sec:\ref{subsubsec:control.attitude.controllers.auxpd}. Recalling that controller structure from Eq:\ref{eq:control-aux-pd}:
\begin{subequations}
\begin{equation}\label{eq:simulation-attitude-auxpd}
\vec{\tau}_{_{XPD}}=\underbrace{\Gamma_2{\widetilde{\Omega}}+\Gamma_3\vec{q}_e-J_b(\hat{u})\dot{\bar{\Omega}}}_{Independent}+\underbrace{\hat{\omega}_b\times J_b(\hat{u})\hat{\omega}_b+\vec{\tau}_b(\hat{u})-\vec{\tau}_g-\vec{\tau}_H}_{Compensation}~~~~\in\mathcal{F}^{b}
\end{equation}
where the auxiliary signals $\widetilde{\Omega}$ and $\bar{\Omega}$ are defined, from Eq:\ref{eq:aux-pd-1} and Eq:\ref{eq:aux-pd-2} respectively, as:
\begin{equation}
\bar{\Omega}\triangleq - \Gamma_1\vec{q}_e~~\text{and}~~\widetilde{\Omega}\triangleq -\vec{\omega}_b-\bar{\Omega}
\end{equation}
\end{subequations}
In Eq:\ref{eq:simulation-attitude-auxpd} both coefficients $\Gamma_2$ and $\Gamma_3$ are $[3\times 3]$ diagonal coefficient matrices, whilst $\Gamma_1$ is a symmetrical $[3\times 3]$ gain matrix. Those coefficients are then structured as follows:
\begin{multline}\label{eq:simulation-attitde-auxpd-coefficients}
\Gamma_1\triangleq \begin{bmatrix}
\Gamma_1(1) & \Gamma_1(4) & \Gamma_1(5)\\
\Gamma_1(4) & \Gamma_1(2) & \Gamma_1(6)\\
\Gamma_1(5) & \Gamma_1(6) & \Gamma_1(3)
\end{bmatrix}~,~~
\Gamma_2\triangleq \begin{bmatrix}
\Gamma_2(1) & 0 & 0\\
0 &\Gamma_2(2) & 0\\
0 & 0 & \Gamma_2(3)
\end{bmatrix}
\\
~~\text{and}~~\Gamma_3\triangleq \begin{bmatrix}
\Gamma_3(1) & 0 & 0\\
0 & \Gamma_3(2) & 0\\
0 & 0 & \Gamma_3(3)
\end{bmatrix}
\end{multline}
Global and local best positions of coefficient particles are found from the error state components on which the particular coefficients act. The first gain matrix $\Gamma_1$ acts on both $\vec{q}_e$ and $\vec{\omega}_e$, so its local best position $\vec{P}_{best}$ is when both errors are at their minimum. The remaining two gain matrices $\Gamma_2$ and $\Gamma_3$ act on $\vec{q}_e$ and $\vec{\omega}_e$ respectively, so their local best positions are when each of those errors are minimized. And finally the globally best performing particle position is when $||\vec{\zeta}||_2$ is minimized. The control coefficients produced after $tx=1000$ iterations are as follows:
\begin{multline}\label{eq:optimized-auxpd}
\Gamma_1=\begin{bmatrix*}[r]
3.5924 & -0.2457 & -0.0277\\
-0.2457 & 3.0666 & -0.0602\\
-0.0277 & -0.0602 & 3.3809
\end{bmatrix*}~,~~\Gamma_2=\begin{bmatrix*}[r]
4.6943 & 0 & 0\\
0 & 4.1642 & 0\\
0 & 0 & 6.4109
\end{bmatrix*}\\
~~\text{and}~~\Gamma_3=\begin{bmatrix*}[r]
1.1007 & 0 & 0\\
0 & 1.3369 & 0 \\
0 & 0 & 1.1331
\end{bmatrix*}
\end{multline}
Besides from the stronger exponential stability, another distinctive feature of auxiliary Controller (Eq:\ref{eq:simulation-attitde-auxpd-coefficients}) is the introduced simulation stiffness to the control structure. Calculations for the control input at each interval became more complex as a result of the introduced error terms. Each iteration of the optimizer took longer to simulate, typically in the order of 70-80\% longer, per step test.
\par
The quaternion attitude step response is shown in Fig:\ref{fig:XPD_Step}, settling in $t_{95}=2.3688~\text{s}$ which is notably faster than previous tested controllers. The improved response time does, however, come at the cost of greater input torques, shown in Fig:\ref{fig:XPD_Torque}. Moreover commanded angular velocity change in $\vec{\omega}_b$ (Fig:\ref{fig:XPD_Angular}) is markedly larger than that of previous controllers.
\begin{figure}[hbtp]
\vspace{-8pt}
\centering
\begin{subfigure}{\textwidth}
\centering
\includegraphics[width=0.7\textwidth]{graphs/XPD_Step}
\vspace{-6pt}
\caption{Quaternion attitude step}
\label{fig:XPD_Step}
\end{subfigure}
\vspace{-14pt}
\end{figure}
\newpage
\begin{figure}[htbp]\ContinuedFloat
\begin{subfigure}{0.49\textwidth}
\centering
\includegraphics[width=\textwidth]{graphs/XPD_Torque}
\vspace{-20pt}
\caption{Plant input torques}
\label{fig:XPD_Torque}
\end{subfigure}
\begin{subfigure}{0.49\textwidth}
\centering
\includegraphics[width=\textwidth]{graphs/XPD_Angular}
\vspace{-20pt}
\caption{Angular velocity}
\label{fig:XPD_Angular}
\end{subfigure}
\begin{subfigure}{\textwidth}
\centering
\includegraphics[width=0.6\textwidth]{graphs/XPD_Input}
\vspace{-6pt}
\caption{Plant actuator inputs}
\label{fig:XPD_Input}
\end{subfigure}
\vspace{-8pt}
\caption{Auxiliary Plant PD}
\end{figure}
Despite the improved (23\% faster) settling time the auxiliary PD controller achieves, neither the applied torque inputs (Fig:\ref{fig:XPD_Torque}) nor the actuator commands (Fig:\ref{fig:XPD_Input}) are as aggressive as those of the higher gain, symmetrical PD controller (Fig:\ref{fig:PD_3x3_Dependent_Torque}). This is a direct consequence of the \emph{guaranteed} exponentially bound error trajectory, proven in Sec:\ref{subsubsec:control.attitude.controllers.auxpd}.
%====================================================
\subsection{Ideal and Adaptive Backstepping Controllers}
%====================================================
The second exponentially stabilizing controller and final attitude controller tested is the Ideal Backstepping Controller. Both Ideal and Adaptive backstepping controllers use the same gain coefficients, the difference in structure between the two is the addition of an adaptive disturbance observer to be used for compensation. That disturbance observer and its explicit coefficients are detailed later in Sec:\ref{subsec:simulation.disturbance.torque}. Reiterating the IBC structure from Eq:\ref{eq:control-ibc}:
\begin{equation}\label{eq:simulation-attitude-ibc}
\vec{\tau}_{_{IBC}}=\underbrace{J_b(\hat{u})\Big((\Gamma_1\Gamma_2+1)\vec{q}_e-\Gamma_2\hat{\omega}_b+\Gamma_1\dot{\vec{q}}_e \Big)}_{\text{Ideal backstepping}}
+\underbrace{\hat{\omega}_b\times J_b(\hat{u})\hat{\omega}_b+\vec{\tau}_b(\hat{u})-\vec{\tau}_g-\vec{\tau}_H}_{\text{Compenstation}}~~~~\in\mathcal{F}^{b}
\end{equation}
Wherein the gain matrices $\Gamma_1$ and $\Gamma_2$ are both positive symmetrical $3\times 3$ coefficient matrices:
\begin{equation}\label{eq:simulation-attitde-ibc-coefficients}
\Gamma_1\triangleq \begin{bmatrix}
\Gamma_1(1) & \Gamma_1(4) & \Gamma_1(5)\\
\Gamma_1(4) & \Gamma_1(2) & \Gamma_1(6)\\
\Gamma_1(5) & \Gamma_1(6) & \Gamma_1(3)
\end{bmatrix}
~~\text{and}~~
\Gamma_2\triangleq \begin{bmatrix}
\Gamma_2(1) & \Gamma_2(4) & \Gamma_2(5)\\
\Gamma_2(4) & \Gamma_2(2) & \Gamma_2(6)\\
\Gamma_2(5) & \Gamma_2(6) & \Gamma_2(3)
\end{bmatrix}
\end{equation}
Both gain coefficient matrices act on the two error vectors $\vec{q}_e$ and $\vec{\omega}_e$, trying to differentiate the local and global coefficient best positions is then problematic. A particle swarm algorithm needs a clear distinction between local and global best performance positions, equating local and global particle positions reduces the directed swarm search to a randomized \emph{Monte Carlo} method of coefficient selection.
\par
To avoid that reduction, $\Gamma_1$ is prioritized to control the quaternion vector error $\vec{q}_e$, similarly $\Gamma_2$ is dedicated to controlling the angular velocity error $\vec{\omega}_e$. It then follows that local and global best positions, $\vec{P}_{best}$ and $\vec{G}_{best}$ respectively, are found in the same way as the symmetrical PD controller. When optimized, the two sets of gain coefficients are:
\begin{equation}\label{eq:optimized-IBC}
\Gamma_1 = \begin{bmatrix*}[r]
5.8631 & 0.0515 & 1.0221\\
0.0515 & 13.8375 & 0.8533\\
1.0221 & 0.8533 & 11.9644
\end{bmatrix*}
~~\text{and}~~
\Gamma_2 = \begin{bmatrix*}[r]
9.1127 & 0.2887 & 0.1353\\
0.2887 & 6.8389 & 0.1971\\
0.1353 & 0.1871 & 2.5294
\end{bmatrix*}
\end{equation}
\begin{figure}[htbp]
\vspace{-10pt}
\centering
\begin{subfigure}{\textwidth}
\centering
\includegraphics[width=0.7\textwidth]{graphs/IBC_Step}
\vspace{-6pt}
\caption{Quaternion attitude step}
\label{fig:IBC_Step}
\end{subfigure}
\begin{subfigure}{0.49\textwidth}
\centering
\includegraphics[width=\textwidth]{graphs/IBC_Torque}
\vspace{-20pt}
\caption{Plant input torques}
\label{fig:IBC_Torque}
\end{subfigure}
\begin{subfigure}{0.49\textwidth}
\centering
\includegraphics[width=\textwidth]{graphs/IBC_Angular}
\vspace{-20pt}
\caption{Angular velocity}
\label{fig:IBC_Angular}
\end{subfigure}
\begin{subfigure}{\textwidth}
\centering
\includegraphics[width=0.6\textwidth]{graphs/IBC_Input}
\vspace{-8pt}
\caption{Plant actuator inputs}
\label{fig:IBC_Input}
\end{subfigure}
\vspace{-8pt}
\caption{Ideal backstepping controller}
\vspace{-10pt}
\end{figure}
\par
The attitude's step response in Fig:\ref{fig:IBC_Step} shows a faster response with notable oscillations introduced at the settling point. The step settles in $t_{95}=1.6403~\text{s}$, almost twice as fast as a basic PD controller. The oscillations produced at the settling point are as a result of the actuator commands (Fig:\ref{fig:IBC_Input}) reaching their rate limits, seen as a large difference between the commanded and physically actuated torque inputs in Fig:\ref{fig:IBC_Torque}. Furthermore, the commanded angular velocity changes for the IBC controller are, on average, twice that of the previous control laws which then results in increased nonlinear torque responses.
\par
The Ideal Backstepping controller is by far the most aggressive control law, which leads to sizable and perhaps unsatisfactory overshoot. That aggression is due to the exact dynamic compensation applied by the attitude control and \emph{not the applied controller gain} as is the case with the previous symmetric PD controller. Saturated rate limits of the actuators then prevent the commanded input being met by the actuated control torque. The Adaptive backstepping controller is tested and discussed later in Sec:\ref{subsec:simulation.disturbance.torque} in the context of robust trajectory stability, rather than stepped controller performances here.
%====================================================
\section{Position Controllers}
\label{sec:simulation.position}
%====================================================
Following the attitude controller optimization, a similar approach is applied to the two proposed position control laws (in Sec:\ref{sec:control.position}). It is important to specify that, for position controller optimization, a plant dependent diagonal PD attitude controller (Sec:\ref{subsubsec:simulation.atttiude.pd.dependent}) is used to stabilize the coupled attitude dynamics. To test each particle's controller coefficient performance, the attitude setpoint was kept at a constant $Q_d=[+1~\vec{0}\hspace{2pt}]^T$, while various position setpoints are applied. The same basic pseudo inversion allocator (Sec:\ref{subsec:allocation.allocators.inverse}) is used for position control to distrubte the virtual control input $\vec{\nu}_d$. Each position setpoint is defined in the inertial frame:
\begin{equation}
\vec{\mathcal{E}}_d(t)\triangleq\begin{bmatrix}
X_d(t)&
Y_d(t)&
Z_d(t)
\end{bmatrix}^T
~~~~\in\mathcal{F}^{I}
\end{equation}
A collection of position setpoints are tested, where each setpoint is positioned on the surface of a sphere at a radius of $C=5~[\text{m}]$ from a central starting point. That starting position is consistently tested at $\vec{\mathcal{E}}_0=[5~5~5]^{T}~[\text{m}]$, relative to the inertial frame's origin. Each setpoint is then stepped away from $\vec{\mathcal{E}}_0$ as per a rotated radial arm:
\begin{equation}
\vec{\mathcal{E}}_d(t)=\vec{\mathcal{E}}_0+R_y(\theta_{y})R_x(\phi_{x})\begin{bmatrix}
0 & 0 & 5
\end{bmatrix}^T
\end{equation}
Test angles $\phi_x$ and $\theta_y$ rotate the radial arm  $C$ for a range $\phi_x\in[-180\text{\textdegree}:180\text{\textdegree}]$ and $\theta_y\in[-90\text{\textdegree}:90\text{\textdegree}]$; both at $30~\text{\textdegree}$ increments. That results in test space position surface illustrated in Fig:\ref{fig:position-setpoint}, with a total of 91 position setpoints to test. Performance of each position step isevaluated with another ITAE integral for the position and translational velocity errors, both transformed into the \emph{body frame}, $\mathcal{F}^{b}$. 
\begin{equation}\label{eq:position-performance}
\vec{\zeta}_{\mathcal{E}}=\int_{t=0}^{15}C_{X}*t*||\vec{X}_e(t)||.dt+\int_{t=0}^{15}C_{v}*t*|\vec{v}_e(t)|.dt
\end{equation}
As with the attitude steps, each position steps simulation is given $t=15~\text{s}$ to reach its settling point when stepped from the starting point. Weighting coefficients $C_X$ and $C_v$ prioritize position and velocity errors respectively, both are weighted at unity. Each particle is then stepped 91 times for position ranges described above, the resultant cost of Eq:\ref{eq:position-performance} is averaged for an overall performance metric. Only plant dependent compensating position controllers are considered and optimized for the position control loop.
\par
Not compensating for the gravitational force acceleration applied to the vehicle in its differential equation of motion, Eq:\ref{eq:quaternion-states-acceleration}, \emph{would} result in instability. An uncompensated gravitational force of $15.45~[\text{N}]$ would drive both Lyapunov function derivatives, Eq:\ref{eq:position-pd-stability} for PD control and Eq:\ref{eq:4.109c} for backstepping control, away from stabilizing negative definite conditions. 
\par
To compare the relative performance of each optimized position controller, a constant position step test is applied in both cases:
\begin{equation}\label{eq:position-step}
\vec{\mathcal{E}}_d=\begin{bmatrix}
X_d&
Y_d&
Z_d
\end{bmatrix}^T=\begin{bmatrix}
7.5&
4&
3
\end{bmatrix}^T~~\text{m},~\in\mathcal{F}^{I}
\end{equation}
\par
\begin{figure}[htbp]
\vspace{-16pt}
\centering
\includegraphics[width=0.72\textwidth]{figs/position-setpoint}
\vspace{-6pt}
\caption{Position setpoint workspace}
\label{fig:position-setpoint}
\vspace{-16pt}
\end{figure}
%====================================================
\subsection{PD}
\label{subsec:simulation.position.pd}
%====================================================
The reference case for position control is the Proportional-Derivative controller, presented in Sec:\ref{subsec:control.position.pd}. The PD position controller designs a control force input, from Eq:\ref{eq:position-pd}:
\begin{equation}\label{eq:optimized-position}
\vec{F}_{_{PD}}=K_p\vec{X}_e + K_d\vec{v}_e + \hat{\omega}_b\times m_b\hat{v}_b-m_b\vec{G}_b~~~~\in\mathcal{F}^{b}
\end{equation}
where $\vec{X}_e$ is the inertial position error $\vec{\mathcal{E}}_e\in\mathcal{F}^{I}$, transformed to the body frame in Eq:\ref{eq:4.80a}. Both $K_p$ and $K_d$ are diagonal gain coefficient matrices. Introducing of symmetric $[3\times 3]$ coefficients to the gain matrices did not yield any improvements for the attitude plant in Sec:\ref{subsubsec:simulation.atttiude.pd.3x3}, so it is not investigated in the context of position control. The two gain coefficients for the PD controller are structured as follows:
\begin{equation}\label{eq:simulation-attitde-pd-diagonal-coefficients}
K_p\triangleq \begin{bmatrix}
K_p(1) & 0 & 0\\
0 & K_p(2) & 0\\
0 & 0 & K_p(3)
\end{bmatrix}
~~\text{and}~~K_d\triangleq \begin{bmatrix}
K_d(1) & 0 & 0\\
0 & K_d(2) & 0\\
0 & 0 & K_d(3)
\end{bmatrix}
\end{equation}
Each coefficient matrix acts on the position error vector, $\vec{X}_e$, and the velocity error vector, $\vec{v}_e$, independently. The following coefficients are the result of the optimization process:
\begin{equation}\label{eq:optimized-position-pd}
K_p = \begin{bmatrix}
2.4167 & 0 & 0\\
0 & 2.1557 & 0\\
0 & 0 & 2.5904
\end{bmatrix}
~~\text{and}~~K_d = \begin{bmatrix}
3.4794 & 0 & 0\\
0 & 3.3846 & 0\\
0 & 0 & 3.8698
\end{bmatrix}
\end{equation}
A step in the position loop's setpoint produces a response shown in Fig:\ref{fig:PD_Position_Step}. Stepping from the initial position to the setpoint $\vec{\mathcal{E}}_d$ described in Eq:\ref{eq:attitude-step-position}. The position step settled in $t_{95}=4.007~\text{s}$ without any overshoot. 
\\
\emph{\color{Gray}Note that $\vec{\mathcal{E}}_d$ in Eq:\ref{eq:attitude-step-position} is defined in the inertial frame, $\mathcal{F}^{I}$. That setpoint is transformed to the body frame as a variable substitution for $\vec{X}_d$ for the controller error in Eq:\ref{eq:optimized-position}.}
\par
Not shown, but still considered, is the effect a position step has on the attitude plant's stability, which still remained stable at the origin with no deviations. Because the attitude setpoint is $Q_d=[+1~\vec{0}\hspace{2pt}]^T$, almost all the force requirement in steady state is to oppose the gravitational downward force acting on the body, Fig:\ref{fig:PD_Position_Force}.
\begin{figure}[htbp]
\vspace{-10pt}
\centering
\begin{subfigure}{\textwidth}
\centering
\includegraphics[width=0.7\textwidth]{graphs/PD_Position_Step}
\vspace{-6pt}
\caption{Position step}
\label{fig:PD_Position_Step}
\end{subfigure}
\begin{subfigure}{0.49\textwidth}
\centering
\includegraphics[width=\textwidth]{graphs/PD_Position_Force}
\vspace{-20pt}
\caption{Plant input forces}
\label{fig:PD_Position_Force}
\end{subfigure}
\begin{subfigure}{0.49\textwidth}
\centering
\includegraphics[width=\textwidth]{graphs/PD_Position_Velocity}
\vspace{-20pt}
\caption{Translational velocity}
\label{fig:PD_Position_Velocity}
\end{subfigure}
\begin{subfigure}{\textwidth}
\centering
\includegraphics[width=0.6\textwidth]{graphs/PD_Position_Input}
\vspace{-8pt}
\caption{Plant actuator inputs}
\label{fig:PD_Position_Input}
\end{subfigure}
\vspace{-8pt}
\caption{Position PD}
\vspace{-34pt}
\end{figure}
%====================================================
\subsection{Ideal and Adaptive Position Backstepping}
\label{subsec:simulation.position.pd}
%====================================================
The second and final position controller to be tested is the Ideal Backstabbing controller, the only exponentially stable position control law reviewed. As is the case with attitude IBC, the coefficients selected for the Ideal Backstepping case are used again for the Adaptive case, the latter is evaluated subsequently in Sec:\ref{subsec:simulation.disturbance.force}. Recalling the position IBC structure from Sec:\ref{subsec:control.position.bacstepping}:
\begin{equation}\label{eq:simulation-position-IBC}
\vec{F}_{_{IBC}}=m_b\Big(\big(1+\Gamma_1\Gamma_2\big)\vec{\zeta}_1-\big(\Gamma_1+\Gamma_2\big)\hat{v}_b\big)\Big)+\hat{\omega}_b\times m_b\hat{v}_b-m_b\vec{G}_b~~~~\in\mathcal{F}^{b}
\end{equation}
with the backstepping variable $\vec{\zeta}_1$ defined $\vec{z}_1\triangleq \vec{X}_e$, from Eq:\ref{eq:102a}. 
\par
The two positive symmetric coefficient gain matrices in Eq:\ref{eq:simulation-position-IBC} are structured as:
\begin{equation}\label{eq:simulation-position-diagonal-coefficients}
\Gamma_1\triangleq \begin{bmatrix}
\Gamma_1(1) & \Gamma_1(4) & \Gamma_1(5)\\
\Gamma_1(4) & \Gamma_1(2) & \Gamma_1(6)\\
\Gamma_1(5) & \Gamma_1(6) & \Gamma_1(3)
\end{bmatrix}
~~\text{and}~~\Gamma_2\triangleq \begin{bmatrix}
\Gamma_2(1) & \Gamma_2(4) & \Gamma_2(5)\\
\Gamma_2(4) & \Gamma_2(2) & \Gamma_2(6)\\
\Gamma_2(5) & \Gamma_2(6) & \Gamma_2(3)
\end{bmatrix}
\end{equation}
Similar to the attitude backstepping controller, the position ideal backstepping controller has gain coefficients which act on both plant's error $\vec{X}_e$ and error rates $\vec{v}_e$. This makes local and global coefficient position selection difficult without adversely affecting the swarm's optimization trajectory process. Using the first coefficient matrix $\Gamma_1$ to prioritize position tracking errors $\vec{X}_e$ and relegating $\Gamma_2$ to settle velocity errors $\vec{v}_e$, the local best positions are chosen where each respective error is minimized. The optimized gain coefficients for $\Gamma_1$ and $\Gamma_2$ were then produced by the PSO algorithm:
\begin{equation}\label{eq:optimized-Position-IBC}
\Gamma_1 = \begin{bmatrix*}
2.3409 & 0.1707 & -0.1644\\
0.1707 & 2.0493 & 0.1060\\
-0.1644 & 0.1060 & 1.7322
\end{bmatrix*}
~~\text{and}~~\Gamma_2= \begin{bmatrix*}
1.5287 & 0.02928 & 0.0816\\
0.0292 & 1.4214 & -0.0410\\
0.0816 & -0.0410 & 1.4753
\end{bmatrix*}
\end{equation}
\begin{figure}[htbp]
\vspace{-20pt}
\centering
\begin{subfigure}{\textwidth}
\centering
\includegraphics[width=0.7\textwidth]{graphs/IBC_Position_Step}
\vspace{-6pt}
\caption{Position step}
\label{fig:IBC_Position_Step}
\end{subfigure}
\begin{subfigure}{0.49\textwidth}
\centering
\includegraphics[width=\textwidth]{graphs/IBC_Position_Force}
\vspace{-20pt}
\caption{Plant input forces}
\label{fig:IBC_Position_Force}
\end{subfigure}
\begin{subfigure}{0.49\textwidth}
\centering
\includegraphics[width=\textwidth]{graphs/IBC_Position_Velocity}
\vspace{-20pt}
\caption{Translational velocity}
\label{fig:IBC_Position_Velocity}
\end{subfigure}
\begin{subfigure}{\textwidth}
\centering
\includegraphics[width=0.6\textwidth]{graphs/IBC_Position_Input}
\vspace{-8pt}
\caption{Plant actuator inputs}
\label{fig:IBC_Position_Input}
\end{subfigure}
\vspace{-8pt}
\caption{Position backstepping controller}
\vspace{-20pt}
\end{figure}
\par
Fig:\ref{fig:IBC_Position_Step} shows how the Ideal Backstepping controller stabilizes and tracks a step change to the translational position setpoint. Note that the position plotted in Fig:\ref{fig:IBC_Position_Step} is the relative position in the inertial frame $\mathcal{F}^{I}$, not the backstepping input $\vec{X}_e\in\mathcal{F}^b$. The Ideal Backstepping controller settles in $t_{95}=2.987~[\text{s}]$, faster than a regular proportional-derivative position controller. The exponentially bound error trajectory improves the controller's performance but, not unexpectedly, commands greater input forces (Fig:\ref{fig:IBC_Position_Force}) from larger spikes in the propeller's rotational speed, shown in Fig:\ref{fig:IBC_Input}. 
\par
The improved performance from the position Ideal Backstepping controller is due to the change in structure, increasing the PD controller's gain by a scale factor of 2 (analogous to the test performed in Fig:\ref{fig:PD_Diagonal_Gain_Step}) \emph{decreases} the step's settling time from $t_{95}=4.007~[\text{s}]$ to $4.379~[\text{s}]$. So improving even the conceptually simpler force position control loop is not as easy as simply adding more gain to the control plant. 
\begin{figure}[htbp]
\vspace{-12pt}
\centering
\begin{subfigure}{\textwidth}
\centering
\includegraphics[width=0.7\textwidth]{graphs/PD_Position_Gain_Step}
\vspace{-6pt}
\caption{Position step}
\label{fig:PD_Position_Gain_Step}
\end{subfigure}
\begin{subfigure}{\textwidth}
\vspace{-4pt}
\centering
\includegraphics[width=0.59\textwidth]{graphs/PD_Position_Gain_Input}
\vspace{-8pt}
\caption{Plant actuator inputs}
\label{fig:PD_Position_Gain_Input}
\end{subfigure}
\vspace{-8pt}
\caption{Increased gain PD position}
\vspace{-36pt}
\end{figure}
%====================================================
\section{Setpoint Control Results}
\label{sec:simulation.autopilot}
%====================================================
None of the proposed attitude or position controllers are unstable, each one achieves the goal of setpoint tracking in the context of stepped inputs. To corroborate dynamic setpoint tracking an increasing (\emph{chirp}) trajectory is commanded, illustrated in Fig:\ref{fig:trajectory}. The trajectory applies an increasing frequency rate of $1/60~[\text{Hz}.s^{-1}]$ starting from zero, such that at $t=60~[\text{s}]$ the trajectory orbits a central point $\vec{C}_0$ and completes a single orbit every minute. Eventually the increasing orbital frequency will push each controller beyond their tracking limit. Only two Proportional Derivative \emph{attitude} controllers are tested here, both with diagonal gain matrices (from Eq:\ref{eq:optimized-pd-independent}), to compare the effects of plant dependent compensation. It was shown previously that symmetric gain coefficients yield no performance improvements for the PD case, so only diagonal coefficient matrices were used. 
\par
Furthermore, Sec:\ref{subsec:simulation.attitude.pd} demonstrated plant independent controllers result in steady state errors, the same is shown to be true for trajectory tracking. Adaptive backstepping controllers and their disturbance rejection properties are only discussed next in Sec:\ref{sec:simulation.disturbance}. Each attitude controller is tested together with a common PD position controller, tracking the orbital XYZ position. Similarly each position controller is tested using a simple diagonal PD controller to track the attitude. The attitude controllers have an initial step to reach the orbital attitude from their starting attitude, $Q_0=[+1~\vec{0}\hspace{2pt}]^T$. Each controller's trajectory setpoint response and its respective tracking error are plotted together in pairs.
\begin{figure}[htbp]
\centering
\vspace{-10pt}
\begin{subfigure}{0.9\textwidth}
\centering
\includegraphics[width=\textwidth]{graphs/PD_Diagonal_Independent_Trajectory}
\vspace{-18pt}
\caption{Attitude trajectory}
\label{fig:independent_diagonal_trajectory}
\end{subfigure}
\begin{subfigure}{0.9\textwidth}
\centering
\includegraphics[width=\textwidth]{graphs/PD_Diagonal_Independent_Error}
\vspace{-18pt}
\caption{Attitude trajectory error}
\label{fig:independent_diagonal_error}
\end{subfigure}
\vspace{-10pt}
\caption{Independent PD attitude controller}
\end{figure}
\begin{figure}[htbp]
\centering
\vspace{-20pt}
\begin{subfigure}{0.9\textwidth}
\centering
\includegraphics[width=\textwidth]{graphs/PD_Diagonal_Dependent_Trajectory}
\vspace{-18pt}
\caption{Attitude trajectory}
\label{fig:dependent_diagonal_trajectory}
\end{subfigure}
\begin{subfigure}{0.9\textwidth}
\centering
\includegraphics[width=\textwidth]{graphs/PD_Diagonal_Dependent_Error}
\vspace{-18pt}
\caption{Attitude trajectory error}
\label{fig:dependent_diagonal_error}
\end{subfigure}
\vspace{-10pt}
\caption{Dependent PD attitude controller}
\end{figure}
\begin{figure}[hbtp]
\centering
\vspace{-20pt}
\begin{subfigure}{0.9\textwidth}
\centering
\includegraphics[width=\textwidth]{graphs/XPD_Trajectory}
\vspace{-18pt}
\caption{Attitude trajectory}
\label{fig:XPD_trajectory}
\end{subfigure}
\vspace{-32pt}
\end{figure}
\newpage
\begin{figure}[htbp]
\centering
\ContinuedFloat
\begin{subfigure}{0.9\textwidth}
\centering
\includegraphics[width=\textwidth]{graphs/XPD_Error}
\vspace{-18pt}
\caption{Attitude trajectory error}
\label{fig:dependent_diagonal_error}
\end{subfigure}
\vspace{-10pt}
\caption{Auxiliary PD attitude controller}
\end{figure}
\begin{figure}[htbp]
\centering
\vspace{-8pt}
\begin{subfigure}{0.9\textwidth}
\centering
\includegraphics[width=\textwidth]{graphs/IBC_Trajectory}
\vspace{-18pt}
\caption{Attitude trajectory}
\label{fig:backstepping_trajectory}
\end{subfigure}
\begin{subfigure}{0.9\textwidth}
\centering
\includegraphics[width=\textwidth]{graphs/IBC_Error}
\vspace{-18pt}
\caption{Attitude trajectory error}
\label{fig:backstepping_error}
\end{subfigure}
\vspace{-10pt}
\caption{Ideal backstepping attitude controller}
\end{figure}
Each control law was successful in tracking a given trajectory each with little to no error, except of course for the plant independent attitude controller, Fig:\ref{fig:independent_diagonal_trjaectory}. Trajectory errors could be improved further with application of higher order state derivative in each controller (wherein $\vec{\omega}_d\not=\vec{0}$ and $\vec{v}_d\not=\vec{0}$). The only notable difference between each attitude controller in Fig:\ref{fig:attitude-trajectory-tracking} is that, at the initial attitude step, the Ideal Backstepping attitude controller (Fig:\ref{fig:attitude-trajectory-ibc-tracking}) oscillates whilst settling due to it's more aggressive response. The independent PD controller actually tracks some trajectory, but with a significant steady state error.
\par
\begin{figure}[hbtp]
\vspace{-11pt}
\begin{subfigure}{\textwidth}
\centering
\includegraphics[width=0.8\textwidth]{graphs/PD_Position_Trajectory}
\vspace{-12pt}
\caption{Diagonal Proportional Derivative Controller}
\end{subfigure}
\begin{subfigure}{\textwidth}
\vspace{-2pt}
\centering
\includegraphics[width=0.85\textwidth]{graphs/IBC_Position_Trajectory}
\vspace{-12pt}
\caption{Ideal Backstepping Controller}
\end{subfigure}
\vspace{-6pt}
\caption{Position Trajectory Tracking}
\label{fig:position-trajectory-tracking}
\vspace{-10pt}
\end{figure}
\par
There is a small lag behind the position controller's trajectories, Fig:\ref{fig:position-trajectory-tracking}. Once again this is a consequence of tracking only first order setpoints; if a velocity setpoint was applied as well that tracking error would be diminished\ldots
%====================================================
\section{Robust Stability and Disturbance Rejection}
\label{sec:simulation.disturbance}
%====================================================
Despite deriving adaptive control laws in Sec:\ref{subsubsec:control.attitude.nonlinear.adaptivebackstep} and Sec:\ref{subsec:control.position.bacstepping} for attitude and position controllers respectively; each of the proposed controls demonstrated acceptable stability under sizeable disturbances. App:\ref{app:disturbance} shows each controller's trajectory response to uncompensated disturbances acting on the vehicle. The torque and force disturbances are now detailed\ldots
%====================================================
\subsection{Torque Disturbance Rejection}
\label{subsec:simulation.disturbance.torque}
%====================================================
Torque turbulences are difficult to define without in-depth accompanying statistical and mathematical analysis. To expedite the stability/disturbance evaluation process, torque turbulences were approximated using a Dryden Gust model,\cite{optimalgust,discretegustmodel}. Alternatively the Von Karman aerospace disturbance model(s) could be implemented but that model is computationally more exhaustive.
\par
Without going into too much detail, the Dryden Wind model produces turbulence signals from white noise filtered through a specified Dryden power spectrum. That power spectrum varies as per an aircraft's orientation, altitude and translational velocity. For the aircraft and trajectory under consideration here such a disturbance model is sufficient for simulating small interference patterns. Recalling then the torque disturbance observer derived for the attitude backstepping plant, from Eq:\ref{eq:asymptotic-disturbance}:
\begin{equation}\label{eq:stability-torque-overserver}
\dot{\hat{L}}=-\Gamma_L J_b^{-1}(u)\big(\Gamma_1\vec{q}_e-\vec{\omega_b}\big)
\end{equation}
The gain adaptivity matrix $\Gamma_L$ was tuned on steady state such that the observer's error deviation from an applied torque $\vec{L}$ was minimized. That resulted in the diagonal adaptivity matrix of $\Gamma_L=diag(29.58,~28.43,~4.60)$. The approximator tracks an applied disturbance as shown in Fig:\ref{fig:torque-observer} over a disturbance range of $\pm 0.2~\text{N.m}$, 20\% of a stepped attitude's control input magnitude, for a short steady state test. Both pitch and roll torque approximator channels ($\phi$ and $\theta$) track the torque with a relatively small error. Greater deviation from the applied torque does, however, occur in the $\psi$ channel about the $\hat{Z}_b$ axis\ldots
\begin{figure}[htbp]
\vspace{-12pt}
\centering
\includegraphics[width=0.8\textwidth]{graphs/torque-observer}
\vspace{-12pt}
\caption{Attitude torque disturbance observer}
\vspace{-16pt}
\label{fig:torque-observer}
\end{figure}
\par
Fig:\ref{fig:ABC_trajectory} then shows the Adaptive Backstepping controller's attitude response over the entire orbital trajectory whilst experiencing a fluctuating torque turbulence. The addition of a torque observer for compensation produces a slight improvement over an uncompensated IBC controller; shown in Fig:\ref{fig:app-attitude-ibc-dist} from App:\ref{app:disturbance}. 
\begin{figure}[hbtp]
\vspace{-6pt}
\centering
\includegraphics[width=0.8\textwidth]{graphs/ABC_trajectory}
\vspace{-12pt}
\caption{Adaptive backstepping attitude trajectory tracking}
\label{fig:ABC_trajectory}
\vspace{-16pt}
\end{figure}
\par
The cost of the disturbance approximator in the attitude plant is obviously more aggressive and greater bandwidth fluctuating input torques applied to the control plant. That could potentially reach actuator rate saturation, but for the trajectories tested here, were not saturation inducing.
%====================================================
\subsection{Disturbance Force Rejection}
\label{subsec:simulation.disturbance.force}
%====================================================
Force disturbances are similarly emulated in simulation using a Dryden Gust model for wind turbulent velocity generation. Additionally, a wind vector field across the inertial frame test space was also used to introduce a constant force offset throughout the trajectory simulation. The force disturbance observer, from Eq:\ref{eq:abc-asymptotic-position}, has an estimate update rule such that:
\begin{equation}
\dot{\hat{D}}=-m_b^{-1}\Gamma_D\Big(\Gamma_1\vec{X}_e-\vec{v}_b\Big)
\end{equation}
Where $\vec{X}_e$ is the inertial position error transformed to the body frame, $\vec{X}_e=Q_b\otimes\vec{\mathcal{E}}_e\otimes Q_b^*$. Then $\Gamma_D$ is the force disturbance observer's adaptivity gain matrix. Using the coefficients $\Gamma_D=diag(4.20,~3.84,~3.97)$ the observer tracks a force disturbance acting on the vehicle over a range of $[-4:8]~\text{N}$. Fig:\ref{fig:force-observer} shows how the force observer adapts to the variable force turbulence applied, the plot is taken over an entire simulation (until $t=120~\text{s}$) to illustrate the vector field effects.
\begin{figure}[hbtp]
\vspace{-6pt}
\centering
\includegraphics[width=0.8\textwidth]{graphs/force-observer}
\vspace{-12pt}
\caption{Position force disturbance observer}
\label{fig:force-observer}
\vspace{-16pt}
\end{figure}
\par
The position adaptive backstepping controller then tracks the inertial frame trajectory as shown in Fig:\ref{fig:ABC_Position_Trajectory}. Again improving the trajectory tracking performance slightly when compared to the Ideal backstepping case from Fig:\ref{fig:app-position-ibc-dist}; but even without adaptive disturbance compensation, the plant is stable throughout the trajectory albeit somewhat noisy. The addition of a vector force field results in a fluctuating offset error from the trajectory, despite the adaptive compensation applied to the control loop.
\begin{figure}[hbtp]
\vspace{-6pt}
\centering
\includegraphics[width=0.8\textwidth]{graphs/ABC_Position_Trajectory}
\vspace{-12pt}
\caption{Adaptive backstepping position trajectory tracking}
\label{fig:ABC_Position_Trajectory}
\vspace{-16pt}
\end{figure}
%====================================================
\section{Allocation Tests}
\label{sec:simulation.allocator}
%====================================================
The various allocation rules, as derived in Ch:\ref{ch:allocation}, implement virtual control inputs to solve for explicit actuator positions. Each of the allocators tested here were compared with basic position and attitude Proportional Derivative controllers commanding a virtual input. 
\par
The abstraction applied to achieve an affine relationship required for inversion allocation (in Eq:\ref{eq:allocator-test}) meant that actuator transfer rates were independent from the allocation rule applied.
\begin{subequations}\label{eq:allocator-test}
\begin{equation}\label{eq:allocator-test.a}
\vec{\nu}_c=B'(\vec{\mathbf{x}},t)u=\begin{bmatrix}
\mathbb{I}_{3\times 3} & \mathbb{I}_{3\times 3} & \mathbb{I}_{3\times 3} & \mathbb{I}_{3\times 3}\\
[\vec{L}_1]_\times & [\vec{L}_2]_\times & [\vec{L}_3]_\times & [\vec{L}_4]_\times
\end{bmatrix}
\begin{bmatrix}
\vec{T}_1&
\ldots&
\vec{T}_4
\end{bmatrix}^T
\end{equation}
\vspace{-10pt}
\begin{equation}
u_i = \begin{bmatrix}
\Omega_i & \lambda_i & \alpha_i
\end{bmatrix}^T = R^\dagger(\vec{\mathbf{x}},\vec{T}_i,t)~~\text{for}~i\in[1:4]
\end{equation}
\end{subequations}
The transfer rate at which physically commanded inputs implement virtually designed control inputs, $\vec{\nu}_c\rightarrow\vec{\nu}_d$, is affected by the thrust version equation $R^\dagger(\vec{\mathbf{x}},t)$, not allocation rules $B'(\vec{\mathbf{x}},t)$. The consequence of this is that, in the context of actuator transfer rates, each allocation rule performed almost identically. Inverse solutions to Eq:\ref{eq:allocation-quadratic} solve for the quadratic least squares minimized actuator positions. That means that each $|\vec{T}_i|$  within the $\mathbb{R}^{1\times 12}$ matrix $|\vec{T}_{1\rightarrow 4}|$ is minimized. The solution is an actuator cost efficient one. In general psuedo inversion, wieghted and priority normalized inverse allocators each stem from Eq:\ref{eq:inversion}: 
\begin{subequations}
\begin{equation}
\begin{bmatrix}
\vec{T}_{1\rightarrow 4}
\end{bmatrix}
=\big(\mathbb{I}_{m\times m}-CB(\vec{\mathbf{x}},t)\big)\vec{T}_p+C\vec{\nu}_d
\end{equation}
\vspace{-10pt}
\begin{equation}\label{eq:allocation-inversion-eq}
C=W^{-1}B^T(\vec{\mathbf{x}},t)\big(B(\vec{\mathbf{x}},t)W^{-1}B^T(\vec{\mathbf{x}},t)\big)^{-1}
\end{equation}
\end{subequations}
A combined step of setpoints for attitude \emph{and} position states is used to compare each allocation rule. A pseudo inversion allocator, Eq:\ref{eq:pseudo-inversion}, is used as the reference case to which subsequent allocator algorithms are evaluated against. The typical setpoint used for both position and attitude steps, with attitude in Euler angles $\vec{\eta}_d$ not quaternions $Q_d$, is:
\begin{equation}\label{eq:simulation-state-step}
\vec{\mathbf{x}}_d=\begin{bmatrix}
\vec{\mathcal{E}}_d\\
\vec{\eta}_d
\end{bmatrix}
=
\begin{bmatrix}
\begin{bmatrix}
7.5 & 4 & 3
\end{bmatrix}^T\\
\begin{bmatrix}
-142 & 167 & -45
\end{bmatrix}^T
\end{bmatrix}~~~~\begin{bmatrix}
[\text{m}]\\
[\text{\textdegree}]
\end{bmatrix}
\end{equation}
A pseudo inverse allocation solves for $\vec{T}_{1\rightarrow 4}$ using $B^\ddagger(\vec{\mathbf{x}},t)\vec{\nu}_d$, from Eq:\ref{eq:pseudo-inversion}. Fig:\ref{fig:pseudo-inverse-step} shows the combined position and attitude step responses. The combined attitude and position step response with a pseudo inverse allocator $B^\ddagger(\vec{\mathbf{x}},t)$ settles for \emph{both} states in $t_{95}=5.6233~\text{s}$ from the state step. 
\begin{figure}[hbtp]
\vspace{-12pt}
\centering
\begin{subfigure}{\textwidth}
\centering
\includegraphics[width=0.8\textwidth]{graphs/pseudo_inverse_attitude}
\vspace{-12pt}
\caption{Attitude Step}
\label{fig:pseudo_inverse_attitude}
\end{subfigure}
\begin{subfigure}{\textwidth}
\vspace{-3pt}
\centering
\includegraphics[width=0.8\textwidth]{graphs/pseudo_inverse_position}
\vspace{-12pt}
\caption{Position Step}
\label{fig:pseudo_inverse_position}
\end{subfigure}
\vspace{-8pt}
\caption{Pseudo-Inverse step response}
\label{fig:pseudo-inverse-step}
\vspace{-24pt}
\end{figure}
\par
The preferred allocator positions, described in Sec:\ref{subsec:allocation.allocators.norminverse}, are hovering conditions defined with respect to either the inertial or body frames, $\mathcal{F}^{I}$ and $\mathcal{F}^{b}$. At steady state with an attitude at the origin, $Q_d=[1~\vec{0}\hspace{2pt}]^T$; the controller commands the virtual control input:
\begin{equation}\label{eq:hover-actuator}
\vec{\nu}_p=\begin{bmatrix}
\vec{F}_p\\
\vec{\tau}_p
\end{bmatrix}
=
\begin{bmatrix}
\begin{bmatrix}
0&0&15.45
\end{bmatrix}^T
\\
\begin{bmatrix}
0.25&
0.50&
-1.89
\end{bmatrix}^T
\end{bmatrix}~~~~\begin{bmatrix}
[\text{N}]\\
[\text{N.mm}]
\end{bmatrix}~~~~\in\mathcal{F}^{b,I}
\end{equation}
The small amount of control torque applied in Eq:\ref{eq:hover-actuator}, about the $\hat{Z}_{I/b}$ axis, is to compensate for net gravitational torque due to the eccentric center of gravity and resultant aerodynamic torque $\vec{\tau}_H$ from the propellers rotational velocity. Applying the pseudo inverse allocation rule to the preferred input $\vec{\nu}_p$ in Eq:\ref{eq:hover-actuator} produces the following actuator positions which command hovering conditions:
\begin{subequations}\label{eq:preffered-hover-inertia}
\begin{equation}
\vec{T}_p^{I}=B^\ddagger(\mathbf{x},t)\vec{\nu}_p=\begin{bmatrix}
T_{1x}&T_{1Y}&T_{1Z}&\ldots~\ldots&T_{4x}&T_{4y}&T_{4z}
\end{bmatrix}
\end{equation}
\vspace{-24pt}
\begin{equation}
=\begin{bmatrix}
\begin{bmatrix}
0.00 & -0.02 & 3.86
\end{bmatrix} 
&
\begin{bmatrix}
0.02 & 0 & 3.86
\end{bmatrix}
&
\begin{bmatrix}
0 & 0.02 & 3.86
\end{bmatrix}
&
\begin{bmatrix}
-0.02 & 0 & 3.86
\end{bmatrix}
\end{bmatrix}^T~~~~\text{N}
\end{equation}
\end{subequations}
Testing the same attitude and position setpoint steps, but with preferred actuator hovering conditions relative to the inertial frame (illustrated in Fig:\ref{fig:hover-inertial}) produces a response shown in Fig:\ref{fig:inertial-norm-step}. The plant settles in $t_{95}=5.617~\text{s}$ with a practically identical response to the pseudo-inverse case presented before in Fig:\ref{fig:pseudo-inverse-step}.
\begin{figure}[hbtp]
\vspace{-11pt}
\centering
\begin{subfigure}{\textwidth}
\centering
\includegraphics[width=0.8\textwidth]{graphs/inertial_norm_attitude}
\vspace{-12pt}
\caption{Attitude Step}
\label{fig:inertial_norm_attitude}
\end{subfigure}
\begin{subfigure}{\textwidth}
\vspace{-2pt}
\centering
\includegraphics[width=0.8\textwidth]{graphs/inertial_norm_position}
\vspace{-6pt}
\caption{Position Step}
\label{fig:inertia_norm_position}
\vspace{-8pt}
\end{subfigure}
\caption{Inertial hover preferred actuator step response}
\label{fig:inertial-norm-step}
\vspace{-18pt}
\end{figure}
\par
Transformation of those hovering conditions in Eq:\ref{eq:hover-actuator} from the inertial frame to the body frame is applied through an instantaneous quaternion transformation:
\begin{equation}
\vec{\nu}_p\text{}\hspace{-2pt}'=\begin{bmatrix}
Q_b\otimes\vec{F}_p\otimes Q_b^*\\
Q_b\otimes\vec{\tau}_p\otimes Q_b^*
\end{bmatrix}~~~~\in\mathcal{F}^{b}
\end{equation}
The hovering conditions are then always a function of the body's instantaneous attitude. However, because the control plant only tracks first order setpoints, the controller naturally tends towards stability at each controller interval, irrespective of the allocator rule applied. Again the plant settles in roughly the same time, $t_{95}=5.614~\text{s}$, with a response shown in Fig:\ref{fig:body-norm-step}. The difference between the two preferred allocator positions has no consequence on the performance of the control loop.
\par
\begin{figure}[hbtp]
\centering
\begin{subfigure}{\textwidth}
\centering
\includegraphics[width=0.8\textwidth]{graphs/body_norm_attitude}
\vspace{-12pt}
\caption{Attitude Step}
\label{fig:body_norm_attitude}
\end{subfigure}
\begin{subfigure}{\textwidth}
\vspace{-3pt}
\centering
\includegraphics[width=0.8\textwidth]{graphs/body_norm_position}
\vspace{-12pt}
\caption{Position Step}
\label{fig:body_norm_position}
\end{subfigure}
\vspace{-8pt}
\caption{Body frame hover preferred actuator step response}
\label{fig:body-norm-step}
\vspace{-16pt}
\end{figure}
\par
The weighted actuator allocation rule, proposed in Sec:\ref{subsec:allocation.allocators.weightedinverse}, prioritizes the use of certain input thrust components in Eq:\ref{eq:allocator-test.a}. The weighting matrix is a $12\times 12$ set of coefficients which bias various allocators as illustrated in Fig:\ref{fig:weighted-matrix}. In order to reduce slack and ensure setpoint tracking , being the primary control objective, each weighting coefficient row and column was constrained to a normalized summation. Furthermore it was proposed that coefficients were selected based on an optimization as per the penalty function Eq:\ref{eq:actuator-penalty}. In practice the weighting coefficients have no bearing on the input plant settling time for each actuator module; to demonstrate a weighted actuator case the following weighting matrix was used for $C=W^{-1}B^T(B.W^{-1}.B^T)^{-1}$ from Eq:\ref{eq:allocation-inversion-eq}:
\begin{equation}\label{eq:inversion-weights}
W\triangleq
\begin{bmatrix}
\begin{bmatrix}
72 & 9 & 9\\
9 & 72 & 9\\
9 & 9 & 72
\end{bmatrix}
&
\begin{bmatrix}
0 & 0 & 0\\
0 & 0 & 0\\
0 & 0 & 0
\end{bmatrix}
&
\begin{bmatrix}
8 & 1 & 1\\
1 & 8 & 1\\
1 & 1 & 8
\end{bmatrix}
&
\begin{bmatrix}
0 & 0 & 0\\
0 & 0 & 0\\
0 & 0 & 0
\end{bmatrix}
\\
\begin{bmatrix}
0 & 0 & 0\\
0 & 0 & 0\\
0 & 0 & 0
\end{bmatrix}
&
\begin{bmatrix}
72 & 9 & 9\\
9 & 72 & 9\\
9 & 9 & 72
\end{bmatrix}
&
\begin{bmatrix}
0 & 0 & 0\\
0 & 0 & 0\\
0 & 0 & 0
\end{bmatrix}
&
\begin{bmatrix}
8 & 1 & 1\\
1 & 8 & 1\\
1 & 1 & 8
\end{bmatrix}
\\
\begin{bmatrix}
8 & 1 & 1\\
1 & 8 & 1\\
1 & 1 & 8
\end{bmatrix}
&
\begin{bmatrix}
0 & 0 & 0\\
0 & 0 & 0\\
0 & 0 & 0
\end{bmatrix}
&
\begin{bmatrix}
72 & 9 & 9\\
9 & 72 & 9\\
9 & 9 & 72
\end{bmatrix}
&
\begin{bmatrix}
0 & 0 & 0\\
0 & 0 & 0\\
0 & 0 & 0
\end{bmatrix}
\\
\begin{bmatrix}
0 & 0 & 0\\
0 & 0 & 0\\
0 & 0 & 0
\end{bmatrix}
&
\begin{bmatrix}
8 & 1 & 1\\
1 & 8 & 1\\
1 & 1 & 8
\end{bmatrix}
&
\begin{bmatrix}
0 & 0 & 0\\
0 & 0 & 0\\
0 & 0 & 0
\end{bmatrix}
&
\begin{bmatrix}
72 & 9 & 9\\
9 & 72 & 9\\
9 & 9 & 72
\end{bmatrix}
\end{bmatrix}\times 10^{-3}
\end{equation}
The weighted allocator's response in Fig:\ref{fig:weighted-inverse-step} shows the minor differences between each of the above allocation rules. The final weighted inversion allocator applied with Eq:\ref{eq:inversion-weights} settled in $t_{95}=5.618$. The inversion based allocator's requirement for an affine effectiveness relationship meant that actuator transfer functions were separated from the allocation block, this made the actuator transfer rates independent from the allocation rule applied. Because of the constraint that each allocator still meets the slack variable requirements of setpoint tracking, each allocator's performance is much of the same. The twelve produced thrust component inputs are, within a margin of error, effectively the same across each allocation rule\ldots
\newpage
\begin{figure}[hbtp]
\centering
\begin{subfigure}{\textwidth}
\centering
\includegraphics[width=0.8\textwidth]{graphs/weighted_inverse_attitude}
\vspace{-10pt}
\caption{Attitude Step}
\label{fig:weighted_inverse_attitude}
\end{subfigure}
\begin{subfigure}{\textwidth}
\vspace{-3pt}
\centering
\includegraphics[width=0.8\textwidth]{graphs/weighted_inverse_position}
\vspace{-10pt}
\caption{Position Step}
\label{fig:weighted_inverse_position}
\end{subfigure}
\vspace{-8pt}
\caption{Weighted actuator allocation step response}
\label{fig:weighted-inverse-step}
\vspace{-16pt}
\end{figure}
%====================================================
\section{Input Saturation}
\label{sec:simulation.saturation}
%====================================================
The introduction of a rotational limit to the actuating servos was a design decision made previously in Sec:\ref{subsec:proto.design.transfer}. The $\pm 90\text{\textdegree}$ limit on the servos is something which can easily be solved by changing the mechanical design to incorporate continuous rotation actuators. The standard state setpoint step used thus far \emph{does not} command any of the actuators beyond their rotational limits, neither does the applied trajectory tracking loop. Alternatively, the following state setpoint is used:
\begin{equation}\label{eq:saturation-state-step}
\vec{\mathbf{x}}_d\text{}\hspace{-2pt}'=
\begin{bmatrix}
\vec{\mathcal{E}}_d\text{}\hspace{-2pt}'\\
\vec{\eta}_d\text{}\hspace{-2pt}'
\end{bmatrix}
=\begin{bmatrix}
\begin{bmatrix}
7.5 & 4 & 3
\end{bmatrix}^T
\\
\begin{bmatrix}
-142 & 35 & -45
\end{bmatrix}^T
\end{bmatrix}
~~~~\begin{bmatrix}
[\text{m}]\\
[\text{\textdegree}]
\end{bmatrix}
\end{equation}
The attitude setpoint, $\vec{\eta}_d\text{}\hspace{-2pt}'$, was chosen because it commands each servo beyond their rotational limit. When using Proprotional Derivative controllers for both position and attitude control loops, Fig:\ref{fig:unsaturated-step} shows that step response for attitude and positions collectively.
\begin{figure}[hbtp]
\vspace{-4pt}
\centering
\begin{subfigure}{\textwidth}
\centering
\includegraphics[width=0.8\textwidth]{graphs/unsaturated-attitude-step}
\vspace{-6pt}
\caption{Attitude step}
\label{fig:unsaturated-attitude-step}
\end{subfigure}
\vspace{-18pt}
\end{figure}
\newpage
\begin{figure}[htbp]
\vspace{-12pt}
\centering
\begin{subfigure}{\textwidth}
\centering
\includegraphics[width=0.8\textwidth]{graphs/unsaturated-position-step}
\vspace{-8pt}
\caption{Position step}
\label{fig:unsaturated-position-step}
\end{subfigure}
\vspace{-10pt}
\caption{Step response without servo limits}
\label{fig:unsaturated-step}
\vspace{-10pt}
\end{figure}
\par
Neither responses in Fig:\ref{fig:unsaturated-step} are anything unexpected. The rotational positions for all eight servos are shown in Fig:\ref{fig:unsaturated-servos}. Noting that each middle ring servo $\alpha_i$ for $i\in[1:4]$ settles to both above or below the $\pm\pi/2$ rotational limit.
\begin{figure}[hbtp]
\vspace{-12pt}
\centering
\includegraphics[width=0.67\textwidth]{graphs/unsaturated-servos}
\vspace{-12pt}
\caption{Servo inputs without limits}
\label{fig:unsaturated-servos}
\vspace{-20pt}
\end{figure}
\par
Then introducing that hard limit to the state step in Eq:\ref{eq:saturation-state-step} produces a step response in Fig:\ref{fig:saturated-step}. The response obviously never reaches a settling point and destabilizes when the actuator saturation limits are applied.
\begin{figure}[hbtp]
\vspace{-12pt}
\centering
\begin{subfigure}{\textwidth}
\centering
\includegraphics[width=0.8\textwidth]{graphs/saturated-attitude-step}
\vspace{-10pt}
\caption{Attitude step}
\label{fig:saturated-attitude}
\vspace{-16pt}
\end{subfigure}
\vspace{-24pt}
\end{figure}
\newpage
\begin{figure}\ContinuedFloat
\vspace{-8pt}
\centering
\begin{subfigure}{\textwidth}
\centering
\includegraphics[width=0.8\textwidth]{graphs/saturated-position-step}
\vspace{-10pt}
\caption{Position step}
\label{fig:saturated-position}
\end{subfigure}
\vspace{-6pt}
\caption{Step response with servo limits}
\label{fig:saturated-step}
\vspace{-18pt}
\end{figure}
\par
Fig:\ref{fig:saturated-servos} shows the limit cycles which the servos get stuck in when an attitude step which exceeds their hard limit is commanded.
\begin{figure}[hbtp]
\vspace{-8pt}
\centering
\includegraphics[width=0.67\textwidth]{graphs/saturated-servos}
\vspace{-6pt}
\caption{Servo inputs without limits}
\label{fig:saturated-servos}
\vspace{-24pt}
\end{figure}
%====================================================
\section{State Estimation}
\label{sec:simulation.state}
%====================================================
The final aspect of the control simulation to consider is the effect state estimation has on the controller's ability to track setpoints. It was proposed, in Ch:\ref{ch:proto}, that a 9-axis inertial measurement unit would produce angular rates and inertial accelerations in the body frame. Then some form of filtration would fuse sensor measurements together to provide state-estimation only for the vehicle's attitude. Position displacement in the inertial frame cannot be approximated only using an IMU due to massive ammounts of integral drift. For that reason a motion capture VICON-like system, \cite{arnold}, was proposed to track the vehicles position under testing conditions. 
\par
Such components of the embedded system loop could indeed be tested in simulation but, owing to the large amounts of noise a physical flight test would induce on those components, would not prove useful in evaluating the efficacy of the net proposed control system. Given the amount of vibrations and disturbances the prototype will undergo, simulations won't be able to accurately approximate such affects. Instead it was deemed pertinent to only apply discretization effects onto the state feedback elements within the control loop.
\par
Quaternion position, angular velocity and translational velocity feedback terms were discretized and sampled at a rate of $70~\text{Hz}$ to emulate an IMU system based on the hardware proposed (Sec:\ref{sec:proto.layout}). In reality those signals would be processed by a Kalman filter and be subject to a relative degree of noise and integral drift. Then inertial position feedback was sampled at $50~\text{Hz}$ to emulate the proposed camera based state estimation from \cite{arnold}.
\par
Both position and attitude control loops, testing with the basic Proportional Derivative controllers in both cases, were stable for the above proposed sampling rates. In fact the entire system was stable for sample rates as slow as $5~\text{Hz}$, whose response for a typical attitude and position step (Eq:\ref{eq:simulation-state-step}) is plotted in Fig:\ref{fig:discrete_step}.
\begin{figure}[hbtp]
\centering
\begin{subfigure}{\textwidth}
\centering
\includegraphics[width=0.8\textwidth]{graphs/discrete_attitude_step}
\vspace{-12pt}
\caption{Attitude step}
\end{subfigure}
\begin{subfigure}{\textwidth}
\vspace{-3pt}
\centering
\includegraphics[width=0.8\textwidth]{graphs/discrete_position_step}
\vspace{-12pt}
\caption{Position step}
\end{subfigure}
\vspace{-8pt}
\caption{Discretized state steps}
\label{fig:discrete_step}
\vspace{-12pt}
\end{figure}
\par
%====================================================
%	CHAPTER 7 - Conclusions
%====================================================
\chapter{Conclusions and Recommendations}
\label{ch:conclusion}
%====================================================
The original objective for the project was to design, simulate and physically test the prototype outlined in Sec:\ref{sec:proto.design}. Modeling the responses of the multibody prototype to obtain the dynamic equations of motion, derived in Sec:\ref{sec:dynamics.nonlinearities}, proved to be dramatically more complex than initially anticipated. 
Time varying moments of inertia (generalized in Eq:\ref{eq:inertial-rate-def}) introduced to the Lagrangian kinematics resulted in an unique problem formulation. Each actuated motor module's response to the net vehicle's dynamics (Eq:\ref{eq:torque-induced-inner},\ref{eq:torque-induced-middle} and \ref{eq:module-response}) required multiple revisions. Each step in the derivation was tested in simulation to ensure the mathematics applied were sound. The difficulties accompanying those derivations pushed back the project's time-line significantly. As a result, it was decided to cancel inclusion of physical flight tests. Physical implementation of the proposed control laws therefore remains open to further work.  With the above being considered, the dynamic model for the system is a significant contribution of this work. The uniqueness of the multibody structure made solving for the differential equations of motion a sizeable task. A consequence of the complex dynamics was an extremely high degree of stiffness in the system which adversely affected the simulation times. Alternatively, relative coordinates could have been implemented in lieu of the used Cartesian coordinates which describe the vehicle and its configuration. Relative descriptions of each state variable could reduce the complexity in calculating instantaneous moments of inertia at each simulation interval. Moreover, implicit Euler integration could have been applied to the simulation, both changes could potentially yield simulation improvements. The cost of such changes would be to reconstruct the entire simulation environment.
\par
The physical tests which corroborated aspects of the dynamic model (Sec:\ref{subsec:dynamics.nonlinearities.torque-tests}) would ideally be extended to physical flight tests. However, considering the complexity of the system and modelling thereof, verification of the dynamics is a useful result. The time varying, non-diagonal inertias of each body  in the multibody system are consequences of the design process and the cost constraint applied to the prototype. In practice, if the rigid component of the frame ($J_y$ from Eq:\ref{eq:inertia.body.c}) was sufficiently larger than that of the actuated (\emph{rotating}) bodies, the relative effect of the multibody interaction responses ($\vec{\tau}_b(\hat{u})$ from Eq:\ref{eq:net-body-response}) on the dynamics would be diminished.
\par
One of the original justifications for the increased platform complexity was the improved actuator bandwidth that would accompany thrust vectoring. The hypothesis was that pitching or rolling a thrust vector would have a faster response than changing the propeller's rotational speed (see actuator transfer functions in Sec:\ref{subsec:proto.design.transfer}). The firmware changes made to the ESCs improved the brushless DC motor's transfer function's time constant significantly. The original firmware which the ESCs used by default produced an exponentially approaching speed curve rather than the standard linear relationship (illustrated in Fig:\ref{fig:rpm-sensor}). Step tests comparing changes before and after the ESC firmware was changed were not be performed. The overall constraint encountered by the actuator plant was rate (\emph{current}) limiting imposed on the rotational servos as a result of their electrical design. That constraint limited the performance of the more aggressive Ideal Backstepping attitude controller, Fig:\ref{fig:IBC_controller_result}. The servo rate limits prevented the motor modules from actuating fast enough, seen in the difference between controller designed and physically commanded inputs in Fig:\ref{fig:IBC_Torque}.
\par
The control solutions presented in this dissertation all stabilize the plant. Respective results for attitude and position controller steps (Sec:\ref{sec:simulation.attitude} and Sec:\ref{sec:simulation.position}) demonstrate the improvement exponential stability yields on a controller plant. All of the control laws proposed were able to track the applied chirp trajectory for low trajectory rates. Each controller's optimization was an ITAE optimization, prioritizing settling times and overshoot errors over aggression or input magnitude. Alternatively, the optimization could apply a penalty to a proposed set of controller coefficients based on energy expenditure or induced torque response, emphasizing stability and smooth transitions over settling times. A particle swarm optimization in Sec:\ref{subsec:simulation.tuning.pso} was chosen due to its simplicity and lack of an explicitly defined gradient function. More complex optimization paradigms could have potentially produced more efficient optimizations.
\par
Certain constraints or assumptions were applied to the model in simulation. It was shown in Sec:\ref{sec:simulation.saturation} that applying rotational limits to the actuation servo broke down the overall setpoint tracking of the control loop. Extending the actuators to accommodate for continuous rotation requires an alteration of the mechanical design and drastically improves the range of motion. The only significant assumption made on the plant's aerodynamics was neglecting to account for any propeller's down-wash becoming incident flow into other propeller. This would have a sizeable impact on the thrust plant model, requiring a complicated fluid dynamics solution to approximate for such effects. The decision to apply nonlinear state space control to the plant prevented the use of Model Predictive control. An MPC control law could potentially better compensate for the vehicle's non-linearities, which were otherwise relegated to feedback compensation.
\par
In conclusion, the non-zero state setpoint tracking goal was achieved by each of the control laws proposed. The control allocation rules applied did not have a notable effect on the plant's performance because of the structure applied in Sec:\ref{sec:allocation.inversion}. Finally, the dynamic model's complexity and the difficulties involved in verification of that model outweighed the control improvements shown. The thrust vectoring accomodated for unique 6-DOF trajectory tracking to be performed, however the same could have been achieved with only a single axis of rotational tilt applied to each lift propeller (similar to related projects described in Sec:\ref{sec:intro.litreview}). The same dynamic complexities led to catastrophic failures with earlier versions of Osprey \cite{ospreywired}, the inspiration for this project. Those complexities led to the subsequent redesign of the Osprey's successor, the V-280 Valor, which has significantly smaller actuator inertias due to fewer moving parts (at the cost of more frequent maintenance).
%*****************************************************
%	APPENDIX
%*****************************************************
\appendix
%-----------------------------------------------------
\chapter{Standard Quadrotor Dynamics}
\label{app:stddynamics}
%-----------------------------------------------------
Following 6-DOF derivations in Section:\ref{subsec:dynamics.rigidbody.lagrange}, the common reductions typically applied those equations for a generic "+" configuration quadrotor are now presented. Reiterating those four differential equations, Eq:\ref{eq:states}, which describe a rigid body's motion (using rotation matrices and not quaternions):
\begin{subequations}
\begin{equation}
\dot{\vec{\mathcal{E}}}=\mathbb{R}_b^I(-\eta)\vec{\nu}~~~~\in\mathcal{F}^I
\end{equation}
\vspace{-10pt}
\begin{equation}
\dot{\vec{v}}=\frac{1}{m}\bigg[-\vec{\omega}_b\times m\vec{v}+m\mathbb{R}_I^b(-\eta)\vec{G}_I+\vec{F}_{net}\bigg]~~~~\in\mathcal{F}^b
\end{equation}
\vspace{-10pt}
\begin{equation}
\dot{\vec{\eta}}=\Psi(\eta)\vec{\omega}_b~~~~\in\mathcal{F}^{v2},\mathcal{F}^{v1},\mathcal{F}^I
\end{equation}
\vspace{-10pt}
\begin{equation}
\dot{\vec{\omega}}_b=\mathbb{I}_b^{-1}\bigg[-\vec{\omega}_b\times\mathbb{I}_b\vec{\omega}_b+\vec{\tau}_{net}\bigg]~~~~\in\mathcal{F}^b
\end{equation}
\end{subequations}
The net heave thrust produced by rotors $i=[1:4]$ is given by:
\begin{subequations}
\begin{equation}
\vec{T}=\sum_{i=1}^4\vec{F}_i 
\end{equation}
The simplified relationship between the thrust force $\vec{F}_i$ and the propellers rotational speed $\Omega_i$ is approximately quadratic:
\begin{equation}
\vec{F}_i=k_1\Omega_i^2
\end{equation}
\end{subequations}
Similarly the aerodynamic torque opposing each rotating propeller is:
\begin{equation}
\tau_{aero}=k_2\Omega_i^2
\end{equation}
The control pitch and  roll torques are generated by opposing differential lift forces and finally the yaw torque is a net response to the rotational aerodynamic propeller torques. The control torque inputs are then defined as:
\begin{subequations}
\begin{equation}
\tau_{\phi}=\vec{L}_{arm}\big(\vec{F}_1-\vec{F}_3\big)
\end{equation}
\vspace{-5pt}
\begin{equation}
\tau_{\theta}=\vec{L}_{arm}\big(\vec{F}_2-\vec{F}_4\big)
\end{equation}
\vspace{-10pt}
\begin{equation}
\tau_{\psi}=\sum_{i=1}^4(-1)^{i}k_2\Omega_i
\end{equation}
\end{subequations}
Simplifying Eq:\ref{.} to component form:
\begin{equation}
\begin{pmatrix}
\dot{u}\\
\dot{v}\\
\dot{w}
\end{pmatrix}
=
\begin{pmatrix}
rv-qw\\
pw-ru\\
qu-pv
\end{pmatrix}
+
\begin{pmatrix}
-g sin(\theta)\\
g cos(\theta)sin(\phi)\\
g cos(\theta)cos(\phi)
\end{pmatrix}
+
\frac{1}{m}\begin{pmatrix}
0\\
0\\
F
\end{pmatrix}
~~~~\in\mathcal{F}^b
\end{equation}
\par
Considering the size of the average angular velocity $\omega_b$, the gyroscopic effects on the body (namely the cross product terms) are negligiable. Assuming too that the body has a (\emph{roughly}) diagonal inertial matrix:
\begin{equation}
\begin{pmatrix}
\frac{\mathbb{I}_y-\mathbb{I}_z}{\mathbb{I}_x}qr\\
\frac{\mathbb{I}_z-\mathbb{I}_x}{\mathbb{I}_y}pr\\
\frac{\mathbb{I}_x-\mathbb{I}_y}{\mathbb{I}_z}pq
\end{pmatrix}
\approx
\vec{0}
\end{equation}
As a result, Eq:\ref{.} then simplifies to:
\begin{equation}
\begin{pmatrix}
\dot{p}\\
\dot{q}\\
\dot{r}
\end{pmatrix}
=
\begin{pmatrix}
\frac{1}{\mathbb{I}_x}\tau_\phi\\
\frac{1}{\mathbb{I}_y}\tau_\theta\\
\frac{1}{\mathbb{I}_z}\tau_\psi
\end{pmatrix}
\end{equation}
Similarly, around the origin and hovering conditions, $\Psi(\eta)\approx\vec{1}$ for $\eta\approx\vec{0}$ then $\dot{\eta}\approx\omega_b$. Or in component form:
\begin{equation}
\begin{pmatrix}
\dot{p}\\
\dot{q}\\
\dot{r}
\end{pmatrix}
\approx
\begin{pmatrix}
\ddot{\phi}\\
\ddot{\theta}\\
\ddot{\psi}
\end{pmatrix}
\end{equation}
Because $\omega_b$ is small, the Coriolis cross-product term in Eq:\ref{.} is depreciated.
\begin{equation}
\begin{pmatrix}
rv-qw\\
pw-ru\\
qu-pv
\end{pmatrix}
\approx\vec{0}
\end{equation}
As such, the differential equations Eq:\ref{.} are then simplied to the following six SISO controllable plants:
\begin{subequations}
\begin{equation}
\ddot{x}=(-cos(\phi)sin(\theta)cos(\psi)-sin(\phi)sin(\psi)\frac{1}{m}F
\end{equation}
\begin{equation}
\ddot{y}=(-cos(\phi)sin(\theta)sin(\psi)+sin(\phi)cos(\psi))\frac{1}{m}F
\end{equation}
\begin{equation}
\ddot{z}=g-(cos(\phi)cos(\theta)\frac{1}{m}F
\end{equation}
\begin{equation}
\ddot{\phi}=\frac{1}{\mathbb{I}_x}\tau_\phi
\end{equation}
\begin{equation}
\ddot{\theta}=\frac{1}{\mathbb{I}_y}\tau_\theta
\end{equation}
\begin{equation}
\ddot{\psi}=\frac{1}{\mathbb{I}_z}\tau_\psi
\end{equation}
\end{subequations} %-----------------------------------------------------
\chapter{Design Bill of Materials}
\label{app:bom}
%-----------------------------------------------------
\section{Parts List}
%-----------------------------------------------------
\begin{table}[htbp]
\centering
\begin{tabularx}{\textwidth}{|X|l|l|}
\hline
\multicolumn{1}{|c|}{Part Name} & No. Used & Unit Weight[g]\\ \hline
\multicolumn{3}{|c|}{Electronics}\\ \hline
SPRacing F3 Deluxe Flight Controller & 1 & 8\\ \hline
OrangeRx 615X 2.4 GHz 6CH Receiver & 1 & 9.8\\ \hline
Signal Converter SBUS-PPM-PWM & 1 & 5.0\\ \hline 
STLink-V2 Debugger & 1 & 3\\ \hline
RotorStar Super Mini S-BEC 10A & 1 & 30\\ \hline
128x96" OLED Display & 1 & 7 \\ \hline
XBee-Pro S1 & 2 & 4 \\ \hline
HobbyWing XRotor 20A Opto ESC & 4 & 15\\ \hline
OrangeRX RPM Sensor & 4 & 2\\ \hline
HobbyKing Multi-Rotor Power Distribution Board & 1 & 49\\ \hline
\multicolumn{3}{|c|}{Motors}\\ \hline
Corona DS-339MG & 8 & 32\\ \hline
Cobra 2208 2000KV Brushlesss DC & 4 & 44.2\\ \hline
\multicolumn{3}{|c|}{Frame Components}\\ \hline
APM Flight Controller Damping Platform & 1 & 7\\ \hline
HobbyKing SK450 Replacement Arm (2 pcs) & 2 & 51\\ \hline
SK450 Extended Landing Skid & 1 & 23.25\\ \hline
Alloy Servo Arm (FUTABA) & 8 & 4\\ \hline
10X18X6 Radial Ball Bearing & 8 & 5\\ \hline
80g Damping Ball & 32 & $\approx 0$\\ \hline
Plastic Retainers for Damping Balls & 32 & $\approx 0$\\ \hline
3/5mm Aluminum Prop Adapter & 4 & $\approx 1$\\ \hline
6x4.5 Gemfam 3-Blade Propeller & 4 & 6\\ \hline
M3 6mm Hex Nylon Spacer & 8 & $\approx 0$\\ \hline
M3 16mm Hex Nylon Spacer & 32 & $\approx 0$\\ \hline
M3 25mm Nylon Screw & 128 & $\approx 0.08$\\ \hline
M2.5x10mm Socket Head Cap Screw & 36 & $\approx 0.2$\\ \hline
M2.5x25mm Socket Head Cap Screw & 20 & $\approx 0.6$\\ \hline
M2.5 A-Lok Nut & 16 & $\approx 0$\\ \hline 
\end{tabularx}
\label{tab:partslist}
\caption{Parts List}
\end{table}
%-----------------------------------------------------
\newpage
%-----------------------------------------------------
\newgeometry{left=1cm,right=1cm,top=2cm,bottom=2cm}
\begin{figure}[hbtp]
\vspace{-20pt}
\centering
\includegraphics[width=0.98\textwidth]{pdfpages/working.pdf}
\captionof{table}{3D Printed Parts}
\end{figure}
\restoregeometry
%-----------------------------------------------------
\newpage
%-----------------------------------------------------
\newgeometry{left=1cm,right=1cm,top=2cm,bottom=1cm}
\begin{table}[htbp]
\label{tab:damping-assemblies.a}
\centering
\begin{tabularx}{\textwidth}{|X|X|}
\hline
\multicolumn{2}{|c|}{Bracket Assemblies 2}\\
\hline
\begin{minipage}{0.5\textwidth}
\vspace{6pt}
\centering
\includegraphics[width=0.7\textwidth]{figs/appendix/assembly-inner-bearing}
\captionof{figure}{Bearing Bracket Inner Ring Assembly}
Parts: A.1, A.2
\end{minipage}
&
\begin{minipage}{0.5\textwidth}
\vspace{6pt}
\centering
\includegraphics[width=0.7\textwidth]{figs/appendix/assembly-inner-servo}
\captionof{figure}{Servo Bracket Inner Ring Assembly}
Parts: B.1, B.2, M3 Servo Horn
\end{minipage}
\\
\hline
\begin{minipage}{0.5\textwidth}
\vspace{6pt}
\centering
\includegraphics[width=0.7\textwidth]{figs/appendix/assembly-middle-servo-bracket}
\captionof{figure}{Servo Bracket Middle Ring Assembly}
Parts: C.1, C.2, C.3, M3 Servo Horn
\end{minipage}
&
\begin{minipage}{0.5\textwidth}
\vspace{6pt}
\centering
\includegraphics[width=0.7\textwidth]{figs/appendix/assembly-middle-bearing-holder}
\captionof{figure}{Bearing Holder Middle Ring Assembly}
Parts: D.1, D.2, D.3, D.4, 18-10 Bearing
\end{minipage}
\\
\hline
\begin{minipage}{0.5\textwidth}
\vspace{6pt}
\centering
\includegraphics[width=0.7\textwidth]{figs/appendix/assembly-middle-servo-mount}
\captionof{figure}{Servo Mount Middle Ring Assembly}
Parts: E.1, E.2, Corona Servo \& Fasteners
\end{minipage}
&
\begin{minipage}{0.5\textwidth}
\vspace{6pt}
\centering
\includegraphics[width=0.7\textwidth]{figs/appendix/assembly-middle-bearing-bracket}
\captionof{figure}{Bearing Shaft Middle Ring Assembly}
Parts: F.1, F.2, F.3
\end{minipage}
\\
\hline
\end{tabularx}
\caption{Inner \& Middle Ring Assemblies}
\end{table}
\restoregeometry
%-----------------------------------------------------
\newpage
%-----------------------------------------------------
% Table 2
\newgeometry{left=1cm,right=1cm,top=2cm,bottom=0.5cm}
\begin{table}[htbp]
\label{tab:damping-assemblies.b}
\centering
\begin{tabularx}{\textwidth}{|X|X|}
\hline
\multicolumn{2}{|c|}{Bracket Assemblies 2}\\
\hline
\begin{minipage}{0.5\textwidth}
\vspace{6pt}
\centering
\includegraphics[width=0.7\textwidth]{figs/appendix/assembly-damping-bearing}
\captionof{figure}{Bearing Holder Damping Assembly}
Parts: G.1, G.2, G.3, G.4, 18-10 Bearing, 80g 
\\
Damping Balls, Bearing Holder Damping Bracket
\end{minipage}
&
\begin{minipage}{0.5\textwidth}
\vspace{6pt}
\centering
\includegraphics[width=0.7\textwidth]{figs/appendix/assembly-damping-servo}
\captionof{figure}{Servo Mount Damping Assembly}
Parts: H.1, H.2, Corona Servo \& Fasteners, 80g Damping Balls, Servo Mount Damping Bracket
\end{minipage}
\\
\hline
\end{tabularx}
\caption{Damping Assemblies}
\end{table}
\par
\begin{table}[htbp]
\label{tab:damping-backets}
\centering
\begin{tabularx}{\textwidth}{|X|X|}
\hline
\multicolumn{2}{|c|}{Laser Cut Brackets}\\
\hline
\begin{minipage}{0.5\textwidth}
\vspace{6pt}
\centering
\includegraphics[width=0.8\textwidth]{figs/appendix/damping-bracket-servo}
\captionof{figure}{Servo Mount Damping Bracket}
\end{minipage}
&
\begin{minipage}{0.5\textwidth}
\vspace{6pt}
\centering
\includegraphics[width=0.8\textwidth]{figs/appendix/damping-bracket-bearing}
\captionof{figure}{Bearing Holder Damping Bracket}
\end{minipage}
\\
\hline
\begin{minipage}{0.5\textwidth}
\vspace{12pt}
\centering
\includegraphics[width=0.9\textwidth]{figs/appendix/damping-arm-mount}
\captionof{figure}{Arm Mount Damping Bracket}
\end{minipage}
&
\begin{minipage}{0.5\textwidth}
\vspace{6pt}
\centering
\includegraphics[width=0.6\textwidth]{figs/appendix/frame-assembly}
\captionof{figure}{Frame Brackets}
\end{minipage}
\\
\hline
\end{tabularx}
\caption{Laser Cut Damping Brackets}
\end{table}

\restoregeometry
%-----------------------------------------------------
\newpage
%-----------------------------------------------------
\newgeometry{left=1cm,right=1cm,top=2cm,bottom=1cm}
\vspace{-20pt}
\begin{figure}[hbtp]
\centering
\includegraphics[width=0.95\textwidth]{pdfpages/rings.pdf}
\captionof{table}{Laser Cut Parts}
\end{figure}
\restoregeometry
%-----------------------------------------------------
\newpage
%-----------------------------------------------------
\newgeometry{left=1cm,right=1cm,top=2cm,bottom=1cm}
\section{F3 Deluxe Schematic Diagram}
\label{app:deluxe-diagram}
{\centering
\fbox{
\begin{minipage}{0.9\textwidth}
\centering
\includegraphics[width=\textwidth]{pdfpages/deluxe-schematic.pdf}
\end{minipage}
}
\captionof{figure}{F3 Deluxe Flight Controller Hardware Schematic}
}
\restoregeometry
%-----------------------------------------------------
\newpage
%-----------------------------------------------------
\chapter{System ID Test Data}
\label{app:systemdat}
%-----------------------------------------------------
\section{Servo Data}
%-----------------------------------------------------
\newpage
%-----------------------------------------------------
\newgeometry{left=1cm,right=1cm,top=2cm,bottom=1cm}
\section{Cobra CM2208-200KV Thrust Data}
\label{app:cobra-test}
\centering
\fbox{
\begin{minipage}{0.9\textwidth}
\centering
\includegraphics[width=\textwidth]{pdfpages/cobra-test.pdf}
\end{minipage}
}
\captionof{figure}{Official Test Results for Cobra Motors}
\restoregeometry
%-----------------------------------------------------
\newpage
%-----------------------------------------------------
\chapter{Inertias}
\label{app:eq}
\begin{subequations}
\begin{equation} \label{eq:app.inertia}
\mathbb{I}_{inner}
\end{equation}
\vspace{-5pt}
\begin{equation}
\mathbb{I}_{middle}
\end{equation}
\begin{equation}
\mathbb{I}_{body}
\end{equation}
\end{subequations}


%=====================================================================
%	BACK MATTER
%=====================================================================

\backmatter

%---------------------------------------------------------------------
%	BIBLIOGRAPHY
%---------------------------------------------------------------------

\bibliographystyle{plain}
\bibliography{thesis}

%---------------------------------------------------------------------
\end{document}
%---------------------------------------------------------------------
%	End of the document
%=====================================================================
