%****************************************************
%	CHAPTER 3 - Dynamics
%****************************************************
\chapter{Kinematics \& Dynamics}
\label{ch:dynamics}
%====================================================
The body's dynamics are first treated as rigid, with appropriate equations derived for generic 6-DOF motion. There after, non-linear aerodynamic and inertial effects, unique to multi-body relative rotations, are investigated and incorporated into the plant's model. Finally a consolidated, quaternion based plant model is presented which is used for the later control plant development next in Chapter:\ref{ch:control}.
%====================================================
\section{Rigid Body Dynamics}
\label{sec:dynamics.rigidbody}
%====================================================
\subsection{Lagrange Derivation}
\label{subsec:dynamics.rigidbody.lagrange}
%====================================================
Fundamentally any body, rigid or otherwise, can undergo two kinds of movements, namely rotational and translation motions. Often a Lagrangian\cite{classicaldynamics,rotationrigidbody} approach for combined angular and translational movements is used to derive the differential equations of motion for each degree of freedom. The Lagrangian principle ensures that translational and rotational kinematic energies and potential energy are conserved throughout the system's trajectory progression. When combined with Euler-Rotational equations, the Euler-Lagrangian\cite{lagrange-formalism} formulation fully defines the aerospace 6-DOF equation set.
\par
Lagrangian formalism is regarded as especially useful in non-cartesian (\emph{spherical etc\ldots}) co-ordinate frames or multi-body systems. With that being said, a cartesian co-ordinate system was already defined in Section:\ref{subsec:proto.conventions.motoraxis}. Rigid body dynamics in a Cartesian co-ordinate frame do lend themselves to Newtonian mechanics. The Newton-Euler or Euler-Lagrange formulations both stipulate the same resultant differential equations of motion. The Lagrangian operator, $\mathcal{L}$, is a term made up of the difference between kinetic and potential energies, $T$ and $U$ respectively. Considering some generalized path co-ordinates $\mathbf{r}(t)$, for both linear $\mathcal{E}$ and angular $\eta$ relative positions;
\begin{equation}\label{eq:generalpath}
\mathbf{r}(t)=\begin{bmatrix}
\mathcal{E} & \eta
\end{bmatrix}^T
\end{equation}
The co-ordinates in Eq:\ref{eq:generalpath} are generalized here, despite being symbols commonly used to represent linear and angular positions. The generalized co-ordinates are later be refined as Cartesian body co-ordinates with respect to the inertial frame. The Lagrangian is, by definition:
\begin{subequations}
\begin{equation}\label{eq:lagrangian.a}
\mathcal{L}(\mathbf{r},\dot{\mathbf{r}},t)=T(\mathbf{r},\dot{\mathbf{r}})-U(\mathbf{r},\dot{\mathbf{r}})
\end{equation}
Introducing a rigid body's general kinetic (angular \& linear) and potential energies, the latter being only gravitational\footnote{Here $G=\begin{bmatrix}
0&0&-9.81
\end{bmatrix}^T~m.s^{-2}$~in the Inertial frame,$\in\mathcal{F}^I$} potential energy in this case:
\begin{equation}\label{eq:lagrangian.b}
\mathcal{L}=\frac{1}{2}\dot{\mathcal{E}}^{T}(m)\dot{\mathcal{E}}+\frac{1}{2}\dot{\eta}^T(\mathbb{I})\dot{\eta}-mGz
\end{equation}
\end{subequations}
\newpage
Noting that $\mathbb{I}$ is the inertial tensor of the body aligned w.r.t to whichever generalized reference frame is used. The Euler-Lagrange formulation equates partial derivatives of the Lagrangian to any generalized forces, $\mathbf{V}$, acting on the system. In this case the generalized forces or, more specifically a net force $F_{net}$ and a net torque $\tau_{net}$.
\begin{equation}\label{eq:euler-lagrange}
\frac{d}{dt}\bigg(\frac{\delta L}{\delta \dot{\mathbf{r}}}\bigg)-\frac{\delta L}{\delta \mathbf{r}} = \mathbf{V} = \begin{bmatrix}
F_{net}\\
\tau_{net}
\end{bmatrix}
\end{equation}
And taking the partial derivatives of Eq:\ref{eq:lagrangian.b} with respect to the path co-ordinates $\mathbf{r}$:
\begin{subequations}
\begin{equation}\label{eq:partial.a}
\frac{\delta L}{\delta \mathbf{r}}=\begin{bmatrix}
mG_x\\
0
\end{bmatrix}
\end{equation}
\vspace{-5pt}
\begin{equation}\label{eq:partial.b}
\frac{d}{dt}\bigg(\frac{\delta L}{\delta \dot{\mathbf{r}}}\bigg)=\bigg[
m\frac{d}{dt}\dot{\mathcal{E}} ~~~ \mathbb{I}\frac{d}{dt}\dot{\eta}\bigg]^T
\end{equation}
\end{subequations}
Where $G_x$ is the gravitation force in whichever reference frame ($\mathcal{F}^x$) the lagrangian is with respect to. In any generalized coordinate system a rotating vector's time derivative which, according to the Reynolds Transportation Theorem\cite{reynolds,conservationequations}, is given by:
\begin{equation}\label{eq:reynolds}
\frac{d\vec{f}}{dt_a}=\frac{d\vec{f}}{dt_b}+\vec{\omega}_{a/b}\times\vec{f}
\end{equation}
So applying that theorem (Eq:\ref{eq:reynolds}) to the partial derivatives in Eq:\ref{eq:partial.b} and further defining the generalized co-ordinates as cartesian body coordinates with respect to an inertial origin (the body frame $\mathcal{F}^b$). Noting that in Eq:\ref{eq:partial.b} the place holders used for linear ($\mathcal{E}$) and angular positions ($\eta$) are in a common shared frame\footnote{In this case $\eta\not=[\phi~\theta~\psi]^T$ seeing that angular position $\eta$ is defined in a common frame. $\eta$ is \underline{NOT} an Euler angle set.}, and hence $[ \dot{\mathcal{E}}~~\dot{\eta} ]^T\equiv [ \nu ~~ \omega]^T\in \mathcal{F}^b$. It then follows that Lagrangian will change:
\begin{subequations}
\begin{equation}
\mathcal{L}=\frac{1}{2}\nu^{T}(m)\nu+\frac{1}{2}\omega^T(\mathbb{I})\omega -mG_b z
\end{equation}
\vspace{-5pt}
\begin{equation}
\frac{d}{dt}\bigg(\frac{\delta L}{\delta \dot{\mathbf{r}}}\bigg)=\bigg[
m\frac{d}{dt}\nu ~~~ \mathbb{I}\frac{d}{dt}\omega\bigg]^T
\end{equation}
\vspace{-5pt}
\begin{equation}
\rightarrow m\frac{d}{dt}\nu_b=m\dot{\nu}+\vec{\omega}_{I/b}\times\nu
\end{equation}
\vspace{-5pt}
\begin{equation}
\rightarrow \mathbb{I}_b \frac{d}{dt}\omega_b=\mathbb{I}_b\dot{\omega}+\vec{\omega}_{I/b}\times\mathbb{I}_b\omega
\end{equation}
\end{subequations}
Which, when substituted back into the Euler-Lagrange formulation Eq:\ref{eq:euler-lagrange}, results in the familiar Newton-Euler equations for linear and angular differentials, both in the body frame;
\begin{subequations}\label{eq:newton}
\begin{equation}\label{eq:newton.a}
F_{net}=m\dot{\nu}+\omega_b\times m \nu - m\mathbb{R}_I^b(-\eta) G
\end{equation}
\vspace{-15pt}
\begin{equation}\label{eq:newton.b}
\tau_{net}=\mathbb{I}_b\dot{\omega}_b+\omega_b\times\mathbb{I}_b\omega
\end{equation}
\end{subequations}
It's important to recall that $\omega_b\not= \dot{\eta}$ where $\eta=[\phi~\theta~\psi]^T$, seeing that Euler Angles are defined in sequentially rotated reference frames. So then four differential equations are often used to completely describe the entire set of state derivatives, namely:
\begin{subequations}\label{eq:states}
\begin{equation}\label{eq:states.a}
\dot{\mathcal{E}}=\mathbb{R}_b^I(-\eta)\nu~~~~\in\mathcal{F}^I
\end{equation}
\vspace{-15pt}
\begin{equation}\label{eq:states.b}
F_{net}=m\dot{\nu}+\omega_b\times m\nu -m \mathbb{R}_I^b(-\eta) G ~~~~\in\mathcal{F}^b
\end{equation}
\vspace{-10pt}
\begin{equation}\label{eq:states.c}
\dot{\eta}=\Psi(\eta)\omega_b~~~~\in\mathcal{F}^{v2},\mathcal{F}^{v1},\mathcal{F}^I
\end{equation}
\vspace{-10pt}
\begin{equation}\label{eq:states.d}
\tau_{net}=\mathbb{I}_b\omega_b+\omega_b\times\mathbb{I}_b\omega~~~~\in\mathcal{F}^b
\end{equation}
\end{subequations}
The state differentials in Eq:\ref{eq:states} can be reduced to a set of two equations. Those differentials are defined only with respect to the reference frames of the state variables which they represent. The non-linear form of those equations substitutes\footnote{Originally introduced in Eq:\ref{eq:angular-rates.e}} $\dot{\eta}=\Phi(\eta)\omega_b$ in the Lagrangian derivative, Eq:\ref{eq:partial.b}.
\begin{equation}
\frac{d}{dt}\bigg(\frac{\delta L}{\delta \dot{\mathbf{r}}}\bigg)=\bigg[m\frac{d}{dt}\nu~~~\mathbb{I}\frac{d}{dt}\dot{\eta}\bigg]^T=\bigg[m\frac{d}{dt}\nu~~~\mathbb{I}\frac{d}{dt}\Phi(\eta)\omega_b\bigg]^T
\end{equation}
This only changes the angular component, and so applying the differential chain rule:
\begin{equation}
\mathbb{I}\frac{d}{dt}\Phi(\eta)\dot{\omega}_b=\mathbb{I}\big(\Phi\dot{(\eta)}\omega_b+\Phi(\eta)\dot{\omega}_b \big)
\end{equation}
Drawing from \cite{autonomousrobotseuler} and recognizing that $\mathbb{I}$ must be transformed to common axes, $\mathbb{J}=\Psi(\eta)^T\mathbb{I}\Psi(\eta)$. The controllable differential equations in Eq:\ref{eq:newton}, then in the inertial frame for force and intermediate Euler frames for each angle becomes\footnote{The relationship $\dot{\Phi}=\Phi\dot{\Psi}\Phi$ was used to simplify Eq:\ref{eq:nonlinear}, the singularity in $\Phi$ still remains\ldots}:
\begin{subequations}\label{eq:nonlinear}
\begin{equation}
M(\eta)\ddot{\eta}+C(\eta,\dot{\eta})\dot{\eta}=\Psi(\eta)\tau
\end{equation}
\vspace{-10pt}
\begin{equation}
M(\eta)=\Psi(\eta)^T\mathbb{I}\Psi(\eta)
\end{equation}
\vspace{-10pt}
\begin{equation}
C(\eta,\dot{\eta})=-\Psi(\eta)\mathbb{I}\Psi\dot{(\eta)}+\Psi(\eta)^T sk(\Psi(\eta)\dot{\eta})\mathbb{I}\Psi(\eta)
\end{equation}
\end{subequations}
Equation \ref{eq:nonlinear} fully describes the state derivative $\ddot{\eta}$ in its own frame(s). The two differential equations which describe the entire bodies motion are then:
\begin{subequations}
\begin{equation}
F=m\dot{\mathcal{E}}+\mathbb{R}_b^I(-\eta)\omega_b \times m \dot{\mathcal{E}}-mG~~~~\in\mathcal{F}^I
\end{equation}
\vspace{-10pt}
\begin{equation}
\Psi(\eta)\tau=M(\eta)\ddot{\eta}+C(\eta,\dot{\eta})~~~~\in\mathcal{F}^{v2,v1,I}
\end{equation}
\end{subequations}
\par
The generalized forces effecting the system, $F(u)$ and $\tau (u)$, are the system's controllable inputs and are going to be affected directly the systems actuators and their associated effectiveness function. In the general case, which is expanded upon in Section:\ref{sec:dynamics.aero}, the control inputs are typically as follows:
\begin{subequations}
The net force acting on the system is just the sum of all thrust vectors produced by rotating propellers, $T(\Omega_i)$.
\begin{equation}
\mu F = \sum \vec{T}_i
\end{equation}
Secondly the net torque is the sum of all torque arms produced from those propeller thrust vectors.
\begin{equation}
\mu \tau = \sum \vec{l}_i \times \vec{T}_i
\end{equation}
\end{subequations}
Where $\vec{T}_i$ is the $i^{th}$ motor's thrust vector, not necessarily in 3 dimensions and typically bound to the $\hat{Z}_b$ axis, $\vec{l}_i$ is that motor's torque arm. The above equations are still applicable to any 6 DOF body, common simplifications applied to the system(s) for quadrotor control are explored in Appendix:\ref{app:stddynamics}. Aspects unique to the quadrotor and tilting quadrotor platforms are now introduced\ldots
%====================================================
\subsection{Rotation Matrix Singularity}\label{subsec:dynamics.rigidbody.singularity}
%====================================================
The Euler Angle singularity is often noted but infrequently proven mathematically, far less common is the demonstration of exactly how that singularity \emph{mathematically} manifests itself. By definition, a singularity occurs when a loss of differentiability exists. In the case of an affixed 3-axis gimbal, when an intermediate axis, for example the rolling angle $\theta$, is at $\pi/2$ then the remaining two axes become co-linear. That being a pitch $\phi$ or yaw $\psi$ rotations will subsequently have the same effect. Such a situation results in the loss of a degree of freedom.
\begin{figure}
\begin{subfigure}{0.5\textwidth}
\centering
\includegraphics[width=\textwidth]{figs/gimbal}
\caption{3-Axis gimbal}
\end{subfigure}
\begin{subfigure}{0.5\textwidth}
\centering
\includegraphics[width=\textwidth]{figs/gimbal-lock}
\caption{Locked gimbal with loss of DOF}
\end{subfigure}
\caption{Gimbal lock}
\end{figure} 
\par
What is obvious in a physical system is not necessarily as clear mathematically. A clear loss of differentiability is manifested in the Euler Matrix $\Psi(\eta)$, Eq:\ref{eq:angular-rates.e} from Section:\ref{subsec:proto.conventions.frames}. The relation between angular velocity, in the inertial frame or inversely in the body frame, and the angular rates of the Euler Angles.
\begin{equation}\label{eq:euler-derivative}
\begin{bmatrix}
\dot{\phi}\\
\dot{\theta}\\
\dot{\psi}
\end{bmatrix}
=\begin{bmatrix}
1 & sin(\phi)tan(\theta) & cos(\phi)tan(\theta)\\
0 & cos(\phi) & -sin(\phi)\\
0 & sin(\phi)sec(\theta) & cos(\phi)sec(\theta)\\
\end{bmatrix}
\begin{bmatrix}
p\\
q\\
r
\end{bmatrix}
=\Phi(\eta)\omega_b
\end{equation}
\begin{equation}
\text{As}~\underset{{\theta \rightarrow \pi /2}}{lim}~sec(\theta)\rightarrow \infty
\end{equation}
Or that $\Phi(\eta)$ is undefined at $\theta=\pi/2$. 
It's clear to see that in Eq:\ref{eq:euler-derivative} there exists an undefined singularity as $\theta\rightarrow\pi/2$. The physical consequence of this is the loss of a degree of freedom. More specifically, if one looks at how the rotation (or transformation) matrices are formulated:
\begin{subequations}
\begin{equation}
\mathbb{R}_I^b = \mathbb{R}_z\mathbb{R}_y\mathbb{R}_x=\begin{bmatrix}
c_\psi & -s_\psi & 0\\
s_\psi & c_\psi & 0\\
0 & 0 & 1
\end{bmatrix}
\begin{bmatrix}
c_\theta & 0 & s_\theta\\
0 & 1 & 0\\
-s_\theta & 0 & c_\theta
\end{bmatrix}
\begin{bmatrix}
1 & 0 & 0\\
0 & c_\phi & -s_\phi\\
0 & s_\phi & c_\phi
\end{bmatrix}
\end{equation}
\begin{equation}
\mathbb{R}_I^b=\begin{bmatrix}
c_\psi c_\theta & c_\psi s_\theta s_\phi - s_\psi c_\phi & c_\psi s_\theta c_\phi + s_\psi s_\phi\\
s_\psi c_\theta & s_\psi s_\theta s_\phi + c_\psi c_\phi & s_\psi s_\theta  c_\phi - c_\psi s_\phi\\
-s_\theta & c_\theta s_\phi & c_\phi c_\theta\\
\end{bmatrix}
\end{equation}
In the case where $\theta=\pi/2$;
\begin{equation}\label{eq:gimbal}
=\begin{bmatrix}
0 & c_\psi s_\phi - s_\psi c_\phi & c_\psi c_\phi + s_\psi s_\phi\\
0 & s_\psi s_\phi + c_\psi c_\phi & s_\psi c_\phi - c_\psi s_\phi\\
-1 & 0 & 0\\
\end{bmatrix}
=
\begin{bmatrix}
0 & s(\phi - \psi) & c(\phi - \psi)\\
0 & c(\phi - \psi) & s(\phi - \psi)\\
-1 & 0 & 0
\end{bmatrix}=\mathbb{R}_{x'}(\phi-\psi)
\end{equation}
\end{subequations}
Where the outcome in Eq:\ref{eq:gimbal} represents an x-axis rotation in a new intermediate frame, post a 90\textdegree ~rotation in the y-axis. Through trigonometric double angles a degree of freedom is lost at $\theta=\pi/2$, when $\phi$ \& $\psi$ effect the same angle.
%====================================================
\subsection{Quaternion Dynamics}
\label{subsec:dynamics.rigidbody.quaternion}
%====================================================
An algorithm proposed in \emph{How To Avoid a Singularity When Using Euler Angles?}\cite{euleranglesingularity} suggested a solution the problem of the Euler Angle singularity. The proposed heuristic was to switch between sequencing conventions (ZYX,ZYZ etc\ldots there are 12 in total) such that the singularity is always avoided. However the implementation of such an algorithm is cumbersome and inefficient. Far more elegant is the use of \emph{quaternion}, $\mathbb{R}^4$, attitude representations (\cite{rotationsequences,quaterniondynamics,spacecraftattitutdequaternions} amongst others\ldots).
\par
A quaternion is analogous to a rotation matrix in that it represents an attitude difference between two reference frames. Without deliberating too much on their proof or details, a quaternion consists or a scalar component, $q_0$, and complex vector component, $\vec{q}\in \mathbb{C}^3$, such that;
\begin{equation}
Q\triangleq 
\begin{bmatrix}
q_0 \\
\vec{q}
\end{bmatrix}
~~\in\mathbb{R}^4
\end{equation}
Any and all quaternions, unless otherwise stated, in this dissertation are all unit quaternions\footnote{Unit quaternions are a subset of the quaternion space}, $Q\in\mathbb{Q}_u$. The need for quaternions with unity magnitude is such to ensure rotational operators don't affect the magnitude of the vector operand. A unit quaternion is defined as:
\begin{equation}
||Q||=\sqrt{{q_0}^2+\vec{q}~^2}=1
\end{equation}
Quaternion multiplication is distributive and associative, but not commutative. Specifically a quaternion multiplciation operation is equivalent to the Hamilton product. For two quaternions, $Q$ \& $P$:
\begin{subequations}
\begin{equation}
Q\otimes P = \begin{bmatrix}
q_0 & \vec{q}
\end{bmatrix}
\otimes
\begin{bmatrix}
p_0 & \vec{p}
\end{bmatrix}
\end{equation}
\vspace{-10pt}
\begin{equation}
=q_0 p_0 - \vec{q}\cdot \vec{p}+p_0 \vec{q} + q_0 \vec{p} + \vec{q}\times\vec{p}
\end{equation}
\end{subequations}
Seeing that the vector component of a quaternion is complex valued, it is natural that there exists a quaternion conjugate property. Namely:
\begin{equation}
Q^*=\begin{bmatrix}
q_0 & -\vec{q}
\end{bmatrix}
\end{equation}
It then follows that\footnote{Disambigation:$\mathbb{I}$ in this context is a $4\times 4$ identity matrix, not an inertial matrix}:
\begin{equation}
Q\otimes Q^* = \mathbb{I}_{4\times 4}
\end{equation}
To apply quaternion rotations to a vector ($\vec{v} \in\mathbb{R}^3$) involves multiplication by two unit quaternions. 
\begin{equation}
\begin{bmatrix}
0 & \vec{v}~'
\end{bmatrix}
=Q\otimes
\begin{bmatrix}
0 & \vec{v}
\end{bmatrix}
\otimes Q^*
\end{equation}
Mostly, the zero scalar components are omitted in a rotation (\emph{or transformation}) operation, such that it is recognized the vector operands are substituted with quaternions.
\begin{equation}
\vec{v}~'=Q \otimes (\vec{v}) \otimes Q^*
\end{equation} 
In the case of rigid body attitude representation, $Q_b$ is the quaternion which represents the difference between $\mathcal{F}^I$ and $\mathcal{F}^b$. A quaternion operator is equivalent to a rotation matrix operation:
\begin{equation}
\mathbb{R}_I^b \underset{Q}{\iff} Q_b \otimes (.) \otimes Q_b^*
\end{equation}
A quaternion time derivative, with $Q_\omega$ being a quaternion with a vector component equal to angular velocity $\omega\in\mathcal{F}^b$ and a zero scalar component, is given by:
\begin{equation}
\frac{d}{dt}Q_b=\frac{1}{2}Q_b\otimes Q_{\omega}=\begin{bmatrix}
-\frac{1}{2}\vec{q}^{~T} \vec{\omega}_b\\
\frac{1}{2}\big([\vec{q}~]_\times+q_0\mathbb{I}\big)\vec{\omega}_b
\end{bmatrix}
\end{equation}
\newpage
Using quaternions to represent attitudes negates the need for an Euler Matrix, $\Phi(\eta)$, to represent attitudes and their rates. A body quaternion is fully defined in the body frame. The first quaternion time derivative replaces Eq:\ref{eq:states.a}\& Eq:\ref{eq:states.c};
\begin{subequations}
\begin{equation}
\dot{\mathcal{E}}=\mathbb{R}_b^I(-\eta)\vec{\nu}\underset{Q}{\iff}Q_b\otimes\vec{\nu}\otimes Q_b^*
\end{equation}
\vspace{-10pt}
\begin{equation}
\dot{\eta}=\Phi(\eta)\vec{\omega}_b\underset{Q}{\iff}\dot{Q}=\frac{1}{2}Q_b\otimes Q_\omega
\end{equation}
\end{subequations}
Second order quaternion derivatives, quaternion accelerations aren't as useful as their velocity counterparts. The second order derivative is mentioned here however it's only relevant to quaternion backstepping in the control section. If possible, quaternion accelerations are avoided due to their complexity;
\begin{equation}
\ddot{Q}\big(\dot{Q},Q,t)=\dot{Q}\otimes Q^* \otimes \dot{Q}+\frac{1}{2}Q\otimes \big[\mathbb{I}_b^-1(\tau-4(Q^*\otimes \dot{Q})\times(\mathbb{I}_b(Q^*\otimes \dot{Q}))\big]
\end{equation}
%====================================================
\subsection{Quaternion Unwinding}
\label{subsec:dynamics.rigidbody.unwinding}
%====================================================
Although quaternions are better than Euler angles and lack the associated singularity, they do contain one caveat. Seeing that a quaternion $Q=[q_0~\vec{q}]^T$ represents a rotation in $\mathbb{R}^3$ using $\mathbb{R}^4$ variables there is whats called dual coverage\cite{unwinding}.
Each unit quaternion, stemming from Euler-Rodriguez theorem, is parameterised such that the quaternion operation represents a single rotation of $\theta$ about an axis $\hat{u}$ such that:
\begin{equation}
Q=\begin{bmatrix}
q_0 & \vec{q}
\end{bmatrix}=\pm
\begin{bmatrix}
cos(\theta/2) & sin(\theta/2)\hat{u}
\end{bmatrix}
\end{equation}
With a unit vector $\hat{u}$. Quaternions relate to an euler angle rotation matrix through the Rodriguez formula:
\begin{equation}
\mathbb{R}(Q)=\mathbb{I}+2q_0[\vec{q}]_\times+2[\vec{q}]^2_\times
\end{equation}
As a result its clear to see that for each unique attitude in 3-Dimensions there exist two quaternions which correlate to the same position. Namely:
\begin{equation}
Q = \begin{bmatrix}
cos(\theta/2) & sin(\theta/2)
\end{bmatrix}
.
\end{equation}
%****************************************************
\section{Non-linearities}
\label{sec:dynamics.nonlinearities}
%****************************************************
\subsection{Gyroscopic Torques}
\label{subsec:dynamics.nonlinearities.gyrotorques}
%****************************************************
\subsection{Coriolis Acceleration}
\label{subsec:dynamics.nonlinearities.coriolis}
%****************************************************
\subsection{Inertial Matrix}
\label{subsec:dynamics.nonlinearities.inertia}
%****************************************************

%****************************************************
\section{Aerodynamics}
\label{sec:dynamics.aero}
%****************************************************
\subsection{Thrust Forces \& Propeller Torques}
\label{subsec:dynamics.aero.bem}
%****************************************************
\subsection{Drag}
\label{subsec:dynamics.aero.drag}
%****************************************************
\subsection{Conning \& Flapping}
\label{subsec:dynamics.aero.flap}
%****************************************************
\subsection{Vortex Ring State}
\label{subsec:dynamics.aero.vrs}
%****************************************************

%****************************************************
\section{Consolidated Model}
\label{sec:dynamics.model}